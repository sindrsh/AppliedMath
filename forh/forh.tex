\input{/home/sindre/P/doc}
\usepackage[T1]{fontenc}
\usepackage[utf8]{luainputenc}
\usepackage{lmodern} % load a font with all the characters
\usepackage{geometry}
\geometry{verbose,paperwidth=16.1 cm, paperheight=24 cm, inner=2.3cm, outer=1.8 cm, bmargin=2cm, tmargin=1.8cm}
\setlength{\parindent}{0bp}
\usepackage{import}
\usepackage[subpreambles=false]{standalone}
\usepackage{amsmath}
\usepackage{amssymb}
\usepackage{esint}
\usepackage{babel}
\usepackage{tabu}
\usepackage[dvipsnames, table]{xcolor}
\makeatother
\makeatletter


%referances
\newcommand{\net}[2]{{\color{blue}\href{#1}{#2}}}

%Spaces
\newcommand{\vsk}{\\[12pt]}
\newcommand{\vs}{\vspace{-12pt}}

% Tabell for opplegg

\newcommand{\ovlist}[1]{
\vspace{-16pt}
\begin{itemize}
	#1
\end{itemize}
}

\newcommand{\lst}[5]{
\rule{\linewidth}{1pt}
\footnotesize
	\textbf{Øvingsområde}\\ #1 
	
	\textbf{Utstyr}\\ #2  \\
	
	\begin{tabular}{@{} p{4cm} l} 
		\textbf{Tid} & \textbf{Elevinndeling} \\
		#3  & #4
	\end{tabular} 

\rule{\linewidth}{1pt}	\vsk
\normalsize
	\textbf{Gjennomføring}\\ #5 \vsk
}
%

\newcounter{opl}
%\numberwithin{opl}{article}

\newcommand{\opl}[1]{
\newpage
{\refstepcounter{opl} %\phantomsection 
\large \textbf{\theopl \;#1} \vsk}
}

% Headlines
\newcommand{\fork}{\textbf{Forkunnskapar}\\}
\newcommand{\forb}{\textbf{Forberedelsar}\\}
\newcommand{\opgvr}{\textbf{Oppgaver}}

\usepackage{datetime2}
\usepackage[]{hyperref}

\begin{document}
\section{Størrelser, enheter og prefikser}
Det vi kan måle og uttrykke med tall, kaller vi \textit{størrelser}. Videre har vi \textit{størrelser med dimensjoner} og \textit{dimensjonsløse størrelser}.\vsk

Et eksempel på en størrelse med dimensjon er ''2 meter''. Dimensjonen er da 'lengde', som vi gjerne måler i meter. Vi sier at meter er en \textit{enhet} for dimensjonen lengde.\vsk

Et eksempel på en størrelse uten dimensjon er ''to hester''. Mens det bare finnes én lengde som er ''2 meter'', ''to hester'' se veldig forskjellig ut, avhengig av hvile to hester det er snakk om.\vsk

\textbf{Regning med dimensjoner}\os
Når vi regner med størrelser med dimensjoner må vi passe på at alle enhetene er like, hvis ikke gir ikke regnestykkene våre mening. I denne boka skal vi se på disse enhetene:
\tbs
\begin{center}
	\begin{tabular}{c|c}
		\textbf{Enhet} & \textbf{Forkortelse} \\ \hline
		meter & m \\\hline
		gram & g \\\hline
		liter & L
	\end{tabular}
\end{center}\tbs
Noen ganger har vi veldig store eller veldig små størrelser, for eksempel er det ca 40\,075\,000\,m rundt ekvator! For så store tall er det vanlig å bruke en \textit{prefiks}, da kan vi si at det er ca 40\,075 km rundt ekvator. Her står 'km' for 'kilometer' og 'kilo' betyr '1\,000'. Så 1\,000 meter er altså 1 kilometer. Her er de viktigste prefiksene:\tbs
\begin{center}
	\begin{tabular}{c|c|c}
		\textbf{Prefiks} & \textbf{Forkortelse}&\textbf{Betydning} \\ \hline
		kilo & k & 1\,000\\\hline
		hekto & h & 100\\\hline
		deka & da & 10\\\hline
		desi & d & 0,1\\\hline
		centi & c & 0,01\\\hline
		milli & m & 0,001\\\hline		
	\end{tabular}
\end{center}\tbs
Bruker vi denne tabellen i kombinasjon med enhetene kan vi for eksempel se at:\vs
\alg{
1000\enh{g}&= 1\enh{kg} \\
0,1 \enh{m} &= 1\enh{dm} \\
0,01 \enh{L} &= 1\enh{cL}
}
Enda ryddigere kan vi få det hvis vi lager en vannrett tabell, med meter, gram eller liter lagt til i midten:
\begin{center}
	\begin{tabular}{|c|c|c|c|c|c|c|c}
		kilo &
		hekto &
		deka & m/g/L &
		desi & 
		centi & 
		milli & 		
	\end{tabular}
\end{center}
Vi har sett hvordan prefiksene egentlig bare betyr et tall, og m, g eller L kan vi si har et 1-tall foran seg ($ {4\cdot1\enh{m}} $ er jo det samme som $ 4\enh{m} $). Vi kan da legge merke til at for å komme fra én rute til en annen i tabellen, er det bare snakk om å flytte komma:
\reg[Omgjøring av prefiks \label{ompref}]{Når vi skal endre prefikser kan vi bruke denne tabellen:
\begin{center}
	\begin{tabular}{|c|c|c|c|c|c|c|c}
		kilo &
		hekto &
		deka & m/g/L &
		desi & 
		centi & 
		milli & 		
	\end{tabular}
\end{center}
Komma må flyttes like mange ganger som antall bokser vi må flytte oss fra opprinnelig prefiks til ny prefiks.\vsk

\textsl{Obs!} For lengde brukes også enheten 'mil' (1 mil er 10\,000\,m). Denne kan legges på til venstre for 'kilo'.
}
\eks[1]{
Gjør om 23,4\,mL til L.

\sv
Vi skriver tabellen vår med L i midten og legger merke til at vi må \textsl{tre bokser til venstre} for å komme oss fra mL til L:
\begin{center}
	\begin{tabular}{|c|c|c|c|c|c|c|c}
		kilo &
		hekto &
		deka & \color{blue}L &
		desi & 
		centi & 
		\color{red} milli & 		
	\end{tabular}
\end{center}
Dét betyr at vi må flytte kommaet vårt tre plasser til venstre for å gjøre om mL til L:
\[ 23,4\enh{mL}=0,0234\enh{L} \]
}
\eks[2]{
Gjør om 30\,hg til cg.

\sv
Vi skriver tabellen vår med g i midten og legger merke til at vi må \textsl{fire bokser til høyre} for å komme oss fra hg til cg:
\begin{center}
	\begin{tabular}{|c|c|c|c|c|c|c|c}
		kilo &
		\color{red}hekto &
		deka & g &
		desi & 
		\color{blue}centi & 
		 milli & 		
	\end{tabular}
\end{center}
Dét betyr at vi må flytte kommaet vårt fire plasser til høyre for å gjøre om hg til cg:
\[ 30\enh{mg}=300\,000\enh{cg} \]
}
\eks[3]{
Gjør om 12\,500\,dm til mil.

\sv
Vi skriver tabellen vår med m i midten, legger til 'mil', og merker oss at vi må \textsl{fem bokser til høyre} for å komme oss fra hg til cg:
\begin{center}
	\begin{tabular}{|c|c|c|c|c|c|c|c|c}
		\color{blue}mil &kilo &
		hekto &
		deka & m &
		\color{red} desi & 
		centi & 
		milli & 		
	\end{tabular}
\end{center}
Dét betyr at vi må flytte kommaet vårt fem plasser til høyre for å gjøre om mil til cg:
\[ 30\enh{dm}=3\,000\,000\enh{mil} \]
\textsl{Merk:} 'mil' er en egen enhet, ikke en prefiks. Vi skriver derfor ikke 'milm', men bare 'mil'.
}



\end{document}

