\input{doc}
\usepackage[T1]{fontenc}
\usepackage[utf8]{luainputenc}
\usepackage{lmodern} % load a font with all the characters
\usepackage{geometry}
\geometry{verbose,paperwidth=16.1 cm, paperheight=24 cm, inner=2.3cm, outer=1.8 cm, bmargin=2cm, tmargin=1.8cm}
\setlength{\parindent}{0bp}
\usepackage{import}
\usepackage[subpreambles=false]{standalone}
\usepackage{amsmath}
\usepackage{amssymb}
\usepackage{esint}
\usepackage{babel}
\usepackage{tabu}
\usepackage[dvipsnames, table]{xcolor}
\makeatother
\makeatletter


%referances
\newcommand{\net}[2]{{\color{blue}\href{#1}{#2}}}

%Spaces
\newcommand{\vsk}{\\[12pt]}
\newcommand{\vs}{\vspace{-12pt}}

% Tabell for opplegg

\newcommand{\ovlist}[1]{
\vspace{-16pt}
\begin{itemize}
	#1
\end{itemize}
}

\newcommand{\lst}[5]{
\rule{\linewidth}{1pt}
\footnotesize
	\textbf{Øvingsområde}\\ #1 
	
	\textbf{Utstyr}\\ #2  \\
	
	\begin{tabular}{@{} p{4cm} l} 
		\textbf{Tid} & \textbf{Elevinndeling} \\
		#3  & #4
	\end{tabular} 

\rule{\linewidth}{1pt}	\vsk
\normalsize
	\textbf{Gjennomføring}\\ #5 \vsk
}
%

\newcounter{opl}
%\numberwithin{opl}{article}

\newcommand{\opl}[1]{
\newpage
{\refstepcounter{opl} %\phantomsection 
\large \textbf{\theopl \;#1} \vsk}
}

% Headlines
\newcommand{\fork}{\textbf{Forkunnskapar}\\}
\newcommand{\forb}{\textbf{Forberedelsar}\\}
\newcommand{\opgvr}{\textbf{Oppgaver}}

\usepackage{datetime2}
\usepackage[]{hyperref}
\geometry{verbose,paperwidth=21 cm, paperheight=29.7 cm, inner=2.3cm, outer=1.8 cm, bmargin=2cm, tmargin=1.8cm}

\begin{document}
\begin{center}
	\begin{tabular}{p{10.5cm} | c | c} 
	\textbf{Kompetasemål 5. klasse} & \textbf{MB} & \textbf{AM}\\ \hline
\shortstack[l]{\\utforske og forklare sammenhenger mellom brøker,\\ desimaltall og prosent og bruke det i hoderegning} &\shortstack{1 \\4} &\shortstack{4} \\ \hline
	
\shortstack[l]{\\beskrive brøk som del av en hel, som del av en mengde\\ og som tall på tallinjen og vurdere og navngi størrelsene} &\shortstack{1\\4} &\shortstack{4} \\ \hline

\shortstack[l]{\\representere brøker på ulike måter og oversette\\ mellom de ulike representasjonene
} &\shortstack{1 \\4} &\shortstack{4} \\ \hline

\shortstack[l]{\\utvikle og bruke ulike strategier for regning med positive \\tall og brøk og forklare tenkemåtene sine
} &\shortstack{1\\4} &\shortstack{4} \\ \hline	

\shortstack[l]{\\formulere og løse problemer fra egen hverdag\\ som har med brøk å gjøre
} &\shortstack{1\\4} &\shortstack{4} \\ \hline

\shortstack[l]{\\diskutere tilfeldighet og sannsynlighet i spill og \\praktiske situasjoner og knytte det til brøk
} &\shortstack{} &\shortstack{7} \\ \hline

\shortstack[l]{\\løse ligninger og ulikheter gjennom logiske resonnementer og\\ forklare hva det vil si at et tall er en løsning på en ligning
} &\shortstack{} &\shortstack{7} \\ \hline

\shortstack[l]{\\lage og løse oppgaver i regneark som \\omhandler personlig økonomi
} &\shortstack{} &\shortstack{6\\E} \\ \hline

\shortstack[l]{\\formulere og løse problemer fra egen hverdag\\ som har med tid å gjøre
} &\shortstack{} &\shortstack{} \\ \hline

\shortstack[l]{\\lage og programmere algoritmer med bruk av \\variabler, vilkår og løkker
} &\shortstack{} &\shortstack{} \\ \hline
\end{tabular}

\begin{itemize}
	\item Ulikheter
	\item Tid
\end{itemize}
\end{center}

\begin{center}
	\begin{tabular}{p{10.5cm} | c | c} 
		\textbf{Kompetasemål 6. klasse} & \textbf{MB} & \textbf{AM}\\ \hline
		\shortstack[l]{\\utforske, navngi og plassere desimaltall \\på tallinjen
		} &\shortstack{1} &\shortstack{} \\ \hline
	
	\shortstack[l]{\\utforske strategier for regning med desimaltall og \\sammenligne med regnestrategier for hele tall
	} &\shortstack{} &\shortstack{1} \\ \hline

\shortstack[l]{\\formulere og løse problemer fra sin egen hverdag som har med \\desimaltall, brøk og prosent å gjøre, og forklare egne\\ tenkemåter
} &\shortstack{} &\shortstack{7\\4} \\ \hline

\shortstack[l]{\\beskrive egenskaper ved og minimumsdefinisjoner av to- og \\tredimensjonale figurer og forklare hvilke egenskaper figurene \\har felles, og hvilke egenskaper som skiller dem fra hverandre
} &\shortstack{} &\shortstack{} \\ \hline

\shortstack[l]{\\utforske og beskrive symmetri i mønstre og utføre \\kongruensavbildninger med og uten koordinatsystem
} &\shortstack{} &\shortstack{} \\ \hline

\shortstack[l]{\\måle radius, diameter og omkrets i sirkler og utforske\\ og argumentere for sammenhengen
} &\shortstack{10} &\shortstack{} \\ \hline

\shortstack[l]{\\utforske mål for areal og volum i praktiske situasjoner \\og representere dem på ulike måter
} &\shortstack{10} &\shortstack{3} \\ \hline

\shortstack[l]{\\bruke ulike strategier for å regne ut areal og omkrets \\og utforske sammenhenger mellom disse
} &\shortstack{6} &\shortstack{3} \\ \hline

\shortstack[l]{\\bruke variabler og formler til å uttrykke sammenhenger\\ i praktiske situasjoner
} &\shortstack{6} &\shortstack{3 \\5} \\ \hline
	\end{tabular}
\end{center}
\begin{itemize}
\item 3D-figurer
\item symmetri
\end{itemize}

\begin{center}
	\begin{tabular}{p{10.5cm} | c | c} 
		\textbf{Kompetasemål 7. klasse} & \textbf{MB} & \textbf{AM}\\ \hline
		\shortstack[l]{\\utvikle og bruke hensiktsmessige strategier i regning med brøk,\\ desimaltall og prosent og forklare tenkemåtene sine
		} &\shortstack{1 \\4} &\shortstack{4} \\ \hline
	
	\shortstack[l]{\\representere og bruke brøk, desimaltall og prosent på ulike \\måter og utforske de matematiske sammenhengene mellom \\disse representasjonsformene
	} &\shortstack{1 \\4} &\shortstack{4} \\ \hline

	\shortstack[l]{\\utforske negative tall i praktiske situasjoner
} &\shortstack{5} &\shortstack{} \\ \hline

	\shortstack[l]{\\bruke tallinje i regning med positive og negative tall
} &\shortstack{5} &\shortstack{} \\ \hline

	\shortstack[l]{\\bruke sammensatte regneuttrykk til å beskrive\\ og utføre utregninger
} &\shortstack{1} &\shortstack{3} \\ \hline

	\shortstack[l]{\\bruke ulike strategier for å løse lineære ligninger og ulikheter \\og vurdere om løsninger er gyldige
} &\shortstack{} &\shortstack{8} \\ \hline

	\shortstack[l]{\\ utforske og bruke hensiktsmessige sentralmål i egne og andres\\ statistiske undersøkelser
} &\shortstack{} &\shortstack{2} \\ \hline

	\shortstack[l]{\\ logge, sortere, presentere og lese data i tabeller og diagrammer\\ og begrunne valget av framstilling
} &\shortstack{} &\shortstack{2} \\ \hline

	\shortstack[l]{\\ lage og vurdere budsjett og regnskap ved å bruke regneark \\med cellereferanser og formler
} &\shortstack{} &\shortstack{6} \\ \hline

\shortstack[l]{\\ bruke programmering til å utforske data i tabeller og datasett
} &\shortstack{} &\shortstack{} \\ \hline	
	\end{tabular}	
\end{center}
\begin{itemize}
\item Ulikheter
\end{itemize}

\begin{center}
	\begin{tabular}{p{10.5cm} | c | c} 
		\textbf{Kompetasemål 8. klasse} & \textbf{MB} & \textbf{AM}\\ \hline
		\shortstack[l]{\\ bruke potenser og kvadratrøtter i utforsking og problemløsing\\ og argumentere for framgangsmåter og resultater
		} &\shortstack{4} &\shortstack{4} \\ \hline
		
		\shortstack[l]{\\representere og bruke brøk, desimaltall og prosent på ulike \\måter og utforske de matematiske sammenhengene mellom \\disse representasjonsformene
		} &\shortstack{1 \\4} &\shortstack{4} \\ \hline
		
		\shortstack[l]{\\utforske negative tall i praktiske situasjoner
		} &\shortstack{5} &\shortstack{} \\ \hline
		
		\shortstack[l]{\\bruke tallinje i regning med positive og negative tall
		} &\shortstack{5} &\shortstack{} \\ \hline
		
		\shortstack[l]{\\bruke sammensatte regneuttrykk til å beskrive\\ og utføre utregninger
		} &\shortstack{1} &\shortstack{3} \\ \hline
		
		\shortstack[l]{\\bruke ulike strategier for å løse lineære ligninger og ulikheter \\og vurdere om løsninger er gyldige
		} &\shortstack{} &\shortstack{8} \\ \hline
		
		\shortstack[l]{\\ utforske og bruke hensiktsmessige sentralmål i egne og andres\\ statistiske undersøkelser
		} &\shortstack{} &\shortstack{2} \\ \hline
		
		\shortstack[l]{\\ logge, sortere, presentere og lese data i tabeller og diagrammer\\ og begrunne valget av framstilling
		} &\shortstack{} &\shortstack{2} \\ \hline
		
		\shortstack[l]{\\ lage og vurdere budsjett og regnskap ved å bruke regneark \\med cellereferanser og formler
		} &\shortstack{} &\shortstack{6} \\ \hline
		
		\shortstack[l]{\\ bruke programmering til å utforske data i tabeller og datasett
		} &\shortstack{} &\shortstack{} \\ \hline	
	\end{tabular}	
\end{center}

\end{document}





