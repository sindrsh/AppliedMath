\input{../doc}
\usepackage[T1]{fontenc}
\usepackage[utf8]{luainputenc}
\usepackage{lmodern} % load a font with all the characters
\usepackage{geometry}
\geometry{verbose,paperwidth=16.1 cm, paperheight=24 cm, inner=2.3cm, outer=1.8 cm, bmargin=2cm, tmargin=1.8cm}
\setlength{\parindent}{0bp}
\usepackage{import}
\usepackage[subpreambles=false]{standalone}
\usepackage{amsmath}
\usepackage{amssymb}
\usepackage{esint}
\usepackage{babel}
\usepackage{tabu}
\usepackage[dvipsnames, table]{xcolor}
\makeatother
\makeatletter


%referances
\newcommand{\net}[2]{{\color{blue}\href{#1}{#2}}}

%Spaces
\newcommand{\vsk}{\\[12pt]}
\newcommand{\vs}{\vspace{-12pt}}

% Tabell for opplegg

\newcommand{\ovlist}[1]{
\vspace{-16pt}
\begin{itemize}
	#1
\end{itemize}
}

\newcommand{\lst}[5]{
\rule{\linewidth}{1pt}
\footnotesize
	\textbf{Øvingsområde}\\ #1 
	
	\textbf{Utstyr}\\ #2  \\
	
	\begin{tabular}{@{} p{4cm} l} 
		\textbf{Tid} & \textbf{Elevinndeling} \\
		#3  & #4
	\end{tabular} 

\rule{\linewidth}{1pt}	\vsk
\normalsize
	\textbf{Gjennomføring}\\ #5 \vsk
}
%

\newcounter{opl}
%\numberwithin{opl}{article}

\newcommand{\opl}[1]{
\newpage
{\refstepcounter{opl} %\phantomsection 
\large \textbf{\theopl \;#1} \vsk}
}

% Headlines
\newcommand{\fork}{\textbf{Forkunnskapar}\\}
\newcommand{\forb}{\textbf{Forberedelsar}\\}
\newcommand{\opgvr}{\textbf{Oppgaver}}

\usepackage{datetime2}
\usepackage[]{hyperref}

\begin{document}

\op{statsentrodd1}
Gitt datasettet
\[ 2\quad12\quad 3\quad 0\quad 2\quad 5\quad 8\quad2\quad 10 \]
Finn \os
\abch{
	\item typetallet \item medianen \item gjennomsnittet 
}

\op{statsentrodd2}
Gitt datasettet
	\[ 9\quad12\quad 3\quad 0\quad 8\quad 5\quad 8\quad4\quad 10\quad 5 \quad 6 \]
Finn \os
\abch{
\item typetallet \item medianen \item gjennomsnittet 
}

\op{statsentrpar1}
Gitt datasettet
\[ 11\quad7\quad 16\quad 0\quad 8\quad 9\quad 8\quad5\quad 16\quad 5 \]
Finn \os
\abch{
	\item typetallet \item medianen \item gjennomsnittet 
}

\op{statsentrpar2}
Gitt datasettet
\[ 6\quad11\quad 14\quad 5\quad 6\quad 9\quad 8\quad5\quad 11\quad 5\quad 11\quad 17 \]
Finn \os
\abch{
	\item typetallet \item medianen \item gjennomsnittet 
}

\op{frkvtb1} \vs
\begin{center}
	\begin{tabular}{c|c}
		|Frukt & Frekvens| \\ \hline
	\end{tabular}
\end{center}
Lag en frekvenstabell for datasettet under
\begin{center}
	banan\quad eple \quad eple\quad eple \quad pære \quad banan \quad eple \quad pære \quad appelsin \quad eple \quad pære \quad pære
\end{center}

\op{frkvtb2} \vs
\begin{center}
	\begin{tabular}{c|c}
		|Tall & Frekvens| \\ \hline
	\end{tabular}
\end{center}
Lag en frekvenstabell for datasettet fra oppgave \ref{statsentrpar2}.

\op{statsoyl} \vs
\abc{
\item Lag et søylediagram for datasettet fra oppgave \ref{frkvtb1}.
\item Lag et søylediagram for datasettet fra oppgave \ref{frkvtb2}.
}

\op{statkak}
Bruk regneark (Excel e.l.) til å 
\abc{
	\item lage et sektordiagram for datasettet fra oppgave \ref{frkvtb1}.
	\item lage et sektordiagram for datasettet fra oppgave \ref{frkvtb2}.
}

\op{stat2}
Av de fire undersøkelsene på side \pageref{undersok}, hvorfor har vi
\abc{
	\item vist frekvenstabell bare for undersøkelse 2?
	\item vist søylediagram bare for undersøkelse  2 og 3?
	\item vist sektordiagram bare for undersøkelse 2 og ?
	\item vist linjediagram bare for undersøkelse 4?
	\item Funnet sentral- og spredningsmål bare for undersøkelse 1, 2 og 4?
}

\op{stat1}
Hvis datasettet har partalls antall svar kan man også finne medianen slik:
\st{
\begin{enumerate}
	\item Finn de to tallene i midten.
	\item Finn differansen mellom tallene, og del denne med 2.
	\item Legg resultatet fra punkt 2 til det laveste av de to tallene i midten.
\end{enumerate}
} \vs
\abc{
	\item Prøv metoden på datasettet fra oppgave \ref{statsentrpar1}.
	\item Hvorfor vil denne metoden alltid fungere? (FLYTT TIL FORMELDEL)
}

\end{document}

