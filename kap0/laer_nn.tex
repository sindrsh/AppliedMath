\input{../doc}
\usepackage[T1]{fontenc}
\usepackage[utf8]{luainputenc}
\usepackage{lmodern} % load a font with all the characters
\usepackage{geometry}
\geometry{verbose,paperwidth=16.1 cm, paperheight=24 cm, inner=2.3cm, outer=1.8 cm, bmargin=2cm, tmargin=1.8cm}
\setlength{\parindent}{0bp}
\usepackage{import}
\usepackage[subpreambles=false]{standalone}
\usepackage{amsmath}
\usepackage{amssymb}
\usepackage{esint}
\usepackage{babel}
\usepackage{tabu}
\usepackage[dvipsnames, table]{xcolor}
\makeatother
\makeatletter


%referances
\newcommand{\net}[2]{{\color{blue}\href{#1}{#2}}}

%Spaces
\newcommand{\vsk}{\\[12pt]}
\newcommand{\vs}{\vspace{-12pt}}

% Tabell for opplegg

\newcommand{\ovlist}[1]{
\vspace{-16pt}
\begin{itemize}
	#1
\end{itemize}
}

\newcommand{\lst}[5]{
\rule{\linewidth}{1pt}
\footnotesize
	\textbf{Øvingsområde}\\ #1 
	
	\textbf{Utstyr}\\ #2  \\
	
	\begin{tabular}{@{} p{4cm} l} 
		\textbf{Tid} & \textbf{Elevinndeling} \\
		#3  & #4
	\end{tabular} 

\rule{\linewidth}{1pt}	\vsk
\normalsize
	\textbf{Gjennomføring}\\ #5 \vsk
}
%

\newcounter{opl}
%\numberwithin{opl}{article}

\newcommand{\opl}[1]{
\newpage
{\refstepcounter{opl} %\phantomsection 
\large \textbf{\theopl \;#1} \vsk}
}

% Headlines
\newcommand{\fork}{\textbf{Forkunnskapar}\\}
\newcommand{\forb}{\textbf{Forberedelsar}\\}
\newcommand{\opgvr}{\textbf{Oppgaver}}

\usepackage{datetime2}
\usepackage[]{hyperref}
\begin{document}
\newpage
\section*{Forord til lærere}
\textbf{Bokas bruksområde}\\
I lag med \net{arg1}{Matematikken sine byggesteinar} (MB) dekker denne boka matematikk for 5.-10. klassse og for VGS-fagen 1P og 2P. Mens MB tar for seg dei teoretiske grunnprinsippene alt er bygd på, er denne boka menit for å vise korleis matematikk kan anvendast i det daglege. Det er likevel med ein viss ambivalens eg bruker ordet ''anvendt''. Eg er hellig overbevist om at dei aller fleste har behov å bruke matematikk i konkrete, praktiske situasjonar for å få opplevinga av at matematikk blir anvendt. Eg håper derfor desse gratis-bøkene kan frigi midlar som kan brukast på utstyr som gjer at elevar (og lærarar) får moglegheita til å måle, estimere, kalkulere og vurdere ut i frå reelle situasjoner.\vsk

\textbf{Boka si disponering} \\
Da boka gapar over matematikk for 5. klasse og heilt til VGS, vil kanskje mange meine at språket er noko avansert, spesielt for dei yngste. Men forenklinger ein tek for å gjere forklaringer lettare å forstå, fører ofte til at ein stadig må vende tilbake til tema for å kommentere tidlegare forenklingar. Eg trur ein i lengda er tjent med å presentere temaa, så langt som råd er, så utfyllande som mogleg, og heller bruke god tid til å forstå dei éin gong for alle.\vsk

Nokon vil kanskje også reagere på at eksempla er veldig enkle, at dei viser få samansatte problem. Éin av grunnane til dette er at slik vil det faktisk vere for dei aller fleste etter endt skulegong; det handler om å bruke formlar direkte. Ein annen grunn er at eg meiner det å meistre likninger er den overlegent beste måten å løyse sammensatte problem på, og derfor handler nesten hele kapittel 6 om problemløysing.\vsk


\textbf{Tilbakemeldingar og eventuelle endringer} \\
Eg håper å høre frå deg med tilbakemeldinger om boka. Merk likevel at alle har sine tankar om korleis ei lærebok ideelt sett bør utformast, så ikkje tolk det som utakksemd viss tilbakemeldingar ikkje blir tatt til etterretning. Husk at kodekilden til både denne \net{https://sindrsh.github.io/AppliedMath/}{boka} og \mb\;ligg åpen for alle på GitHub; med litt kunnskapar om Git og \LaTeX kan du enkelt gjere endringer akkurat slik det passer deg og klassen din!


\newpage
\textbf{Gjøreliste} \\
Prosjektet som denne boka er ein viktig del av er under stadig utvikling. Her er ei liste med komande gjereremål, i prioritert rekkefølge:
\begin{itemize}
	\item Korrigere skrivefeil. Dette gjerast kontinuerleg, gir du beskjed om feil funne til {\tt sindre.heggen@gmail.com}, vil korrigering som oftast bli utført samme dag.
	\item Nynorskutgave vil være klar innen 22. august.
	\item Legge til fleire oppgåver både i denne boka og i \mb. (mykje vil vere på plass innan 22. august)
	\item Legge til fasit (mykje vil vere på plass innan 22. august)
	\item Lage ei pensumoversikt for denne boka og \mb\,sett opp mot kompetansemålene f.o.m. 5. klasse og t.o.m. 2P. (mykje vil vere på plass innan 22. august)
	\item Legge til forklaringar for volumet til tredimensjonale figurer.
	\item Videreutvikle \net{https://hellandmatte.netlify.app/}{nettside} med læringsvideoar, undervisningsopplegg og meir. 
\end{itemize}¨
\newpage

\end{document}