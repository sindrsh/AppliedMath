\input{../doc}
\usepackage[T1]{fontenc}
\usepackage[utf8]{luainputenc}
\usepackage{lmodern} % load a font with all the characters
\usepackage{geometry}
\geometry{verbose,paperwidth=16.1 cm, paperheight=24 cm, inner=2.3cm, outer=1.8 cm, bmargin=2cm, tmargin=1.8cm}
\setlength{\parindent}{0bp}
\usepackage{import}
\usepackage[subpreambles=false]{standalone}
\usepackage{amsmath}
\usepackage{amssymb}
\usepackage{esint}
\usepackage{babel}
\usepackage{tabu}
\usepackage[dvipsnames, table]{xcolor}
\makeatother
\makeatletter


%referances
\newcommand{\net}[2]{{\color{blue}\href{#1}{#2}}}

%Spaces
\newcommand{\vsk}{\\[12pt]}
\newcommand{\vs}{\vspace{-12pt}}

% Tabell for opplegg

\newcommand{\ovlist}[1]{
\vspace{-16pt}
\begin{itemize}
	#1
\end{itemize}
}

\newcommand{\lst}[5]{
\rule{\linewidth}{1pt}
\footnotesize
	\textbf{Øvingsområde}\\ #1 
	
	\textbf{Utstyr}\\ #2  \\
	
	\begin{tabular}{@{} p{4cm} l} 
		\textbf{Tid} & \textbf{Elevinndeling} \\
		#3  & #4
	\end{tabular} 

\rule{\linewidth}{1pt}	\vsk
\normalsize
	\textbf{Gjennomføring}\\ #5 \vsk
}
%

\newcounter{opl}
%\numberwithin{opl}{article}

\newcommand{\opl}[1]{
\newpage
{\refstepcounter{opl} %\phantomsection 
\large \textbf{\theopl \;#1} \vsk}
}

% Headlines
\newcommand{\fork}{\textbf{Forkunnskapar}\\}
\newcommand{\forb}{\textbf{Forberedelsar}\\}
\newcommand{\opgvr}{\textbf{Oppgaver}}

\usepackage{datetime2}
\usepackage[]{hyperref}
\begin{document}
\newpage
\section*{Forord til lærere}
\textbf{Bokas bruksområde}\\
Sammen med \net{arg1}{Matematikkens byggesteiner} (MB) dekker denne boka matematikk for 5.-10. klassse og for VGS-fagen 1P og 2P. Mens MB tar for seg de teoretiske grunnprinsippene alt er bygd på, er denne boka ment for å vise hvordan matematikk kan anvendes i det daglige. Det er likevel med en viss ambivalens jeg bruker ordet ''anvendt''. Jeg er hellig overbevist om at de aller fleste har behov å bruke matematikk i konkrete, praktiske situasjoner for å få opplevelsen av at matematikk blir anvendt. Jeg håper derfor disse gratis-bøkene kan frigi midler som kan brukes på utstyr som gjør at elever (og lærere) får muligheten til å måle, estimere, kalkulere og vurdere ut i fra reelle situasjonar.\vsk

\textbf{Bokas disponering} \\
Da boka gaper over matematikk for 5. klasse og helt til VGS, vil kanskje mange mene at språket er noe avansert, spesielt for de yngste. Men forenklinger man tar for å gjøre forklaringer lettere å forstå, fører ofte til at man stadig må vende tilbake til tema for å kommentere tidligere forenklingener. Jeg tror man i lengden er tjent med å presentere temaene så utfyllende som, innenfor rimelighetens grenser, mulig, og heller bruke god tid til å forstå dem én gang for alle.\vsk

Noen vil kanskje også reagere på at eksemplene er veldig enkle, at de viser få sammensatte problemer. Én av grunnene til dette er at slik vil det faktisk være for de aller fleste etter endt skolegang; det handler om å bruke formler direkte. En annen grunn er at jeg mener det å meistre likninger er den overlegent beste måten å løse sammensatte problemer på, og derfor handler nesten hele kapittel 6 om problemløsing.\vsk

\textbf{Tilbakemeldinger og eventuelle endringer} \\
Jeg håper å høre fra deg med tilbakemeldinger om boka. Merk likevel at alle har sine tanker om hvordan ei lærebok ideelt sett bør utformes, så ikke tolk det som utakknemlighet hvis tilbakemeldinger ikke tas til etterretning. Husk at kodekilden til både denne \net{https://sindrsh.github.io/AppliedMath/}{boka} og \mb\;ligger åpen for alle på GitHub; med litt kunnskaper om Git og \LaTeX kan du enkelt gjøre endringer akkurat slik det passer deg og din klasse!


\newpage
\textbf{Gjøreliste} \\
Prosjektet som denne boka er en viktig del av er under stadig utvikling. Her er en liste med kommende gjøremål, i prioritert rekkefølge:
\begin{itemize}
	\item Korrigere skrivefeil. Dette gjøres kontinuerlig, gir du beskjed om feil funnet til {\tt sindre.heggen@gmail.com}, vil korrigering som oftest bli utført samme dag.
	\item Nynorskutgave vil være klar innen 22. august.
	\item Legge til flere oppgaver både i denne boka og i \mb. (mye vil være på plass innen 22. august)
	\item Legge til fasit (mye vil være på plass innen 22. august)
	\item Lage en pensumoversikt for denne boka og \mb\,sett opp mot kompetansemålene f.o.m. 5. klasse og t.o.m. 2P. (mye vil være på plass innen 22. august)
	\item Legge til forklaringer for volumet til tredimensjonale figurer.
	\item Videreutvikle \net{https://hellandmatte.netlify.app/}{nettside} med læringsvideoer, undervisningsopplegg og mer. 
\end{itemize}
\newpage

\end{document}