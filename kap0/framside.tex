\input{/home/sindre/P/doc}
\usepackage[T1]{fontenc}
\usepackage[utf8]{luainputenc}
\usepackage{lmodern} % load a font with all the characters
\usepackage{geometry}
\geometry{verbose,paperwidth=16.1 cm, paperheight=24 cm, inner=2.3cm, outer=1.8 cm, bmargin=2cm, tmargin=1.8cm}
\setlength{\parindent}{0bp}
\usepackage{import}
\usepackage[subpreambles=false]{standalone}
\usepackage{amsmath}
\usepackage{amssymb}
\usepackage{esint}
\usepackage{babel}
\usepackage{tabu}
\usepackage[dvipsnames, table]{xcolor}
\makeatother
\makeatletter


%referances
\newcommand{\net}[2]{{\color{blue}\href{#1}{#2}}}

%Spaces
\newcommand{\vsk}{\\[12pt]}
\newcommand{\vs}{\vspace{-12pt}}

% Tabell for opplegg

\newcommand{\ovlist}[1]{
\vspace{-16pt}
\begin{itemize}
	#1
\end{itemize}
}

\newcommand{\lst}[5]{
\rule{\linewidth}{1pt}
\footnotesize
	\textbf{Øvingsområde}\\ #1 
	
	\textbf{Utstyr}\\ #2  \\
	
	\begin{tabular}{@{} p{4cm} l} 
		\textbf{Tid} & \textbf{Elevinndeling} \\
		#3  & #4
	\end{tabular} 

\rule{\linewidth}{1pt}	\vsk
\normalsize
	\textbf{Gjennomføring}\\ #5 \vsk
}
%

\newcounter{opl}
%\numberwithin{opl}{article}

\newcommand{\opl}[1]{
\newpage
{\refstepcounter{opl} %\phantomsection 
\large \textbf{\theopl \;#1} \vsk}
}

% Headlines
\newcommand{\fork}{\textbf{Forkunnskapar}\\}
\newcommand{\forb}{\textbf{Forberedelsar}\\}
\newcommand{\opgvr}{\textbf{Oppgaver}}

\usepackage{datetime2}
\usepackage[]{hyperref}
\begin{document}
	\pagecolor{Aquamarine!30}
	\begin{titlepage}
		\begin{center}
			\vspace*{1cm}
			
			{\fontsize{30}{30}{\textbf{Anvend matematikk \\[12pt]for grunnskule og VGS}}}
			
			\vspace{2.45cm} 
			%\Large  Matematikk R2
			\begin{figure}[H]
				\centering
				\qquad\includegraphics[scale=1.4]{\asym{venn3e}}
			\end{figure}           
			\vfill
			\raggedleft Sindre Sogge Heggen   \end{center}
	\end{titlepage}
	\pagenumbering{arabic}
	\pagecolor{white}
	\newpage\thispagestyle{empty} \phantom{}

	\begin{picture}(100,80)
	\put(100,0){\begin{minipage}[l]{0.8\columnwidth}
		\textit{ ''Wahrlich es ist nicht das Wissen, sondern das Lernen, nicht das Besitzen, sondern das Erwerben, nicht das Da-Seyn, sondern das Hinkommen, was den grössten Genuss gewährt'' }
		\vsk  
		
		\textit{ ''Det er ikke kunnskapen, men læringen, ikke besittelsen, men ervervelsen, ikke oppholdet, men ankomsten, som gir den største gleden.''}
		\vsk
		
		{\hfill --- Carl Friedrich Gauss}
		\end{minipage}}
	\end{picture}
	\vfill       
	Alt innhold er laget av Sindre Sogge Heggen. Teksten er skrevet i \LaTeX\;og figurene er lagd vha. \LaTeX, GeoGebra og Asymptote.\vsk
	
	Dokumentet er beskyttet av åndsverkloven, videreformidling må godkjennes av forfatter.
	\begin{center}
		
		
		28.08.2017
	\end{center}
	
\end{document}
\textit{        ''Det er ikkje kunnskapen, men læringa, ikkje eigedomen, men ervervinga, ikkje opphaldet, men ankomsten, som gir den største gleda.''} \\

\textit{        ''Det er ikkje å vite, men å lære, ikke å eige, men å  erverve, ikke å være der, men å komme dit, som gir den største gleda.''}

