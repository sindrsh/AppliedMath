\input{/home/sindre/P/doc}
\usepackage[T1]{fontenc}
\usepackage[utf8]{luainputenc}
\usepackage{lmodern} % load a font with all the characters
\usepackage{geometry}
\geometry{verbose,paperwidth=16.1 cm, paperheight=24 cm, inner=2.3cm, outer=1.8 cm, bmargin=2cm, tmargin=1.8cm}
\setlength{\parindent}{0bp}
\usepackage{import}
\usepackage[subpreambles=false]{standalone}
\usepackage{amsmath}
\usepackage{amssymb}
\usepackage{esint}
\usepackage{babel}
\usepackage{tabu}
\usepackage[dvipsnames, table]{xcolor}
\makeatother
\makeatletter


%referances
\newcommand{\net}[2]{{\color{blue}\href{#1}{#2}}}

%Spaces
\newcommand{\vsk}{\\[12pt]}
\newcommand{\vs}{\vspace{-12pt}}

% Tabell for opplegg

\newcommand{\ovlist}[1]{
\vspace{-16pt}
\begin{itemize}
	#1
\end{itemize}
}

\newcommand{\lst}[5]{
\rule{\linewidth}{1pt}
\footnotesize
	\textbf{Øvingsområde}\\ #1 
	
	\textbf{Utstyr}\\ #2  \\
	
	\begin{tabular}{@{} p{4cm} l} 
		\textbf{Tid} & \textbf{Elevinndeling} \\
		#3  & #4
	\end{tabular} 

\rule{\linewidth}{1pt}	\vsk
\normalsize
	\textbf{Gjennomføring}\\ #5 \vsk
}
%

\newcounter{opl}
%\numberwithin{opl}{article}

\newcommand{\opl}[1]{
\newpage
{\refstepcounter{opl} %\phantomsection 
\large \textbf{\theopl \;#1} \vsk}
}

% Headlines
\newcommand{\fork}{\textbf{Forkunnskapar}\\}
\newcommand{\forb}{\textbf{Forberedelsar}\\}
\newcommand{\opgvr}{\textbf{Oppgaver}}

\usepackage{datetime2}
\usepackage[]{hyperref}

\begin{document}
\phantomsection 
\addcontentsline{toc}{section}{Oppgaver} 
\section*{Oppgaver for Excel}

\obs \textsl{I eksamensoppgaver vil du oppdage at skattesystemer er presentert på en litt annen måte enn i denne boka. Dette er blant annet fordi skattereglene kan forandre seg fra år til år, og i denne boka har vi valgt å presentere skattereglene for 2018. Det i }

\ope{gjorop}
Gjør oppgave \ref{borge4} og \ref{nora} i Excel.

\ope{serielaan}
\textbf{a)} Sett opp et serielån hvor:
\begin{itemize}
	\item Lånesummen er 300\,000\,kr
	\item Renten er 2,1\%
	\item Lånet skal betales med 15 årlige terminbeløp.
\end{itemize}
Avrund alle kronebeløp til hele kroner.\os

\textbf{b)} Hvor mye koster lånet totalt? (Summen av alle terminbeløpene.)

\ope{anulaan}
\textbf{a)} Sett opp et annuitetslån hvor:
\begin{itemize}
	\item Lånesummen er 300\,000\,kr
	\item Renten er 2,1\%
	\item Lånet skal betales med 15 årlige terminbeløp, som er 23\,523\,kr.
\end{itemize}
Avrund alle kronebeløp til hele kroner.\os

\textbf{b)} Hvor mye koster lånet totalt? \os
\textbf{c)} Sammenlign svaret du fikk i oppgave b) med svaret fra oppgave E\ref{serielaan}b, hvilket lån koster mest penger?\os

\ope{sjekk}
Sjekk at du i oppgave E\ref{serielaan} og E\ref{anulaan} har fåt samme svar som nettsiden \net{https://www.laanekalkulator.no/}{laanekalkulator.no}. (Velg \textsl{Tinglysning: Ingen} og sett alle gebyrer til 0).
\newpage
\ope{h17_d2_9}
(Oppgaven er hentet fra del 2, eksamen høsten 2017.)\vsk \\
\includegraphics[scale=0.9]{h17_d2_9}

\newpage
\ope{h16_d2_5}
(Oppgaven er hentet fra del 2, eksamen våren 2016.)
\begin{figure}
	\centering
	\includegraphics[scale=0.75]{v16_d2_5}
\end{figure}
\newpage
\ope{h16_d2_6}
(Oppgaven er hentet fra del 2, eksamen våren 2016.)
\begin{figure}
	\centering
	\includegraphics[scale=0.75]{v16_d2_6}
\end{figure}
\end{document}