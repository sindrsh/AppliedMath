\input{/home/sindre/P/doc}
\usepackage[T1]{fontenc}
\usepackage[utf8]{luainputenc}
\usepackage{lmodern} % load a font with all the characters
\usepackage{geometry}
\geometry{verbose,paperwidth=16.1 cm, paperheight=24 cm, inner=2.3cm, outer=1.8 cm, bmargin=2cm, tmargin=1.8cm}
\setlength{\parindent}{0bp}
\usepackage{import}
\usepackage[subpreambles=false]{standalone}
\usepackage{amsmath}
\usepackage{amssymb}
\usepackage{esint}
\usepackage{babel}
\usepackage{tabu}
\usepackage[dvipsnames, table]{xcolor}
\makeatother
\makeatletter


%referances
\newcommand{\net}[2]{{\color{blue}\href{#1}{#2}}}

%Spaces
\newcommand{\vsk}{\\[12pt]}
\newcommand{\vs}{\vspace{-12pt}}

% Tabell for opplegg

\newcommand{\ovlist}[1]{
\vspace{-16pt}
\begin{itemize}
	#1
\end{itemize}
}

\newcommand{\lst}[5]{
\rule{\linewidth}{1pt}
\footnotesize
	\textbf{Øvingsområde}\\ #1 
	
	\textbf{Utstyr}\\ #2  \\
	
	\begin{tabular}{@{} p{4cm} l} 
		\textbf{Tid} & \textbf{Elevinndeling} \\
		#3  & #4
	\end{tabular} 

\rule{\linewidth}{1pt}	\vsk
\normalsize
	\textbf{Gjennomføring}\\ #5 \vsk
}
%

\newcounter{opl}
%\numberwithin{opl}{article}

\newcommand{\opl}[1]{
\newpage
{\refstepcounter{opl} %\phantomsection 
\large \textbf{\theopl \;#1} \vsk}
}

% Headlines
\newcommand{\fork}{\textbf{Forkunnskapar}\\}
\newcommand{\forb}{\textbf{Forberedelsar}\\}
\newcommand{\opgvr}{\textbf{Oppgaver}}

\usepackage{datetime2}
\usepackage[]{hyperref}
\begin{document}

\section{Potensenes oppbygging}
\begin{figure}[hbt]
\centering \includegraphics{pot} 
\end{figure}\vs
En potens består av et \textit{grunntall} og en \textit{eksponent}. For eksempel er $2^{3}$ en potens med grunntall 2 og
eksponent 3. Eksponenten til potensen sier hvor mange eksemplarer
av grunntallet vi skal gange sammen. For potensen $2^{3}$ sier vi
``\textit{to opphøyd i tre}'' eller bare ``\textit{to i tredje}'',
som vi kan skrive slik: 
\[
2^{3}=2\cdot2\cdot2
\]
Prinsippet er akkurat det samme, selv om grunntallet er en bokstav:
\[
a^{3}=a\cdot a\cdot a
\]
For alle potensregler vi skal se på i dette kapitlet, spiller det ingen rolle om grunntallet eller eksponenten er sifre eller en bokstav.\regv
\reg[Potenstall]{
$ {a^n} $ er et potenstall med grunntall $ a $ og eksponent $ n $. Dette betyr at $ n $ stykker av $ a $ skal ganges sammen.
}
\eks[1]{\vs \vs
\algv{
5^3 &= 5\cdot5\cdot5 \\
&= 125
}
}
\eks[2]{\vs \vs
	\[ c^4 = c\cdot c \cdot c \cdot c \]
}
\eks[3]{ \vs \vs
\algv{
(-7)^2 &= (-7)\cdot(-7) \\
&= 49
}
}
\section{Multiplikasjon av potenser}
Når vi jobber med potenser er det ønskelig å skrive et uttrykk så enkelt som mulig. La oss se på tilfellet 
\[ 2^{2}\cdot2^{3} \]
Vi har at: \vs
\begin{align*}
2^{2} & =2\cdot2\\
2^{3} & =2\cdot2\cdot2
\end{align*}


Med andre ord kan vi skrive: 
\begin{align*}
2^{2}\cdot2^{3} & =\underbrace{2\cdot2}_{2^{2}}\cdot\underbrace{2\cdot2\cdot2}_{2^{3}}\\
 & =2^{5}
\end{align*}
\reg[Multiplikasjon av potenser]{
\begin{equation}
a^{m}\cdot a^{n}=a^{m+n}\label{pmul}	
\end{equation}
}
\eks[1]{\vs \vs
\algv{3^{5}\cdot3^{2}&=3^{5+2}\\&=3^{7}}
}
\eks[2]{\vs \vs
\algv{
4^{3}\cdot4^{7}\cdot4^{4}&=4^{3+7+4} \\ &=3^{14}
}}
\eks[3]{\vs \vs
\algv{
b^4\cdot b^{11}&= b^{3+11}\\
&=b^{14}
}
}
\eks{ \vs \vs
\algv{
a^5\cdot a^{-7} &= a^{5-7} \\
&= a^{-2} 
}	
} 
\section{Potensbrøker}

Hva nå med en potens delt på en annen med samme grunntall? La
oss undersøke brøken
\[ \frac{3^{4}}{3^{2}} \]
Vi starter med å skrive
ut potensen over og under brøkstreken: 
\begin{align*}
\frac{3^{4}}{3^{2}} & =\frac{\overbrace{3\cdot3\cdot3\cdot3}^{3^{4}}}{\underbrace{3\cdot3}_{3^{2}}}\br
 & =\frac{\bcancel{3}\cdot\bcancel{3}\cdot3\cdot3}{\bcancel{3}\cdot\bcancel{3}}\br
 & =3\cdot3\\
 & =3^{2}
\end{align*}
Men denne litt lange utregningen kunne vi skrevet på en mye kortere måte:
\begin{align*}
\frac{3^{4}}{3^{2}} & =3^{4-2}\\
 & =3^{2}
\end{align*}
\reg[Potens delt på potens]{\vs
\begin{equation}\label{ppp}
\frac{a^{m}}{a^{n}}=a^{m-n}
\end{equation} }

\eks[1]{\vspace{-20 pt}
\[
\frac{3^{5}}{3^{2}}=3^{5-2}=3^{3}
\]
} 
\eks[2]{\vspace{-20 pt}\[
\frac{2^{4}\cdot5^{3}}{5^{2}\cdot2^{2}}=2^{4-2}\cdot5^{3-2}=2^{2}\cdot5
\]}
\section[Spesialtilfellet $ a^0 $]{\boldmath Spesialtilfellet $ a^0 $}
Husk at når et tall deles på seg selv, blir
svaret alltid 1. For eksempel er ${\frac{4}{4}=1}$, ${\frac{45}{45}=1}$ og ${\frac{1,2}{1,2}=1}$. En potens er også bare et tall. Deler vi derfor $6^{4}$ med seg selv,
er svaret 1:
\[
\frac{6^{4}}{6^{4}}=1
\]
Hvis vi gjør samme regnestykket om igjen, men nå bruker ligning \eqref{ppp},
finner vi at:
\begin{align*}
\frac{6^{4}}{6^{4}} & =6^{4-4}\\
 & =6^{0}
\end{align*}
Siden ${\frac{6^{4}}{6^{4}}=1}$ og ${\frac{6^{4}}{6^{4}}=6^{0}}$,
må dette bety at ${6^{0}=1}$. 
\reg[Spesialtilfellet \boldmath $a^0$]{\vs \vs
\[
a^{0}=1
\]
}
\eks[1]{\vs \vs\[
1000^{0}=1
\]}
\eks[2]{\vs \vs\[
(-4)^{0}=1
\]}
\section{Negativ eksponent}
La oss nå prøve å forenkle uttrykket 
\[ \frac{2^{2}}{2^{4}} \]
Denne brøken kan vi skrive som: 
\begin{align*}
\frac{2^{2}}{2^{4}} & =\frac{\bcancel{2}\cdot\bcancel{2}}{\bcancel{2}\cdot\bcancel{2}\cdot2\cdot2}\\
 & =\frac{1}{2^{2}}
\end{align*}
Hvis vi istedenfor bruker ligning \eqref{ppp}, får vi: 
\begin{align*}
\frac{2^{2}}{2^{4}} & =2^{2-4}\\
 & =2^{-2}
\end{align*}
Siden ${\frac{2^{2}}{2^{4}}=\dfrac{1}{2^{2}}}$ og ${\frac{2^{2}}{2^{4}}=2^{-2}}$,
må altså:
\[ 2^{-2}=\frac{1}{2^{2}} \]
\reg[Potens med negativ eksponent]{\vs
\begin{equation}\label{pneg}
a^{-m}=\frac{1}{a^{m}}
\end{equation}}

\eks[1]{\vspace{-20 pt}
\[
3^{-5}=\frac{1}{3^{5}}
\] \vspace{-10 pt}
}
\eks[2]{\vspace{-20 pt}\[
c^{-7}=\frac{1}{c^{7}}
\]\vspace{-10 pt}}
\section{Brøk som grunntall}
Vi skal nå se på tilfellet der grunntallet til potensen er en
brøk, for eksempel $\left(\frac{2}{3}\right)^{3}$. Eksponenten 3
sier oss at tre stykker av $\frac{2}{3}$ skal ganges sammen:
\begin{align*}
\frac{2}{3}\cdot\frac{2}{3}\cdot\frac{2}{3} & =\frac{2\cdot2\cdot2}{3\cdot3\cdot3}\\
 & =\frac{2^{3}}{3^{3}}
\end{align*}
\reg[Brøk som grunntall]{\vs
\begin{equation}\label{pbrg}
\left(\frac{a}{b}\right)^{m}=\frac{a^{m}}{b^{m}}
\end{equation}}

\section{Faktorer som grunntall}
\parbox[l][][l]{0.6\linewidth}{
Hva nå med to faktorer ganget sammen, opphøyd i en eksponent? La oss
se på ${(2\cdot x)^{3}}$. Nå er ${2\cdot x}$ grunntallet (oftest skriver
vi dette bare som $2x$), og vi får: 
\[
(2\cdot x)^{3}=(2\cdot x)\cdot(2\cdot x)\cdot(2\cdot x)
\] }\qquad
\parbox[r][][l]{0.3\linewidth}{\begin{shaded}%
		\textsl{Husk}: Faktorer er tall som ganges sammen.I regnestykkt \y{2\cdot a \cdot 5} kaller vi 2, $ a $ og 5 for faktorer.\end{shaded}}
Fordi det i regnestykket over bare inngår multiplikasjon, er parantesene overflødige og kan derfor fjernes. Samtidig kan vi multiplisere i den rekkefølgen vi selv ønsker, så lenge alle faktorene er med:
\begin{align*}
(2\cdot x)^{3} & =(2\cdot x)\cdot(2\cdot x)\cdot(2\cdot x)\\
 & =2\cdot2\cdot2\cdot x\cdot x\cdot x\\
 & =2^{3}\cdot x^{3}
\end{align*}
\reg[Faktorer som grunntall]{
\begin{equation}\label{key}
\left(ab\right)^{m}=a^{m}b^{m}
\end{equation}
}\regv
\textbf{\textsl{Obs!}} Når vi har uttrykk som $ab$ eller $a^{m}b^{m}$ står det egentlig et gangetegn mellom faktorene (${ab=a\cdot b}$ og ${a^{m}b^{n}=a^{m}\cdot b^{m}}$),
men det er vanlig å sløyfe dette når vi bruker bokstaver.\regv

\eks[1]{\vs\vs
\[
(ab)^{4}=a^{4}b^{4}
\]}

\eks[2]{\vs
\algvv{
(5x)^{2}&=5^{2}x^{2}\\&=25x^2
}
}

\section{Potens som grunntall}
Det siste tilfellet vi skal se på er når grunntallet også er
et potenstall. Dette er tilfellet for tallet $\left(7^{3}\right)^{4}$, som vi kan skrive som:
\begin{align*}
\left(7^{3}\right)^{4} & =7^{3}\cdot7^{3}\cdot7^{3}\cdot7^{3}
\end{align*}


Vi kan nå bruke ligning \eqref{pmul}, og får da: 
\begin{align*}
7^{3}\cdot7^{3}\cdot7^{3}\cdot7^{3} & =7^{3+3+3+3}\\
 & =7^{3\cdot4}\\
 & =7^{12}
\end{align*}
\reg[Potens som grunntall]{\vs
\begin{equation}\label{key}
\left(a^{m}\right)^{n}=a^{m\cdot n}
\end{equation}}

\section{Tall på standardform}
La oss se på tallet 
\[ 4500000 \]
\prbxl{0.6}{Dette er et stort tall med mange sifre, og som vi kan ønske å skrive på en mer kompakt
	måte. }\qquad
\prbxr{0.3}{Tallet 347 består av sifrene 3, 4 og 7.}
Som en start omskriver vi tallet vårt ved å legge til '',0'':
\[
4500000=4500000,0
\]
Vi skal nå utnytte en egenskap ved tallet 10. For om vi flytter komma én plass til venstre og deretter ganger tallet
vårt med 10, forblir verdien den samme:
\[ 4500000,0=450000,0\cdot10 \]
Og sånn kan vi holde på; vi kan flytte komma så mange ganger vi vil
til venstre, så lenge vi multipliserer tilsvarende ganger med 10.
Om vi nå skriver tallet på \textit{standardform} stopper vi forflytningen når vi har ett siffer igjen foran komma. Da får vi: 
\begin{align*}
4500000,0 & =450000,0\cdot10=450000,0\cdot10^{1}\\
 & =45000,0\cdot10\cdot10=45000,0\cdot10^{2}\\
 & =4500,0\cdot10\cdot10\cdot10=4500,0\cdot10^{3}\\
 & =450,0\cdot10\cdot10\cdot10\cdot10=450,0\cdot10^{4}\\
 & =45,0\cdot10\cdot10\cdot10\cdot10\cdot10=45,0\cdot10^{5}\\
 & =4,5\cdot10\cdot10\cdot10\cdot10\cdot10\cdot10=4,5\cdot10^{6}
\end{align*}
Altså har vi at:
\[ \text{4500000 på standardform}=4,6\cdot10^6 \] \vsk

\prbxl{0.64}{La oss også skrive $0,00056$ på standardform. Om vi flytter komma én plass til høyre og deler på 10, forblir verdien den samme:
	\[ 0,00056=0,0056\cdot\frac{1}{10} \]
 }\quad
\prbxr{0.3}{\textsl{Husk}: Å dele på 10 er det samme som å gange med $ \frac{1}{10} $.}
I dette tilfellet flytter vi komma og deler med 10 helt til vi har ett siffer foran komma som ikke er 0:
\begin{align*}
0.00056 & =0,0056\cdot\frac{1}{10}=0,0056\cdot10^{-1}\\
 & =0,056\cdot\frac{1}{10}\cdot\frac{1}{10}=0,056\cdot10^{-2}\\
 & =0,56\cdot\frac{1}{10}\cdot\frac{1}{10}\cdot\frac{1}{10}=0,56\cdot10^{-3}\\
 & =5,6\cdot\frac{1}{10}\cdot\frac{1}{10}\cdot\frac{1}{10}\cdot\frac{1}{10}=5,6\cdot10^{-4}
\end{align*}
0,000056 skrevet på standardform er altså $5,6\cdot10^{-4}$.\regv
\reg[Tall på standardform \label{std}]{
For å skrive tall på standardform flytter vi komma helt til vi har ett siffer, som ikke er 0, foran komma.
\begin{itemize}
\item Flytter vi komma til venstre, må vi gange med 10 opphøyd i antall flytt.
\item Flytter vi komma til høyre, må vi gange med 10 opphøyd i minus antall flytt.
\end{itemize}}
\eks[1]{ \vs \vs
\[ 3560,3 = 3,5603\cdot10^3 \]
}
\eks[2]{ \vs \vs
\[
0,0645=6,45\cdot10^{-2}
\]} 
\eks[3]{ \vs \vs \vs
	\begin{align*}
	-7600 & =-7600,0\\
	& =-7,6\cdot10^{3}
	\end{align*}}
\subsection{Multiplikasjon av tall på standardform}
\reg[Multiplikasjon av tall på standardform]{ 
\vs 	
\[
a\cdot b\cdot10^{m}\cdot10^{n}=a\cdot b\cdot10^{m+n}
\] }

\eks{\vspace{-35 pt}
\begin{align*}
2.3\cdot10^{5}\cdot3.4\cdot10^{-2} & =7.82\cdot10^{5-2}\\
 & =7.82\cdot10^{3}
\end{align*}\vspace{-10 pt}}


 
\end{document}
