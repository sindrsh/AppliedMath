\input{/home/sindre/P/doc}
\usepackage[T1]{fontenc}
\usepackage[utf8]{luainputenc}
\usepackage{lmodern} % load a font with all the characters
\usepackage{geometry}
\geometry{verbose,paperwidth=16.1 cm, paperheight=24 cm, inner=2.3cm, outer=1.8 cm, bmargin=2cm, tmargin=1.8cm}
\setlength{\parindent}{0bp}
\usepackage{import}
\usepackage[subpreambles=false]{standalone}
\usepackage{amsmath}
\usepackage{amssymb}
\usepackage{esint}
\usepackage{babel}
\usepackage{tabu}
\usepackage[dvipsnames, table]{xcolor}
\makeatother
\makeatletter


%referances
\newcommand{\net}[2]{{\color{blue}\href{#1}{#2}}}

%Spaces
\newcommand{\vsk}{\\[12pt]}
\newcommand{\vs}{\vspace{-12pt}}

% Tabell for opplegg

\newcommand{\ovlist}[1]{
\vspace{-16pt}
\begin{itemize}
	#1
\end{itemize}
}

\newcommand{\lst}[5]{
\rule{\linewidth}{1pt}
\footnotesize
	\textbf{Øvingsområde}\\ #1 
	
	\textbf{Utstyr}\\ #2  \\
	
	\begin{tabular}{@{} p{4cm} l} 
		\textbf{Tid} & \textbf{Elevinndeling} \\
		#3  & #4
	\end{tabular} 

\rule{\linewidth}{1pt}	\vsk
\normalsize
	\textbf{Gjennomføring}\\ #5 \vsk
}
%

\newcounter{opl}
%\numberwithin{opl}{article}

\newcommand{\opl}[1]{
\newpage
{\refstepcounter{opl} %\phantomsection 
\large \textbf{\theopl \;#1} \vsk}
}

% Headlines
\newcommand{\fork}{\textbf{Forkunnskapar}\\}
\newcommand{\forb}{\textbf{Forberedelsar}\\}
\newcommand{\opgvr}{\textbf{Oppgaver}}

\usepackage{datetime2}
\usepackage[]{hyperref}
\begin{document}
Mål for opplæringen er at eleven skal kunne	
\begin{itemize}
	\item tolke og bruke formler som gjelder dagligliv og yrkesliv
\end{itemize}
\newpage
\section{Generelt}
Tenk at du har en jobb der du tjener 200 kr i timen, og at du jobber 5 timer hver arbeidsdag. Regnestykket for hvor mange kroner du tjener på en arbeidsdag (dagslønnen) er dette:
\alg{
\text{dagslønn}&= 200\cdot5\\
&= 1000
}
Hvis du istedenfor tjener 500 kr i timen og jobber 3 timer hver dag, blir regnestykket seende slik ut: 
\alg{
\text{dagslønn}&= 500\cdot3\\
&= 1500
}
\prbxl{0.6}{Saken er at selv om timelønnen og timetallet forandrer seg, er selve \textsl{regnemetode} for dagslønnen akkurat den samme: \textsl{Vi ganger timelønnen med timetallet}. Når en regnemetode forblir den samme, selv om tallene forandrer seg, sier vi at vi har en \textit{formel}. En formel forteller oss hvordan vi skal regne ut det vi ønsker å vite. Når vi regnet ut dagslønnen vår ganget vi timelønnen med timeantallet, formelen for dagslønnen kan vi da skrive slik:}\qquad
\prbxr{0.3}{I de to regnestykkene 
	\[2\cdot3 = 6  \]
og 
\[ 4\cdot5 = 20 \]
er \textsl{regnemetoden} den samme (vi ganger to tall), men ikke \textsl{resultatet}.
}
\[ \text{dagslønn}=\text{timelønnn}\cdot\text{timetall} \]
For å gjøre formlene våre enda kortere bruker vi også å forkorte størrelsene, gjerne med bokstaver som har sammenheng med navnet på størrelsen. For eksempel kan vi kalle dagslønnen for $ D $, timelønnen for $ L $ og timetallet for $ T $, da blir formelen vår seende ut som dette
\[ D = T\cdot L \]
Fordi $ D $ står alene på den ene siden av ''$ = $'-tegnet, sier vi at dette er en formel for $ D $.\regv
\reg[Formler\label{test}]{En formel viser sammenhengen mellom størrelser.}
\eks[1]{
Hvis du kjører med den samme farten hele tiden, finner du lengden du har kjørt ved å gange farten med tiden. Kall lengden du har kjørt for $ l $, farten for $ f $ og tiden for $ t $.\os
Lag en formel for $ l $.  

\sv 
Oppgaveteksten forteller oss at vi finner $ l $ ved å gange $ f $ med $ t $:
\[ l = f\cdot t \]
Dette er altså formelen for $ l $.
}
\eks[2]{En vennegjeng ønsker å spleise på en bil som koster 50\,000 kr, men det er usikkert hvor mange personer som skal være med på å spleise.\os 
\textbf{a)} Kall ''antall personer som blir med på å spleise'' for $ P $ og ''utgift per person i kroner'' for $ U $  og lag en formel for $ U $.\os

\textbf{b)} Finn utgiften per person hvis 20 personer blir med.

\sv
\textbf{a)} Siden prisen på bilen skal deles på antall personer som er med i spleiselaget, må formelen bli:
\[ U = \frac{50\,000}{P} \]

\textbf{b)} Vi erstatter $ P $ med 20, og får:
\alg{
U &= \frac{50\,000}{20}\\
&= 2\,500
}
Utgiften per person er altså 2\,500 kr.
}
\section{Omgjøring av formler}
\ref{test}
Vi har sett (\hyperref[farteks]{\textsl{Eksempel 1}}, s. \pageref{farteks}) at lengden $ l $ vi har kjørt, farten $ f $ vi har holdt og tiden $ t $ vi har brukt kan settes i sammenheng via formelen:
\[ l = f\cdot t \] 
Ut ifra denne formelen kan vi altså finne lengden hvis vi vet hvor fort og hvor lenge vi har kjørt. Men hva om vi isteden vet hvor langt og hvor lenge vi har kjørt, men ikke hvor fort?\vsk

Det vi må gjøre, er å skrive om formelen så det blir en formel for $ f $ istedenfor $ l $. Det vi nå må ha med oss, er at $ l $, $ f $ og $ t$ er alle tall, derfor kan vi bruke punktene fra \hr{lsmlig} for å gjøre om på ligningen vår. Og fordi vi ønsker en formel for $ f $, ønsker vi at $ f $ skal stå alene på den ene siden av likhetstegnet:
\alg{
l &= f\cdot t \br
\frac{l}{t}&=\frac{f\cdot \bcancel{t}}{\bcancel{t}} \br
\frac{l}{t}&=f
}
\reg[Omforming av formler]{Når vi skal omforme en størrelse, bruker vi ligningsreglene fra \hr{lsmlig} for å få størrelsen vi ønsker til å stå på én side av likhetstegnet.}
\eks[1 \phs{farteks}]{
\textit{Ohms lov} sier at strømmen $ I $ gjennom en metallisk leder (med konstant temeperatur) er gitt ved formelen:
\[ I = \frac{U}{R} \]
hvor $ U $ er spenningen og $ R $ er resistansen. \os
\textbf{a)} Skriv om formelen til en formel for $ R $.
\vsk

Strøm måles i Ampere (A), spenning i Volt (V) og motstand i Ohm ($ \Omega $).\os
\textbf{b)} Hvis strømmen er 2\,A og spenningen 12\,V, hva er da resistansen?

\sv
\textbf{a)} Vi gjør om formelen slik at $ R $ står alene på én side av likhetsregnet:\vs
\alg{
I\cdot R&=\frac{U\cdot \cancel{R}}{\cancel{R}} \br
I\cdot R &= U \br
\frac{\cancel{I}\cdot R}{\cancel{I}} &= \frac{U}{I}\br 
R &= \frac{U}{I}
}
\textbf{b)} Vi bruker formelen vi fant i a) og får at:
\alg{
R &= \frac{U}{I} \br
&= \frac{12}{2} \\
&= 6
}
Resistansen er altså $ 6\,\Omega $.
}
\eks[2]{
Si vi har målt en temperatur $ T_C $ i grader Celsius ($ ^\circ C $). Temperaturen $ T_F $ målt i Fahrenheit ($ ^\circ F $) er da gitt ved formelen:
\[ T_F = \frac{9}{5}\cdot T_C+32 \]
\textbf{a)} Skriv om formelen til en formel for $ T_C $.\os
\textbf{b)} Hvis en temperatur er målt til 59$ ^\circ F $, hva er da temperaturen målt i $ ^\circ C $?

\sv
\textbf{a)} \alg{
T_F &= \frac{9}{5}\cdot T_C+32 \\
T_F-32 &= \frac{9}{5}\cdot T_C \\
5(T_F-32) &= \cancel{5}\cdot\frac{9}{\cancel{5}}\cdot F_C \\
5(T_F-32) &= 9T_C \\
\frac{5(T_F-32)}{9} &= \frac{\cancel{9}T_C}{\cancel{9}} \\
\frac{5(T_F-32)}{9} &= T_C
}
\textbf{b)} Vi bruker formelen fra a), og finner at:
\alg{
T_C&= \frac{5(59-32)}{9} \br
&= \frac{5(27)}{9} \br
&= 5\cdot 3 \\
&= 15
}
}

\end{document}


