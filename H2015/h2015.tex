%\documentclass[11pt,english]{report}
%\usepackage[T1]{fontenc}
\usepackage[utf8]{luainputenc}
\usepackage{lmodern} % load a font with all the characters
\usepackage{geometry}
\geometry{verbose,paperwidth=16.1 cm, paperheight=24 cm, inner=2.3cm, outer=1.8 cm, bmargin=2cm, tmargin=1.8cm}
\setlength{\parindent}{0bp}
\usepackage{import}
\usepackage[subpreambles=false]{standalone}
\usepackage{amsmath}
\usepackage{amssymb}
\usepackage{esint}
\usepackage{babel}
\usepackage{tabu}
\usepackage[dvipsnames, table]{xcolor}
\makeatother
\makeatletter


%referances
\newcommand{\net}[2]{{\color{blue}\href{#1}{#2}}}

%Spaces
\newcommand{\vsk}{\\[12pt]}
\newcommand{\vs}{\vspace{-12pt}}

% Tabell for opplegg

\newcommand{\ovlist}[1]{
\vspace{-16pt}
\begin{itemize}
	#1
\end{itemize}
}

\newcommand{\lst}[5]{
\rule{\linewidth}{1pt}
\footnotesize
	\textbf{Øvingsområde}\\ #1 
	
	\textbf{Utstyr}\\ #2  \\
	
	\begin{tabular}{@{} p{4cm} l} 
		\textbf{Tid} & \textbf{Elevinndeling} \\
		#3  & #4
	\end{tabular} 

\rule{\linewidth}{1pt}	\vsk
\normalsize
	\textbf{Gjennomføring}\\ #5 \vsk
}
%

\newcounter{opl}
%\numberwithin{opl}{article}

\newcommand{\opl}[1]{
\newpage
{\refstepcounter{opl} %\phantomsection 
\large \textbf{\theopl \;#1} \vsk}
}

% Headlines
\newcommand{\fork}{\textbf{Forkunnskapar}\\}
\newcommand{\forb}{\textbf{Forberedelsar}\\}
\newcommand{\opgvr}{\textbf{Oppgaver}}

\usepackage{datetime2}
\usepackage[]{hyperref}
%\usepackage[T1]{fontenc}
\usepackage[utf8]{luainputenc}
\usepackage{lmodern} % load a font with all the characters
\usepackage{geometry}
\geometry{verbose,paperwidth=16.1 cm, paperheight=24 cm, inner=2.3cm, outer=1.8 cm, bmargin=2cm, tmargin=1.8cm}
\setlength{\parindent}{0bp}
\usepackage{import}
\usepackage[subpreambles=false]{standalone}
\usepackage{amsmath}
\usepackage{amssymb}
\usepackage{esint}
\usepackage{babel}
\usepackage{tabu}
\usepackage[dvipsnames, table]{xcolor}
\makeatother
\makeatletter


%referances
\newcommand{\net}[2]{{\color{blue}\href{#1}{#2}}}

%Spaces
\newcommand{\vsk}{\\[12pt]}
\newcommand{\vs}{\vspace{-12pt}}

% Tabell for opplegg

\newcommand{\ovlist}[1]{
\vspace{-16pt}
\begin{itemize}
	#1
\end{itemize}
}

\newcommand{\lst}[5]{
\rule{\linewidth}{1pt}
\footnotesize
	\textbf{Øvingsområde}\\ #1 
	
	\textbf{Utstyr}\\ #2  \\
	
	\begin{tabular}{@{} p{4cm} l} 
		\textbf{Tid} & \textbf{Elevinndeling} \\
		#3  & #4
	\end{tabular} 

\rule{\linewidth}{1pt}	\vsk
\normalsize
	\textbf{Gjennomføring}\\ #5 \vsk
}
%

\newcounter{opl}
%\numberwithin{opl}{article}

\newcommand{\opl}[1]{
\newpage
{\refstepcounter{opl} %\phantomsection 
\large \textbf{\theopl \;#1} \vsk}
}

% Headlines
\newcommand{\fork}{\textbf{Forkunnskapar}\\}
\newcommand{\forb}{\textbf{Forberedelsar}\\}
\newcommand{\opgvr}{\textbf{Oppgaver}}

\usepackage{datetime2}
\usepackage[]{hyperref}
\usepackage[T1]{fontenc}
\usepackage[utf8]{luainputenc}
\usepackage{lmodern} % load a font with all the characters
\usepackage{geometry}
\geometry{verbose,paperwidth=16.1 cm, paperheight=24 cm, inner=2.3cm, outer=1.8 cm, bmargin=2cm, tmargin=1.8cm}
\setlength{\parindent}{0bp}
\usepackage{import}
\usepackage[subpreambles=false]{standalone}
\usepackage{amsmath}
\usepackage{amssymb}
\usepackage{esint}
\usepackage{babel}
\usepackage{tabu}
\usepackage[dvipsnames, table]{xcolor}
\makeatother
\makeatletter


%referances
\newcommand{\net}[2]{{\color{blue}\href{#1}{#2}}}

%Spaces
\newcommand{\vsk}{\\[12pt]}
\newcommand{\vs}{\vspace{-12pt}}

% Tabell for opplegg

\newcommand{\ovlist}[1]{
\vspace{-16pt}
\begin{itemize}
	#1
\end{itemize}
}

\newcommand{\lst}[5]{
\rule{\linewidth}{1pt}
\footnotesize
	\textbf{Øvingsområde}\\ #1 
	
	\textbf{Utstyr}\\ #2  \\
	
	\begin{tabular}{@{} p{4cm} l} 
		\textbf{Tid} & \textbf{Elevinndeling} \\
		#3  & #4
	\end{tabular} 

\rule{\linewidth}{1pt}	\vsk
\normalsize
	\textbf{Gjennomføring}\\ #5 \vsk
}
%

\newcounter{opl}
%\numberwithin{opl}{article}

\newcommand{\opl}[1]{
\newpage
{\refstepcounter{opl} %\phantomsection 
\large \textbf{\theopl \;#1} \vsk}
}

% Headlines
\newcommand{\fork}{\textbf{Forkunnskapar}\\}
\newcommand{\forb}{\textbf{Forberedelsar}\\}
\newcommand{\opgvr}{\textbf{Oppgaver}}

\usepackage{datetime2}
\usepackage[]{hyperref}

\begin{document}
\section{Eksamen 2015}
\subsection{Oppgave 1}
Siden prisen på varen er ukjent, kaller vi denne for $ x $:
\[ \text{Pris på ukjent vare}=x \]
Prisen er redusert med 30\%, noe som betyr at prosentfaktoren er 0,3. Vekstfaktoren når prisen er \textit{satt} ned blir da:
\[ \text{Vekstfaktor = 1-0,3 = 0,7} \]
Vi vet at den nye prisen (altså 280 kr) er det samme som den gamle prisen ($ x $) ganget med vekstfaktoren (0,7):
\algv{
	x\cdot 0,7&= 280 \\
	x &= \frac{280}{0.7}
	}
$ \dfrac{280}{0.7} $ kan vi skrive som 2800/7:
\algv{&2800/7= 400\\
	&\underline{40} \\
&\,\;0 \\
&\,\;\underline{0} \\
&\,\;00 \\
&\,\;\,\;\underline{0}
}
\subsection{Oppgave 2}
Vi starter med å gange sammen $ 3,4\cdot4,0 $ \vspace{-11 pt}
\begin{center}
	$ 3,4 \cdot4 $ \\
	\begin{tabular}{c@{\,}c@{\,}c@{\,}c}
		\hline
		&  & 1 & 6 \\
		 & 1  & 2 &  \\
		\hline
		& 1 & 3, & 6 \\
	\end{tabular}
\end{center}
Hele regnestykket blir da:
\algv{ 3,4 \cdot 10^9 \cdot 4,0 \cdot 10^{-3} &= 13,6 \cdot 10^{9-3} \\
&= 1,36 \cdot 10^7
	}

\subsection{Oppgave 3}
\algv{
\frac{4^3\cdot 2^{-6}}{4^0\cdot 2^{-4}} &= \frac{(2^2)^3\cdot2^{-6}}{2^{-4}} \\
&= 2^{6-6-(-4)} \\
&= 2^4 \\
&= 16
	}

\subsection{Oppgave 5}
Siden vi ikke vet hvor mye gevinsten er, kaller vi denne for $ x $:
\[ \text{Gevinst}=x \]
En rente på 3,2\% svarer til en prosentfaktor på 0,032. Siden pengene har \textit{økt} blir vekstfaktoren:
\[ \text{Vekstfaktor}=1+0,032=1,032\] 
At gevinsten har økt til 500138 etter 10 år, må bety at:
\alg{
	x\cdot1,032^{10} &= 500 138 \\
	x &= \frac{500138}{1,032^{10}}
	}

\subsection{Oppgave 5}
Omkretsen 40000 km tilsvarer $ 4\cdot10^7 $ m. For å finne ut hvor mange armlengder på 1,6 m som går på $ 4\cdot10^7 $ m, må vi dele:
\[ \frac{4\cdot10^7}{1,6} \]

\algv{&40/16= 2,5\\
	&\underline{32} \\
	&\,\;80 \\
	&\,\;\underline{80} \\
	&\,\;00 \\
	&\,\;\,\;\underline{0}
}
\[ \frac{4\cdot10^7}{1,6} 	= 2,5\cdot10^7 \]
\end{document}


