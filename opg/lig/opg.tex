\input{/home/sindre/P/doc}
\usepackage[T1]{fontenc}
\usepackage[utf8]{luainputenc}
\usepackage{lmodern} % load a font with all the characters
\usepackage{geometry}
\geometry{verbose,paperwidth=16.1 cm, paperheight=24 cm, inner=2.3cm, outer=1.8 cm, bmargin=2cm, tmargin=1.8cm}
\setlength{\parindent}{0bp}
\usepackage{import}
\usepackage[subpreambles=false]{standalone}
\usepackage{amsmath}
\usepackage{amssymb}
\usepackage{esint}
\usepackage{babel}
\usepackage{tabu}
\usepackage[dvipsnames, table]{xcolor}
\makeatother
\makeatletter


%referances
\newcommand{\net}[2]{{\color{blue}\href{#1}{#2}}}

%Spaces
\newcommand{\vsk}{\\[12pt]}
\newcommand{\vs}{\vspace{-12pt}}

% Tabell for opplegg

\newcommand{\ovlist}[1]{
\vspace{-16pt}
\begin{itemize}
	#1
\end{itemize}
}

\newcommand{\lst}[5]{
\rule{\linewidth}{1pt}
\footnotesize
	\textbf{Øvingsområde}\\ #1 
	
	\textbf{Utstyr}\\ #2  \\
	
	\begin{tabular}{@{} p{4cm} l} 
		\textbf{Tid} & \textbf{Elevinndeling} \\
		#3  & #4
	\end{tabular} 

\rule{\linewidth}{1pt}	\vsk
\normalsize
	\textbf{Gjennomføring}\\ #5 \vsk
}
%

\newcounter{opl}
%\numberwithin{opl}{article}

\newcommand{\opl}[1]{
\newpage
{\refstepcounter{opl} %\phantomsection 
\large \textbf{\theopl \;#1} \vsk}
}

% Headlines
\newcommand{\fork}{\textbf{Forkunnskapar}\\}
\newcommand{\forb}{\textbf{Forberedelsar}\\}
\newcommand{\opgvr}{\textbf{Oppgaver}}

\usepackage{datetime2}
\usepackage[]{hyperref}

\begin{document}
\opgt
\nes

\op{lig1}
Løs ligningene: \os
\begin{tabular}{@{}l l l l}
	\textbf{a)} $ x-4=9 $ &  \textbf{b)} $ 3x-7=2x-12 $ 
\end{tabular}
\nes

\op{lig2}
Løs ligningene: \os
\begin{tabular}{@{}l l l l}
	\textbf{a)} $ 5x=100 $ &  \textbf{b)} $ 3x+8=24-x $ 
\end{tabular}

\nes

\op{lig3}
Løs ligningene: \os
\begin{tabular}{@{}l l l l}
	\textbf{a)} $ \dfrac{x}{7}=3 $ & 
	\textbf{b)} $ \dfrac{x}{6}=\dfrac{1}{3} $ &   \textbf{c)} $ \dfrac{3x}{5}=9 $
\end{tabular}

\nes
\op{lig4}
Løs ligningene (skriv svarene som hele tall eller brøk): \os
\begin{tabular}{@{}l l l l}
	\textbf{a)} $3x-(2-2x)=8x-3(2^2+4) $ & 
	\textbf{b)} $\displaystyle \frac{3x}{5} - \frac{4}{10} = \frac{2x}{10} +2 $ & \\[8 pt]
	\textbf{c)} $ \displaystyle \frac{1}{2}(x-6)=\frac{1}{6}(5x + 12)$ &
	\textbf{d)} $ \displaystyle \frac{1}{x}+2 = \frac{5}{x} $
\end{tabular}

\nes
\op{lig5}
Samarbeid minimum to og to:\os
\begin{itemize}
	\item Hver person lager et regnestykke med nummeret til dagen de har bursdag på. Regnestykket må ha med minst to regnearter \\($ +,-,\cdot $ og $ : $).\os \textsl{Eksempel:} Har du bursdag 9. mars kan et regnestykke være: \textit{Dagen min ganget med 4, og etterpå fratrekt 6, blir 30.}
	\item Hver person lager et regnestykke med nummeret til måneden de har bursdag på. Regnestykk må ha med minst to regnearter.
	\item Gi regnestykkene til hverandre. Sett opp regnestykkene som ligninger og finn bursdagene.
\end{itemize}
\newpage
\op{lig6}
Ola og Kari tilbyr et matematikk-kurs for dårlige lærere. For hvert kurs tjener de 12000 kr. Ola er assistenten til Kari, og skal Kari ha dobbelt så mye av inntekten som Ola. \os

Hvor mye tjener Ola og hvor mye tjener Kari for hvert kurs?

\op{lig7}
Du skal snekre et gjerde som er 3,4 m langt. For å lage gjerdet skal du bruke 8 planker som er 0,25 m breie, som vist i figuren under:
\fig{gj}
Det skal være den samme avstanden mellom alle plankene. Hvor lang er denne avstanden?

\op{lig8}
Etter å ha blitt satt ned med 35\%, koster en vare nå 845 kr. Hva kostet varen før prisen ble satt ned?


\end{document}

