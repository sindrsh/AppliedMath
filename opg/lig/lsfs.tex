\input{/home/sindre/P/doc}
\usepackage[T1]{fontenc}
\usepackage[utf8]{luainputenc}
\usepackage{lmodern} % load a font with all the characters
\usepackage{geometry}
\geometry{verbose,paperwidth=16.1 cm, paperheight=24 cm, inner=2.3cm, outer=1.8 cm, bmargin=2cm, tmargin=1.8cm}
\setlength{\parindent}{0bp}
\usepackage{import}
\usepackage[subpreambles=false]{standalone}
\usepackage{amsmath}
\usepackage{amssymb}
\usepackage{esint}
\usepackage{babel}
\usepackage{tabu}
\usepackage[dvipsnames, table]{xcolor}
\makeatother
\makeatletter


%referances
\newcommand{\net}[2]{{\color{blue}\href{#1}{#2}}}

%Spaces
\newcommand{\vsk}{\\[12pt]}
\newcommand{\vs}{\vspace{-12pt}}

% Tabell for opplegg

\newcommand{\ovlist}[1]{
\vspace{-16pt}
\begin{itemize}
	#1
\end{itemize}
}

\newcommand{\lst}[5]{
\rule{\linewidth}{1pt}
\footnotesize
	\textbf{Øvingsområde}\\ #1 
	
	\textbf{Utstyr}\\ #2  \\
	
	\begin{tabular}{@{} p{4cm} l} 
		\textbf{Tid} & \textbf{Elevinndeling} \\
		#3  & #4
	\end{tabular} 

\rule{\linewidth}{1pt}	\vsk
\normalsize
	\textbf{Gjennomføring}\\ #5 \vsk
}
%

\newcounter{opl}
%\numberwithin{opl}{article}

\newcommand{\opl}[1]{
\newpage
{\refstepcounter{opl} %\phantomsection 
\large \textbf{\theopl \;#1} \vsk}
}

% Headlines
\newcommand{\fork}{\textbf{Forkunnskapar}\\}
\newcommand{\forb}{\textbf{Forberedelsar}\\}
\newcommand{\opgvr}{\textbf{Oppgaver}}

\usepackage{datetime2}
\usepackage[]{hyperref}
\usepackage{xr}
\externaldocument{/home/sindre/P/bok1P}

\begin{document}

\section*{Løsningsforslag til kapittel \ref*{Lig}}
\opr{lig8}\\ Denne oppgaven kan løses på flere måter:\os
\textit{Løsningsmetode 1}

Når vi tar bort 35\% av et tall sitter vi igjen med $ {100\%-35\%=65\%} $ av tallet. Vi vet derfor at 845 utgjør 65\% av originalprisen:
\[ 65\%\text{ av } \textit{originalpris}=845 \] 
Å finne 65\% av et tall er det samme som å gange med 0,65 (se \hrs{vekstfaktor}{seksjon}). Hvis vi skriver $ x $ istedenfor \textit{originalpris}, får vi:
\alg{
65\% \textit{ av }x &= 845 \\
0,65\cdot x &= 845 \\
\frac{\cancel{0,65} x}{\cancel{0,65}} &= \frac{845}{0,65} \\
x &= 1300
}

\textit{Løsningsmetode 2}\\
Når vi tar bort 35\% av et tall sitter vi igjen med $ {100\%-35\%=65\%} $ av tallet. Vi vet derfor at 845 utgjør 65\% av originalprisen:
\alg{
845 &= 65\% \text{ av } \textit{originalprisen} \\
\frac{845}{65} &= \frac{\bcancel{65}\% \text{ av } \textit{originalprisen}}{\bcancel{65}}\\
13 &= 1\% \text{ av } \textit{originalprisen} 
}
Fordi 13 er 1\% av originalprisen, må originalprisen være $ {13\cdot100}=1300 $, altså 1300 kr. 
\end{document}

