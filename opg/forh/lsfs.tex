\input{/home/sindre/P/doc}
\usepackage[T1]{fontenc}
\usepackage[utf8]{luainputenc}
\usepackage{lmodern} % load a font with all the characters
\usepackage{geometry}
\geometry{verbose,paperwidth=16.1 cm, paperheight=24 cm, inner=2.3cm, outer=1.8 cm, bmargin=2cm, tmargin=1.8cm}
\setlength{\parindent}{0bp}
\usepackage{import}
\usepackage[subpreambles=false]{standalone}
\usepackage{amsmath}
\usepackage{amssymb}
\usepackage{esint}
\usepackage{babel}
\usepackage{tabu}
\usepackage[dvipsnames, table]{xcolor}
\makeatother
\makeatletter


%referances
\newcommand{\net}[2]{{\color{blue}\href{#1}{#2}}}

%Spaces
\newcommand{\vsk}{\\[12pt]}
\newcommand{\vs}{\vspace{-12pt}}

% Tabell for opplegg

\newcommand{\ovlist}[1]{
\vspace{-16pt}
\begin{itemize}
	#1
\end{itemize}
}

\newcommand{\lst}[5]{
\rule{\linewidth}{1pt}
\footnotesize
	\textbf{Øvingsområde}\\ #1 
	
	\textbf{Utstyr}\\ #2  \\
	
	\begin{tabular}{@{} p{4cm} l} 
		\textbf{Tid} & \textbf{Elevinndeling} \\
		#3  & #4
	\end{tabular} 

\rule{\linewidth}{1pt}	\vsk
\normalsize
	\textbf{Gjennomføring}\\ #5 \vsk
}
%

\newcounter{opl}
%\numberwithin{opl}{article}

\newcommand{\opl}[1]{
\newpage
{\refstepcounter{opl} %\phantomsection 
\large \textbf{\theopl \;#1} \vsk}
}

% Headlines
\newcommand{\fork}{\textbf{Forkunnskapar}\\}
\newcommand{\forb}{\textbf{Forberedelsar}\\}
\newcommand{\opgvr}{\textbf{Oppgaver}}

\usepackage{datetime2}
\usepackage[]{hyperref}
\usepackage{xr}
\externaldocument{/home/sindre/P/bok1P}

\begin{document}

\section*{Løsningsforslag til kapittel \ref*{Forh}}

\opr{?}\\
Sirupen utgjør $ \frac{2}{11} $ av saften, noe som betyr at:
\[ \text{100\,g saft inneholder }\frac{2}{11}\cdot100\text{\,g sirup} \]
Og siden det er 44 karbohydrater per 100\,g sirup, får vi at:
\[ \text{100\,g saft inneholder }\frac{2}{11}\cdot\text{44 karbohydrater}\]
\[ \text{100\,g saft inneholder 8 karbohydrater}\]
Altså 2\,g mindre enn Coca-Cola. \\ \vsk
\opr{Gruble?}\\ 
Sirupen utgjør $ \frac{1}{6} $ av 100\,g saft, noe som betyr at:
\[ \frac{100}{6}\text{\,g sirup inneholder 12,5\,g sukker} \]
For å finne hvor mye sukker 100\,g sirup inneholder, ganger vi med 6:
\[ \frac{100}{6}\cdot6\text{\,g sirup inneholder $ 12,5\cdot6 $\,g sukker} \]
\[ \text{100\,g sirup inneholder 75\,g sukker.} \]
\end{document}

