\input{../../doc}
\usepackage[T1]{fontenc}
\usepackage[utf8]{luainputenc}
\usepackage{lmodern} % load a font with all the characters
\usepackage{geometry}
\geometry{verbose,paperwidth=16.1 cm, paperheight=24 cm, inner=2.3cm, outer=1.8 cm, bmargin=2cm, tmargin=1.8cm}
\setlength{\parindent}{0bp}
\usepackage{import}
\usepackage[subpreambles=false]{standalone}
\usepackage{amsmath}
\usepackage{amssymb}
\usepackage{esint}
\usepackage{babel}
\usepackage{tabu}
\usepackage[dvipsnames, table]{xcolor}
\makeatother
\makeatletter


%referances
\newcommand{\net}[2]{{\color{blue}\href{#1}{#2}}}

%Spaces
\newcommand{\vsk}{\\[12pt]}
\newcommand{\vs}{\vspace{-12pt}}

% Tabell for opplegg

\newcommand{\ovlist}[1]{
\vspace{-16pt}
\begin{itemize}
	#1
\end{itemize}
}

\newcommand{\lst}[5]{
\rule{\linewidth}{1pt}
\footnotesize
	\textbf{Øvingsområde}\\ #1 
	
	\textbf{Utstyr}\\ #2  \\
	
	\begin{tabular}{@{} p{4cm} l} 
		\textbf{Tid} & \textbf{Elevinndeling} \\
		#3  & #4
	\end{tabular} 

\rule{\linewidth}{1pt}	\vsk
\normalsize
	\textbf{Gjennomføring}\\ #5 \vsk
}
%

\newcounter{opl}
%\numberwithin{opl}{article}

\newcommand{\opl}[1]{
\newpage
{\refstepcounter{opl} %\phantomsection 
\large \textbf{\theopl \;#1} \vsk}
}

% Headlines
\newcommand{\fork}{\textbf{Forkunnskapar}\\}
\newcommand{\forb}{\textbf{Forberedelsar}\\}
\newcommand{\opgvr}{\textbf{Oppgaver}}

\usepackage{datetime2}
\usepackage[]{hyperref}

\begin{document}

\opl{Kast terning og fyll brettet} 
	\lst
	{
	\ovlist{
	\item Forstå gonging som gjentatt addisjon.
	\item Koble gonging opp mot areal	
	}	
	}
	{
		\textsl{Med digitale hjelpemiddel:} Showbie, Explain Everything og terningar\\
		\textsl{Utan digitale hjelpemiddel:} Utskrift av ruteark, fargeblyantar og terningar}
	{45 min\,+}
	{Grupper på 2-3}
{
\vs
\begin{enumerate}
	\item Kvar elev får utdelt eit $ 10\cdot10 $ ruteark. 
	\begin{center}
		\includegraphics[scale=0.075]{gong1}
	\end{center}
	Utskriftsversjon finn du \net{her}{her}.
	
	\textit{Ved bruk av digitale hjelpemiddel; del fila i Showbie (e.l.), og be elevane laste den inn i Exlain Everything.}
	\item Elevane byttar på å kaste to terningar, desse dannar gangestykker som elevane skal teikne inn i sine ruteark.
\end{enumerate}

	\textit{Eksempel} \\
	I sin første runde får Ola 3 og 2 på sine to terningar. Han kan derfor teikne inn ein boks som er 2 ruter høg og 3 ruter brei, eller omvend, der han sjølv vil:
\begin{center}
	\includegraphics[scale=0.075]{gong1b}
\end{center}
\begin{itemize}
	\item Elevane kan plassere boksane kor dei vil på brettet, men boksar kan ikkje overlappe.
\item Spelet er ferdig når alless deltakarar i samme spelerunde ender opp med boksar dei ikkje har plass til.
\end{itemize}
\textit{For å få med alle gangestykker kan ein bruke 3 terningar i staden for berre to. Da kan elevane gonge summen av to terningar med den resterande terningen så lenge summen blir mindre eller lik 10.}
}

\opl{Heisann}


\end{document}

	
\end{document}