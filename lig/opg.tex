\input{../doc}
\usepackage[T1]{fontenc}
\usepackage[utf8]{luainputenc}
\usepackage{lmodern} % load a font with all the characters
\usepackage{geometry}
\geometry{verbose,paperwidth=16.1 cm, paperheight=24 cm, inner=2.3cm, outer=1.8 cm, bmargin=2cm, tmargin=1.8cm}
\setlength{\parindent}{0bp}
\usepackage{import}
\usepackage[subpreambles=false]{standalone}
\usepackage{amsmath}
\usepackage{amssymb}
\usepackage{esint}
\usepackage{babel}
\usepackage{tabu}
\usepackage[dvipsnames, table]{xcolor}
\makeatother
\makeatletter


%referances
\newcommand{\net}[2]{{\color{blue}\href{#1}{#2}}}

%Spaces
\newcommand{\vsk}{\\[12pt]}
\newcommand{\vs}{\vspace{-12pt}}

% Tabell for opplegg

\newcommand{\ovlist}[1]{
\vspace{-16pt}
\begin{itemize}
	#1
\end{itemize}
}

\newcommand{\lst}[5]{
\rule{\linewidth}{1pt}
\footnotesize
	\textbf{Øvingsområde}\\ #1 
	
	\textbf{Utstyr}\\ #2  \\
	
	\begin{tabular}{@{} p{4cm} l} 
		\textbf{Tid} & \textbf{Elevinndeling} \\
		#3  & #4
	\end{tabular} 

\rule{\linewidth}{1pt}	\vsk
\normalsize
	\textbf{Gjennomføring}\\ #5 \vsk
}
%

\newcounter{opl}
%\numberwithin{opl}{article}

\newcommand{\opl}[1]{
\newpage
{\refstepcounter{opl} %\phantomsection 
\large \textbf{\theopl \;#1} \vsk}
}

% Headlines
\newcommand{\fork}{\textbf{Forkunnskapar}\\}
\newcommand{\forb}{\textbf{Forberedelsar}\\}
\newcommand{\opgvr}{\textbf{Oppgaver}}

\usepackage{datetime2}
\usepackage[]{hyperref}

\begin{document}
\opgt
\nes

\op{lig6}
Ola og Kari tilbyr et matematikk-kurs. For hvert kurs tjener de til sammen 12\,000\,kr. Ola er assistenten til Kari, og Kari skal ha dobbelt så mye av inntekten som Ola. \os

Hvor mye tjener Ola og hvor mye tjener Kari for hvert kurs?

\op{lig7}
Du skal snekre et gjerde som er 3,4 m langt. For å lage gjerdet skal du bruke 8 planker som er 0,25 m breie, som vist i figuren under:
\fig{gj}
Det skal være den samme avstanden mellom alle plankene. Hvor lang er denne avstanden?

\op{lig8}
Etter å ha blitt satt ned med 35\%, koster en vare nå 845 kr. Hva kostet varen før prisen ble satt ned?

\op{frm1} For å regne ut et veldig kjent tall kan vi starte med å gjøre dette: \os
\begin{enumerate}
	\item Start med tallet 2.
	\item Gang så med to 2ere og del med 1 ganger 3.
	\item Gang så med to 4ere og del med 3 ganger 5.
	\item Gang så med to 6ere og del med 5 ganger 7.
\end{enumerate}
Verdien til tallet vi søker får vi hvis vi fra punkt 4 fortsetter uendelig mange punkt videre!\os
\textbf{a)} Skriv opp punkt 5 og 6. \os
\textbf{b)} Gjør som punkt 1 til 6 sier. Hvilket tall tror du vi snakker om?

\op{frm2}
\textit{Makspuls} er et mål på hvor mange hjerteslag hjertet maksimalt kan slå i løpet av et minutt. På siden \href{http://www.trening.no/utholdenhet/ny-formel-for-beregning-av-makspuls/}{\color{blue}trening.no} kan man lese dette:\os
''Den tradisjonelle metoden å estimere maksimalpuls er å ta utgangspunkt i 220 og deretter trekke fra alderen.''\os

\textbf{a)} Kall ''maksimalpuls'' for $ m $ og ''alder'' for $ a $ og lag en formel for $ m $, som beskrevet i sitatet. \os
\textbf{b)} Bruk formelen fra a) til å regne ut makspulsen din.\vsk

På den samme siden kan vi lese at en ny og bedre metode er slik:\os
''Ta din alder og multipliser dette med 0,64. Deretter trekker du dette fra 211.''\os

\textbf{c)} Lag en formel for $ m $, som beskrevet i sitatet.\os

\textbf{d)} Bruk formelen fra c) til å regne ut makspulsen din.

\vsk
For å fysisk måle makspulsen din kan du gjøre dette:
\begin{itemize}
	\item Hopp opp og ned opp og ned i ca. 10 sekunder (da vil hjertet ditt omtrent slå så raskt det kan en liten stund etter).
	\item Tell hjerteslag umiddelbart etter hoppingen.
	\item Tell i 15 sekunder.
\end{itemize}
\textbf{e)} Kall ''antall hjerteslag i løpet av 15 sekunder'' for $ A $ og lag en formel for $ m $.\os
\textbf{f)} Bruk formelen fra e) til å regne ut makspulsen din.\os
\textbf{g)} Sammenlign resultatene fra b), d) og f), er de like eller forskjellige?

\nes
\op{frm3}
På nettsiden \net{http://www.viivilla.no/}{viivilla.no} får vi vite at dette er formelen for å lage en perfekt trapp:\os
''2 ganger opptrinn (trinnhøyde) pluss 1 gang inntrinn (trinndybde) bør bli 62 centimeter (med et slingringsmonn på et par centimeter).'' \os
\textbf{a)} Kall ''trinnhøyden'' for $ h $ og ''trinndybden'' for $ d $ og skriv opp formelen i sitatet (uten slingringsmonn).\os
\textbf{b)} Sjekk trappene på skolen, er formelen oppfylt eller ikke?\os

\textbf{c)} Hvis ikke: Hva måtte trinnhøyden vært for at formelen skulle blitt oppfylt?\os

\textbf{d)} Skriv om formelen til en formel for $ h $.

\begin{comment}
\op{frm4}
Formelen for BMI (Body Mass Index) ser slik ut:
\[ \text{BMI}=\frac{m}{h^2} \]
hvor $ m $ betyr en persons vekt (i kg) og $ h $ er personens høyde (i meter).\os

\textbf{a)} Hvis en person har $ {\text{BMI}=29} $ og er 2\,m høy, hvor mye veier da personen?\os
\textbf{b)} Hvis en person har $ {\text{BMI}=23} $ og veier 75\,kg, hvor høy er personen?\os
{\small 
\obs BMI er et mål som er lagd for å studere relasjonen mellom vekt og høyde \textsl{for store folkegrupper}. Det kan være interessant å vite hva BMI-en til 1000 mennesker er, men BMI-en til enkeltmennesker sier svært lite om personen. For eksempel blir både Aksel Lund Svindal og Ragnhild Mowinckel, to av Norges best trente atleter, definert som overvektige på BMI-skalaen.
\begin{center}
\includegraphics[scale=1]{aksel}
\includegraphics[scale=0.145]{rag}
\end{center}
Vi har brukt BMI-en til enkeltpersoner i oppgaven fordi det gir oss øving i formelregning.}
\end{comment}
\op{frm4}
Effekten $ P $ (målt i Watt) i en elektrisk krets er gitt ved formelen:
\[ P=R\cdot I^2 \]
hvor $ R $ er motstanden og $ I $ er strømmen i kretsen.\os
\textbf{a)} Hvis $ {R=5\,\Omega} $ og $ {I=10\,A} $, hva er da effekten?\os
\textbf{b)} Skriv om formelen til en formel for $ I^2 $.

\op{frm5}
På
\net{http://www.klikk.no/foreldre/smabarn/regn-ut-barnets-hoyde-som-voksen-2446226}{klikk.no} finner man disse formelene for å regne ut hvor høy et barn kommer til å bli:\os

\textit{For jenter:}
\begin{enumerate}
	\item Legg sammen mors høyde i cm + fars høyde i cm
	\item Trekk fra 13 cm
	\item Del tallet på to
\end{enumerate}

\textit{For gutter:}
\begin{enumerate}
	\item Legg sammen mors høyde i cm + fars høyde i cm
	\item Legg til 13 cm
	\item Del tallet på to
\end{enumerate}
Kall barnets (fremtidige) høyde for $ B $, mors høyde for $ M $ og fars høde for $ F $.\os
\textbf{a)} Lag en formel for $ B $ når barnet er ei jente.\os
\textbf{b)} Lag en formel for $ B $ når barnet er en gutt.\os
\textbf{c)} Gjør om formelen fra a) til en formel for $ F $.\os
\textbf{d)} Ei jente har en mor som er 165\,cm. Når jenta er utvokst kommer hun til å bli 171\,cm. Hvor høy er faren til jenta?

\end{document}

