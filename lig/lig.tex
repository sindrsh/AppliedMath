\input{/home/sindre/P/doc}
\usepackage[T1]{fontenc}
\usepackage[utf8]{luainputenc}
\usepackage{lmodern} % load a font with all the characters
\usepackage{geometry}
\geometry{verbose,paperwidth=16.1 cm, paperheight=24 cm, inner=2.3cm, outer=1.8 cm, bmargin=2cm, tmargin=1.8cm}
\setlength{\parindent}{0bp}
\usepackage{import}
\usepackage[subpreambles=false]{standalone}
\usepackage{amsmath}
\usepackage{amssymb}
\usepackage{esint}
\usepackage{babel}
\usepackage{tabu}
\usepackage[dvipsnames, table]{xcolor}
\makeatother
\makeatletter


%referances
\newcommand{\net}[2]{{\color{blue}\href{#1}{#2}}}

%Spaces
\newcommand{\vsk}{\\[12pt]}
\newcommand{\vs}{\vspace{-12pt}}

% Tabell for opplegg

\newcommand{\ovlist}[1]{
\vspace{-16pt}
\begin{itemize}
	#1
\end{itemize}
}

\newcommand{\lst}[5]{
\rule{\linewidth}{1pt}
\footnotesize
	\textbf{Øvingsområde}\\ #1 
	
	\textbf{Utstyr}\\ #2  \\
	
	\begin{tabular}{@{} p{4cm} l} 
		\textbf{Tid} & \textbf{Elevinndeling} \\
		#3  & #4
	\end{tabular} 

\rule{\linewidth}{1pt}	\vsk
\normalsize
	\textbf{Gjennomføring}\\ #5 \vsk
}
%

\newcounter{opl}
%\numberwithin{opl}{article}

\newcommand{\opl}[1]{
\newpage
{\refstepcounter{opl} %\phantomsection 
\large \textbf{\theopl \;#1} \vsk}
}

% Headlines
\newcommand{\fork}{\textbf{Forkunnskapar}\\}
\newcommand{\forb}{\textbf{Forberedelsar}\\}
\newcommand{\opgvr}{\textbf{Oppgaver}}

\usepackage{datetime2}
\usepackage[]{hyperref}
\begin{document}
Mål for opplæringen er at eleven skal kunne	
\begin{itemize}
	\item forenkle uttrykk med flere ledd og løse liginger av første grad og enkle potensligninger
\end{itemize}
\newpage
\textsl{Obs!} Hvis du syns du forstår deg på ligninger, men bare trenger litt repetisjon, kan du hoppe til \hrs{ligsaml}{seksjon}
\section{Introduksjon}
Si at vi ønsker å finne et tall som er slik at hvis vi legger til $3$, så får vi $7$. Når et tall er ukjent for oss, er det vanlig å kalle det for $ x $. \textit{Ligningen} for dette tallet blir:
\[ x+4=7 \]
Tegnet ''$ = $'', som betyr ''er lik'', gir oss en veldig viktig informasjon. Det forteller oss at \textsl{venstresiden er akkurat like stor som høyresiden}, selv om vi ikke vet hva verdien til $ x $ er!
\[ \underbrace{x+4}_\text{venstreside}\,\underbrace{=}_\text{er lik }\underbrace{7}_\text{høyreside} \]
Som en figur kan vi se for oss en vekt med $ x+4 $ på den ene siden og 7 på den andre. ''$ = $''-tegnet kan vi bruke så lenge vekten er den samme på begge sider.
\begin{figure}
\centering
\includegraphics[]{\asym{lig}}\quad
\includegraphics[]{\asym{lig1}}
\end{figure}
Når vi skal løse ligninger er det alltid lov til å se eller prøve seg fram til hva som er rett svar. Kanskje har du allerede merket at $ {x=3} $ er løsningen på vår ligning fordi: 
\[ 3+4=7 \]
Men de fleste ligninger er det vanskelig å se svaret på, derfor skal vi i de fire neste seksjonene se på andre metoder vi kan bruke for å finne verdien til $ x $.
\section{Tall som skifter side}
\textbf{Første eksempel}\os
Vi starter med å se hvordan vi kan løse ligningen
\[ x+4=7 \]
ved at et \textsl{tall skifter side}.\\
\begin{comment}
\info{Hva er et ledd?}{Et ledd }
\end{comment}

\prbxl{0.5}{
Vi skal fortsatt bruke figurer av ligningene våre, men det blir tungvint å måtte tegne vekter hele tiden. Isteden tegner vi nå ligningen vår slik:
}\qquad
\prbxr{0.4}{På høyesiden i figuren  har vi skrevet $ +7 $, men det er vanligst å bare skrive 7 (slik som i ligningen).}
\fig{lig2}\vspace{2pt}
Prinsippet er likevel det samme som før: Vi må hele tiden passe på at begge sider veier like mye.\vsk

Det blir tydelig hva vekten til $ x $ er hvis $ x $ står alene på én av sidene. $ x $ blir for seg selv på venstresiden hvis vi tar bort de 4 rutene som står der. Men skal vi ta bort 4 ruter fra venstresiden må vi ta bort 4 ruter fra høyresiden også (for at begge sider skal veie like mye):
\fig{lig3}
Dette skriver vi som:\
\algv{
x+\overbrace{4-{\color{red}4}}^0&=7-{\color{red}4}  \\
x &= 3
}
Men fordi $ {4-{\color{red}4}=0} $, er det egentlig unødvendig å skrive dette. Hele veien til løsningen skriver vi da slik:

\prbxl{0.5}{
	\alg{
		x+4 &= 7 \\
		x&= 7-4 \\
		x &= 3
}}
\prbxr{0.5}{Mellom første og andre linje er det vanlig å si at \textsl{4 har skiftet side, og derfor også fortegn (fra $ + $ til $ - $).}}
\textbf{Andre eksempel}\os
La oss gå videre til å se på en litt vanskeligere ligning:
\[ 4x-2=3x+5 \]
\prbxl{0.6}{Vi har forsatt lyst til å bruke figurer for å hjelpe oss med å forstå hvordan vi kan løse ligningen, men hvordan kan vi tegne $ -2 $? Jo, vi tenker oss at rosa kuler er ballonger som løfter vekten vår oppover.}\qquad
\prbxr{0.3}{Tenk på de rosa kulene som heliumballonger.}

\prbxl{0.6}{Én rosa kule er akkurat nok til å løfte én blå rute, altså: \textsl{Én blå rute legger til 1 på vekten vår, mens én rosa kule trekker ifra 1}. Ligningen blir da seende slik ut:{}}\qquad
\prbxr{0.3}{Å trekke ifra 1 er det samme som å skrive $ -1 $.}

\fig{lig4}
Vi kan først legge merke til at vi kan ta bort tre $ x$-er på begge sider:
\fig{lig5}
For å utligne ballongene, legger vi til to ruter på venstre side. Da må vi også legge til to ruter på høyre side:
\fig{lig6}
Men fordi de to rutene og de to ballongene til sammen veier 0, kan vi liksågodt ta dem bort:
\fig{lig7}
Det vi har tegnet i figurene våre kan vi oppsummere slik:
\begin{flalign*}
&& 4x-2&=3x+5 && \llap{1. figur} \\
&& 4x-{\color{red}3x}-2&=3x-{\color{red}3x}+5 && \\
&& x -2 &= 5 &&\llap{2. figur}\\
&& x-2+\color{blue}2&=  5+\color{blue}2 &&\llap{3. figur}\\
&& x &= 7 &&\llap{4. figur}
\end{flalign*}
Fordi \y{3x-{\color{red}3x}=0} og \y{-2+{\color{blue}2}=0} kunne vi droppet å skrive disse leddene. Isteden sier vi altså at $ 3x $ og 2 har skiftet side, og derfor også fortegn. Dette skiftet gjør vi helst samtidig for at utregningen skal bli kortere:
\alg{
4x-2&=3x+5 \\
4x-{\color{red}3x}&=5+\color{blue}2\\
x &= 7
}

\reg[Flytting av tall over likhetstegnet \label{bytt}]{I en ligning ønsker vi å samle alle $x$-er og alle kjente tall på hver sin side av likhetstegnet. Når side skiftes, må også fortegn skiftes.}

\eks[1]{Løs ligningen:
\[ 3x+3 =2x+5 \]
\sv	
	 \vs \vs \vs \vs
	\begin{align*}
	3x-2x &=5-3 \\
	x &=2
	\end{align*}  \vspace{-20pt}}

\eks[2]{Løs ligningen:
\[ -4x-3 =-5x+12 \\ \]	
\sv
	 \vs \vs \vs \vs
	\begin{align*}
	-4x+5x &=12+3 \\
	x &=15
	\end{align*}  \vspace{-20pt}	}
\section{Deling på begge sider av likhetstegnet}	
Hittil har vi sett på ligninger hvor vi endte opp med én $x $ på den ene siden. Andre ganger ender vi med flere $ x $-er, som for eksempel i ligningen:
\[ 3x=6 \]
\fig{lig8}
Deler vi venstresiden vår i tre like biter, får vi én $ x $ i hver bit. Om vi også deler høyresiden i tre like biter blir det tydeligere hva $ x $ må være:
\fig{lig9}
\fig{lig10}
Figurene vi har tegnet over skriver vi som:\\ \vs
\prbxl{0.6}{\begin{flalign*}
	&& 3x&=6 && \llap{1. figur} \br
	&& \frac{3x}{3}&=\frac{6}{3} && \llap{2. figur} \br
	&& x&=2 && \llap{3. figur}
	\end{flalign*}}\qquad
\prbxr{0.3}{Fra \hr{kans} husker du kanskje også at vi gjerne skriver:
\[ \frac{\cancel{3}x}{\cancel{3}} \] \vs
}

\reg[Deling på begge sider av likhetstegnet \label{ligdel}]{Hvis vi har et tall ganget med $ x $ på én side av en ligning, kan vi dele begge sider med tallet for å finne $ x $.}
\eks[1]{ Løs ligningen:
	\[ 	4x = 20  \]
	\sv \vs \vs \vsb
	\begin{align*}
	\frac{\bcancel{4}x}{\bcancel{4}}&=\frac{20}{4} \\
	x &=5
	\end{align*}
	\vspace{-20 pt}
	}
	
\eks[2]{Løs ligningen:
	\[2x+6 =3x-2 \]
\sv \vs \vs \vs \vs
	\begin{flalign*}
	&& 2x-3x &= -2-6 &&\\
	&&-x &= -8&& \\
	&& \frac{\bcancel{-1}x}{\bcancel{-1}} &= \frac{-8}{-1} &&\cm{$-x=-1x$}\\
	&& x &= 8&&
	\end{flalign*}\vspace{-20 pt}}

\section{Multiplikasjon på begge sider av likhetstegnet}
Det siste tilfellet vi skal se på, er når ligninger forteller oss noe om brøkdeler av den ukjente, som for eksempel denne:
\[ \frac{x}{3}=4 \]
Som vi tegner slik:
\fig{lig11}
Igjen er ligningen vår løst hvis én $ x $-rute står alene, noe vi får om vi legger til to stykker av $ \frac{x}{3} $ på venstresiden. Og ligningen forteller oss at $ \frac{x}{3} $ er det samme som 4; for hver $ \frac{x}{3} $ vi legger til på venstresiden, må vi derfor legge til 4 ekstra på høyresiden for at sidene skal veie det samme:
\fig{lig12}
\fig{lig13}
Legg nå merke til at \y{\frac{x}{3}+\frac{x}{3}+\frac{x}{3}=\frac{x}{3}\cdot3} og at \y{4\cdot3=12}, figurene vi har tegnet kan vi derfor skrive som:
\begin{flalign*}
&& \frac{x}{3}&=4 && \llap{1. figur} \br
&& \frac{x}{3}\cdot 3&=4\cdot3 && \llap{2. figur} \\
&& x&=12 && \llap{3. figur}
\end{flalign*}
\reg[Brøker med \boldmath$ x $ som teller]{
Hvis vi har en brøkdel av $ x $ på den éne siden i en ligning, kan vi gange begge sider med nevneren for å finne $ x $.
}
\eks[1]{
Løs ligningen:
\[ \frac{x}{5}=2 \]
\sv \vsb
\algvv{
\frac{x}{\cancel{5}}\cdot\cancel{5}&=2\cdot5 \\
x &= 10
}
}

\section{Løsningsmetodene oppsummert \label{ligsaml}}
En ligning er løst når vi har $ x $ alene på én side av likhetstegnet, og vi har mange veier vi kan gå for å få til dette. Det vi har prøvd å skape et bilde av i de tre forrige seksjonene er at: \textsl{Vi kan legge til, trekke ifra, gange eller dele med et hvilket som helst tall, så lenge vi gjør det på begge sider av likhetstegnet}. \vsk

Vi har også sett at istedenfor å legge til eller trekke ifra det samme tallet på begge sider, flytter vi tall vi allerede har fra den éne siden til den andre. Og da må vi huske å skifte fortegnet til tallet.\vsk

\textsl{Obs!} Husk at $ x $ også er et tall.\regv 

\reg[Løsningsmetoder for ligninger \label{lsmlig}]{
I en ligning kan vi alltid:
\begin{itemize}
\item Flytte tall fra den ene siden til den andre, så lenge vi også skifter fortegn på leddet.
\item Dele både venstre og høyre side med det samme tallet.
\item Gange både venstre og høyre side med det samme tallet.
\end{itemize}
}
\eks{ Løs ligningen:
	\[ \frac{1}{3}x+\frac{1}{6}=\frac{5}{12}x+2  \vs\]	
	
	\sv
	Får å unngå brøker, ganger vi begge sider med fellesnevneren 12:
	\begin{align*}
	\left(\frac{1}{3}x+\frac{1}{6}\right)12&=\left(\frac{5}{12}x+2\right)12 \br	
\frac{1}{3}x\cdot12+\frac{1}{6}\cdot12&=\frac{5}{12}x\cdot12+2\cdot12 \tag{$ \ast $}\br	
	4x+2 &= 5x+24 \\
	4x-5x &= 24-2 \\
	-x &= 22 \\
	\frac{\cancel{-1}\,x}{\cancel{-1}} &= \frac{22}{-1} \\
	x &= -22
	\end{align*} 
}
\info{Tips}{
Noen liker å lage regelen om at \textsl{vi kan gange eller dele alle ledd med det samme tallet}. I eksempelet over kunne vi da hoppet direkte til andre linje i utregningen (markert med $ (\ast) $).
}
\eks[]{
Løs ligningen:
\[ 3-\frac{6}{x}= 2+\frac{5}{x} \vs\]

\sv
I alle ligninger ønsker vi å få $ x $ alene på én side og over aller brøkstreker. For få $ x $ over alle brøkstreker, ganger vi med $ x $ i alle ledd:
\alg{
3\cdot x-\frac{6}{\cancel{x}}\cdot \cancel{x}= 2\cdot x+\frac{5}{\cancel{x}}\cdot \cancel{x} \\
3x -6 = 2x+5
}
Nå er veien kort til å få $ x $ på én side:
\alg{
3x-2x &= 5+6 \\
x &= 11
}
}
\section{Potensligninger}\vs
\prbxl{0.6}{Noen ganger har vi ligninger der $ x $ er ganget flere ganger med seg selv, som for eksempel denne:}\prbxr{0.3}{ $ {x\cdot x }$ skriver vi som $ x^2 $}\vs 
\[ x^2 = 9 \]
Fordi $ x $ er grunntallet i en potens, kaller vi dette en \textit{potensligning}.
Spørsmålet vårt blir nå: \textsl{Hvilket tall ganget med seg selv blir 9}? Svaret er \textsl{både} 3 og $ (-3) $! Lignigen har altså to løsninger. Derfor sier vi at både $ {x=3} $ og $ {x=-3} $ er løsninger av ligningen.\vsk

Hva med ligningen:
\[ x^3 = 8\]
Her er $ {x=2} $ eneste løsning, for det er bare hvis vi ganger sammen tre 2ere at vi får 8.\vsk

Potensligninger kan selvsagt se mye verre ut enn de to vi har sett på, men heldigvis kan vi bruke akkurat de samme reglene som i \hr{lsmlig}:
\reg[Potensligninger]{En ligning med uttrykket $ x^n $ kalles en potensligning. Ligningen løses ved å bruke metodene fra \hr{lsmlig} for å isolere $ x^n $ på én side av likhetstegnet.}
\eks[1]{
Løs ligningen: \[ x^2+5= 21\]
\sv \vs \vs \vs
\algv{
x^2+5&= 21\\
x^2 &= 21-5\\
x^2 &= 16
}
Fordi $ {4\cdot4 =16} $ og $ {(-4)\cdot(-4)=16} $ er både $ {x=4} $ og ${ x=-4} $ løsninger av ligningen.
}
\eks[2]{
Løs ligningen:
\[ 5x^3 -35 = 4x^3-8 \]
\sv \vs \vs
\algv{
5x^3 -35 &= 4x^3-8 \\
5x^3-4x^3 &= -8+35 \\
x^3 &= 27 
}
Fordi $ {3\cdot3\cdot3=27} $ er $ {x=3} $ løsningen av ligningen.
}

\section{Å lage ligninger}
Hver gang vi i hverdagen må utføre en regneoperasjon, løser vi egentlig en ligning! \vsk

Tenk for eksempel at du skal kjøpe 2 kg epler i butikken, og eplene koster 10 kr/kg. Regnestykket ditt blir da dette:
\small
\[ \textit{hva jeg må betale for eplene}=\textit{antall kg epler}\cdot \textit{kiloprisen for epler} \]
\normalsize
Hvis vi bestemmer oss for at $ x $ betyr det samme som \textit{hva jeg må betale for eplene}, blir ligningen vår seende slik ut:
\[ x=\textit{antall kg epler}\cdot \textit{kiloprisen for epler} \]
Og fordi vi vet både hva \textit{antall kg epler} og \textit{kiloprisen for epler} er, kan vi finne svaret:\vs
\alg{
x &= 2\cdot10 \\
 &= 20
}
Vi må altså betale 20 kr for eplene. \vsk 

Her kunne vi selvsagt regnet ut prisen for eplene direkte, men for lengre utregninger er det lurt å lage en ligning. Og det blir oftere lettere for oss å lage ligningen hvis vi gjør som i det korte eksempelet med eplene.\regv
\reg[Å lage en ligning]{
Når vi skal beskrive et spørsmål som en ligning kan det være lurt å gjøre følgende:
\begin{itemize}
	\item Sette opp regnestykket i ord.
	\item Erstatte den ukjente størrelsen med $ x $.
\end{itemize}
}
\eks[1]{
Tenk at klassen ønsker å dra på en klassetur som til sammen koster 11\,000\,kr. For å dekke utgiftene har dere allerede skaffet 2\,000\,kr, resten skal skaffes gjennom loddsalg. For hvert lodd som selges, tjener dere 25\,kr.\os

\textbf{a)} Lag en ligning for hvor mange lodd klassen må selge for å få råd til klasseturen.\os
\textbf{b)} Løs ligningen.

\sv
\textbf{a)} Vi starter med å tenke oss regnestykket i ord:
\small
\[ \textit{penger allerede skaffet}+\textit{antall lodd}\cdot\textit{penger per lodd}=\textit{prisen på turen} \]
\normalsize
Den eneste størrelsen vi ikke vet om er \textit{antall lodd}. Vi erstatter derfor \textit{antall lodd} med $ x $, og setter inn verdien til de andre:
\[ 2\,000+x\cdot25 = 11\,000 \]

\textbf{b)} \\ \vspace{-20pt}
\prbxl{0.7}{\alg{
		25x &= 11\,000-2\,000\\
		25 x &= 9\,000\\
		\frac{\cancel{25} x}{\cancel{25}} &= \frac{9\,000}{25} \\
		x &= 360
}}
\prbxr{0.25}{$ {x\cdot25 }$ er skrevet om  til $ 25x $.}
}
\eks[2]{
''Broren min er dobbelt så gammel som meg. Til sammen er vi 9 år gamle. Hvor gammel er jeg?''.

\sv
''Broren min er dobbelt så gammel som meg.'' betyr at:
\[ \textit{brors alder}=2\cdot\textit{min alder} \]
''Til sammen er vi 9 år gamle.'' betyr at:
\[ \textit{brors alder}+\textit{min alder}=\textit{9 år} \]
Erstatter vi \textit{brors alder} med ''$2\cdot\textit{min alder} $'', får vi:
\[ 2\cdot\textit{min alder}+\textit{min alder}=\text{9 år} \]
Det som er ukjent for oss er \textit{min alder}:
\alg{
2x+x &= 9 \\
3x &= 9\\
\frac{\cancel{3}x}{\cancel{3}}&= \frac{9}{3} \\
x &= 3
}
''Jeg'' er altså 3 år gammel.
}

\end{document}


