\input{../doc}
\usepackage[T1]{fontenc}
\usepackage[utf8]{luainputenc}
\usepackage{lmodern} % load a font with all the characters
\usepackage{geometry}
\geometry{verbose,paperwidth=16.1 cm, paperheight=24 cm, inner=2.3cm, outer=1.8 cm, bmargin=2cm, tmargin=1.8cm}
\setlength{\parindent}{0bp}
\usepackage{import}
\usepackage[subpreambles=false]{standalone}
\usepackage{amsmath}
\usepackage{amssymb}
\usepackage{esint}
\usepackage{babel}
\usepackage{tabu}
\usepackage[dvipsnames, table]{xcolor}
\makeatother
\makeatletter


%referances
\newcommand{\net}[2]{{\color{blue}\href{#1}{#2}}}

%Spaces
\newcommand{\vsk}{\\[12pt]}
\newcommand{\vs}{\vspace{-12pt}}

% Tabell for opplegg

\newcommand{\ovlist}[1]{
\vspace{-16pt}
\begin{itemize}
	#1
\end{itemize}
}

\newcommand{\lst}[5]{
\rule{\linewidth}{1pt}
\footnotesize
	\textbf{Øvingsområde}\\ #1 
	
	\textbf{Utstyr}\\ #2  \\
	
	\begin{tabular}{@{} p{4cm} l} 
		\textbf{Tid} & \textbf{Elevinndeling} \\
		#3  & #4
	\end{tabular} 

\rule{\linewidth}{1pt}	\vsk
\normalsize
	\textbf{Gjennomføring}\\ #5 \vsk
}
%

\newcounter{opl}
%\numberwithin{opl}{article}

\newcommand{\opl}[1]{
\newpage
{\refstepcounter{opl} %\phantomsection 
\large \textbf{\theopl \;#1} \vsk}
}

% Headlines
\newcommand{\fork}{\textbf{Forkunnskapar}\\}
\newcommand{\forb}{\textbf{Forberedelsar}\\}
\newcommand{\opgvr}{\textbf{Oppgaver}}

\usepackage{datetime2}
\usepackage[]{hyperref}
\begin{document}

\section{Å finne størrelser}
Likninger, formler og funksjoner (og utttrykk) er begreper som dukker i forskjellige sammenhenger, men i bunn og grunn handler de om det samme; \textsl{de uttrykker relasjoner mellom størrelser}. Når alle størrelsen utenom den ene er kjent, kan vi finne den siste størrelsen enten direkte eller indirekte.\vsk


\subsection{Å finne størrelser direkte}
\subsubsection{Når formelen er kjent}
Når formler er gitt, er det snakk om å sette inn verdier og regne ut. Formler har vi brukt i mange kapitler allerede, omtrent alle regelboksene i boka inneholder en formel. 
Men så langt har vi brukt formler med dimensjonsløse størrelser. Når vi har størrelser med enheter er det helt avgjørende at vi passer på at enhetene som er involvert er de samme.

\regv

\eks[1]{
Hvis du kjører med konstant fart, er strekningen du har kjørt etter en viss tid gitt ved formelen
\[ \text{strekning}=\text{fart}\cdot \text{tid} \]

\abc{
\item Hvor langt kjører en bil som holder farten 50\enh{km/h} i 3 timer?
\item Hvor langt kjører en bil som holder farten  90\enh{km/h} i 45 minutt?
}

\sv
\abc{
\item I formelen er nå  farten $ 50 $ og tiden $ 3 $, og da er
\[ \text{strekning}=50\cdot3=150 \]
Altså har bilen kjørt 150\enh{km} 
\item Her har vi to forskjellige enheter for tid involvert; timer og minutt. Da må vi enten gjøre om farten til km/min eller tiden til timer. Vi velger å gjøre om minutt til timer:
\alg{
45\enh{minutt}&=\frac{45}{60}\enh{timer} \br
&=\frac{3}{4}\enh{timer}
}
}
I formelen er nå farten 90 og tiden $ \dfrac{3}{4} $, og da er
\[ \text{strekning}=90\cdot\frac{3}{4}=67.5\]
Altså har bilen kjørt 67.5\enh{km}.
}
\eks[2]{
\textit{Kiloprisen} til en vare er hva en vare koster per kg. For en hvilken som helst vare har vi at
\[ \text{kilopris}=\frac{\text{pris}}{\text{vekt}} \]
\abc{
	\item 10\enh{kg} tomater koster 35\enh{kr}. Hva er kiloprisen til tomatene?
	\item Safran går for å være verdens dyreste krydder, 5\enh{g} kan koste 600\enh{kr}. Hva er da kiloprisen på safran?
}
\sv
\abc{
\item I formelen er nå prisen 35 og vekten 10, og da er
\[ \text{kilopris}=\frac{35}{10}=3,5 \]
Altså er kiloprisen på tomater 3,5\enh{kr/kg}
\item Her har vi to forskjellige enheter for vekt involvert; kg og gram. Vi gjør om antall g til antall kg (se ??):
\[ 5\enh{g}=0,005\enh{kg} \]
}
I formelen vår er nå prisen 600 og vekten 0,005, og da er
\[ \text{kilopris}=\frac{600}{0,005}=120\,000 \]
Altså koster safran 120\,000\enh{kr/kg}.
}
\begin{comment}
	\info{Merk}{
I de to eksemplene over har vi brukt fulle ord i formlene våre, men det lønner seg å bli vant til å bare bruke enkeltbokstaver. I \textsl{Eksempel 1} kunne vi kalt farten $ f $, strekningen $ s $, og tiden $ t $. Da ville formelen sett slik ut:
\[ s=f\cdot t \]
Legg også merke til at du i seksjon ?? og ?? allerede har reknet mye med formler gitt ved enkeltbokstaver!
}
\end{comment}
\subsubsection{Når formlene er ukjente}
Når vi bare får en beskrivelse av en situasjon, må vi selv lage formlene. Da gjelder det å identifisere hvilke størrelser som er ukjente, og finne relasjonen mellom dem.\regv
\eks[1]{
For en taxi er det følgende kostnader:
\begin{itemize}
	\item Du må betale 50\enh{kr} uansett hvor langt du blir kjørt.
	\item I tillegg betaler du 15\enh{kr} for hver kilometer du blir kjørt.
\end{itemize}
\abc{
\item Sett opp et uttrykk for hvor mye taxituren koster for hver kilometer du blir kjørt.
\item Hva koster en taxitur på 17\enh{km}?
}

\sv
\abc{
\item Her er det to ukjente størrelser; \textsl{kostnaden for taxituren} og \textsl{antall kilometer kjørt}. Relasjonen mellom dem er denne:
	\[ \text{kostnaden for taxituren}=50+15\cdot\text{antall kilometer kjørt} \]
\item Vi har nå at
\[ \text{kostnaden for taxituren}=50+15\cdot17= 305 \]
Taxituren koster altså 305\enh{kr}.
}
}
\info{Tips}{
Ved å la enkeltbokstaver representere størrelser får man kortere uttrykk. La $ k $ stå for \textsl{kostnad for taxituren} og $ x $ stå for \textsl{antall kilometer kjørt}. Da blir uttrykket fra \textsl{Eksempel 1} over dette:
\[ k=50+15x \]
I tillegg kan man gjerne bruke skrivemåten for funksjoner:
\[ k(x)=50+15x \] 
}
\subsection{Å finne størrelser indirekte}
\subsubsection{Når formlene er kjente}
\eks[1]{
	Vi har sett (\hyperref[farteks]{\textsl{Eksempel 1}}, s. \pageref{farteks}) at strekningen $ s $ vi har kjørt, farten $ f $ vi har holdt og tiden $ t $ vi har brukt kan settes i sammenheng via formelen\footnote{$ \text{strekning}=\text{fart}\cdot \text{tid} $}:
	\[ s = f\cdot t \] 
	Siden $ s $ står alene på én side av likhetstegnet, sier vi at dette er \textsl{en formel for} $ s $. Ønsker vi en formel for $ f $, kan vi gjøre om formelen ved å følge prinsippene for likninger (se \mb, s. ??):
	\alg{
		s &= f\cdot t \br
		\frac{s}{t}&=\frac{f\cdot \bcancel{t}}{\bcancel{t}} \br
		\frac{s}{t}&=f
	}
}
\eks[2]{
	\textit{Ohms lov} sier at strømmen $ I $ gjennom en metallisk leder (med konstant temeperatur) er gitt ved formelen
	\[ I = \frac{U}{R} \]
	hvor $ U $ er spenningen og $ R $ er resistansen. \os
	\textbf{a)} Skriv om formelen til en formel for $ R $.
	\vsk
	
	Strøm måles i Ampere (A), spenning i Volt (V) og motstand i Ohm ($ \Omega $).\os
	\textbf{b)} Hvis strømmen er 2\,A og spenningen 12\,V, hva er da resistansen?
	
	\sv
	\textbf{a)} Vi gjør om formelen slik at $ R $ står alene på én side av likhetsregnet:\vs
	\alg{
		I\cdot R&=\frac{U\cdot \cancel{R}}{\cancel{R}} \br
		I\cdot R &= U \br
		\frac{\cancel{I}\cdot R}{\cancel{I}} &= \frac{U}{I}\br 
		R &= \frac{U}{I}
	}
	\textbf{b)} Vi bruker formelen vi fant i a) og får at:
	\alg{
		R &= \frac{U}{I} \br
		&= \frac{12}{2} \\
		&= 6
	}
	Resistansen er altså $ 6\,\Omega $.
}
\eks[3]{
	Si vi har målt en temperatur $ T_C $ i grader Celsius ($ ^\circ C $). Temperaturen $ T_F $ målt i Fahrenheit ($ ^\circ F $) er da gitt ved formelen:
	\[ T_F = \frac{9}{5}\cdot T_C+32 \]
	\textbf{a)} Skriv om formelen til en formel for $ T_C $.\os
	\textbf{b)} Hvis en temperatur er målt til 59$ ^\circ F $, hva er da temperaturen målt i $ ^\circ C $?
	
	\sv
	\textbf{a)} \alg{
		T_F &= \frac{9}{5}\cdot T_C+32 \\
		T_F-32 &= \frac{9}{5}\cdot T_C \\
		5(T_F-32) &= \cancel{5}\cdot\frac{9}{\cancel{5}}\cdot F_C \\
		5(T_F-32) &= 9T_C \\
		\frac{5(T_F-32)}{9} &= \frac{\cancel{9}T_C}{\cancel{9}} \\
		\frac{5(T_F-32)}{9} &= T_C
	}
	\textbf{b)} Vi bruker formelen fra a), og finner at:
	\alg{
		T_C&= \frac{5(59-32)}{9} \br
		&= \frac{5(27)}{9} \br
		&= 5\cdot 3 \\
		&= 15
	}
}

\subsubsection{Når formlene er ukjente}
\eks[1]{
Tenk at klassen ønsker å dra på en klassetur som til sammen koster 11\,000\,kr. For å dekke utgiftene har dere allerede skaffet 2\,000\,kr, resten skal skaffes gjennom loddsalg. For hvert lodd som selges, tjener dere 25\,kr.\os

\textbf{a)} Lag en ligning for hvor mange lodd klassen må selge for å få råd til klasseturen.\os
\textbf{b)} Løs ligningen.

\sv
\textbf{a)} Vi starter med å tenke oss regnestykket i ord:
\small
\[ \textit{penger allerede skaffet}+\textit{antall lodd}\cdot\textit{penger per lodd}=\textit{prisen på turen} \]
\normalsize
Den eneste størrelsen vi ikke vet om er \textit{antall lodd}. Vi erstatter derfor \textit{antall lodd} med $ x $, og setter inn verdien til de andre:
\[ 2\,000+x\cdot25 = 11\,000 \]

\textbf{b)} \\ \vspace{-20pt}
\prbxl{0.7}{\alg{
		25x &= 11\,000-2\,000\\
		25 x &= 9\,000\\
		\frac{\cancel{25} x}{\cancel{25}} &= \frac{9\,000}{25} \\
		x &= 360
}}
\prbxr{0.25}{$ {x\cdot25 }$ er skrevet om  til $ 25x $.}
}
\eks[2]{
''Broren min er dobbelt så gammel som meg. Til sammen er vi 9 år gamle. Hvor gammel er jeg?''.

\sv
''Broren min er dobbelt så gammel som meg.'' betyr at:
\[ \textit{brors alder}=2\cdot\textit{min alder} \]
''Til sammen er vi 9 år gamle.'' betyr at:
\[ \textit{brors alder}+\textit{min alder}=\textit{9 år} \]
Erstatter vi \textit{brors alder} med ''$2\cdot\textit{min alder} $'', får vi:
\[ 2\cdot\textit{min alder}+\textit{min alder}=\text{9 år} \]
Det som er ukjent for oss er \textit{min alder}:
\alg{
2x+x &= 9 \\
3x &= 9\\
\frac{\cancel{3}x}{\cancel{3}}&= \frac{9}{3} \\
x &= 3
}
''Jeg'' er altså 3 år gammel.
}

\eks[2]{En vennegjeng ønsker å spleise på en bil som koster 50\,000 kr, men det er usikkert hvor mange personer som skal være med på å spleise.\os 
	\textbf{a)} Kall ''antall personer som blir med på å spleise'' for $ P $ og ''utgift per person i kroner'' for $ U $  og lag en formel for $ U $.\os
	
	\textbf{b)} Finn utgiften per person hvis 20 personer blir med.
	
	\sv
	\textbf{a)} Siden prisen på bilen skal deles på antall personer som er med i spleiselaget, må formelen bli:
	\[ U = \frac{50\,000}{P} \]
	
	\textbf{b)} Vi erstatter $ P $ med 20, og får:
	\alg{
		U &= \frac{50\,000}{20}\\
		&= 2\,500
	}
	Utgiften per person er altså 2\,500 kr.
}
\section{Funksjoners egenskaper}
\subsection{Funksjoner med samme verdi; skjæringspunkt}

\reg[Skjæringspunktene til grafer]{
	Et punkt hvor to funksjoner har samme verdi kalles et \textit{skjæringspunkt} til funksjonene.
}
\eks[1]{
Gitt de to funksjonene
\alg{
	f(x) &= 2x+1 \vn 
	g(x) &= x+4
}
Finn skjæringspunktet til $ f(x) $ og $ g(x) $? 

\sv

Vi kan finne skjæringspunktet både ved en \textsl{grafisk} og en \textsl{algebraisk} metode. \vsk

\textit{Grafisk metode} \os

Vi tegner grafene\footnote{For hvordan tegne en graf, se \mb, s. ?? og ??. Hvor langt $ x $-aksen bør strekke seg vet man ikke på forhånd, men kan avgjøres ved å sette inn enkle $ x $-verdier i funksjonene.} til funksjonene inn i det samme koordinatsystemet:
\fig{lig1}
Vi leser av at funksjonene har samme verdi når $ {x=3} $, og da har begge funksjonene verdien 7. Altså er skjæringspunktet $ (3, 7) $. \vsk

\textit{Algebraisk metode} \os
At $ f(x) $ og $ g(x) $ har samme verdi gir likningen
\alg{
f(x)&=g(x) \\
2x+1 &=x+4 \\
x&=3
}
Videre har vi at
\alg{
f(3)&=2\cdot3+1=7 \vn 
g(3)&=3+4=7
}
Altså er $ (3, 7) $ skjæringspunktet til grafene.\vsk

\mer Det hadde selvsagt holdt å bare finne én av $ f(3) $ og $ g(3) $.
}

\eks[2]{
En klasse planlegger en tur som krever bussreise. De får tilbud fra to busselskap:
\begin{itemize}
	\item \textbf{Busselskap 1} \\
	Klassen betaler 10\,000\,kr uansett, og 10\enh{kr} per km.
	\item \textbf{Busselskap 2} \\
	Klassen betaler 4\,000\,kr uansett, og 30\enh{kr} per km.
\end{itemize}
For hvilken lengde kjørt tilbyr busselskapene same pris?

\sv

Vi innfører følgende variabler:
\begin{itemize}
	\item $ x=\text{antall kilometer kjørt} $ \\
	\item $ f(x)=\text{pris for Busselskap 1} $ \\
	\item $ g(x)=\text{pris for Busselskap 2} $
\end{itemize}
Da er
\alg{
	f(x)&=10x+10\,000\vn
	g(x)&=30x+4\,000
}
Videre løser vi nå oppgaven både med en grafisk og en algebraisk metode. \vsk

\textit{Grafisk metode}\os
Vi tegner grafene til funksjonene inn i samme koordinatsystem:
\fig{lig2}
Vi leser av at funksjonene har samme verdi når $ {x=200} $. Dette betyr at busselskapene tilbyr samme pris hvis klassen skal kjøre 200\enh{km}.\vsk

\textit{Algebraisk metode} \os
Busselskapene har samme pris når
\alg{
f(x)&=g(x) \\
10x+10\,000&=30x+6\,000 \\
4\,000&=20x \\
x&=200
}
Busselskapene tilbyr altså samme pris hvis klassen skal kjøre 200\enh{km}.
}

\subsection{Null-, bunn- og toppunkt}
\reg[Null-, bunn- og toppunkt]{\begin{itemize}
	\item \textbf{Nullpunkt} \\
	En $ x $-verdi som gir funksjonsverdi 0.
	\item \textbf{Bunnpunkt} \\
	Punkt hvor funksjonen har sin laveste verdi.
	\item \textbf{Toppunkt} \\
	Punkt hvor funksjonen har sin høyeste verdi.
\end{itemize}
}
\eks[1]{
Funksjonen 
\[ f(x)=x^2-6x+8 \qquad,\qquad x\in[0, 10]\]
har
\begin{itemize}
	\item Nullpunkt $ {x=1} $ og $ {x=5} $.
	\item Bunnpunkt $ (3, -4) $.
	\item Toppunkt $ (7, 12) $.
\end{itemize}
\fig{lig3}
}
\info{Hvorfor er nullpunkt en verdi?}{
Det kan kanskje virke litt rart at vi kaller $ x $-verdier for nullpunkt, punkt har jo både en $ x $-verdi og en $ y $-verdi. Men når det er snakk om nullpunkt, er det underforstått at $ {y=0} $, og da er det tilstrekkelig å få vite $ x $-verdien for å avgjøre hvilket punkt det er snakk om.  
}
\section{Likningssett}
Vi har så langt sett på likninger med ett ukjent tall, men ofte er det to eller flere tall som er ukjente. Som regel er det slik at 
\begin{itemize}
	\item er det to ukjente, trengs to likninger for å finne løsninger som er konstanter.
	\item er det tre ukjente, trengs tre likninger for å finne løsninger som er konstanter.
\end{itemize}
Og slik fortsetter det. Likningene som gir oss den nødvendige informasjonen om de ukjente, kalles et \textit{likningssett}. I denne boka skal vi konsentrere oss om likninger med to ukjente.
\subsection{Innsettingsmetoden}
\reg[Innsettingsmetoden]{
Et likningssett bestående av to ukjente, $ x $ og $ y $, kan løses ved å 
\begin{enumerate}
	\item bruke den éne likningen til å finne et uttrykk for $ x $.
	\item sette uttrykket fra punkt 1 inn i den andre likningen, og løse denne med hensyn på $ y $.
	\item sette løsningen for $ y $ inn i uttrykket for $ x $.
\end{enumerate}
{\footnotesize \mer I punktene over kan (selvsagt) $ x $ og $ y $ bytte roller.}
}
\newpage
\eks[1]{
Løs likningssettet, og sett prøve på svaret.
\alg{
	x-y&=5 \tag{I} \label{eks4a}\vn
	x+y&=9 \tag{II} \label{eks4b}
}
\sv
Av \eqref{eks4a} har vi at \vn
\alg{
x-y &= 5 \\
x&=5+y
}
Vi setter dette uttrykket for $ x $ inn i \eqref{eks4b}:
\alg{
5+y+y &=9 \\
2y&=9-5 \\
2y&=4 \\
y&=2
}
Vi setter løsningen for $ y $ inn i uttrykket for $ x $:
\alg{
x&=5+y \\
&=5+2 \\
&=7
}
Altså er $ x=7 $ og $ y=2 $. \vsk

Vi setter prøve på svaret:
\alg{
x-y&=7-2=5 \vn
x+y &=7+2=9
} 
}
\newpage
\eks[1]{
Løs likningssettet \vs
\alg{
	7x-5y&=-8 \tag{I} \label{eks1a}\vn
	5x-2y&=4x-5 \tag{II} \label{eks1b}
}
\sv

Ved innsettingsmetoden kan man ofte spare seg for en del utregning ved å velge likningen og den ukjente som gir det fineste uttrykket innledningsvis. Vi observerer at \eqref{eks1b} gir et fint uttrykk for $ x $:
\alg{
7x-5y&=-6 \\
x&=2y-5
}
Vi setter dette uttrykket for $ x $ inn i \eqref{eks1a}:
\alg{
	7x-5y&=-8\\
7(2y-5)-5y&=-8 \\
14y-35-5y&=-8 \\
9y &=27 \\
y&=3
}
Vi setter løsningen for $ y $ inn i uttrykket for $ x $:
\alg{
x&=2y-5 \\
&=2\cdot 3-5\\
&=1
}
Altså er $ x=1 $ og $ y=3 $.
}
\newpage
\eks[2]{
Løs likningssettet \vs
\alg{
3x-4y&=-2 \tag{I} \label{eks2a}\\
9y-5x&=6x+y \tag{II} \label{eks2b}
}
\sv

Vi velger her å bruke \eqref{eks1a} til å finne et uttrykk for $ y $:
\alg{
3x-4y&=-2 \\
3x+2&=4y \\
\frac{3x+2}{4}&=y
}
Vi setter dette uttrykket for $ y $ inn i \eqref{eks2b}:
\alg{
9y-5x&=6x+y \\	
9\cdot\frac{3x+2}{4}-5x&=6x+\frac{3x+2}{4} \\
9(3x+2)-20x&=24x+3x+2 \\
27x+18-20x &=24x+3x+2 \\
-20x&=-16 \\
x&=\frac{4}{5}
}
Vi setter løsningen for $ x $ inn i uttrykket for $ y $:
\alg{
y&=\frac{3x+2}{4} \br
&=\frac{3\cdot\frac{4}{5}+2}{4} \br
&=\frac{\frac{22}{5}}{4} \\
&=\frac{11}{10}
}
Altså er $ x=\frac{4}{5} $ og $ y=\frac{11}{10} $.
}
\subsection{Grafisk metode}
\reg[Grafisk løsning av likningssett]{
Et likningssett bestående av to ukjente, $ x $ og $ y $, kan løses ved å 
	\begin{enumerate}
		\item omskrive de to likningene til uttrykk for to linjer.
		\item finne skjæringspunktet mellom linjene.
\end{enumerate}
}
\eks[1]{
Løs likningsettet\vs
\alg{
	x-y&=5 \tag{I} \label{eks3a}\vn
	x+y&=9 \tag{II} \label{eks3b}
}\vs 
\sv
Av \eqref{eks3a} har vi at \vs
\alg{
x-y&=5 \\
y&=x-5
}
Av \eqref{eks3b} har vi at \vs
\alg{
x+y&=9 \\
y &= 9-x 
}
Vi tegner disse to linjene inn i et koordinatsystem:
\fig{ligset1}
Altså er $ x=7 $ og $ y=2 $.
}
\end{document}


