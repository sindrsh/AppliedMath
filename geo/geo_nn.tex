\input{../doc}
\usepackage[T1]{fontenc}
\usepackage[utf8]{luainputenc}
\usepackage{lmodern} % load a font with all the characters
\usepackage{geometry}
\geometry{verbose,paperwidth=16.1 cm, paperheight=24 cm, inner=2.3cm, outer=1.8 cm, bmargin=2cm, tmargin=1.8cm}
\setlength{\parindent}{0bp}
\usepackage{import}
\usepackage[subpreambles=false]{standalone}
\usepackage{amsmath}
\usepackage{amssymb}
\usepackage{esint}
\usepackage{babel}
\usepackage{tabu}
\usepackage[dvipsnames, table]{xcolor}
\makeatother
\makeatletter


%referances
\newcommand{\net}[2]{{\color{blue}\href{#1}{#2}}}

%Spaces
\newcommand{\vsk}{\\[12pt]}
\newcommand{\vs}{\vspace{-12pt}}

% Tabell for opplegg

\newcommand{\ovlist}[1]{
\vspace{-16pt}
\begin{itemize}
	#1
\end{itemize}
}

\newcommand{\lst}[5]{
\rule{\linewidth}{1pt}
\footnotesize
	\textbf{Øvingsområde}\\ #1 
	
	\textbf{Utstyr}\\ #2  \\
	
	\begin{tabular}{@{} p{4cm} l} 
		\textbf{Tid} & \textbf{Elevinndeling} \\
		#3  & #4
	\end{tabular} 

\rule{\linewidth}{1pt}	\vsk
\normalsize
	\textbf{Gjennomføring}\\ #5 \vsk
}
%

\newcounter{opl}
%\numberwithin{opl}{article}

\newcommand{\opl}[1]{
\newpage
{\refstepcounter{opl} %\phantomsection 
\large \textbf{\theopl \;#1} \vsk}
}

% Headlines
\newcommand{\fork}{\textbf{Forkunnskapar}\\}
\newcommand{\forb}{\textbf{Forberedelsar}\\}
\newcommand{\opgvr}{\textbf{Oppgaver}}

\usepackage{datetime2}
\usepackage[]{hyperref}

\begin{document}
\section{Symmetri}

\begin{figure}
	\centering
	\includegraphics[scale=0.19]{\fpath{sym}}\;
	\includegraphics[scale=0.16]{\fpath{symb}}\;
	\includegraphics[scale=0.16]{\fpath{symc}}
	\caption*{Bilder henta fra \net{https://freesvg.org}{freesvg.org}.}
\end{figure}
Mange figurar kan delast inn i minst to delar der den éine delen berre er ei forskyvd, speilvend eller rotert utgåve av den andre. Dette kallast \textit{symmetri}\index{symmetri}. Dei tre komande regelboksane definerer dei tre variantene for symmetri, men merk dette: Symmetri blir som regel intuitivt forstått ved å studere figurar, men er omstendeleg å skildre med ord. Her vil det derfor, for mange, vere ein fordel å hoppe rett til eksempla. \vsk

\reg[Translasjonssymmetri (parallellforskyvning)]{
Ein symmetri der minst to deler er forskyvde utgåver av kvarandre kallast en \textit{translasjonssymmetri}. \vsk

Når ei form forskyvast, blir kvart punkt på forma flytta langs den samme \vs \text{vektoren}\footnote{Ein vektor er eit linjestykke med retning.}.
}
\eks[1]{
Figuren under viser ein translasjonssymmetri som består av to sommerfuglar.
\begin{figure}
	\centering
	\subfloat{\includegraphics[scale=0.2]{\fpath{btfly0}}}\quad
	\subfloat{\includegraphics[scale=0.2]{\fpath{btfly0}}}
	\caption*{Bilde henta fra \net{https://freesvg.org}{freesvg.org}.}
\end{figure}
}
\newpage
\eks[2]{
Under visast $ \triangle ABC $ og ein blå vektor.
\fig{trans1a}
Under visast $ \triangle ABC $ forskyvd med den blå vektoren. 
\fig{trans1}
}
\reg[Speiling]{Ein symmetri der minst to delar er vende utgåver av kvarandre kallast ein \textit{speilingssymmetri} og har minst éin \textit{symmetrilinje} (\textit{symmetriakse}).\vsk

Når eit punkt speilast, blir det forskjyvd vinkelrett på symmetrilinja, fram til det nye og det opprinnelege punktet har samme avstand til symmetrilinja.
} 
\newpage
\eks[1]{
Sommerfuglen er ein speilsymmetri, med den raude linja som symmetrilinje.
\begin{figure}
	\centering
	\includegraphics[scale=0.3]{\fpath{btfly}}
\end{figure}
}
\eks[2]{
	Den raude linja og den blå linja er begge symmetrilinjer til det grøne rektangelet.
	\fig{sym2}
}
\eks[3]{
Under visast ei form laga av punkta $ A, B, C, D, E $ og $ F $, og denne forma speila om den blå linja.
\fig{sym3}
}
\reg[Rotasjonssymmetri]{
Ein symmetri der minst to delar er ei rotert utgåve av kvarandre kallast ein \textit{rotasjonssymmetri} og har alltid eit tilhørande \textit{rotasjonspunkt} og ein \textit{rotasjonsvinkel}. \vsk

Når eit punkt roterast vil det nye og det opprinnelege punktet
\begin{itemize}
	\item ligge langs den same sirkelbogen, som har sentrum i rotasjonspunktet. 
	\item med rotasjonspunktet som toppunkt danne rotasjonsvinkelen.
\end{itemize} 
Viss rotasjonsvinkelen er eit positivt tal, vil det nye punktet forflyttast langs sirkelbogen \textsl{mot} klokka. Hvis rotasjonsvinkelen er eit negativt tall, vil det nye punktet forflyttast langs sirkelbogen \textsl{med} klokka.
}
\eks[1]{
Mønsteret under er rotasjonssymmetrisk. Rotasjonssenteret er i midten av figuren og rotasjonsvinkelen er $ 120^\circ $
\begin{figure}
	\centering
	\includegraphics[scale=0.2]{\fpath{rot0}}
	\caption*{Bilde henta fra \net{https://freesvg.org}{freesvg.org}.}
\end{figure}
}
\newpage
\eks[2]{
Figuren under viser $ \triangle ABC $ rotert $ 80^\circ $ om rotasjonspunktet $ P $.
\fig{rot1}
Da er
\[ PA='PA \quad,\quad PB=PB'\quad,\quad PC=PC' \]
og
\[ \angle APA'=\angle BPB'=\angle CPC'=80^\circ \]
} \vsk

\spr{
Ei form som er ei forskyvd, speilvend eller rotert utgåve av ei anna form, kallast ei \textit{kongruensavbilding}.
}
\section{Tredimensjonal geometri}
I \mb \;har vi sett på todimensjonale figurar som trekantar, firkantar, sirklar o.l. Alle todimensjonale figurar kan teiknast inn i et koordinatsystem med to akser.
\fig{3da}
For å teikne \textit{tredimensjonale} figurar trengs derimot tre aksar:
\fig{3db}
Mens eit rektangel seiast å ha ei breidde og ei høgde, kan vi seie at boksen over har ei bredde, ei høgde \textsl{og} ei lengde (dybde). \vsk

Området som ''ligg utanpå'' ein tredimensjonal figur kallar vi \textit{overflata}. Overflata til boksen over består av 6 rektangel. Mangekantar som er delar av ei overflate kallast \textit{sideflater}.

\reg[Tredimensjonale figurer]{
\parbox[l][][l]{0.4\linewidth}{
	\centering
	\fig{fprism}	
}
\parbox[r][][l]{0.6\linewidth}{
	\textbf{Firkanta prisme}\\
	Har to like og fire like rektangel som sideflater. Alle sideflatene som er i kontakt, står vinkelrette på kvarandre.
}
\parbox[l][][l]{0.4\linewidth}{
	\centering
	\fig{kube}	
}
\parbox[r][][l]{0.6\linewidth}{
	\textbf{Kube}\\
	Firkanta prisme med kvadrat som sideflater.
}
\parbox[l][][l]{0.4\linewidth}{
	\centering
	\fig{tprism}	
}
\parbox[r][][l]{0.6\linewidth}{
	\textbf{Trekanta prisme}\\
	To av sideflatene er like trekanter som er parallelle. Har tre sideflater som er trekantar.
}
\parbox[l][][l]{0.4\linewidth}{
	\centering
	\fig{fpyr}	
}
\parbox[r][][l]{0.6\linewidth}{
	\textbf{Firkanta pyramide}\\
	Har ett rektangel og fire trekanter\\som sideflater.
}
\parbox[l][][l]{0.4\linewidth}{
	\centering
	\fig{tpyr}	
}
\parbox[r][][l]{0.6\linewidth}{
	\textbf{Trekanta pyramide}\\
	Har fire trekanter som sideflater.
}
\parbox[l][][l]{0.4\linewidth}{
	\centering
	\fig{kjegle}	
}
\parbox[r][][l]{0.6\linewidth}{
	\textbf{Kjegle}\\
	Ein del av overflata er ein sirkel, den resterende delen er ein samanbretta sektor.
}
}
\newpage
\info{Tips}{Det er ikkje så lett å se for seg hva en \textsl{sammenbretta sektor} er, men prøv dette: 
\begin{enumerate}
	\item Teikn ein sektor på eit ark. Klipp ut sektoren, og føy saman dei to kantene på sektoren. Da har du ei kjegle utan bunn.
\end{enumerate}
}
\section{Volum}
Når vi ønsker å seie noko om kor mykje det er plass til inni ein gjenstand, snakkar vi om \textit{volumet} til den. Som eit mål på volum tenker vi oss ei kube med sidelengde 1.
\fig{vol1}
Ei slik kube kan vi kalle 'einarkuba'. Sei vi har ei firkanta prisme med breidde 3, lengde 4 og høgde 2.
\fig{vol2}
I denne er det plass til akkurat 24 einarkuber.
\fig{vol2a}
Dette kunne vi ha rekna slik:
\[ 3\cdot 4\cdot 2=24\]
Altså
\[\text{breidde}\cdot\text{lengde}\cdot\text{høgde} \]\vsk
\newpage
\subsubsection{Grunnflate}
For å rekne ut volumet til dei mest elementære figurane vi har, kan det være lurt å bruke omgrepet \textit{grunnflate}. Slik som for ei grunnlinje\footnote{sjå \mb, s. 81.}, er det vårt valg av grunnflate som avgjer korleis vi skal rekne ut høgda. For prisma fra førre side, er det naturleg å velge flata som ligg horisontalt til å vere grunnflata, og for å indikere dette skriv ein ofte bokstaven $ G $:
\fig{vol2b}
Grunnflata har arealet $ 3\cdot4=12 $, mens høgda er 2. Volumet til heile prisma er grunnflata sitt areal gonga med høgda:
\alg{
	 V &= 3\cdot 4 \cdot 2 \\
	 &= G\cdot 2\\
	 &= 24
}
\info{Grunnflata eller grunnflatearealet?}{
I teksten over har vi først kalla sjølve grunnflata for $ G $, og så brukt $ G $ for \textsl{grunnflatearealet}. I denne boka er omgrepet \textsl{grunnflate} så sterkt knytt til \textsl{grunnflatearealet} at vi ikkje skiller mellom desse to.
}
\reg[Volum \label{volforml}]{ 
Volumet $ V $ til ei firkanta prisme eller ein sylinder med grunnflate $ G $ og høgde $ h $ er
\[ V = G\cdot h \]
\begin{figure}
	\centering
	\footnotesize
	\stackunder[6pt]{\includegraphics[scale=0.7]{\fpath{vol3c}}}{Firkanta prisme}\qquad\qquad
	\stackunder[6pt]{\includegraphics[scale=0.7]{\fpath{vol3b}}}{Sylinder}
\end{figure}
Volumet $ V $ til ei kjegle eller ei pyramide med grunnflate $ G $ og høgde $ h $ er
\[ V = \frac{G\cdot h}{3} \]
\begin{figure}
	\centering
\footnotesize
\stackunder[6pt]{\includegraphics[scale=0.7]{\fpath{vol3}}}{Kjegle}\qquad \qquad    
\stackunder[6pt]{\includegraphics[scale=0.7]{\fpath{vol3a}}}{Firkanta pyramide}
\end{figure}
}\vsk
\info{Merk}{
Formlane frå \rref{volforml} gjeld også for prismer, sylindrar, kjegler og pyramider som heller (er skeive). Vis grunnflata er plassert horisontalt, er høgda den vertikale avstanden mellom grunnflata og toppen til figuren.
\fig{vol4} 
(For spisse gjenstandar som kjegler og pyramider finst det sjølvsagt bare eitt valg av grunnflate.)
}
\newpage
\eks[1]{
\fig{vol5}
Ein sylinder har radius 7 og høgde 5.
\abc{
\item Finn grunnflata til sylinderen.
\item Finn volumet til sylinderen.
}

\sv \vs
\abc{
\item Vi har at\footnote{se \mb, s. 140.}:
\algv{
	\text{grunnflate}&= \pi \cdot 7^2  \\
	&= 49 \pi
} 
\item Dermed er
\algv{
	\text{volumet til sylinderen}&= 49\pi\cdot 6 \\
	&= 294\pi
}
}
}
\newpage
\eks[2]{
Ei firkanta pyramide har lengde 2, bredde 3 og høgde 5.
\fig{vol6}
\abc{
\item Finn grunnflata til pyramiden.
\item Finn volumet til pyramiden.
}
\sv  \vs
\abc{
\item Vi har at\footnote{se \mb, s. 140.}
\algv{
\text{grunnflate}&=2\cdot 3 \\
&= 6 
}
\item Dermed er
\algv{
\text{volumet til pyramiden}&= 6\cdot 5 \\
&= 30
}
}
} \vsk

\reg[Volumet til ei kule]{
	Volumet $ V $ til ei kule med radius $ r $ er:
	\[ V = \frac{4\cdot\pi\cdot r^3}{3} \]
\begin{figure}
	\centering
	\includegraphics[scale=0.7]{\fpath{vol3d}}
\end{figure}
}
\section{Omkrets, areal og volum med enheter}
Når vi måler lengder med linjal eller liknande, må vi passe på å ta med nemningane i svaret vårt. \regv

\eks[1]{ \vs
\begin{figure}
	\centering
	\includegraphics[scale=0.04]{\fpath{2t5}}
\end{figure}
\alg{
	\text{omkretsen til rektangelet} &= 5\enh{cm}+2\enh{cm}+5\enh{cm}+2\enh{cm} \\
	&= 14\enh{cm}
} \vs

\prbxl{0.65}{\alg{
		\text{arealet til rektangelet}&=2\enh{cm}\cdot5\enh{cm} \\
		&= 2\cdot 5\enh{cm}^2\\
		&= 10\enh{cm}^2
}}
\prbxr{0.3}{Vi skriv 'cm$ ^2 $' fordi vi har gonga sammen 2 lengder som vi har målt i 'cm'.}
}
\eks[2]{
Ein sylinder har radius $ 4\enh{m} $ og høgde $ 2\enh{m} $. Finn volumet til sylinderen.

\sv
Så lenge vi er sikre på at størrelsane vår har same nemning (i dette tilfellet 'm'), kan vi først rekne uten størrelser:
\alg{
\text{grunnflate til sylinderen}&=\pi\cdot 4^2 \\
&= 16 \pi
}
\alg{
\text{volumet til sylinderen}&= 16\pi \cdot2 \\
&= 32\pi
}
Vi har her ganget sammen tre lengder (to faktorer lik 4\enh{m} og én faktor lik $ 2\enh{m} $) med meter som enhet, altså er volumet til sylinderen $ 32\pi\enh{m}^3 $.
} \vsk

\info{Merk}{
Når vi finn volumet til gjenstandar, måler vi gjerne lengder som høgde, breidde, radius og liknande. Desse lengdene har eininga 'meter'. Men i det daglege oppgir vi gjerne volum med eininga 'liter'. Da er det verd å ha med seg at
\[ 1\enh{L} = 1\enh{dm}^3 \]
}
\end{document}


