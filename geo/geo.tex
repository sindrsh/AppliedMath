\input{../doc}
\usepackage[T1]{fontenc}
\usepackage[utf8]{luainputenc}
\usepackage{lmodern} % load a font with all the characters
\usepackage{geometry}
\geometry{verbose,paperwidth=16.1 cm, paperheight=24 cm, inner=2.3cm, outer=1.8 cm, bmargin=2cm, tmargin=1.8cm}
\setlength{\parindent}{0bp}
\usepackage{import}
\usepackage[subpreambles=false]{standalone}
\usepackage{amsmath}
\usepackage{amssymb}
\usepackage{esint}
\usepackage{babel}
\usepackage{tabu}
\usepackage[dvipsnames, table]{xcolor}
\makeatother
\makeatletter


%referances
\newcommand{\net}[2]{{\color{blue}\href{#1}{#2}}}

%Spaces
\newcommand{\vsk}{\\[12pt]}
\newcommand{\vs}{\vspace{-12pt}}

% Tabell for opplegg

\newcommand{\ovlist}[1]{
\vspace{-16pt}
\begin{itemize}
	#1
\end{itemize}
}

\newcommand{\lst}[5]{
\rule{\linewidth}{1pt}
\footnotesize
	\textbf{Øvingsområde}\\ #1 
	
	\textbf{Utstyr}\\ #2  \\
	
	\begin{tabular}{@{} p{4cm} l} 
		\textbf{Tid} & \textbf{Elevinndeling} \\
		#3  & #4
	\end{tabular} 

\rule{\linewidth}{1pt}	\vsk
\normalsize
	\textbf{Gjennomføring}\\ #5 \vsk
}
%

\newcounter{opl}
%\numberwithin{opl}{article}

\newcommand{\opl}[1]{
\newpage
{\refstepcounter{opl} %\phantomsection 
\large \textbf{\theopl \;#1} \vsk}
}

% Headlines
\newcommand{\fork}{\textbf{Forkunnskapar}\\}
\newcommand{\forb}{\textbf{Forberedelsar}\\}
\newcommand{\opgvr}{\textbf{Oppgaver}}

\usepackage{datetime2}
\usepackage[]{hyperref}

\begin{document}
\section{Symmetri}

\begin{figure}
	\centering
	\includegraphics[scale=0.19]{\fpath{sym}}\;
	\includegraphics[scale=0.16]{\fpath{symb}}\;
	\includegraphics[scale=0.16]{\fpath{symc}}
	\caption*{Bilder hentet fra \net{https://freesvg.org}{freesvg.org}.}
\end{figure}
Mange figurer kan deles inn i minst to deler hvor den éne delen bare er en forskjøvet, speilvendt eller rotert utgave av den andre. Dette kalles \textit{symmetri}\index{symmetri}. De tre kommende regelboksene definerer de tre variantene for symmetri, men merk dette: Symmetri blir som regel intuitivt forstått ved å studere figurer, men er omstendelig å beskrive med ord. Her vil det derfor, for mange, være en fordel å hoppe rett til eksemplene. \vsk

\reg[Translasjonssymmetri (parallellforskyvning)]{
En symmetri hvor minst to deler er forskjøvne utgaver av hverandre kalles en \textit{translasjonssymmetri}. \vsk

Når en form forskyves, blir hvert punkt på formen flyttet langs den samme \vs \text{vektoren}\footnote{En vektor er et linjestykke med retning.}.
}
\eks[1]{
Figuren under viser en translasjonssymmetri som består av to sommerfugler.
\begin{figure}
	\centering
	\subfloat{\includegraphics[scale=0.2]{\fpath{btfly0}}}\quad
	\subfloat{\includegraphics[scale=0.2]{\fpath{btfly0}}}
	\caption*{Bilde hentet fra \net{https://freesvg.org}{freesvg.org}.}
\end{figure}
}
\newpage
\eks[2]{
Under vises $ \triangle ABC $ og en blå vektor.
\fig{trans1a}
Under vises $ \triangle ABC $ forskjøvet med den blå vektoren. 
\fig{trans1}
}
\reg[Speiling]{En symmetri hvor minst to deler er vendte utgaver av hverandre kalles en \textit{speilingssymmetri} og har minst én \textit{symmetrilinje} (\textit{symmetriakse}).\vsk

Når et punkt speiles, blir det forskjøvet vinkelrett på symmetrilinja, fram til det nye og det opprinnelige punktet har samme avstand til symmetrilinja.
} 
\newpage
\eks[1]{
Sommerfuglen er en speilsymmetri, med den røde linja som symmetrilinje.
\begin{figure}
	\centering
	\includegraphics[scale=0.3]{\fpath{btfly}}
\end{figure}
}
\eks[2]{
	Den røde linja og den blå linja er begge symmetrilinjer til det grønne rektangelet.
	\fig{sym2}
}
\eks[3]{
Under vises en form laget av punktene $ A, B, C, D, E $ og $ F $, og denne formen speilet om den blå linja.
\fig{sym3}
}
\reg[Rotasjonssymmetri]{
En symmetri hvor minst to deler er en rotert utgave av hverandre kalles en \textit{rotasjonssymmetri} og har alltid et tilhørende \textit{rotasjonspunkt} og en \textit{rotasjonsvinkel}. \vsk

Når et punkt roteres vil det nye og det opprinnelige punktet
\begin{itemize}
	\item ligge langs den samme sirkelbuen med sentrum i rotasjonspunktet. 
	\item med rotasjonspunktet som toppunkt danne rotasjonsvinkelen.
\end{itemize} 
Hvis rotasjonsvinkelen er et positivt tall, vil det nye punktet forflyttes langs sirkelbuen \textsl{mot} klokka. Hvis rotasjonsvinkelen er et negativt tall, vil det nye punktet forflyttes langs sirkelbuen \textsl{med} klokka.
}
\eks[1]{
Mønsteret under er rotasjonssymmetrisk. Rotasjonssenteret er i midten av figuren og rotasjonsvinkelen er $ 120^\circ $
\begin{figure}
	\centering
	\includegraphics[scale=0.2]{\fpath{rot0}}
	\caption*{Bilder hentet fra \net{https://freesvg.org}{freesvg.org}.}
\end{figure}
}
\newpage
\eks[2]{
Figuren under viser $ \triangle ABC $ rotert $ 80^\circ $ om rotasjonspunktet $ P $.
\fig{rot1}
Da er
\[ PA='PA \quad,\quad PB=PB'\quad,\quad PC=PC' \]
og
\[ \angle APA'=\angle BPB'=\angle CPC'=80^\circ \]
} \vsk

\spr{
En form som er en forskjøvet, speilvendt eller rotert utgave av en annen form, kalles en \textit{kongruensavbilding}.
}
\section{Tredimensjonal geometri}
I \mb har vi sett på todimensjonale figurer som trekanter, firkanter, sirkler o.l. Alle todimensjonale figurer kan tegnes inn i et koordinatsystem med to akser.
\fig{3da}
For å tegne inn \textit{tredimensjonale} figurer trengs derimot tre akser:
\fig{3db}
Mens et rektangel sies å ha en bredde og en høgde, kan vi si at boksen over har en bredde, en høgde \textsl{og} en lengde (dybde). \vsk

Området som ''ligger utenpå'' en tredimensjonal figur kaller vi \textit{overflaten}. Overflaten til boksen over består av 6 rektangler. Rektangler eller trekanter som er deler av en overflate kalles \textit{sideflater}.

\reg[Tredimensjonale figurer]{
\parbox[l][][l]{0.4\linewidth}{
	\centering
	\fig{fprism}	
}
\parbox[r][][l]{0.6\linewidth}{
	\textbf{Firkantet prisme}\\
	Har to like og fire like rektangler som sideflater. Alle sideflatene som er i kontakt, står vinkelrette på hverandre.
}
\parbox[l][][l]{0.4\linewidth}{
	\centering
	\fig{kube}	
}
\parbox[r][][l]{0.6\linewidth}{
	\textbf{Kube}\\
	Firkantet prisme med kvadrater som sideflater.
}
\parbox[l][][l]{0.4\linewidth}{
	\centering
	\fig{tprism}	
}
\parbox[r][][l]{0.6\linewidth}{
	\textbf{Trekantet prisme}\\
	To av sideflatene er like trekanter som er parallelle. Har tre sideflater som er trekanter.
}
\parbox[l][][l]{0.4\linewidth}{
	\centering
	\fig{fpyr}	
}
\parbox[r][][l]{0.6\linewidth}{
	\textbf{Firkantet pyramide}\\
	Har ett rektangel og fire trekanter\\som sideflater.
}
\parbox[l][][l]{0.4\linewidth}{
	\centering
	\fig{tpyr}	
}
\parbox[r][][l]{0.6\linewidth}{
	\textbf{Trekantet pyramide}\\
	Har fire trekanter som sideflater.
}
\parbox[l][][l]{0.4\linewidth}{
	\centering
	\fig{kjegle}	
}
\parbox[r][][l]{0.6\linewidth}{
	\textbf{Kjegle}\\
	En del av overflaten er en sirkel, den resterende delen er en sammenbrettet sektor.
}
}
\newpage
\info{Tips}{Det er ikke så lett å se for seg hva en sammenbrettet sektor er, men prøv dette: 
\begin{enumerate}
	\item Tegn en sektor på et ark. Klipp ut sektoren, og føy sammen de to kantene på sektoren. Da har du en kjegle uten bunn (sirkel).
\end{enumerate}
}
\section{Volum}
Når vi ønsker å si noe om hvor mye det er plass til inni en gjenstand, snakker vi om \textit{volumet} av den. Som et mål på volum tenker vi oss en kube med sidelengde 1.
\fig{vol1}
En slik kube kan vi kalle 'enerkuben'. Si vi har en firkantet prisme med bredde 3, lengde 4 og høgde 2.
\fig{vol2}
I denne er det plass til akkurat 24 enerkuber.
\fig{vol2a}
Dette kunne vi ha regnet ut slik:
\[ 3\cdot 4\cdot 2=24\]
Altså
\[\text{bredde}\cdot\text{lengde}\cdot\text{høgde} \]\vsk
\newpage
\subsubsection{Grunnflate}
For å regne ut volumet av de mest elementære figurene vi har, kan det være lurt å bruke begrepet \textit{grunnflate}. Slik som for en grunnlinje (se \mb, s. ??), er det vårt valg av grunnflate som bestemer hvordan vi skal regne ut høgden. For prismen fra forrige side, er det naturlig å velge flaten som ligger horisontalt til å være grunnflaten, og for å indikere dette brukes ofte $ G $:
\fig{vol2b}
Grunnflaten har arealet $ 3\cdot4=12 $, mens høgden er 2. Volumet av hele prismen er grunnflatens areal ganget med høgden:
\alg{
	 V &= 3\cdot 4 \cdot 2 \\
	 &= G\cdot 2\\
	 &= 24
}
\info{Grunnflaten eller grunnflatearealet?}{
I teksten over har vi først kalt selve grunnflaten for $ G $, og deretter brukt $ G $ for \textsl{grunnflatearealet}. I denne boka er begrepet grunnflate så sterkt knyttet til grunnflatearealet at vi ikke kommer til å skille mellom disse to begrepene.
}
\reg[Volum \label{volforml}]{ 
Volumet $ V $ til en firkantet prisme eller en sylinder med grunnflate $ G $ og høgde $ h $ er
\[ V = G\cdot h \]
\begin{figure}
	\centering
	\footnotesize
	\stackunder[6pt]{\includegraphics[scale=0.7]{\fpath{vol3c}}}{Firkantet prisme}\qquad\qquad
	\stackunder[6pt]{\includegraphics[scale=0.7]{\fpath{vol3b}}}{Sylinder}
\end{figure}
Volumet $ V $ til en kjegle eller en pyramide med grunnflate $ G $ og høgde $ h $ er
\[ V = \frac{G\cdot h}{3} \]
\begin{figure}
	\centering
\footnotesize
\stackunder[6pt]{\includegraphics[scale=0.7]{\fpath{vol3}}}{Kjegle}\qquad \qquad    
\stackunder[6pt]{\includegraphics[scale=0.7]{\fpath{vol3a}}}{Firkantet pyramide}
\end{figure}
}\vsk
\info{Merk}{
Formlene fra \rref{volforml} gjelder også for prismer, sylindre, kjegler og pyramider som heller (er skjeve). Hvis grunnflaten er plassert horisontalt, er høgden den vertikale avstanden mellom grunnflaten og toppen til figuren.
\fig{vol4} 
(For spisse gjenstander som kjegler og pyramider finnes det selvsagt bare ett valg av grunnflate.)
}
\newpage
\eks[1]{
\fig{vol5}
En sylinder har radius 7 og høgde 5. Finn volumet til sylinderen.
\abc{
\item Finn grunnflaten til sylinderen.
\item Finn volumet til sylinderen.
}

\sv \vs
\abc{
\item Vi har at (se regel ?? i \mb):
\alg{
	\text{grunnflate}&= \pi \cdot 7^2  \\
	&= 49 \pi
} 
\item Dermed er
\algv{
	\text{volumet til sylinderen}&= 49\pi\cdot 6 \\
	&= 294\pi
}
}
}
\newpage
\eks[2]{
En firkantet pyramide har lengde 2, bredde 3 og høgde 5.
\fig{vol6}
\abc{
\item Finn grunnflaten til pyramiden.
\item Finn volumet til pyramiden.
}
\sv  \vs
\abc{
\item Vi har at (se regel ?? i \mb)
\alg{
\text{grunnflate}&=2\cdot 3 \\
&= 6 
}
\item Dermed er
\alg{
\text{volumet til pyramiden}&= 6\cdot 5 \\
&= 30
}
}
} \vsk

\reg[Volumet til ei kule]{
	Volumet $ V $ til ei kule med radius $ r $ er:
	\[ V = \frac{4\cdot\pi\cdot r^3}{3} \]
\begin{figure}
	\centering
	\includegraphics[scale=0.7]{\fpath{vol3d}}
\end{figure}
}
\section{Omkrets, areal og volum med enheter}
Når vi måler lengder med linjal eller lignende, må vi passe på å ta med enhetene i svaret vårt. \regv

\eks[1]{ \vs
\begin{figure}
	\centering
	\includegraphics[scale=0.04]{\fpath{2t5}}
\end{figure}
\alg{
	\text{Omkretsen til rektangelet} &= 5\enh{cm}+2\enh{cm}+5\enh{cm}+2\enh{cm} \\
	&= 14\enh{cm}
} \vs

\prbxl{0.65}{\alg{
		\text{Arealet til rektangelet}&=2\enh{cm}\cdot5\enh{cm} \\
		&= 2\cdot 5\enh{cm}^2\\
		&= 10\enh{cm}^2
}}
\prbxr{0.3}{Vi skriver cm$ ^2 $ fordi vi har ganget sammen 2 lengder som vi har målt i cm.}
}
\eks[2]{
En sylinder har radius $ 4\enh{m} $ og høgde $ 2\enh{m} $. Finn volumet til sylinderen.

\sv
Så lenge vi er sikre på at størrelsene vår har samme enhet (i dette tilfellet meter), kan vi først rekne uten størrelser:
\alg{
\text{grunnflate til sylinderen}&=\pi\cdot 4^2 \\
&= 16 \pi
}
\alg{
\text{volumet til sylinderen}&= 16\pi \cdot2 \\
&= 32\pi
}
Vi har her ganget sammen tre lengder (to faktorer lik 4\enh{m} og én faktor lik $ 2\enh{m} $) med meter som enhet, altså er volumet til sylinderen $ 32\pi\enh{m}^3 $.
} \vsk

\info{Merk}{
Når vi skal finne volumet til gjenstander, måler vi lengder som høgde, bredde, radius osv., men i det daglige oppgir vi gjerne volum i liter. Da er det verdt å ha med seg at
\[ 1\enh{L} = 1\enh{dm}^3 \]
}
\end{document}


