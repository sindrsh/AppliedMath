\input{/home/sindre/P/doc}
\usepackage[T1]{fontenc}
\usepackage[utf8]{luainputenc}
\usepackage{lmodern} % load a font with all the characters
\usepackage{geometry}
\geometry{verbose,paperwidth=16.1 cm, paperheight=24 cm, inner=2.3cm, outer=1.8 cm, bmargin=2cm, tmargin=1.8cm}
\setlength{\parindent}{0bp}
\usepackage{import}
\usepackage[subpreambles=false]{standalone}
\usepackage{amsmath}
\usepackage{amssymb}
\usepackage{esint}
\usepackage{babel}
\usepackage{tabu}
\usepackage[dvipsnames, table]{xcolor}
\makeatother
\makeatletter


%referances
\newcommand{\net}[2]{{\color{blue}\href{#1}{#2}}}

%Spaces
\newcommand{\vsk}{\\[12pt]}
\newcommand{\vs}{\vspace{-12pt}}

% Tabell for opplegg

\newcommand{\ovlist}[1]{
\vspace{-16pt}
\begin{itemize}
	#1
\end{itemize}
}

\newcommand{\lst}[5]{
\rule{\linewidth}{1pt}
\footnotesize
	\textbf{Øvingsområde}\\ #1 
	
	\textbf{Utstyr}\\ #2  \\
	
	\begin{tabular}{@{} p{4cm} l} 
		\textbf{Tid} & \textbf{Elevinndeling} \\
		#3  & #4
	\end{tabular} 

\rule{\linewidth}{1pt}	\vsk
\normalsize
	\textbf{Gjennomføring}\\ #5 \vsk
}
%

\newcounter{opl}
%\numberwithin{opl}{article}

\newcommand{\opl}[1]{
\newpage
{\refstepcounter{opl} %\phantomsection 
\large \textbf{\theopl \;#1} \vsk}
}

% Headlines
\newcommand{\fork}{\textbf{Forkunnskapar}\\}
\newcommand{\forb}{\textbf{Forberedelsar}\\}
\newcommand{\opgvr}{\textbf{Oppgaver}}

\usepackage{datetime2}
\usepackage[]{hyperref}

\begin{document}
\tableofcontents
\newpage
Mål for opplæringen er at eleven skal kunne
\begin{itemize}
\item bruke og begrunne bruken av formlikhet, målestokk og Pytagoras’ setning til beregninger og i praktisk arbeid
\item løse problemer som gjelder lengde, vinkel, areal og volum
\item regne med ulike måleenheter, bruke ulike måleredskaper, vurdere hvilke måleredskaper som er hensiktsmessige, og vurdere måleusikkerheten
\item tolke, lage og bruke skisser og arbeidstegninger på problemstillinger fra kultur- og yrkesliv og presentere og begrunne løsninger
\end{itemize}
\newpage
\section{Repetisjon av begreper om trekanter}
Vi navngir punkt på trekanter med store bokstaver fra alfabetet (\textit{A, B, C,} osv). En trekant som går mellom punktene $ A $, $ B $ og $ C $ skriver vi som $ \triangle ABC $. Vi kaller siden mellom $ A $ og $ B $ for $ AB $, siden mellom $ B $ og $ C $ for $ BC $ og siden mellom $ A $ og $ C $ for $ AC $.
Videre kaller vi gjerne vinkelen tilhørende \textit{A} for $ \angle A $, vinkelen tilhørende \textit{B} for $ \angle B $ og vinkelen tilhørende $ C $ for $ \angle C $. 
\fig{geo1}
Men noen ganger har vi flere enn tre punkt i trekanten vi ønsker å studere, slik som i denne figuren:
\fig{geo2}
For å være helt tydelig på hvilken vinkel vi mener, bruker vi da tre bokstaver:
\begin{itemize}
	\item $ \angle DCB $ betyr vinkelen mellom siden  $ DC $ og $ BC $.
	\item $ \angle ACD $ betyr vinkelen mellom siden  $ DC $ og $ AC $.
\end{itemize}\vsk 
\textbf{Vinkelverdier}\os
Vinkler måler vi i \textit{grader}. Hvis vi har at $ \angle A=30^\circ $ sier vi at \textit{vinkelverdien} til $ \angle A $ er $ 30^\circ $.
Og én ting må vi virkelig huske angående vinklene i en trekant, nemlig dette:
\reg[Summen av vinklene i en trekant\label{180}]{I alle trekanter er summen av vinklelverdiene $ 180^\circ $. For en trekant med vinklene $ \angle A $, $ \angle B $ og $ \angle C $ kan vi altså skrive:
	\[ \angle A+ \angle B +\angle C = 180^\circ \]
\begin{figure}
	\centering
	\includegraphics[scale=0.8]{\asym{geo1}}
\end{figure}	
	
}\vsk
\section{Formlike trekanter}
\reg[Formlike trekanter I]{Hvis tre vinkelrverdier finnes i både $ \triangle ABC $ og $ \triangle CDE $, så er trekantene \textit{formlike}.} \regv

En fantastisk sak med \hr{180} er at hvis vi vet verdien til to vinkler i en trekant, så vet vi verdien til den siste vinkelen også! Dette betyr at hvis vi kan finne to vinkelverdier i to forskjellige trekanter, må den ''siste'' vinkelen også finnes i begge trekantene!\regv

\reg[Formlike trekanter II]{Trekanter er formlike hvis de har to vinkler som er like.
\begin{figure}
	\centering
	\includegraphics[scale=0.8]{\asym{geo4}}\qquad
	\includegraphics[scale=0.8]{\asym{geo3}}
\end{figure}
$ \triangle ABC $ er formlik med $ \triangle DEF $ hvis:
\begin{itemize}
\item $ \angle A = \angle D$ eller $ \angle A = \angle E $
\item $ \angle B $ er lik den av $ \angle D $ og $ \angle E$ som $ \angle A $ eventuelt \textit{ikke} er lik.
\end{itemize}
}
\section{Samsvar og forhold i formlike trekanter}
\textbf{Samsvarende sider}\os
Det vil ofte være slik at formlike trekanter ikke er rotert likens. For eksempel er trekantene $ \triangle ABC $ og $ \triangle DEF $ i figuren under formlike, men rotert forskjellig:
\begin{figure}
	\centering
	\includegraphics[]{\asym{tri1}}\qquad
		\includegraphics[]{\asym{tri2}}
\end{figure}
Noen ganger er det viktig å holde styr på hvilken side som ''hører til'' hvilken vinkel. Dette ser vi lettest ved å tegne en pil fra vinkelhjørnet og over til motsatt side:
\begin{figure}
	\centering
	\phantom{}\;\;\includegraphics[]{\asym{tri1a}}\qquad
	\includegraphics[]{\asym{tri2a}}
\end{figure}
For de to trekantene observerer vi nå dette:
\begin{multicols}{2}
\quad Trekant $ \triangle ABC $
\begin{itemize}
	\item $ BC $ hører til $ \angle A$.
	\item $ AC $ hører til $ \angle B$.
	\item $ AB $ hører til $ \angle C$.	
\end{itemize}
\quad Trekant $ \triangle DEF $

\begin{itemize}
	\item $ DE $ hører til $ \angle A$.
	\item $ EF $ hører til $ \angle B$.
	\item $ DF $ hører til $ \angle C$.		
\end{itemize}

\end{multicols}

\begin{comment}
\begin{itemize}
\item I trekanten $ \triangle ABC $ er $ \angle A $ vinkelen mellom sidene $ AC $ og $ AB $. Da sier vi at $ BC $ er siden som hører til $ \angle A $. 	
\end{itemize}
Om vi holder oss til $ \triangle ABC $ kan vi fortsette slik:
\begin{itemize}
\item  $ \angle B $ er vinkelen mellom sidene $ BC $ og $ BA $. Derfor er $ AC $ siden som hører til $ \angle B $. 
\item  $ \angle C $ er vinkelen mellom sidene $ CA $ og $ CB $. Derfor er $ AB $ siden som hører til $ \angle B $. 	
\end{itemize}
\end{comment}
Sider som hører til de samme vinklene kaller vi \textit{samsvarende} sider:
\begin{itemize}
	\item $ BC $ og $ DE $ er samsvarende sider fordi begge hører til $ \angle A $.
	\item $ AC $ og $ EF $ er samsvarende sider fordi begge hører til $ \angle B $.
	\item $ AB $ og $ DE $ er samsvarende sider fordi begge hører til $ \angle C $.	
\end{itemize}
\reg[Samsvarende sider]{Sidene som hører til de samme vinklene i formlike trekanter, kaller vi samsvarende sider.}\vsk

\textbf{Forhold i formlike trekanter}\os
En viktig grunn til at vi er så opptatt av samsvarende sider, er at vi vet noe om forholdet mellom dem:\regv
\reg[Forhold i formlike trekanter\label{forform}]{
I to trekanter som er formlike, er forholdet mellom samsvarende sider det samme:
\[ \frac{AB}{DE}=\frac{AC}{DF}=\frac{BC}{EF} \]
\begin{figure}
	\centering
	\includegraphics[scale=0.8]{\asym{geo5}}\qquad
	\includegraphics[scale=0.8]{\asym{geo6}}
\end{figure}
}
\eks[]{
Trekantene i figuren under er formlike. Finn lengden til $EF $.
\begin{figure}
	\centering
	\includegraphics[scale=1]{\asym{tri3a}}\quad
	\includegraphics[scale=1]{\asym{tri3b}}
\end{figure}
\sv
Vi observerer at $ AB $ samsvarer med $ DE $, $ BC $ med $ EF $ og $ AC $ med $ DF $. Det betyr at:
\alg{
\frac{DE}{AB} &= \frac{EF}{BC} \br
\frac{7,5}{5}&= \frac{EF}{3} \br
\frac{7,5}{5}\cdot3&=\frac{EF}{\cancel{3}}\cdot\cancel{3}\\
4,5 &= EF
}
}
\section{Omkrets}
\textbf{Mangekanter}\os \vs
\prbxl{0.6}{
Hvor langt det er rundt en figur, kaller vi \textit{omkretsen} til figuren. For figurer som består bare av rette kanter, legger vi sammen lengden av sidene for å finne omkretsen.
}\qquad
\prbxr{0.3}{Samlebetegnelsen for trekanter, firkanter, femkanter osv. er \textit{mangekanter}.}\vs
\begin{figure}
	\centering
\subfloat[]{\includegraphics[]{\asym{tri23b}}}\qquad
\subfloat[]{\includegraphics[]{\asym{tri23a}}}
\end{figure}
Dette betyr at:
\begin{itemize}
	\item I figur (a) er omkretsen $ {2+1+3+2=9} $
	\item I figur (b) er omkretsen $ {5+2+4=11} $	
\end{itemize}
\reg[Omkrets av mangekanter]{Omkretsen av en mangekant finner vi ved å legge sammen lengden av alle sidene.}\vsk
\textbf{Sirkler}\\
For sirkler er det å finne omkretsen litt verre, for her har vi ingen sider vi kan legge sammen! Da må vi isteden ty til en formel:\regv
\reg[Omkretsen av en sirkel]{Omkretsen $ O $ av en sirkel med radius $ r $ er:
	\[ O = 2\pi r \]
	\fig{tri22}
}
\eks[1]{
Finn omkretsen til en sirkel med radius 3.

\sv \vs \vs \vs
\alg{ 
O &= 2\pi r \\
&= 2 \pi \cdot 3 \\
&= 6\pi 
}
Omkretsen er $6\pi  $.
}
\eks[2]{
En sirkel har omkretsen $ 22\pi $. Hvor lang er radiusen til sirkelen?

\sv
Vi starter med å gjøre om formelen for omkretsen til en formel for radiusen:\vs
\alg{
O &= 2\pi r \\
\frac{O}{2\pi}&=\frac{\cancel{2\pi}}{\cancel{2\pi}} \\
\frac{O}{2\pi}&=r
}
Så setter vi omkretsen inn i den nye formelen vår:
\alg{
r &= \frac{22\pi}{2\pi} \\
&= 11
}
Radiusen til sirkelen er altså 11.
}
\section{Areal}
Overalt rundt oss kan vi se det vi kaller \textit{overflater}. Gulvet vi går på, pulten vi sitter ved eller arket vi skriver på har alle overflater det er lett å legge merke til. Når vi ønsker å si noe om hvor stor en overflate er, bruker vi begrepet \textit{areal}. Idéen bak begrepet areal er denne:\vsk

Vi tenker oss et kvadrat med bredde 1 og høyde 1, som vi kan kalle for ''enerkvadradet''.
\fig{tri10}
Så ser vi på overflaten vi ønsker å finne arealet av og spør oss:\os

\textsl{Hvor mange enerkvadrat er det plass til i denne overflaten?}\os

Si nå at vi har et rektangel med 5 som bredde og 4 som høyde:
\fig{tri11a}
Hvis vi inni rektangelet streker opp linjer som har 1 i avstand bortover og 1 i avstand oppover, observerer vi at vi får plass til 20 enerkvadrat i rektangelet vårt:
\fig{tri11}
Dette betyr at arealet av rektangelet er 20, noe vi kunne regnet ut slik:
\[ 5\cdot 4=20 \]
\reg[Arealet av rektangler \label{arfir}]{
Arealet $ A $ av et rektangel med bredde $ b $ og høyde $ h $ er:
\[A= b\cdot h \]
\fig{tri12}
}\vsk

\textbf{Høyden i en trekant}\os
Når vi vet hvordan å finne arealet av en firkant, er ikke veien lang til å finne arealet av en trekant. Men først må vi forstå hva vi mener med \textit{høyden, grunnlinjen} og \textit{toppunktet} i en trekant.
\fig{tri14a}
Vi kan alltids rotere en trekant slik at den éne siden ligger rett horisontalt, trekanten $ \triangle ABC $ over kan vi rotere slik:\vs
\begin{figure}
\centering
\subfloat[]{\includegraphics[]{\asym{tri14b}}}
\quad
\subfloat[]{\includegraphics[]{\asym{tri14c}}}
\quad
\subfloat[]{\includegraphics[]{\asym{tri14d}}}
\end{figure}
Da er dette gjeldende:
\begin{itemize}
	\item I figur (a) er $ AB $ grunnlinjen og $ C $ toppunktet.
	\item I figur (b) er $ CA $ grunnlinjen og $ B $ toppunktet.
	\item I figur (c) er $ BC $ grunnlinjen og $ A $ toppunktet.
\end{itemize}
Med høyden i en trekant mener vi den vertikale avstanden mellom toppunktet og grunnlinjen:
\begin{itemize}
	\item I figur (a) er $ AC $ høyden.
	\item I figur (b) er $ CA $ høyden.
	\item I figur (c) kan vi tegne inn høyden $ h $ som i figuren under:
\end{itemize}
\fig{tri14e}
\begin{itemize}
\item I trekanter hvor toppunktet ligger bortenfor grunnlinjen kan vi tegne inn høyden $ h $ slik:
\end{itemize}
\fig{tri15}
\reg[Høyden i en trekant]{
	Høyden $ h  $ i en trekant er den vertikale avstanden mellom grunnlinja $ g $ og toppunktet.
	\begin{figure}
		\centering
		\includegraphics[scale=0.8]{\asym{tri16}}
		\quad
		\includegraphics[scale=0.8]{\asym{tri16a}}
		\quad
		\includegraphics[scale=0.8]{\asym{tri16b}}
	\end{figure}
}\vsk
\textbf{Arealet av en trekant}\os
La oss starte med en rettvinklet trekant som er plassert slik at $ 90^\circ $-graderen ''ligger'' på grunnlinja:
\fig{tri16}
Av slike trekanter kan vi alltid tegne et rektangel med bredde $ g $ og høyde $ h $:
\fig{tri17}
Av \hr{arfir} vet vi at arealet av rektangelet er $ {g\cdot h} $. Trekanten som består at de stiplede linjene og diagonalen er identisk med vår opprinnelige trekant. Dette betyr at trekanten med $ g $ som grunnlinje og $ h $ som høyde har et areal som er halvparten så stort som arealet av rektangelet. Arealet $ A $ til trekanten blir derfor:
\[ A = \frac{g\cdot h}{2} \]
Når vi vet arealet av rette trekanter, kan vi også finne formelen for arealet av alle andre trekanter. Vi skal her komme fram til formlene ved å bruke tall istedenfor symboler.\vsk

Trekanten under har grunnlinje med lengde 3 og høyde 2. 
\fig{tri18}
Også ''rundt'' denne trekanten tegner vi et rektangel:
\fig{tri18a}
Vi gir nå arealene følgende navn:
\alg{
\text{Areal av rektangelet}&= R\\
\text{Areal av oransje trekant} &= O\\
\text{Areal av grønn trekant}& = G\\
\text{Areal av blå trekant} &= B\\
}
Og da har vi at:
\alg{
 R&= 7\cdot2=14\\
O&= \frac{7\cdot2}{2}=7\\ 
G &= \frac{4\cdot 2}{2}=4 \\
}
Det blå arealet er det samme som arealet av rektangelet minus arealet av det oransje og grønne området:
\alg{
B &= R-O-G \\
&= 14-7-4 \\&=3
}
\fig{tri20}
\fig{tri20a}
\reg[Arealet av en trekant]{
Arealet $ A $ av en trekant med grunnlinje $ g $ og høyde $ h $ er:
\[ A = \frac{g\cdot h}{2} \]
	\begin{figure}
		\centering
		\includegraphics[scale=0.8]{\asym{tri16}}
		\quad
		\includegraphics[scale=0.8]{\asym{tri16a}}
		\quad
		\includegraphics[scale=0.8]{\asym{tri16b}}
	\end{figure}
}
\eks[1]{
\textbf{a)} Skriv om arealformelen for en trekant til en formel for høyden $ h $.\os 
\textbf{b)} En trekant med grunnlinje 5 har arealet 40. Hvor lang er grunnlinjen til trekanten?

\sv
\textbf{a)} \vsb
\alg{
2\cdot A &= \bcancel{2}\frac{g\cdot h}{\bcancel{2}} \br
\frac{2A}{g\cdot h} &= \frac{\bcancel{g}}{\bcancel{g}} \br
\frac{2A}{g} &= h
}
\textbf{b)} Vi setter høyden og arealet inn i formelen vi fant i a):
\alg{
h &= \frac{2\cdot 40}{5} \\
&= 2\cdot 8 \\
&= 16
}
}
\vsk
\begin{comment}
	\begin{figure}
	\centering
	\includegraphics[]{\asym{tri8}}\qquad 	
	\includegraphics[]{\asym{tri9}}
	\end{figure}
\end{comment}
\textbf{Arealet av et parallellogramet}\os
Et parallellogram er en firkant der de motstående sidene er parallelle og like lange.
\fig{tri21}
Vi kan alltid lage oss to like trekanter ved å tegne inn én av diagonalene.
\fig{tri21c}
Trekantene på bildet over har grunnlinje $ b $ og høyde $ h $. Da vet vi at:
\alg{\text{Arealet til én av trekantene}=\frac{b\cdot h}{2}}
Og da må begge trekantene til sammen ha arealet:
 \[ \frac{b\cdot h}{2}+\frac{b\cdot h}{2} =b\cdot h \]
\reg[Arealet av et parallellogram]{
Arealet $ A $ av et parallellogram med bredde $ b $ og høyde $ h $ er:
\[ A = b\cdot h  \]
\fig{tri21}
}\vsk
\textbf{Trapeset}\os
I et trapes er minst to sider parallelle, men alle sidene kan gjerne ha forskjellig lengde:
\fig{tri21a}
Men også for et trapes får vi to trekanter hvis vi tegner én av diagonalene:
\fig{tri21d}
På figuren over har vi at:
\alg{
	\text{Arealet av blå trekant} &= \frac{a\cdot h}{2} \br
	\text{Arealet av grønn trekant} &= \frac{b\cdot h}{2}
}
\reg[Arealet av et trapes]{
Arealet $ A $ av et trapes med øvre bredde $ a $ og nedre bredde $ b $ og høyde $ h $ er:
\[ A = \frac{(a+b)\cdot h}{2}\]
\fig{tri21a}
}


\section{Arealet av en sirkel}
I figuren nedenfor har vi delt opp en sirkel i 4, 10 og 50 (like store) biter, og lagt disse bitene etter hverandre.
\fig{tri19}
\fig{tri19a}
\fig{tri19b}
I hvert tilfelle må de små buelengdene til sammen utgjøre hele buelengden til sirkelen. Hvis sirkelen har radius $ r $ betyr dette at buelengdene til sammen må bli $ 2\pi r $. Og når vi har like mange biter med buen vendt opp som biter med buen vendt ned, må lengden være $ \pi r $ både oppe og nede. \vsk

Men jo flere biter vi deler sirkelen inn i, jo mer begynner de sammenlagte bitene å ligne et rektangel (i figuren under har vi 100 biter). Da blir det mer og mer riktig å si at høyden \textit{nesten} er lik $ r $. Og fordi lengden til de øverste buene er $ \pi r $, må bredden til rektangelet være \textit{nesten} det samme.
\fig{tri19c}
 Arealet  $ A $ til dette rektangelet blir da (ca.):
\[ A\approx b\cdot h = \pi r\cdot r = \pi r^2 \]
Jo flere små biter vi deler sirkelen vår inn i, jo bedre blir tilnærmingen, og sånn kan vi si at dette er et uttrykk for arealet av sirkelen.\regv
\reg[Arealet av en sirkel]{
Arealet $ A $ av en sirkel med radius $ r $ er:
\[ A = \pi r^2 \]
\fig{tri22a}
}
\section{Areal med enheter}
Når vi snakker om arealer i virkeligheten må vi alltid ta med enheten til arealet (det gir ingen mening å si at arealet til en fotballbane er ca. 80\,000). Hvis du ser tilbake på arealformlene, vil du se at alle innebærer at vi ganger sammen to lengder. Og akkurat som for vanlige tall kan vi også skrive to like enheter ganget sammen som en potens:
\alg{
\text{m}\cdot \text{m} &= \text{m}^2 \\
\text{dm}\cdot \text{dm} &= \text{dm}^2
}
Husk at vi alltid kan velge selv hvilken enhet vi ønsker å måle noe i. La oss for eksempel tenke oss at vi har et rom som er 2\,m bredt og 3\,m langt (vi bruker her ordet langt istedenfor høyde, fordi vi tenker oss at vi ser gulvet ovenfra):
\fig{tri24}
Arealet av dette rommet er da:
\[ 3\enh{m}\cdot 2\enh{m}=6\enh{m}^2 \]
Men vi kunne selvsagt oppgitt lengden til rommet i desimeter istedenfor:
\fig{tri24a}
Og da blir arealet:
\[ 30\enh{dm}\cdot 20\enh{dm}=600\enh{dm}^2 \]
Vi vet at 10\,dm = 1\,m. Når vi går fra en enhet (fra m til dm) som er 10 ganger \textsl{mindre}, blir altså tallverdien til arealet 100 ganger \textsl{større} (fra 6 til 600)! Vi kan altså følge samme metode som i \hr{ompref}, bare at vi nå må flytte komma \textsl{to} plasser hver gang vi skifter rute (bruk gjerne ''\,$ ^2 $\,'' som en påminner om dette).
\reg[Enheter for areal]{
Når vi skal endre enheter for areal kan vi bruke denne tabellen:
\begin{center}
	\begin{tabular}{|c|c|c|c|c|c|c|c}
		km$ ^2 $ &
		hm$ ^2 $ &
		dam$ ^2 $ & 
		m$ ^2 $ &
		dm$ ^2 $ & 
		cm$ ^2 $ & 
		mm$ ^2 $& 		
	\end{tabular}
\end{center}
Komma må flyttes to ganger for hver rute vi flytter oss.
}
\eks{
\textbf{a)} Gjør om 0,2\,m$ ^2 $ til et areal målt i dm$ ^2 $.\os
\textbf{b)} Gjør om 45\,000\,dm$ ^2 $ til et areal målt i km$ ^2 $.

\sv
\textbf{a)} For å komme oss fra m$ ^2 $ til dm$ ^2 $ må vi flytte oss én rute til høyre i tabellen. Det betyr at vi må flytte komma $ {1\cdot2=2} $ plasser til høyre:
\[ 0,2\enh{m}^2=0,02\enh{cm}^2 \]
\textbf{b)} For å komme oss fra dm$ ^2 $ til km$ ^2 $ må vi flytte oss fire ruter til venstre i tabellen. Det betyr at vi må flytte komma $ {4\cdot2=8} $ plasser til venstre:
\[ 45\,000\enh{dm}^2=0,00045\enh{km}^2 \]
}
\section{Pytagoras' setning}
Ved hjelp av arealformlelen for et kvadrat og en trekant skal vi nå komme fram til én av de mest kjente ligningene i matematikk. På figuren under har vi tegnet to kvadrater som er like store, men som er delt inn i forskjellige former.
\fig{tri26d}
Vi observerer nå dette:
\begin{itemize}
	\item Arealet av et rødt kvadrat er $ a^2 $, arealet av et lilla kvadrat er $ b^2 $ og arealet av det blå kvadratet er $ c^2 $.
	\item Arealet av en grønn trekant er halvparten av arealet til et oransje rektangel.
	\item Om vi tar bort de to oransje rektanglene og de fire grønne trekantene, sitter vi igjen med like mye areal i venstre figur som i høyre figur.
	\fig{tri26e}
	\item Dette må bety at:
	\[ a^2+b^2=c^2 \]
\end{itemize}
\prbxl{0.5}{Denne ligningen kalles \textit{Pytagoras'} setning, og oftest bruker vi den når vi skal finne lengder i rettvinklete trekanter. Dette er fordi vi alltid kan lage figurer som de over, så lenge trekanten vår er rettvinklet: }\qquad
\prbxr{0.4}{Pytagoras' (ca. 580-500 f.kr.) var en gresk matematiker. Han var trolig langt ifra den første som oppdaget denne sammenhengen, og det finnes over 100 forskjellige bevis for den!}
\fig{tri26f}
\reg[Pytagoras' setning]{
Arealet av den lengste siden i en rettvinklet trekant er alltid lik summen av arealene til de to korteste sidene:\regv

\parbox[l][][l]{0.7\linewidth}{\[ a^2+b^2=c^2 \]
}
\parbox[r]{0.2\linewidth}{\includegraphics[]{\asym{tri26g}}}
}
\eks[1]{
Finn lengden av siden $ c $ i trekanten under:
\fig{tri26h}
\sv
Vi vet at:
\[ c^2=a^2+b^2 \]
hvor $ a $ og $ b $ er lengden til de korteste sidene i trekanten. Derfor få vi at:
\alg{
c^2 &= 4^2 + 3^2 \\
&= 16+9 \\
&=25
}
Fordi $ \sqrt{25}=5 $, må lengden til $ c $ være 5.
}
\eks[2]{
	Finn lengden av siden $ x $ i trekanten under:
	\fig{tri26i}
	\sv
	Vi vet at:
	\[ c^2=a^2+x^2 \]
	hvor $ c $ er lengden til den lengste siden og $ a $ lengden til den andre kortsiden. Derfor få vi at:
	\alg{
		17^2 &= 13^2 + x^2 \\
		289-225&=x^2  \\
		64&=x^2
	}
	Fordi $ \sqrt{64}=8 $, må lengden til $ x $ være 8 cm.
}
\section{Volum}
Når vi ønsker å si noe om hvor mye det er plass til inni en gjenstand, snakker vi om \textit{volumet} av den. Som et mål på volum tenker vi oss \textit{en kube} som har 1 som både bredde, lengde og høyde:
\fig{vol1}
En slik kube kan vi kalle''enhetskuben''. Si vi har en firkantet boks med bredde 3, lengde 4 og høyde 2:
\fig{vol2}
Vi kan må merke oss at vi har plass til akkurat 24 enhetskuber i denne boksen:
\fig{vol2a}
Og dette kunne vi ha regnet ut slik:
\[ 3\cdot 4\cdot 2=24\]
Altså:
\[\text{bredde}\cdot\text{lengde}\cdot\text{høyde} \]\vsk

\textbf{Grunnflate}\\
For å regne ut volumet av de mest elementære figurene vi har, kan det være lurt å bruke begrepet \textit{grunnflate}. Slik som for en grunnlinje, er det vårt valg av grunnflate som bestemer hvordan vi skal regne ut høyden. For en slik boks som vi akkurat så på, er det naturlig å velge flaten som ''ligger ned'' til å være grunnflaten, og for å indikere dette brukes ofte $ G $:
\begin{figure}
		\centering
	\includegraphics[]{\asym{vol2b}}\qquad
	\includegraphics[]{\asym{vol2c}}
\end{figure}
Grunnflaten har arealet $ 3\cdot4=12 $, mens høyden er 2. Volumet av hele boksen er grunnflaten ganger høyden:
\alg{
	 V &= G\cdot h\\
	 &= 12\cdot h\\
	 &= 24
}
\reg[Volum]{
Volumet $ V $ av en firkantet boks eller en sylinder med grunnflate $ G $ og høyde $ h $ er:
\[ V = G\cdot h \]
\begin{figure}
	\centering
	\footnotesize
	\stackunder[6pt]{\includegraphics[scale=0.7]{\asym{vol3c}}}{Boks}\qquad
	\stackunder[6pt]{\includegraphics[scale=0.7]{\asym{vol3b}}}{Sylinder}
\end{figure}
Volumet $ V $ av en kjegle eller en pyramide med grunnflate $ G $ og høyde $ h $ er:
\[ V = \frac{G\cdot h}{3} \]
\begin{figure}
	\centering
\footnotesize
\stackunder[6pt]{\includegraphics[scale=0.7]{\asym{vol3}}}{Kjegle}\qquad
\stackunder[6pt]{\includegraphics[scale=0.7]{\asym{vol3a}}}{Pyramide}
\end{figure}
}
\textbf{Volumet av ei kule} \\
Som vanlig skiller ting seg ut når vi snakker om renit sirkelformede figurer, og ei \textit{kule} er ikke noe unntak. For den spesielt interesserte kan et bevis for volumformelen leses \net{https://drive.google.com/open?id=0B9-bzK2nA0X2Ty1yNElYQzRETjQ}{her}, men det er altså helt lov til å bykse rett på formelen:\regv
\reg[Volumet av ei kule]{
	Volumet $ V $ av ei kule med radius $ r $ er:
	\[ V = \frac{4\cdot\pi\cdot r^3}{3} \]
\begin{figure}
	\centering
	\includegraphics[scale=0.7]{\asym{vol3d}}
\end{figure}
}
\end{document}


