\input{../doc}
\usepackage[T1]{fontenc}
\usepackage[utf8]{luainputenc}
\usepackage{lmodern} % load a font with all the characters
\usepackage{geometry}
\geometry{verbose,paperwidth=16.1 cm, paperheight=24 cm, inner=2.3cm, outer=1.8 cm, bmargin=2cm, tmargin=1.8cm}
\setlength{\parindent}{0bp}
\usepackage{import}
\usepackage[subpreambles=false]{standalone}
\usepackage{amsmath}
\usepackage{amssymb}
\usepackage{esint}
\usepackage{babel}
\usepackage{tabu}
\usepackage[dvipsnames, table]{xcolor}
\makeatother
\makeatletter


%referances
\newcommand{\net}[2]{{\color{blue}\href{#1}{#2}}}

%Spaces
\newcommand{\vsk}{\\[12pt]}
\newcommand{\vs}{\vspace{-12pt}}

% Tabell for opplegg

\newcommand{\ovlist}[1]{
\vspace{-16pt}
\begin{itemize}
	#1
\end{itemize}
}

\newcommand{\lst}[5]{
\rule{\linewidth}{1pt}
\footnotesize
	\textbf{Øvingsområde}\\ #1 
	
	\textbf{Utstyr}\\ #2  \\
	
	\begin{tabular}{@{} p{4cm} l} 
		\textbf{Tid} & \textbf{Elevinndeling} \\
		#3  & #4
	\end{tabular} 

\rule{\linewidth}{1pt}	\vsk
\normalsize
	\textbf{Gjennomføring}\\ #5 \vsk
}
%

\newcounter{opl}
%\numberwithin{opl}{article}

\newcommand{\opl}[1]{
\newpage
{\refstepcounter{opl} %\phantomsection 
\large \textbf{\theopl \;#1} \vsk}
}

% Headlines
\newcommand{\fork}{\textbf{Forkunnskapar}\\}
\newcommand{\forb}{\textbf{Forberedelsar}\\}
\newcommand{\opgvr}{\textbf{Oppgaver}}

\usepackage{datetime2}
\usepackage[]{hyperref}

\begin{document}
\section{Symmetri}

\begin{figure}
	\centering
	\includegraphics[scale=0.19]{\fpath{sym}}\;
	\includegraphics[scale=0.16]{\fpath{symb}}\;
	\includegraphics[scale=0.16]{\fpath{symc}}
	\caption*{Bilder hentet fra \net{https://freesvg.org}{freesvg.org}.}
\end{figure}
Mange figurer kan deles inn i minst to deler hvor den éne delen bare er en forskjøvet, speilvendt eller rotert utgave av den andre. Dette kalles \textit{symmetri}\index{symmetri}. De tre kommende regelboksene definerer de tre variantene for symmetri, men merk dette: Symmetri blir som regel intuitivt forstått ved å studere figurer, men er omstendelig å beskrive med ord. Her vil det derfor, for mange, være en fordel å hoppe rett til eksemplene. \vsk

\reg[Translasjonssymmetri (parallellforskyvning)]{
En symmetri hvor minst to deler er forskjøvne utgaver av hverandre kalles en \textit{translasjonssymmetri}. \vsk

Når en form forskyves, blir hvert punkt på formen flyttet langs det samme linjestykket.
}
\eks[1]{
Figuren under viser en translasjonssymmetri som består av to sommerfugler.
\begin{figure}
	\centering
	\subfloat{\includegraphics[scale=0.2]{\fpath{btfly0}}}\quad
	\subfloat{\includegraphics[scale=0.2]{\fpath{btfly0}}}
	\caption*{Bilde hentet fra \net{https://freesvg.org}{freesvg.org}.}
\end{figure}
}
\newpage
\eks[2]{
Under vises $ \triangle ABC $ og et blått linjestykke.
\fig{trans1a}
Under vises $ \triangle ABC $ forskjøvet med det blå linjestykket. 
\fig{trans1}
}
\reg[Speiling]{En symmetri hvor minst to deler er vendte utgaver av hverandre kalles en \textit{speilingssymmetri} og har minst én \textit{symmetrilinje} (\textit{symmetriakse}).\vsk

Når en et punkt speiles, blir det forskjøvet vinkelrett på symmetrilinja fram til det nye og det opprinnelige punktet har samme avstand til symmetrilinja.
} 
\newpage
\eks[1]{
Sommerfuglen er en speilsymmetri, med den røde linja som symmetrilinje.
\begin{figure}
	\centering
	\includegraphics[scale=0.3]{\fpath{btfly}}
\end{figure}
}
\eks[2]{
	Den røde linja og den blå linja er begge symmetrilinjer til det grønne rektangelet.
	\fig{sym2}
}
\eks[3]{
Under vises en form laget av punktene $ A, B, C, D, E $ og $ F $, og denne formen speilet om den blå linja.
\fig{sym3}
}
\reg[Rotasjonssymmetri]{
En symmetri hvor minst to deler er en rotert utgave av hverandre kalles en \textit{rotasjonssymmetri} og har alltid et tilhørende \textit{rotasjonspunkt} og en \textit{rotasjonsvinkel}. \vsk

Når et punkt roteres vil det nye og det opprinnelige punktet
\begin{itemize}
	\item ligge langs den samme sirkelbuen med sentrum i rotasjonspunktet. 
	\item med rotasjonspunktet som toppunkt danne rotasjonsvinkelen.
\end{itemize} 
Hvis rotasjonsvinkelen er et positivt tall, vil det nye punktet forflyttes langs sirkelbuen \textsl{mot} klokka. Hvis rotasjonsvinkelen er et negativt tall, vil det nye punktet forflyttes langs sirkelbuen \textsl{med} klokka.
}
\eks[1]{
Mønsteret under er rotasjonssymmetrisk. Rotasjonssenteret er i midten av figuren og rotasjonsvinkelen er $ 120^\circ $
\begin{figure}
	\centering
	\includegraphics[scale=0.2]{\fpath{rot0}}
\end{figure}
}
\newpage
\eks[2]{
Figuren under viser $ \triangle ABC $ rotert $ 80^\circ $ om rotasjonspunktet $ P $.
\fig{rot1}
Da er
\[ PA='PA \quad,\quad PB=PB'\quad,\quad PC=PC' \]
og
\[ \angle APA'=\angle BPB'=\angle CPC'=80^\circ \]
} \vsk

\spr{
En form som er en forskjøvet, speilvendt eller rotert utgave av en annen form, kalles en \textit{kongruensavbilding}.
}
\section{Størrelser, enheter og prefikser}
Det vi kan måle og uttrykke med tall, kaller vi \textit{størrelser}. Videre har vi \textit{størrelser med dimensjoner} og \textit{dimensjonsløse størrelser}.\vsk

Et eksempel på en størrelse med dimensjon er ''2 meter''. Dimensjonen er da 'lengde', som vi gjerne måler i meter. Vi sier at meter er en \textit{enhet} for dimensjonen lengde.\vsk

Et eksempel på en størrelse uten dimensjon er ''to hester''. Mens det bare finnes én lengde som er ''2 meter'', ''to hester'' se veldig forskjellig ut, avhengig av hvile to hester det er snakk om.\vsk

\textbf{Regning med dimensjoner}\os
Når vi regner med størrelser med dimensjoner må vi passe på at alle enhetene er like, hvis ikke gir ikke regnestykkene våre mening. I denne boka skal vi se på disse enhetene:
\tbs
\begin{center}
	\begin{tabular}{c|c|c}
		\textbf{Enhet} & \textbf{Forkortelse} &\textbf{Dimensjon}\\ \hline
		meter & m &lengde\\\hline
		gram & g &masse\\\hline
		liter & L & volum
	\end{tabular}
\end{center}\tbs
Noen ganger har vi veldig store eller veldig små størrelser, for eksempel er det ca 40\,075\,000\,m rundt ekvator! For så store tall er det vanlig å bruke en \textit{prefiks}, da kan vi si at det er ca 40\,075 km rundt ekvator. Her står 'km' for 'kilometer' og 'kilo' betyr '1\,000'. Så 1\,000 meter er altså 1 kilometer. Her er prefiksene man oftest\footnote{Rett nok er 'deka' en veldig lite brukt prefiks, men vi har tatt den med fordi den kompletterer tallmønsteret.} møter på i hverdagen:\tbs
\begin{center}
	\begin{tabular}{c|c|c}
		\textbf{Prefiks} & \textbf{Forkortelse}&\textbf{Betydning} \\ \hline
		kilo & k & 1\,000\\\hline
		hekto & h & 100\\\hline
		deka & da & 10\\\hline
		desi & d & 0,1\\\hline
		centi & c & 0,01\\\hline
		milli & m & 0,001\\\hline		
	\end{tabular}
\end{center}
\newpage
Bruker vi denne tabellen i kombinasjon med enhetene kan vi for eksempel se at:\vs
\alg{
	1000\enh{g}&= 1\enh{kg} \\
	0,1 \enh{m} &= 1\enh{dm} \\
	0,01 \enh{L} &= 1\enh{cL}
}
Enda ryddigere kan vi få det hvis vi lager en vannrett tabell, med meter, gram eller liter lagt til i midten\footnote{Legg merke til at meter, gram og liter er \textsl{enheter med dimensjoner}, mens kilo, hekto osv. er \textsl{dimensjonsløse tall}. Det kan derfor virke litt rart å sette dem opp i samme tabell, men for dette formålet fungerer det helt fint.}: \regv

\reg[\ompref \label{ompref}]{Når vi skal endre prefikser kan vi bruke denne tabellen:
	\begin{center}
		\begin{tabular}{|c|c|c|c|c|c|c|c}
			kilo &
			hekto &
			deka & m/g/L &
			desi & 
			centi & 
			milli & 		
		\end{tabular}
	\end{center}
	Komma må flyttes like mange ganger som antall ruter vi må flytte oss fra opprinnelig prefiks til ny prefiks.\vsk
	
	{\footnotesize For lengde brukes også enheten 'mil' (1 mil $ = $ 10\,000\,m). Denne kan legges på til venstre for 'kilo'.}
}
\eks[1]{
	Skriv om 23,4\,mL til antall L.
	
	\sv
	Vi skriver tabellen vår med L i midten, og legger merke til at vi må \textsl{tre ruter til venstre} for å komme oss fra mL til L:
	\begin{center}
		\begin{tabular}{|c|c|c|c|c|c|c|c}
			kilo &
			hekto &
			deka & \color{blue}L &
			desi & 
			centi & 
			\color{red} milli & 		
		\end{tabular}
	\end{center}
	Det betyr at vi må flytte kommaet vårt tre plasser til venstre for å gjøre om mL til L:
	\[ 23,4\enh{mL}=0,0234\enh{L} \]
}
\eks[2]{
	Skriv om 30\,hg til antall cg.
	
	\sv
	Vi skriver tabellen vår med g i midten og legger merke til at vi må \textsl{fire ruter til høyre} for å komme oss fra hg til cg:
	\begin{center}
		\begin{tabular}{|c|c|c|c|c|c|c|c}
			kilo &
			\color{red}hekto &
			deka & g &
			desi & 
			\color{blue}centi & 
			milli & 		
		\end{tabular}
	\end{center}
	Dét betyr at vi må flytte kommaet vårt fire plasser til høyre for å gjøre om hg til cg:
	\[ 30\enh{mg}=300\,000\enh{cg} \]
}
\eks[3]{
	Gjør om 12\,500\,dm til antall mil.
	
	\sv
	Vi skriver tabellen vår med m i midten, legger til 'mil', og merker oss at vi må \textsl{fem ruter til høyre} for å komme oss fra hg til cg:
	\begin{center}
		\begin{tabular}{|c|c|c|c|c|c|c|c|c}
			\color{blue}mil &kilo &
			hekto &
			deka & m &
			\color{red} desi & 
			centi & 
			milli & 		
		\end{tabular}
	\end{center}
	Dét betyr at vi må flytte kommaet vårt fem plasser til høyre for å gjøre om mil til cg:
	\[ 30\enh{dm}=3\,000\,000\enh{mil} \]
}

\fork{\ompref}{
Omgjøring av prefikser handler om å gange/dele med 10, 100 osv. (se seksjon ??) \vsk

La oss som første eksempel skrive om $ 3,452\enh{km} $ til antall meter. Vi har at
\algv{
3,452\enh{km}&= 3,452\cdot1000 \enh{m} \\
&=3\,452\enh{m}
}
La oss som andre eksempel skrive om 47\enh{mm} til antall meter. Vi har at
\algv{
47\enh{mm}&=47\cdot\frac{1}{1000} \enh{m} \\
&= (47:1000) \enh{m}\\
&=0,047\enh{m}
}
}
\section{Volum}
Når vi ønsker å si noe om hvor mye det er plass til inni en gjenstand, snakker vi om \textit{volumet} av den. Som et mål på volum tenker vi oss \textit{en kube} som har 1 som både bredde, lengde og høgde:
\fig{vol1}
En slik kube kan vi kalle''enhetskuben''. Si vi har en firkantet boks med bredde 3, lengde 4 og høgde 2:
\fig{vol2}
Vi kan må merke oss at vi har plass til akkurat 24 enhetskuber i denne boksen:
\fig{vol2a}
Og dette kunne vi ha regnet ut slik:
\[ 3\cdot 4\cdot 2=24\]
Altså:
\[\text{bredde}\cdot\text{lengde}\cdot\text{høgde} \]\vsk

\textbf{Grunnflate}\\
For å regne ut volumet av de mest elementære figurene vi har, kan det være lurt å bruke begrepet \textit{grunnflate}. Slik som for en grunnlinje, er det vårt valg av grunnflate som bestemer hvordan vi skal regne ut høgden. For en slik boks som vi akkurat så på, er det naturlig å velge flaten som ''ligger ned'' til å være grunnflaten, og for å indikere dette brukes ofte $ G $:
\begin{figure}
		\centering
	\includegraphics[]{\fpath{vol2b}}\qquad
	\includegraphics[]{\fpath{vol2c}}
\end{figure}
Grunnflaten har arealet $ 3\cdot4=12 $, mens høgden er 2. Volumet av hele boksen er grunnflatens areal ganger høgden:
\alg{
	 V &= 3\cdot 4 \cdot 2 \\
	 &= G\cdot 2\\
	 &= 24
}
\info{Grunnflaten eller grunnflatearealet?}{
I teksten over har vi først kalt selve grunnflaten for $ G $, og deretter brukt $ G $ for \textsl{grunnflatearealet}. I denne boka er begrepet grunnflate så sterkt knyttet til grunnflatearealet at vi ikke kommer til å skille mellom disse to begrepene.
}
\reg[Volum \label{volforml}]{ 
Volumet $ V $ til en firkantet prisme eller en sylinder med grunnflate $ G $ og høgde $ h $ er
\[ V = G\cdot h \]
\begin{figure}
	\centering
	\footnotesize
	\stackunder[6pt]{\includegraphics[scale=0.7]{\fpath{vol3c}}}{Firkantet prisme}\qquad\qquad
	\stackunder[6pt]{\includegraphics[scale=0.7]{\fpath{vol3b}}}{Sylinder}
\end{figure}
Volumet $ V $ til en kjegle eller en pyramide med grunnflate $ G $ og høgde $ h $ er
\[ V = \frac{G\cdot h}{3} \]
\begin{figure}
	\centering
\footnotesize
\stackunder[6pt]{\includegraphics[scale=0.7]{\fpath{vol3}}}{Kjegle}\qquad \qquad    
\stackunder[6pt]{\includegraphics[scale=0.7]{\fpath{vol3a}}}{Firkantet pyramide}
\end{figure}
}
\info{Obs!}{
Formlene fra \rref{volforml} gjelder også for prismer, sylindre, kjegler og pyramider som heller (er skjeve). Hvis grunnflaten er plassert horisontalt, er høgden den vertikale avstanden mellom grunnflaten og toppen til figuren.
\fig{vol4} 
}
\newpage
\eks[1]{
\fig{vol5}
En sylinder har radius 7 og høgde 5. Finn volumet til sylinderen.
\abc{
\item Finn grunnflaten til sylinderen.
\item Finn volumet til sylinderen.
}

\sv \vs
\abc{
\item Vi har at (se regel ?? i \mb):
\alg{
	\text{grunnflate}&= \pi \cdot 7^2  \\
	&= 49 \pi
} 
\item Dermed er
\algv{
	\text{volumet til sylinderen}&= 49\pi\cdot 6 \\
	&= 294\pi
}
}
}
\newpage
\eks[2]{
En firkantet pyramide har lengde 2, bredde 3 og høgde 5.
\fig{vol6}
\abc{
\item Finn grunnflaten til pyramiden.
\item Finn volumet til pyramiden.
}
\sv  \vs
\abc{
\item Vi har at (se regel ?? i \mb)
\alg{
\text{grunnflate}&=2\cdot 3 \\
&= 6 
}
\item Dermed er
\alg{
\text{volumet til pyramiden}&= 6\cdot 5 \\
&= 30
}
}
}
\textbf{Volumet av ei kule} \\
Som vanlig skiller ting seg ut når vi snakker om renit sirkelformede figurer, og ei \textit{kule} er ikke noe unntak. For den spesielt interesserte kan et bevis for volumformelen leses \net{https://drive.google.com/open?id=0B9-bzK2nA0X2Ty1yNElYQzRETjQ}{her}, men det er altså helt lov til å bykse rett på formelen:\regv
\reg[Volumet av ei kule]{
	Volumet $ V $ av ei kule med radius $ r $ er:
	\[ V = \frac{4\cdot\pi\cdot r^3}{3} \]
\begin{figure}
	\centering
	\includegraphics[scale=0.7]{\fpath{vol3d}}
\end{figure}
}
\section{Omkrets, areal og volum med enheter}
Når vi måler lengder med linjal eller lignende, må vi passe på å ta med enhetene i svaret vårt. \regv

\eks[1]{
\begin{figure}
	\centering
	\includegraphics[scale=0.06]{\fpath{2t5}}
\end{figure}
\alg{
	\text{Omkretsen til rektangelet} &= 5\enh{cm}+2\enh{cm}+5\enh{cm}+2\enh{cm} \\
	&= 14\enh{cm}
}

\prbxl{0.65}{\alg{
		\text{Arealet til rektangelet}&=2\enh{cm}\cdot5\enh{cm} \\
		&= 2\cdot 5\enh{cm}^2\\
		&= 10\enh{cm}^2
}}
\prbxr{0.3}{Vi skriver cm$ ^2 $ fordi vi har ganget sammen 2 lengder som vi har målt i cm.}
}
\eks[2]{
En sylinder har radius $ 4\enh{m} $ og høgde $ 2\enh{m} $. Finn volumet til sylinderen.

\sv
Så lenge vi er sikre på at størrelsene vår har samme enhet (i dette tilfellet meter), kan vi først rekne uten størrelser:
\alg{
\text{grunnflate til sylinderen}&=\pi\cdot 4^2 \\
&= 16 \pi
}
\alg{
\text{volumet til sylinderen}&= 16\pi \cdot2 \\
&= 32\pi
}
Vi har her ganget sammen tre lengder (to faktorer lik 4\enh{m} og én faktor lik $ 2\enh{m} $) med meter som enhet, altså er volumet til sylinderen $ 32\pi\enh{m}^3 $
} \vsk

\info{Merk}{
Når vi skal finne volumet til gjenstander, måler vi lengder som høgde, bredde, radius osv., men i det daglige oppgir vi gjerne volum i liter. Da er det verdt å ha med seg at
\[ 1\enh{L} = 1\enh{dm}^3 \]
}
\end{document}


