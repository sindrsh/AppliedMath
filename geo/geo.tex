\input{/home/sindre/P/doc}
\usepackage[T1]{fontenc}
\usepackage[utf8]{luainputenc}
\usepackage{lmodern} % load a font with all the characters
\usepackage{geometry}
\geometry{verbose,paperwidth=16.1 cm, paperheight=24 cm, inner=2.3cm, outer=1.8 cm, bmargin=2cm, tmargin=1.8cm}
\setlength{\parindent}{0bp}
\usepackage{import}
\usepackage[subpreambles=false]{standalone}
\usepackage{amsmath}
\usepackage{amssymb}
\usepackage{esint}
\usepackage{babel}
\usepackage{tabu}
\usepackage[dvipsnames, table]{xcolor}
\makeatother
\makeatletter


%referances
\newcommand{\net}[2]{{\color{blue}\href{#1}{#2}}}

%Spaces
\newcommand{\vsk}{\\[12pt]}
\newcommand{\vs}{\vspace{-12pt}}

% Tabell for opplegg

\newcommand{\ovlist}[1]{
\vspace{-16pt}
\begin{itemize}
	#1
\end{itemize}
}

\newcommand{\lst}[5]{
\rule{\linewidth}{1pt}
\footnotesize
	\textbf{Øvingsområde}\\ #1 
	
	\textbf{Utstyr}\\ #2  \\
	
	\begin{tabular}{@{} p{4cm} l} 
		\textbf{Tid} & \textbf{Elevinndeling} \\
		#3  & #4
	\end{tabular} 

\rule{\linewidth}{1pt}	\vsk
\normalsize
	\textbf{Gjennomføring}\\ #5 \vsk
}
%

\newcounter{opl}
%\numberwithin{opl}{article}

\newcommand{\opl}[1]{
\newpage
{\refstepcounter{opl} %\phantomsection 
\large \textbf{\theopl \;#1} \vsk}
}

% Headlines
\newcommand{\fork}{\textbf{Forkunnskapar}\\}
\newcommand{\forb}{\textbf{Forberedelsar}\\}
\newcommand{\opgvr}{\textbf{Oppgaver}}

\usepackage{datetime2}
\usepackage[]{hyperref}

\begin{document}
\tableofcontents
\newpage

\section{Volum}
Når vi ønsker å si noe om hvor mye det er plass til inni en gjenstand, snakker vi om \textit{volumet} av den. Som et mål på volum tenker vi oss \textit{en kube} som har 1 som både bredde, lengde og høyde:
\fig{vol1}
En slik kube kan vi kalle''enhetskuben''. Si vi har en firkantet boks med bredde 3, lengde 4 og høyde 2:
\fig{vol2}
Vi kan må merke oss at vi har plass til akkurat 24 enhetskuber i denne boksen:
\fig{vol2a}
Og dette kunne vi ha regnet ut slik:
\[ 3\cdot 4\cdot 2=24\]
Altså:
\[\text{bredde}\cdot\text{lengde}\cdot\text{høyde} \]\vsk

\textbf{Grunnflate}\\
For å regne ut volumet av de mest elementære figurene vi har, kan det være lurt å bruke begrepet \textit{grunnflate}. Slik som for en grunnlinje, er det vårt valg av grunnflate som bestemer hvordan vi skal regne ut høyden. For en slik boks som vi akkurat så på, er det naturlig å velge flaten som ''ligger ned'' til å være grunnflaten, og for å indikere dette brukes ofte $ G $:
\begin{figure}
		\centering
	\includegraphics[]{\asym{vol2b}}\qquad
	\includegraphics[]{\asym{vol2c}}
\end{figure}
Grunnflaten har arealet $ 3\cdot4=12 $, mens høyden er 2. Volumet av hele boksen er grunnflaten ganger høyden:
\alg{
	 V &= G\cdot h\\
	 &= 12\cdot h\\
	 &= 24
}
\reg[Volum]{
Volumet $ V $ av en firkantet boks eller en sylinder med grunnflate $ G $ og høyde $ h $ er:
\[ V = G\cdot h \]
\begin{figure}
	\centering
	\footnotesize
	\stackunder[6pt]{\includegraphics[scale=0.7]{\asym{vol3c}}}{Boks}\qquad
	\stackunder[6pt]{\includegraphics[scale=0.7]{\asym{vol3b}}}{Sylinder}
\end{figure}
Volumet $ V $ av en kjegle eller en pyramide med grunnflate $ G $ og høyde $ h $ er:
\[ V = \frac{G\cdot h}{3} \]
\begin{figure}
	\centering
\footnotesize
\stackunder[6pt]{\includegraphics[scale=0.7]{\asym{vol3}}}{Kjegle}\qquad
\stackunder[6pt]{\includegraphics[scale=0.7]{\asym{vol3a}}}{Pyramide}
\end{figure}
}
\textbf{Volumet av ei kule} \\
Som vanlig skiller ting seg ut når vi snakker om renit sirkelformede figurer, og ei \textit{kule} er ikke noe unntak. For den spesielt interesserte kan et bevis for volumformelen leses \net{https://drive.google.com/open?id=0B9-bzK2nA0X2Ty1yNElYQzRETjQ}{her}, men det er altså helt lov til å bykse rett på formelen:\regv
\reg[Volumet av ei kule]{
	Volumet $ V $ av ei kule med radius $ r $ er:
	\[ V = \frac{4\cdot\pi\cdot r^3}{3} \]
\begin{figure}
	\centering
	\includegraphics[scale=0.7]{\asym{vol3d}}
\end{figure}
}
\end{document}


