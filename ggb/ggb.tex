\input{../doc}
\usepackage[T1]{fontenc}
\usepackage[utf8]{luainputenc}
\usepackage{lmodern} % load a font with all the characters
\usepackage{geometry}
\geometry{verbose,paperwidth=16.1 cm, paperheight=24 cm, inner=2.3cm, outer=1.8 cm, bmargin=2cm, tmargin=1.8cm}
\setlength{\parindent}{0bp}
\usepackage{import}
\usepackage[subpreambles=false]{standalone}
\usepackage{amsmath}
\usepackage{amssymb}
\usepackage{esint}
\usepackage{babel}
\usepackage{tabu}
\usepackage[dvipsnames, table]{xcolor}
\makeatother
\makeatletter


%referances
\newcommand{\net}[2]{{\color{blue}\href{#1}{#2}}}

%Spaces
\newcommand{\vsk}{\\[12pt]}
\newcommand{\vs}{\vspace{-12pt}}

% Tabell for opplegg

\newcommand{\ovlist}[1]{
\vspace{-16pt}
\begin{itemize}
	#1
\end{itemize}
}

\newcommand{\lst}[5]{
\rule{\linewidth}{1pt}
\footnotesize
	\textbf{Øvingsområde}\\ #1 
	
	\textbf{Utstyr}\\ #2  \\
	
	\begin{tabular}{@{} p{4cm} l} 
		\textbf{Tid} & \textbf{Elevinndeling} \\
		#3  & #4
	\end{tabular} 

\rule{\linewidth}{1pt}	\vsk
\normalsize
	\textbf{Gjennomføring}\\ #5 \vsk
}
%

\newcounter{opl}
%\numberwithin{opl}{article}

\newcommand{\opl}[1]{
\newpage
{\refstepcounter{opl} %\phantomsection 
\large \textbf{\theopl \;#1} \vsk}
}

% Headlines
\newcommand{\fork}{\textbf{Forkunnskapar}\\}
\newcommand{\forb}{\textbf{Forberedelsar}\\}
\newcommand{\opgvr}{\textbf{Oppgaver}}

\usepackage{datetime2}
\usepackage[]{hyperref}

\begin{document}
\subsection{Introduksjon}
Når du åpner GeoGebra får du et bilde som dette:
\begin{figure}[H]
	\centering
	\includegraphics[scale=0.1]{ggbalgoggraf}
\end{figure}
Feltet hvor det står ''Skriv inn'' kalles \textit{inntastingsfeltet}. Dette feltet og det blanke feltet under utgjør \textit{algebrafeltet}. Koordinatsystemet til høyre kalles \textit{grafikkfeltet}.

\subsection{Skrive inn en funksjon}
Si vi har funksjonen 
\[f(x)= \frac{3}{2} x^2 + 3x \]
For å bruke $ f(x) $ i GeoGebra, skriver vi:
\g{3/2*x\^{}2+3x}
Når vi ikke gir funksjonen noen navn, vil GeoGebra automatisk gi funksjonen navnet $ f $. I algebrafeltet får vi derfor
\begin{figure}[H]
	\centering
	\includegraphics[scale=0.5]{skrivf}
\end{figure}
I grafikkfeltet får vi grafen til $ f $. \vsk

Hvis vi isteden har funksjonen
\[ P(x)= 0,15x^3 - 0,4 x\]
er det to ting vi må passe på. Det første er at \textsl{alle desimaltall må skrives med punktum istedenfor komma} i GeoGebra
. Det andre er at vi ønsker å gi funksjonen navnet $ P(x) $. Vi skriver da
\g{P(x) = 0.15x\^{}3 - 0.4x}
og får \vspace{-5pt}
\begin{figure}[H]
	\centering
	\includegraphics[scale=0.5]{pfig}
\end{figure}
\info{Obs!}{
Man kan aldri gi funksjoner navnet $ {y(x)} $ i GeoGebra. $ y $ kan bare brukes når man skriver inn uttrykk for en rett linje, altså $ {y=ax +b} $, hvor $ a $ og $ b $ er to valgfrie tall.
}

\subsection{Vannette og loddrette linjer}

Ønser vi å lage ei linje som går vannrett gjennom verdien 3 på $ y $-aksen og ei linje som går loddrett gjennom verdien 2 på $ x $-aksen skriver vi:
\g{y = 3}
og 
\g{x = 2 }
Da får vi denne figuren:
\begin{figure}[H]
	\centering
	\includegraphics[scale=0.5]{23}
\end{figure}

\subsection{Å finne verdien til en funksjon}
Si vi har funksjonen
\[H(x)= x^2 + 3x -3 \]
Hvis vi ønsker å vite hva $ H(2) $ er, skriver vi
\g{H(2)}
som resulterer i dette
\begin{figure}[H]
	\centering
	\includegraphics[scale=0.5]{H}
\end{figure}
Da vet vi at $ H(2)=7 $.\vsk


\newpage
\subsection{Rette linjer}
Det anbefales på det sterkeste at du bruker funksjonsuttrykk når du behandler linjer i GeoGebra, men det kan være greit å vite om den andre måten man kan gjøre det på. \vsk

La oss se på de to linjene \vs
\alg{
	y&= x-3 \vn
	y&= -2x+1
}
Vi skriver disse inn i GeoGebra, og får
\begin{figure}[H]
	\centering
	\includegraphics[scale=0.5]{fglin1}
\end{figure}
Ønsker vi nå å finne hva verdien til $ {y=x-3} $ er når $ {x=2} $, må vi legge merke til at GeoGebra har kalt denne linja for $ f $. Svaret vi søker får vi da ved å skrive $ f(2) $. Ønsker vi samtidig å vite hva $ {y=-2x+1} $ er når $ {x=0} $ må vi skrive $ g(0) $:
\begin{figure}[H]
	\centering
	\includegraphics[scale=0.6]{fglin2}
\end{figure}

\st{
\textbf{Videoer}
\begin{itemize}
\item \net{}{Finne nullpunktet til en graf.}	
\end{itemize}

}
\vedlg{Finne skjæringspunkt}
Se videoen 
\vedlg{Finne nullpunkt}
Se videon
\net{https://drive.google.com/open?id=12aJaWEhLFlleFm6ajQTEAuNtam1UAjPz}{nullpkt}.
\vedlg{Finne topp- eller bunnpunkt}
Se videoen \net{https://drive.google.com/open?id=17DbHhH24a9zcdr8xPD3339sFh1dE1KQ3}{ekstrmpkt}.
\vedlg{Tegne linjen mellom to punkt}
Se videoen \net{https://drive.google.com/open?id=12RrLFBpSoHwj9CZQcmIytz27QmKOoGbA}{linpkt}.

\begin{comment}
	\vedlg{Regresjon}
	Når vi har en samling med \\
	
	\begin{tabular}{c|c}
	\textbf{Måned (1. i hver)} &| \textbf{Dagslengde (timer)}\\
	\hline
	1&	5.18 \\
	2&	7.50 \\
	3&	10.28 \\
	4&	13.42 \\
	5&	16.45 \\
	6&	19.23 \\
	7&	19.78 \\
	8&	17.45 \\
	9&	14.38 \\
	10&	11.4 \\ 
	11&	8.32 \\
	12&	5.70\\	
	\end{tabular}
\end{comment}
\vedlg{Tegne graf på gitt intervall}
I denne
\net{https://drive.google.com/open?id=1dOrz90-JgeaxTq7cklIcqxG65PdOnjn4}{videoen} her vi tegnet inn funksjonen:
\[ f(x)=0.0.0013x^3-0.59x^2+61x+2000\quad,\quad0\leq x\leq300 \]

\newpage
\vedlg{Oppgaver}
\textbf{G.1}\bs
\textbf{a)} Skriv den lineære funksjonen $ {f(x)=2x+4} $ og linja $ {y=2x+2} $ inn i GeoGebra. Lag $ f(x) $ blå og $ y $ grønn. Hva ser du ut ifra grafen til de to linjene?\bs
\textbf{b)} Finn verdien til $ f(x) $ når $ {x=4} $.\bs
\textbf{c)} Finn verdien til $ y $ når $ {x=-3} $.\vsk

\textbf{G.2}\bs
\textbf{a)} Tegn punktene $ (-1,2) $ og $ (2,8) $.\bs
\textbf{b)} Finn uttrykket til linja som går gjennom disse punktene.\vsk

\textbf{G.3}\bs
\textbf{a)} Skriv inn funksjonen $ {f(x)=x^2+2x-3} $.\bs
\textbf{b)} Finn $ f(4) $.\bs
\textbf{c)} Finn nullpunktene til $ f(x) $.\bs
\textbf{d)} Finn bunnpunktet til $ f(x) $.\bs
\textbf{e)} Finn skjæringspunktet mellom $ f(x) $ og linja $ y=5 $.



\end{document}

