\input{/home/sindre/P/doc}
\usepackage[T1]{fontenc}
\usepackage[utf8]{luainputenc}
\usepackage{lmodern} % load a font with all the characters
\usepackage{geometry}
\geometry{verbose,paperwidth=16.1 cm, paperheight=24 cm, inner=2.3cm, outer=1.8 cm, bmargin=2cm, tmargin=1.8cm}
\setlength{\parindent}{0bp}
\usepackage{import}
\usepackage[subpreambles=false]{standalone}
\usepackage{amsmath}
\usepackage{amssymb}
\usepackage{esint}
\usepackage{babel}
\usepackage{tabu}
\usepackage[dvipsnames, table]{xcolor}
\makeatother
\makeatletter


%referances
\newcommand{\net}[2]{{\color{blue}\href{#1}{#2}}}

%Spaces
\newcommand{\vsk}{\\[12pt]}
\newcommand{\vs}{\vspace{-12pt}}

% Tabell for opplegg

\newcommand{\ovlist}[1]{
\vspace{-16pt}
\begin{itemize}
	#1
\end{itemize}
}

\newcommand{\lst}[5]{
\rule{\linewidth}{1pt}
\footnotesize
	\textbf{Øvingsområde}\\ #1 
	
	\textbf{Utstyr}\\ #2  \\
	
	\begin{tabular}{@{} p{4cm} l} 
		\textbf{Tid} & \textbf{Elevinndeling} \\
		#3  & #4
	\end{tabular} 

\rule{\linewidth}{1pt}	\vsk
\normalsize
	\textbf{Gjennomføring}\\ #5 \vsk
}
%

\newcounter{opl}
%\numberwithin{opl}{article}

\newcommand{\opl}[1]{
\newpage
{\refstepcounter{opl} %\phantomsection 
\large \textbf{\theopl \;#1} \vsk}
}

% Headlines
\newcommand{\fork}{\textbf{Forkunnskapar}\\}
\newcommand{\forb}{\textbf{Forberedelsar}\\}
\newcommand{\opgvr}{\textbf{Oppgaver}}

\usepackage{datetime2}
\usepackage[]{hyperref}

\begin{document}

\textbf{Mål for opplæringen er at eleven skal kunne}
\begin{itemize}
	\item gjøre overslag over svar, regne praktiske oppgaver, med og utan digitale verktøy
	\item redegjøre for omgrepet lineær vekst, vise gangen i slik vekst og bruke dette i praktiske eksempel, også digitalt
\end{itemize}
\newpage
\vedlg{Skrive inn en funksjon}
\textbf{Funksjon}\bs
Si vi har funksjonen 
\[f(x)= \frac{3}{2} x^2 + 3x -3 \]
For å bruke $ f(x) $ i GeoGebra, skriver vi:
\g{3/2x\^{}2+3x}
Når vi ikke gir funksjonen noen navn, vil GeoGebra automatisk gi funksjonen navnet $ f $. I algebrafeltet får vi derfor:
\begin{figure}[H]
	\centering
	\includegraphics[scale=0.6]{skrivf}
\end{figure}

Hvis vi istedenfor har funksjonen
\[ P(x)= 0,15x^3 - 0,4 x\]
er det to ting vi må passe på. Det første er at \textsl{alle desimaltall må skrives med punktum istedenfor komma} i GeoGebra
. Det andre er at vi ønsker å gi funksjonen navnet $ P(x) $. Vi skriver da:
\g{P(x)= 0.15x\^{}3 - 0.4x}
og får:
\begin{figure}[H]
	\centering
	\includegraphics[scale=0.6]{pfig}
\end{figure}
{\color{red} \textbf{ADVARSEL}:} Man kan aldri gi funksjoner navnet $ {y(x)} $ i GeoGebra. $ y $ kan bare brukes når man skriver inn uttrykk for en rett linje, altså $ {y=ax +b} $, hvor $ a $ og $ b $ er to valgfrie tall.\vsk

\textbf{Linje}\bs
Si vi har dette uttrykket for ei linje:
\[ y = 2x+4 \]
I GeoGebra lager vi denne linjen ved å skrive akkurat det samme:
\g{
2x+4
}
Og får dette:
\begin{figure}[H]
	\centering
	\includegraphics[scale=0.6]{lin1}
\end{figure}
Ønser vi å lage ei linje som går vannrett gjennom verdien 3 på $ y $-aksen og ei linje som går loddrett gjennom verdien 2 på $ x $-aksen skriver vi:
\g{y = 3}
og 
\g{x =2 }
Da får vi denne figuren:
\begin{figure}[H]
	\centering
	\includegraphics[scale=0.6]{23}
\end{figure}

\begin{comment}
	\vedlg{Tegne grafen på eksamen}
	En veldig vanlig oppgave på eksamen er å tegne en graf på et bestemt intervall. La oss bruke Oppgave 5 fra del 2 på 1P-eksamen 2015 som eksempel:
	\begin{figure}[H]
	\centering
	\includegraphics[scale=0.6]{opg5}
	\end{figure}
	$ 0\leq x \leq 300 $ betyr at vi bare skal tegne grafen hvor $ x $ ligger mellom 0 og 300. For å få til dette bruker vi kommandoen \texttt{Funksjon[ <Funksjon>, <Start>, <Slutt> ]}:
	\g{f(x)=Funksjon[0.0013x\^{}3-0.59x\^{}2+61x+2000, 0, 300]}
	Grafen er nå tegnet inn i grafikkfeltet, men det er ikke alltid vi ser den med en gang. Dette er fordi vi ikke har justert aksene våre riktig. For å justere disse må vi bruke knappen \texttt{Flytt grafikkfelt}
	\begin{figure}[H]
	\centering
	\includegraphics[scale=0.35]{flytt}
	\end{figure}
	Siden vi ikke kan ha negative dager, bør vi her flytte aksekorset slik at origo (punktet $ (0, 0) $) ligger helt nede i venstre hjørne. Etterpå bruker vi \texttt{Flytt grafikkfelt}-knappen og drar i $ x $-aksen slik at bare tallene fra 0$ - $300 er synlige. Etterpå justerer vi $ y $-aksen slik at grafen til $ f $ er synlig for på hele dette intervallet (grafen bør dekke så mye av feltet som mulig):
	\begin{figure}[H]
	\centering
	\includegraphics[scale=0.35]{opg5a}
	\end{figure}
	Algebrafeltet tar vi med slik at både vi og sensor kan dobbeltsjekke at uttrykket vårt er skrevet inn riktig.
	
	Vi bør også skrive på hva de forskjellige aksene forteller oss. Vi leser i oppgaven at $ f(x) $ er vannstanden i mm, mens $ x $ er antall dager. For å skrive på dette bruker vi \texttt{Tekst}-knappen:
	\begin{figure}[H]
	\centering
	\includegraphics[scale=0.35]{tekst}
	\end{figure}
	Navnelappene plasserer vi i enden av hver akse:
	\begin{figure}[H]
	\centering
	\includegraphics[scale=0.35]{tekst2}
	\end{figure}
\end{comment}

\vedlg{Finne verdien til en funksjon/linje}
\textbf{Funksjon}\bs
Si vi har funksjonen
\[H(x)= x^2 + 3x -3 \]
Hvis ønsker å vite hva $ H(2) $ er, skriver vi
\g{H(2)}
som resulterer i dette:
\begin{figure}[H]
	\centering
	\includegraphics[scale=0.6]{H}
\end{figure}
Da vet vi at $ H(2)=7 $.\vsk

\textbf{Linje}\bs
Skriver vi inn ei linje blir saken litt annerledes, noe vi her skal vise ved å bruke de to linjene gitt ved uttrykkene:
\alg{
y&= x-3 \\
y&= -2x+1
}
Vi skriver disse  linjene inn i GeoGebra og får:
\begin{figure}[H]
	\centering
	\includegraphics[scale=0.6]{fglin1}
\end{figure}
Ønsker vi nå å finne hva verdien til $ {y=x-3} $ er når $ {x=2} $, må vi legge merke til at GeoGebra har kalt denne linja for $ f $. Svaret vi søker får vi da ved å skrive $ f(2) $. Ønsker vi samtidig å vite hva $ {y=-2x+1} $ er når $ {x=0} $ må vi skrive $ g(0) $:
\begin{figure}[H]
	\centering
	\includegraphics[scale=0.6]{fglin2}
\end{figure}
\vedlg{Finne skjæringspunkt}
Se videoen \net{https://drive.google.com/open?id=1cNkoEsprqnYvKIjX2aouhrr_PGTfTdHc}{skj}.
\vedlg{Finne nullpunkt}
Se videon
\net{https://drive.google.com/open?id=12aJaWEhLFlleFm6ajQTEAuNtam1UAjPz}{nullpkt}.
\vedlg{Finne topp- eller bunnpunkt}
Se videoen \net{https://drive.google.com/open?id=17DbHhH24a9zcdr8xPD3339sFh1dE1KQ3}{ekstrmpkt}.
\vedlg{Tegne linjen mellom to punkt}
Se videoen \net{https://drive.google.com/open?id=12RrLFBpSoHwj9CZQcmIytz27QmKOoGbA}{linpkt}.

\begin{comment}
	\vedlg{Regresjon}
	Når vi har en samling med \\
	
	\begin{tabular}{c|c}
	\textbf{Måned (1. i hver)} &| \textbf{Dagslengde (timer)}\\
	\hline
	1&	5.18 \\
	2&	7.50 \\
	3&	10.28 \\
	4&	13.42 \\
	5&	16.45 \\
	6&	19.23 \\
	7&	19.78 \\
	8&	17.45 \\
	9&	14.38 \\
	10&	11.4 \\ 
	11&	8.32 \\
	12&	5.70\\	
	\end{tabular}
\end{comment}
\vedlg{Tegne graf på gitt intervall}
I denne
\net{https://drive.google.com/open?id=1dOrz90-JgeaxTq7cklIcqxG65PdOnjn4}{videoen} her vi tegnet inn funksjonen:
\[ f(x)=0.0.0013x^3-0.59x^2+61x+2000\quad,\quad0\leq x\leq300 \]

\newpage
\vedlg{Oppgaver}
\textbf{G.1}\bs
\textbf{a)} Skriv den lineære funksjonen $ {f(x)=2x+4} $ og linja $ {y=2x+2} $ inn i GeoGebra. Lag $ f(x) $ blå og $ y $ grønn. Hva ser du ut ifra grafen til de to linjene?\bs
\textbf{b)} Finn verdien til $ f(x) $ når $ {x=4} $.\bs
\textbf{c)} Finn verdien til $ y $ når $ {x=-3} $.\vsk

\textbf{G.2}\bs
\textbf{a)} Tegn punktene $ (-1,2) $ og $ (2,8) $.\bs
\textbf{b)} Finn uttrykket til linja som går gjennom disse punktene.\vsk

\textbf{G.3}\bs
\textbf{a)} Skriv inn funksjonen $ {f(x)=x^2+2x-3} $.\bs
\textbf{b)} Finn $ f(4) $.\bs
\textbf{c)} Finn nullpunktene til $ f(x) $.\bs
\textbf{d)} Finn bunnpunktet til $ f(x) $.\bs
\textbf{e)} Finn skjæringspunktet mellom $ f(x) $ og linja $ y=5 $.



\end{document}

