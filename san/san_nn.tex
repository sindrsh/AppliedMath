\input{../doc}
\usepackage[T1]{fontenc}
\usepackage[utf8]{luainputenc}
\usepackage{lmodern} % load a font with all the characters
\usepackage{geometry}
\geometry{verbose,paperwidth=16.1 cm, paperheight=24 cm, inner=2.3cm, outer=1.8 cm, bmargin=2cm, tmargin=1.8cm}
\setlength{\parindent}{0bp}
\usepackage{import}
\usepackage[subpreambles=false]{standalone}
\usepackage{amsmath}
\usepackage{amssymb}
\usepackage{esint}
\usepackage{babel}
\usepackage{tabu}
\usepackage[dvipsnames, table]{xcolor}
\makeatother
\makeatletter


%referances
\newcommand{\net}[2]{{\color{blue}\href{#1}{#2}}}

%Spaces
\newcommand{\vsk}{\\[12pt]}
\newcommand{\vs}{\vspace{-12pt}}

% Tabell for opplegg

\newcommand{\ovlist}[1]{
\vspace{-16pt}
\begin{itemize}
	#1
\end{itemize}
}

\newcommand{\lst}[5]{
\rule{\linewidth}{1pt}
\footnotesize
	\textbf{Øvingsområde}\\ #1 
	
	\textbf{Utstyr}\\ #2  \\
	
	\begin{tabular}{@{} p{4cm} l} 
		\textbf{Tid} & \textbf{Elevinndeling} \\
		#3  & #4
	\end{tabular} 

\rule{\linewidth}{1pt}	\vsk
\normalsize
	\textbf{Gjennomføring}\\ #5 \vsk
}
%

\newcounter{opl}
%\numberwithin{opl}{article}

\newcommand{\opl}[1]{
\newpage
{\refstepcounter{opl} %\phantomsection 
\large \textbf{\theopl \;#1} \vsk}
}

% Headlines
\newcommand{\fork}{\textbf{Forkunnskapar}\\}
\newcommand{\forb}{\textbf{Forberedelsar}\\}
\newcommand{\opgvr}{\textbf{Oppgaver}}

\usepackage{datetime2}
\usepackage[]{hyperref}

\begin{document}
\section{Grunnprinnsippet}
Sjølve prinsippet bak sannsynsregning er at vi spør kor mange \textit{gunstige utfall}  vi har i eit utvalg av \textit{moglege utfall}. Sannsynetet for ei \textit{hending} er da gitt som eit forholdstal mellom desse. \regv
\reg[Sannsynet for ei hending \label{grnprsp}]{
\[ \text{sannsynet for ei hending}=\frac{\text{antall gunstige utfall}}{\text{antall moglege utfall}} \]
} \vsk
Når vi kastar ein terning, kallar vi 'å få ein firar' ei hending. Og da ein terning har seks forskjellige sider, er det seks moglege utfall.
\fig{san1}
Viss vi ønsker 'å få ein firar', er det bare 1 av desse 6 utfalla som gir oss det vi ønsker, altså er
\[ \text{sannsyn for å få ein firar}=
 \frac{1}{6} \]
\prbxl{0.6}{For å unngå lange uttrykk bruker vi gjerne enkeltbokstavar for å indikere ei hending. I staden for å skrive 'å få ein firar', kan vi bruke bokstaven \textit{F}, og for å indikere at vi snakkar om sannsynet for ei hending, bruker vi bokstaven \textit{P}.}\qquad \prbxr{0.3}{$ P $ kommer av det engelske ordet for sannsyn, \textit{probability}.} \\[5pt]
Når vi skriv $P(S)$ betyr dette 'sannsynet for å få ein firar':
\[ P(S)=\frac{1}{6} \]
Kva med det motsatte, altså sannsynet for å \textsl{ikke} å få ein firar? For å uttrykke at noko er motsett av ei hending, sett vi ein strek over namnet. Hendinga 'å \textsl{ikkje} få eun firar' skriv vi altså som $ \bar{F} $. Det 'å \textsl{ikkje} få ein firar' er det same som 'å få \textsl{enten} ein einar, ein toar, ein trear, ein femmar \textsl{eller} ein seksar', derfor har denne hendinga 5 gunstige utfall. Det betyr at
\[ P(\bar{S})=\frac{5}{6} \]
\reg[Symboler for sannsyn]{
$ P(A) $ er sannsynet for at hending $ A $	skjer. \vsk

$ A $ og $ \bar{A} $ er motsette hendingr. \vsk

$ P(\bar{A}) $ er sannsynet for at $ A $ \textsl{ikkje} skjer, og omvend.
} \vsk

\info{Obs!}{
 Som regel er det ei god vane å forkorte brøkar når det let seg gjere, men i sannsynsrekning vil det ofte lønne seg å la vere. Du vil derfor oppdage at mange brøkar i komande seksjonar kunne vore forkorta.
}
\section{Hendingar med og utan felles utfall}
\subsection{Hendingar utan felles utfall} \vspace{-5pt}
\prbxl{0.6}{La oss kalle hendinga 'å få ein trear' (på ein terning) for \textit{T}. Hendinga 'å få ein trear \textsl{eller} ein firar' skriv vi da som $ T\cup F $.} \qquad
\prbxr{0.3}{Symbolet \sym{$ \cup $} kallast \textit{union}.}
Det er 2 av 6 sider på ein terning som er tre \textsl{eller} fire, sannsynet for 'å få ein trear \textsl{eller} ein firar' er derfor $ \frac{2}{6} $:
\[ P(F\,\cup\,S)=\frac{2}{6} \]
\fig{san1a}

Det same svaret får vi ved å legge saman $P(F)$ og $P(S)$:
\[ P(T\,\cup\,F)=P(T)+P(F)=\frac{1}{6}+\frac{1}{6}=\frac{2}{6} \]
Å finne $ P(T\cup F) $ ved å summere $ P(T) $ og $ P(F) $ kan vi gjere da $ T $ og $ F $ ikkje har nokon \textit{felles utfall}. Dette fordi ingen sider på trekanten viser \textsl{både} en trear og en firar. \regv

\reg[Hendingar utan felles utfall \label{ufutf}]{For to hendingar \textit{A} og \textit{B} utan felles utfall, er
	\[ P(A\,\cup\,B)=P(A)+P(B) \] \vs\vs}
\newpage
\eks{
Du trekk opp ei kule frå ein bolle der det ligg éin raud, to blå og éi grøn kule. Hva er sannsynet for at du trekk opp ei raud \textsl{eller} ei blå kule?

\sv
Vi kaller hendinga 'å få ei raud kule' for $ R $ og hendinga 'å få ei blå kule' for $ B $.
\begin{itemize}
	\item Det er i alt 4 moglege utfall (kuler). 
	\item Sidan alle kulene berre har éi farge, er det ingen av hendingane $ R $ og $ B $ som har felles utfall.
	\item Sannsynet for å trekke ei raud kule er
	\[ P(R)=\frac{1}{4} \] 
	\item Sannsynet for å trekke ei blå kule er
	\[ P(B)=\frac{2}{4} \] 
\end{itemize}
	Sannsynet for å få ei raud \textsl{eller} ei blå kule er dermed 
	 \alg{
		P(R\cup B)&=P(R)+P(B)\br
		&=\frac{1}{4}+\frac{2}{4} \br
		&= \frac{3}{4} 	 
	 }
}
\newpage
\subsection{Summen av alle sannsyn er 1}
Tenk at vi kastar ein terning og at vi held både 'å få ein firar' og 'å ikkje få ein firar' for gunstige hendingar . Vi har tidlegare sett at $ {P(F)=\frac{1}{6}} $, $ {P(\bar{F})=\frac{5}{6} }$, og at $ F $ og $ \bar{F} $ ikkje har felles utfall. Av \rref{ufutf} har vi da at
\algv{
P(F\cup\bar{F})&=P(F)+P(\bar{F})\br
&= \frac{1}{6}+\frac{5}{6} \\
&= 1
}
Enten så skjer $ F $, eller så skjer den ikkje. Og skjer den ikkje, så skjer $ \bar{F} $. Viss vi sier at \textsl{både} $ F $ og $ \bar{F} $ er gunstige hendingr, seier vi altså at alle moglege utfall er gunstige, og da gir  \rref{grnprsp} eit sannsyn lik 1. \regv
 
\reg[Summen av alle sannsyner \label{sumer1}]{
Summen av sannsyna for alle moglege hendingar er alltid lik 1.
} \vsk

Ei hending $ A $ og den motsette hendinga $ \bar A $ vil til saman alltid utgjere alle hendingar. Av \rref{sumer1} har vi da at
\alg{
P(A)+P(\bar{A})&=1 \\
P(A) &= 1-P(\bar{A})
}
\reg[Motsatte hendingar \label{motsatt}]{
For ei hending $ A $ er
\[ P(A)=1-P(\bar{A}) \]
}
\newpage
\eks{
I ein klasse med 25 elevar er det 12 jenter og 13 gutar. Ein elev skal tilfeldig trekkast ut til å vere med i ein matematikkonkurranse. \vsk

\abc{
\item Kva er sannsynet for at ein gut blir trukke? 
\item Kva er sannsynet for at ein gut \textsl{ikkje} blir trukke?}

\sv
Vi kallar hendinga 'ein gutt blir trukke' for $ G $.
\abc{
\item Sannsynet for at ein gut blir trukke er
\[ P(G)=\frac{13}{25} \]
\item Sannsynet for at ein gut \textsl{ikke} blir trukket er
\alg{P(\bar{G})&=1-P(G) \\
	&=	1-\frac{13}{25} \\
	&= \frac{12}{25}
}
\mer At ein gut \textsl{ikkje} blir trukke er det same som at ei jente blir trukke.
}
}
\newpage
\subsection{Felles utfall}	
	Nokon gongar er det slik at to hendingar kan ha \textit{felles utfall}. La oss sjå på ein vanleg kortstokk med $52$ kort som er likt delt inn i typane spar, hjerter, ruter og kløver.  Kort som er av sorten knekt, dame, kong eller ess kallast \textit{honnørkort}.	
	\begin{figure}[H]
		\centering
		\includegraphics[scale=0.45]{kort}
	\end{figure}  

\prbxl{0.5}{Tenk at vi trekk opp eit kort frå ein blanda kortstokk. 
	Vi ønsker å finne sannsynet for 'å trekke kløverkort \textsl{eller} honnørkort'.
	Vi startar med å telle opp dei gunstige utfalla for kløverkort, og finn at antalet er 13.}
\qquad
\prbxr{0.4}{Eit kort som kløver kong er eit kløverkort, men det er også eit honnørkort, og derfor er det begge deler; \textsl{både} kløverkort \textsl{og} honnørkor.}
	
	\begin{figure}[H]
		\centering
		\includegraphics[scale=0.45]{kort1}
	\end{figure}
	Etterpå tell vi opp gunstige utfall for honnørkort, og finn at antalet er 16. \\
	\begin{figure}[H]
		\centering
		\includegraphics[scale=0.45]{kort2}
	\end{figure}
Til saman har vi telt ${13+16=29}$ gunstige utfall, men no støter vi på eit problem. For da vi fann alle kløverkort, telte vi blant anna kløver knekt, dame, kong og ess. Desse fire korta telte vi også da vi fann alle honnørkort, noko som betyr at vi har telt dei same korta to gongar! \\
	\begin{figure}[H]
		\centering
		\includegraphics[scale=0.45]{kort4}
	\end{figure}
Det finst no for eksempel ikkje to kløver ess i ein kortstokk, så skal vi rekne ut kvor mange kort som oppfyller kravet om å vere kløver \textsl{eller} honnør, så må vi trekke ifrå antalet kort vi har telt dobbelt:
$$ 13+16-4=25 $$
	\begin{figure}[H]
		\centering
		\includegraphics[scale=0.45]{kort3}
	\end{figure}

La \textit{K} vere hendinga 'å trekke eit kløverkort' og \textit{H} være hendinga 'å trekke eit honnørkort'. Sidan det er 25 kort som er kløverkort \textsl{eller} honnørkort av i alt 52 kort, har vi at
$$P(K\cup\,H)=\frac{25}{52}$$

Sidan vi har 13 kløverkort og 16 honnørkort, får vi vidare at
$$P(K)=\frac{13}{52} \text{\; og \;} P(H)=\frac{16}{52}$$
\prbxl{0.6}{
Vi har sett at fire kort er \textsl{både} kløver \textsl{og} honnørkort, dette skriv vi som}\qquad
\prbxr{0.3}{Symbolet \sym{$ \cap $} kallast \textit{snitt}.} \vs
\[ K\,\cap\,H=4 \]
Vi seier da at \textit{K} og \textit{H} har 4 \textit{felles utfall}.
Vidare er
\[ P(K\,\cap\,H)=\frac{4}{52} \]

No som vi har funne $ P(K), P(H)$ og $P(K\cup H)$ kan vi igjen finne  $P(K\,\cap\,H)$ på følgande måte:
\begin{align*}
P(K\,\cup\,H)&=P(K)+P(H)-P(K\,\cap\,H) \br
&= \frac{13}{52}+\frac{16}{52}-\frac{4}{52} \br
&= \frac{25}{52}
\end{align*}

\reg[Hendingar med felles utfall \label{mfutf}]{For to hendingar $ A $ og $ B $ er
	\[ P(A\,\cup\,B)=P(A)+P(B)-P(A\,\cap\,B) \]\vs\vs}
\info{Merk}{
	Viss ein anvender \rref{mfutf} på to hendingar uten felles utfall, ender ein opp med \rref{ufutf}. 
}
\newpage
\eks{ \label{eksfothand}
I ein klasse på 20 personar spelar 7 personar fotball og 10 personar spelar handball. Av desse er det 4 som spelar både fotball og handball. Om ein trekk ut éin person frå klassen, kva er sannsynet for at denne personen spelar fotball \textsl{eller} handball?
		
\sv
Vi lar $ F $ være hendinga 'spelar fotball' og $ H $ vere hendinga 'spelar handball'.
\begin{itemize} 
\item Sannsynet for at ein person spelar fotball er
\[ P(F)=\frac{7}{20} \]
\item Sannsynet for at ein person spelar handball er
\[ P(H)=\frac{10}{20} \]
\item Sannsynet for at ein person spelar \textsl{både} fotball og handball er
\[ P(F\cap H)=\frac{4}{20} \]	
\end{itemize}

Sannsynet for at ein person spelar fotball \textsl{eller} handball er derfor 
\algv{
P(F\cup H)&= P(F)+P(H)-P(F\cap H)\br
&=\frac{7}{20}+\frac{10}{20}-\frac{4}{20}\br
&=\frac{13}{20}
} 
}

\subsection{Venndiagram}
Målet med eit venndiagram er å lage ein figur som illustrerer antalet av dei \textsl{særskilde} utfalla og dei \textsl{felles} utfalla. La oss bruke eksempelet på side \pageref{eksfothand} til å lage ein slik figur. For klassen der nokon spelar fotball, nokon handball og nokon begge deler, kan vi lage eit venndiagram som vist under.
\begin{figure}
	\centering
	\includegraphics[]{\fpath{venn}}
\end{figure}
Den grøne ellipsen\footnote{Ei ellipse er ein ''strekt'' sirkel.} representerer dei som spelar fotball ($ F $) og den blå dei som spelar handball ($ H $). Da nokre spelar \textsl{begge} sportane ($ F\cap H $), har vi teikna ellipsane litt over i kvarandre. Videre veit vi at 7 spelar fotball, 10 spelar handball og 4 av disse gjer \textsl{begge} deler. Dette illustrerast slik:
\begin{figure}
	\centering
	\includegraphics[]{\fpath{vennb}}
\end{figure}
Diagrammet fortel no at 3 personar spelar \textsl{berre} fotball og 6 spelar \textsl{berre} handball. I tilleg spelar 4 personar \textsl{både} fotball og handball. (Til saman er det derfor 7 som spelar fotball og 10 som spelar handball.) 
\newpage
\eks[1]{I en skuleklasse er det 31 elevar. I denne klassen er det 15 elevar som tek buss til skulen og 9 elever som ter båt. Av desse er det 3 stykker som ter både buss og båt.

\abc{
\item Sett opp eit venndiagram som illustrerer gitt informasjon.\br

\item Éin person trekkast tilfeldig ut av klassen. Kva er sannsynet for at denne personen tar buss \textsl{eller} båt til skulen?
}
\sv
\abc{
\vs	
\item Sidan 3 elevar tek \textsl{både} buss og båt, er det \y{15-3=12} som \textsl{berre} tek buss og \y{9-3=6} som \textsl{berre} tek båt. Vi let $ A $ bety 'tar buss' og $ B $ bety 'tar båt', venndiagrammet vårt blir da sjåande slik ut:
\fig{venne}
\item Sannsynet for at ein person tek buss \textsl{eller} båt kan vi skrive som \y{P(A\cup B) }. Sidan 15 elevar tek buss, 9 ter båt og 3 tek begge deler, er det i alt \y{15+9-3=21} elever som tek buss \textsl{eller} båt. Da det er 31 elevar i alt å velge mellom, er
\[ P(A\cup B)=\frac{21}{31} \]\vs}
}
\newpage
\eks[2]{
Om en klasse med 29 elever vet vi følgende:
\begin{itemize}
	\item 10 elever spiller fotball
	\item 8 elever spiller handball
	\item 6 elever spiller volleyball
	\item 2 elever spiller både fotball og handball, men ikke volleyball
	\item 3 elever spiller både fotball og volleyball, men ikke handball
	\item ingen elever spiller både handball og volleyball, men ikke fotball.	
	\item 1 elev spiller alle tre sportene.
\end{itemize}
\textbf{a)} Sett opp et venndiagram som beskriver fordelingen av de tre sportene i klassen. La $ F $ bety \textit{spiller fobtall}, $ H $ bety \textit{spiller handball} og $ V $ bety \textit{spiller volleyball}.\br

\textbf{b)} Én person trekkes tilfeldig ut av klassen. Hva er sannsynet for at denne personen spiller enten fotball, handball eller volleyball?\br

\textbf{c)} Personen som trekkes ut viser seg å spille fotball. Hva er sjansen for at denne personen også spiller handball?

\sv
Når vi skal lage et venndiagram er det lurt å skrive inn de felles utfallene først. Ut ifra fjerde til syvende punkt kan vi tegne dette:
\begin{figure}
	\centering
	\includegraphics[]{\fpath{venn3ea}}
\end{figure}
Da ser vi videre at \y{10-2-1-3=4} elever spiller \textsl{bare} fotball, \y{8-2-1=5} spiller \textsl{bare} handball og \y{6-3-1-0=2} spiller \textsl{bare} volleyball:
\begin{figure}
	\centering
	\includegraphics[]{\fpath{venn3e}}
\end{figure} 
\textbf{b)} Av diagrammet vårt ser vi at det er $ 5+2+4+3+1+2+0=17 $ uniker elever som spiller én eller flere av sportene. Sjansen for å trekke én av disse 17 i en klasse med 29 elever er $ \frac{17}{29} $. \br

\textbf{c)} Vi leser av diagrammet at av de totalt 10 som spiller fotball, er det \y{2+1=3} som også spiller handball. Sjansen for at personen som er trukket ut spiller handball er derfor $ \frac{3}{7} $.
}
\subsection{Krysstabell}
Når det er snakk om to hendingr kan vi også sette opp en \textit{krysstabell} for å skaffe oss oversikt. Si at det på en skole med 300 elever deles ut melk og epler til de elevene som ønsker det i lunsjen. Si videre at 220 av elevene får melk, mens 250 får eple. Av disse er det 180 som får både melk og eple. Viss vi lar $ M $ bety \textit{får melk} og $ E $ bety \textit{får eple}, vil krysstabellen vår først se slik ut:
\begin{center}
	\renewcommand{\arraystretch}{1.5}
	\begin{tabular}{|c|c|c|c}
		& M &$ \bar{M} $ & sum \\
		\hline$ E $ & & \\
		\hline$ \bar{E} $ & &\\
		\hline sum & &
	\end{tabular}
\end{center}


Deretter fyller vi inn tabellen ut ifra informasjonene vi har:
\begin{itemize}
	\item får \textsl{både} melk og epple: \y{M\cap E = 180}
	\item får melk, men ikke eple: \y{M\cap \bar{E} = 220-180=40}
	\item får eple, men ikke melk: \y{E\cap M=250-180=70}
	\item får hverken melk eller eple: \y{\bar M \cap\bar{E}=300-180-40-70=10}
\end{itemize}

\begin{center}
	\renewcommand{\arraystretch}{1.5}
	\begin{tabular}{|c|c|c|c}
				& M &$ \bar{M} $ & sum \\
		\hline$ E $ & 180& 70&250 \\
		\hline$ \bar{E} $ & 40 &10&50\\
		\hline sum & 220& 80& 300
	\end{tabular}
\end{center}

\section{Gjentatte trekk \label{komb}}
\subsection{Permutasjoner}
\begin{figure}[H]
	\centering
	\includegraphics[scale=0.8]{\fpath{bolle}}
	\vs
\end{figure}
La oss tenke oss at vi har en bolle med fire kuler som er nummererte fra 1 til 4. I et forsøk trekker vi opp én og én kule fram til vi har trukket opp tre kuler. Viss vi for eksempel først trekker kule 2, deretter kule 4, og så kule 3, får vi \textit{permutasjonen} $2\; 4\; 3$. \vsk
 
Hvor mange forskjellige permutasjoner kan vi få? La oss lage en figur som hjelper oss. Ved første trekning er det 4 kuler å plukke av, vi kan derfor si at vi har 4 veier å gå. Enten trekker vi kule 1, eller kule 2, eller kule 3, eller kule 4:
\begin{figure}[H]
\centering
\includegraphics[scale=0.8]{\fpath{perm1a}}
\end{figure}
Kula vi trekker opp legger vi ut av bollen, og trekker så for andre gang. For hver av de 4 veiene vi kunne gå i første trekning får vi nå 3 nye veier å følge. Altså har vi nå  $3\cdot4=12$ veier vi kan gå.

\begin{figure}[H]
	\centering
	\includegraphics[scale=0.8]{\fpath{perm1b}}
\end{figure}
 
Den andre kula vi trekker opp legger vi også ut av bollen,  så  for hver av de 12 veiene fra 2. trekning, får vi nå to nye moglege veier å gå. Totalt antall veier (permutasjoner) blir derfor $12\cdot2=24$. 

\begin{figure}[H]
\centering
\includegraphics[scale=0.8]{\fpath{perm1c}}
\end{figure}

Denne utregningen kunne vi også ha skrevet slik:
\[ 4\cdot3\cdot2=24 \]
\reg[Produktregelen for permutasjoner]{Når vi foretar flere trekninger etter hverandre, finner vi alle moglege permutasjoner ved å gange sammen de moglege utfallene i hver trekning.
}

\eks{	
	Av de 29 bokstavene i alfabetet ønsker vi å lage et ord som består av 3 bokstaver. Vi godkjenner ord som ikke har noen betydning, men en bokstav kan bare brukes én gang i ordet.\os
	
	Hvor mange ord kan vi lage?

	\sv
	Først har vi 29 bokstaver å trekke fra, deretter 28 bokstaver, og til slutt 27 bokstaver. Dermed er antall permutasjoner gitt som
	\[ \underbrace{29}_{\substack{\text{moglege utfall}\\\text{1. trekning}}}\cdot\underbrace{28}_{\substack{\text{moglege utfall}\\\text{2. trekning}}}\cdot\underbrace{27}_{\substack{\text{moglege utfall}\\\text{3. trekning}}}=21\,924 \]
	Vi kan altså lage 21\,924 forskjellige ord.
}
\eks[2]{
	Vi kaster om krone eller mynt fire ganger etter hverandre. Hvor mange permutasjoner har vi da?
	
	\sv
	Hver gang vi kaster om krone eller mynt, har vi to moglege utfall. Antall permutasjoner er derfor gitt som
	\[ 2\cdot2\cdot2\cdot2=16 \]\vs}
\newpage
\info{Kombinasjoner}{
I dagligtale blir ofte ordet \textit{kombinasjoner} brukt istedenfor permutasjoner, men innenfor sannsynsregning har kombinasjoner og permutasjoner forskjellig betydning. Den store forskjellen er at permutasjoner tar hensyn til rekkefølge, mens kombinasjoner ikke gjør det. \vsk

Si vi ønsker å danne et ord med to bokstaver ved hjelp av med bokstavene $ A $, $ B $ og $ C $, og at vi godtar gjenbruk av bokstav. Da har vi 9 moglege permutasjoner:
\[ AA, AB, AC, BB, BA, BC, CC, CA, CB \]
Kombinasjoner derimot viser til en unik sammensetting når rekkefølge ikke tas hensyn til, for eksempel er $ AB $ og $ BA $ den samme kombinasjonen. I dette tilfelle har vi altså 6 kombinasjoner
\[ AA, AB, AC, BB, BC, CC \]
}
\newpage
\subsection{Sannsyn ved gjentatte trekk}
\prbxl{0.5}{Tenk at vi har en med bolle sju kuler. Tre av dem er grønne, to er blå og to er røde. Si at vi tar opp først én kule av bollen, og deretter én til. Hva er sannsynet for at vi trekker opp to grønne kuler?} \qquad
\fgbxr{0.4}{
\fig{bolle2}
}
Viss vi lar $ G $ bety 'å trekke en grønn kule', kan vi skrive denne sannsynet som $ P(GG) $. For å komme fram til et svar, starter vi med å finne ut hvor mange \textsl{gunstige} permutasjoner vi har. Siden vi i første trekning har 3 gunstige utfall, og i andre trekning 2 gunstige utfall, har vi $3\cdot2=6$ gunstige perutasjoner. Totalt velger vi blant 7 kuler i første trekning og 6 kuler i andre trekning. Antall \textsl{moglege} permutasjoner er derfor $7\cdot6=42$\,. Sannsynet for å få to grønne kuler blir da
\begin{equation}
P(GG)=\frac{3\cdot2}{7\cdot6}=\frac{6}{42}=\frac{1}{7} \label{trekk}
\end{equation}
\rule{\linewidth}{1pt}
La oss også finne sannsynet for å få en grønn kule for hver trekning isolert sett. I første trekning har vi 3 grønne av i alt 7 kuler, altså er
\[ P(G)=\frac{3}{7} \]
\prbxl{0.5}{I andre trekning tas det for gitt at en grønn kule er plukket opp ved første trekning, og dermed er ute av bollen. Vi har da 2 av 6 kuler som er grønne:
\[ P(G|G)=\frac{2}{6} \]
}
\qquad
\prbxr{0.4}{Symbolet \sym{$ | $} betyr \textit{\textsl{gitt} at ... har skjedd}. $ P(G|G) $ er derfor en forkortelse for 'sannsynet for å trekke en grønn kule, \textsl{gitt} at en grønn kule er trukket'.}


Viss vi ganger sammen sannsynet fra første trekning med sannsynet fra andre trekning, blir regnestykket det samme som i ligning \eqref{trekk}:
\[ P(GG)=\frac{3}{7}\cdot\frac{2}{6}=\frac{6}{42}=\frac{1}{7} \]

\reg[Sannsyn ved gjentatte trekk \label{permsans}]{Sannsynet for at $ A $ vil skje, \textsl{gitt} at $ B$ har skjedd, skrives som \y{P(A|B)}. \vsk\\

Sannsynet for at $ A $ skjer først, deretter $ B $, deretter $ C $, og så videre ($ ... $) er
\[ P(ABC...)=P(A)\cdot P(B|A)\cdot P(C|AB)\cdot... \]
}

\eks{
	I en bolle ligger to blå og to røde kuler.  Vi trekker én og én kule opp av bollen, fram til vi har hentet opp tre kuler. Hva er sannsynet for at vi først trekker en blå kule, deretter en rød, og til slutt en blå? 
		
		\sv 
		Vi lar  $ B $ bety  'å trekke blå kule' og  $ R $ bety 'å trekke rød kule'.	Sannsynet for først en blå, så en rød, og så en blå kule, skriver vi da som $P(BRB)$.
		\begin{itemize}
			\item 	Sannsynet for \textit{B} i første trekning er $P(B)=\frac{2}{4}$.
			\item Sannsynet for \textit{R} i andre trekning, \textsl{gitt} \textit{B} i første er
			\[ P(R|B)=\frac{2}{3} \]
			
			\item Sannsynet for \textit{B} i tredje trekning, \textsl{gitt} \textit{B} i første og \textit{R} i andre er
			\[ P(B|RB)=\frac{1}{2} \]
		\end{itemize}
	  Altså har vi at
		\begin{align*}
		P(BRB)&=P(B)\cdot P(R|B)\cdot P(B|RB)\br
		&= \frac{2}{4}\cdot\frac{2}{3}\cdot\frac{1}{2} \br
		&= \frac{4}{24} \br
		&= \frac{1}{6}
		\end{align*}}

\subsection{Valgtre}
Vi kan utnytte \rref{permsans} for å lage en hjelpetegning når vi har å gjøre med gjentatte trekk. Tegningen vi her skal ende opp med kalles et \textit{valgtre}. Vi tegner da en lignende figur som vi gjorde i delkapittel \ref{komb}, men langs alle veier skriver vi nå på sannsynet for utfallet veien leder oss til. 
\begin{figure}[H]
\centering
\includegraphics{\fpath{bolle2}}
\end{figure}
La oss igjen se på bollen med de syv kulene. Trekk av grønn, blå eller rød kule betegner vi henholdsvis med bokstavene \textit{G}, \textit{B} og \textit{R}.

Ved første trekning er sjansen for å trekke en grønn kule $ \frac{3}{7} $, derfor skriver vi denne brøken på veien som fører oss til \textit{G}. Gitt at vi har trukket en grønn kule, er sannsynet for også å trekke en grønn kule i andre trekning lik $ \frac{2}{6} $. Denne brøken skriver vi derfor langs veien som fører oss fra \textit{G} til \textit{G}.

Og sånn fortseter vi til vi har ført opp alle sannsynete til hver vei:  \vs
\begin{figure}[H]
\centering
\includegraphics[]{\fpath{tree}}
\end{figure}

For å få en rask oversikt over de forskjellige permutasjonene veiene fører til, kan det være lurt å skrive opp disse under hver ende av treet: 

La oss nå bruke valgtret over for å finne sannsynet for å trekke én grønn og én blå kule. \textit{GB} og \textit{BG} er da de gunstige permutasjonene. Ved å gange sammen sannsynete langs veien til \textit{GB}, finner vi at:
\[ P(GB)=\frac{3}{7}\cdot\frac{2}{6}=\frac{6}{42} \]
På samme måte kan vi finne \textit{P(BG)}:
\[ P(BG)=\frac{2}{7}\cdot\frac{3}{6}=\frac{6}{42} \]
Sannsynet for at '\textit{GB} \textsl{eller} \textit{BG}' inntreffer er (se \rref{mfutf}):
\begin{align*}
P(GB\cup BG)&=P(GB)+P(BG) \\
		&=\frac{6}{42}+\frac{6}{42} \br
		&= \frac{12}{42} \br
		&= \frac{2}{7}
\end{align*}


\reg[Permutasjoner på et valgtre]{	For å finne sannsynete til en permutasjon på et valgtre, ganger vi sammen sannsynete langs veien vi må følge for å komme til permutasjonen. }
\eks{I en bolle med 10 kuler er tre kuer grønne, to er blå og fem er røde. Du trekker to kuler ut av bollen. La $ G, B $ og $ R $ henholdsvis bety å trekke en blå, grønn eller rød kule.\vsk

\textbf{a)}  Tegn et valgtre som skisserer hvilke permutasjoner av $ B $, $ G $ og $ R $ du kan få.	\br
\textbf{b)} Hva er sannsynet for at du trekker to røde kuler? \br
\textbf{c)} Hva er sannsynet for at du trekker én blå og én grønn kule? \br
\textbf{d)} Hva er sannsynet for at du trekker \textsl{minst} én blå \textsl{eller} én grønn kule?

\sv
\textbf{a)}
\begin{figure}[H]
	\centering
	\includegraphics{\fpath{treee}}
\end{figure}
\textbf{b)} Av valgtreet vårt ser vi at:
\alg{
	P(RR)&=\frac{2}{10}\cdot\frac{1}{9}\br
	&= \frac{2}{90} \br
	&= \frac{1}{45}
}
\textbf{c)} Både permutasjonen $ GB $ og $ BG $ gir oss én blå og én grønn kule. Vi starter med å finne sannsynet for hver av dem:
\alg{
P(GB)&= \frac{3}{10}\cdot\frac{5}{9} \br
&= \frac{15}{90} \br
&= \frac{1}{6}
}
\alg{
P(BG) &= \frac{5}{10}\cdot\frac{3}{9} \br
&= \frac{1}{6}
}
Sannsynet for $ GB $ \textsl{eller} $ BG $ er summen av $ P(GB) $ og $ P(BG) $:
\alg{
P(GB\cup BG) &= P(GB)+P(BG) \\
&= \frac{1}{6}+\frac{1}{6} \br
&= \frac{2}{6} \br
&= \frac{1}{3}
}
\textbf{d)} For å svare på denne oppgaven kan vi selvsagt legge sammen sannsynet for permutasjonene $ GG $, $ GB $, $ GR$, $ BG $, $ BB $, $ BR $, $ RG $ og $ RB $, men vi sparer oss veldig mye arbeid viss vi merker oss dette: Å få \textsl{minst} én blå eller én grønn kule er det motsatte av å \textsl{bare} få røde kuler. Sjansen for dette, å få to røde kuler, fant vi i oppgave b). Av \rref{motsatt} har vi at
\alg{
P(\bar{R}) &= 1-P(R) \br
&= 1- \frac{1}{45} \br
&= \frac{45}{45}-\frac{1}{45} \br
&= \frac{44}{45}
}
Sjasen for å få \textsl{minst} én blå \textsl{eller} én grønn kule er altså $ \frac{44}{45} $.
}

\end{document}


