\input{/home/sindre/P/doc}
\usepackage[T1]{fontenc}
\usepackage[utf8]{luainputenc}
\usepackage{lmodern} % load a font with all the characters
\usepackage{geometry}
\geometry{verbose,paperwidth=16.1 cm, paperheight=24 cm, inner=2.3cm, outer=1.8 cm, bmargin=2cm, tmargin=1.8cm}
\setlength{\parindent}{0bp}
\usepackage{import}
\usepackage[subpreambles=false]{standalone}
\usepackage{amsmath}
\usepackage{amssymb}
\usepackage{esint}
\usepackage{babel}
\usepackage{tabu}
\usepackage[dvipsnames, table]{xcolor}
\makeatother
\makeatletter


%referances
\newcommand{\net}[2]{{\color{blue}\href{#1}{#2}}}

%Spaces
\newcommand{\vsk}{\\[12pt]}
\newcommand{\vs}{\vspace{-12pt}}

% Tabell for opplegg

\newcommand{\ovlist}[1]{
\vspace{-16pt}
\begin{itemize}
	#1
\end{itemize}
}

\newcommand{\lst}[5]{
\rule{\linewidth}{1pt}
\footnotesize
	\textbf{Øvingsområde}\\ #1 
	
	\textbf{Utstyr}\\ #2  \\
	
	\begin{tabular}{@{} p{4cm} l} 
		\textbf{Tid} & \textbf{Elevinndeling} \\
		#3  & #4
	\end{tabular} 

\rule{\linewidth}{1pt}	\vsk
\normalsize
	\textbf{Gjennomføring}\\ #5 \vsk
}
%

\newcounter{opl}
%\numberwithin{opl}{article}

\newcommand{\opl}[1]{
\newpage
{\refstepcounter{opl} %\phantomsection 
\large \textbf{\theopl \;#1} \vsk}
}

% Headlines
\newcommand{\fork}{\textbf{Forkunnskapar}\\}
\newcommand{\forb}{\textbf{Forberedelsar}\\}
\newcommand{\opgvr}{\textbf{Oppgaver}}

\usepackage{datetime2}
\usepackage[]{hyperref}
%\usepackage[T1]{fontenc}
\usepackage[utf8]{luainputenc}
\usepackage{lmodern} % load a font with all the characters
\usepackage{geometry}
\geometry{verbose,paperwidth=16.1 cm, paperheight=24 cm, inner=2.3cm, outer=1.8 cm, bmargin=2cm, tmargin=1.8cm}
\setlength{\parindent}{0bp}
\usepackage{import}
\usepackage[subpreambles=false]{standalone}
\usepackage{amsmath}
\usepackage{amssymb}
\usepackage{esint}
\usepackage{babel}
\usepackage{tabu}
\usepackage[dvipsnames, table]{xcolor}
\makeatother
\makeatletter


%referances
\newcommand{\net}[2]{{\color{blue}\href{#1}{#2}}}

%Spaces
\newcommand{\vsk}{\\[12pt]}
\newcommand{\vs}{\vspace{-12pt}}

% Tabell for opplegg

\newcommand{\ovlist}[1]{
\vspace{-16pt}
\begin{itemize}
	#1
\end{itemize}
}

\newcommand{\lst}[5]{
\rule{\linewidth}{1pt}
\footnotesize
	\textbf{Øvingsområde}\\ #1 
	
	\textbf{Utstyr}\\ #2  \\
	
	\begin{tabular}{@{} p{4cm} l} 
		\textbf{Tid} & \textbf{Elevinndeling} \\
		#3  & #4
	\end{tabular} 

\rule{\linewidth}{1pt}	\vsk
\normalsize
	\textbf{Gjennomføring}\\ #5 \vsk
}
%

\newcounter{opl}
%\numberwithin{opl}{article}

\newcommand{\opl}[1]{
\newpage
{\refstepcounter{opl} %\phantomsection 
\large \textbf{\theopl \;#1} \vsk}
}

% Headlines
\newcommand{\fork}{\textbf{Forkunnskapar}\\}
\newcommand{\forb}{\textbf{Forberedelsar}\\}
\newcommand{\opgvr}{\textbf{Oppgaver}}

\usepackage{datetime2}
\usepackage[]{hyperref}
\begin{document}

\section{Grunnprinnsippet}
\setlength\itemsep{0 pt}
Selve prinsippet bak sannsynlighetsregning er at vi spør oss hvor mange \textit{gunstige utfall}  vi har i et utvalg av \textit{mulige utfall} . Sannsynligheten for en \textit{hendelse} er da gitt som et forholdstall mellom disse:
$$\text{sannsynlighet for en hendelse}=\frac{\text{gunstige utfall}}{\text{mulige utfall}}$$

Når vi kaster en terning, kaller vi det \textit{å få en sekser} for en hendelse. Og fordi en terning har seks forskjellige sider, sier vi at det er seks mulige utfall. Av disse sidene er det bare én som er en sekser, derfor har vi ett gunstig utfall. 
\begin{figure}[H]
	\centering
	\includegraphics[scale=0.3]{tern}
\end{figure}Sannsynligheten for å få en sekser er altså $ \frac{1}{6} $:
\alg{
\text{sannsynlighet for å få en sekser}&= \frac{\text{gunstige utfall}}{\text{mulige utfall}}\br
&=\frac{\text{sider med sekser på}}{\text{totalt antall sider}} \br
&= \frac{1}{6}
}
For å unngå lange uttrykk bruker vi ofte enkeltbokstaver for å indikere en hendelse. Istedenfor å skrive \textit{få en sekser}, kan vi bruke bokstaven \textit{S}, og for å indikere at vi spør om sannsynligheten for en hendelse bruker vi bokstaven \textit{P}. Når vi skriver $P(S)$ betyr dette \textit{sannsynligheten for å få en sekser}:
\[ P(S)=\frac{1}{6} \]
Hva med det motsatte, altså sannsynligheten for å \textsl{ikke} få en sekser? For å uttrykke at noe er motsatt av en hendelse, setter vi en strek over navnet. Hendelsen \textit{å ikke få en sekser} skriver vi altså som $ \bar{S} $. \textit{Å ikke få en sekser} er det samme som \textit{å få enten en ener, en toer, en treer, en firer eller en femmer}, derfor har denne hendelsen fem gunstige utfall. Det betyr at:
\[ P(\bar{S})=\frac{5}{6} \]
\section{Hendelser med og uten felles utfall}
\subsection{Hendelser uten felles utfall}
\begin{wrapfigure}[4]{r}{0.25\linewidth}
	\begin{shaded*}
		Symbolet $ \cup $ kalles \textit{union}.
	\end{shaded*}
\end{wrapfigure}
La oss nå i tillegg kalle hendelsen å \textit{få en femmer} for \textit{F}. Hendelsen \textit{å få en femmer \textsl{eller} en sekser} skriver vi da som $ F\cup S $ . Det er to av seks sider på en terning som er fem \textsl{eller} seks, sannsynligheten for \textit{å få en femmer \textsl{eller} en sekser} er derfor $ \frac{2}{6} $:
$$P(F\,\cup\,S)=\frac{2}{6}$$ 
\begin{figure}[H]
	\centering
	\includegraphics[scale=0.3]{tern2}
\end{figure}

Det samme svaret hadde vi fått ved å legge sammen sannsynlighetene $P(F)$ og $P(S)$:
\[ P(F\,\cup\,S)=P(F)+P(S)=\frac{1}{6}+\frac{1}{6}=\frac{2}{6} \]
Å finne $ P(F\cup S) $ ved å legge sammen $ P(F) $ og $ P(S) $ kan vi gjøre fordi $ F $ og $ S $ ikke har noen \textit{felles utfall}. $ F $ og $ S $ har ingen felles utfall på grunn av dette: Hvis man får en femmer, kan man umulig ha fått en sekser samtidig. Og får man en sekser, kan man umulig ha fått en femmer samtidig.
\reg[Hendelser uten felles utfall \label{uf}]{For to hendelser \textit{A} og \textit{B} som ikke har noen felles utfall, er sannsynligheten for $ A $ \textsl{eller} $ B $ lik sannsynligheten for \textit{A} pluss sannsynligheten for \textit{B}:
	\[ P(A\,\cup\,B)=P(A)+P(B) \] \vs\vs}
\eks{
Du trekker opp en kule fra en bolle hvor det ligger én rød, to blå og én grønn kule. Hva er sannsynligheten for at du trekker opp en rød $ (R) $ \textsl{eller} en blå $ (B) $ kule?

\sv \vs
\begin{itemize}
	\item Det er i alt fire mulige utfall. 
	\item Siden alle kulene bare har én farge, er det ingen av hendelsene $ R $, $ B $ eller $ G $ som har felles utfall.
	\item Sannsynligheten for å trekke en rød kule er: $ P(R)=\frac{1}{4}$ 
	\item Sannsynligheten for å trekke en blå kule er: $P(B)=\frac{2}{4} $. 
\end{itemize}
	Sannsynligheten for å få en rød \textsl{eller} en blå kule blir derfor: 
	 \alg{
		P(R\cup B)&=P(R)+P(B)\br
		&=\frac{1}{4}+\frac{2}{4} \br
		&= \frac{3}{4} 	 
	 }
}
\subsection{Summen av alle sannsynligheter er 1}
Tenk nå at vi kaster en terning og at vi godtar hendelsen \textit{å få en sekser} $ (S) $ \textsl{eller} hendelsen \textit{å ikke få en sekser} $ (\bar{S}) $. Vi har tidligere sett at $ {P(S)=\frac{1}{6}} $ og at $ {P(\bar{S})=\frac{5}{6} }$. Av ?? vet vi at sannsynligheten for \textit{å få en sekser} \textsl{eller} \textit{å ikke få en sekser} er:
\alg{
P(S\cup\bar{S})&=P(S)+P(\bar{S})\br
&= \frac{1}{6}+\frac{5}{6} \\
&= 1
}
\begin{wrapfigure}[6]{r}{0.4\linewidth}
	\vs\vs
	\begin{shaded}
		Alle hendelser som kan skje? Hva med \textit{få en ener}, \textit{få en toer} osv.? Jo, de ligger alle innbakt i $ \bar{S} $.
	\end{shaded}
\end{wrapfigure}
Dette er egentlig ikke så overraskende, for kaster vi en terning er det to ting som kan skje: Enten får vi en sekser, eller så får vi det ikke. Summen $ P(S)+P(\bar{S}) $ er derfor summen av sannsynlighetene for alle hendelser som kan skje. Hvis vi ''godtar'' alle hendelser, så er alle utfall gunstige. Antall gunstige utfall blir derfor det samme som antall mulige utfall, og forholdet mellom dem blir da 1: 

\reg[Summen av alle sannsynligheter \label{sumer1}]{
Summen av sannsynlighetene for alle hendelser er alltid lik 1.
}
Slik som i tilfellet av \textit{å få en sekser}/\textit{å ikke få en sekser}, vil det alltid være slik at en hendelse $ A $ og den motsatte hendelsen $ \bar A $ til sammen utgjør alle hendelser. Av ?? har vi da at:
\alg{
P(A)+P(\bar{A})&=1 \\
P(\bar{A}) &= 1-P(A)
}
\reg[Motsatte hendelser \label{motsatt}]{
Sannsynligheten for at en hendelse $ A $ \textsl{ikke} vil skje er gitt som:
\[ P(\bar{A})=1-P(A) \]\vs\vs
}
\eks{
I en klasse med 25 elever er det 12 jenter og 13 gutter. Èn elev skal tilfeldig trekkes ut til å være med i en matematikkonkurranse. \vsk

\textbf{a)} Hva er sannsynligheten for at en gutt $ (G) $ blir trukket? \br
\textbf{b)} Hva er sannsynligheten for at en gutt \textsl{ikke} blir trukket?

\sv
\textbf{a)} Sannsynligheten for at en gutt blir trukket er: $ P(G)=\frac{13}{25} $ \br
\textbf{b)} Sannsynligheten for at en gutt \textsl{ikke} blir trukket er:
\alg{P(\bar{G})&=1-P(G) \\
&=	1-\frac{13}{25} \\
	&= \frac{12}{25}
}
\mer At en gutt \textsl{ikke} blir trukket er det samme som at en jente blir trukket.
}
\subsection{Felles utfall}	
	Noen ganger er det slik at to hendelser kan ha \textit{felles utfall}. La oss se på en vanlig kortstokk med $52$ kort som er likt delt inn i typene spar, hjerter, ruter og kløver.  Kort som er av arten knekt, dame, kong eller ess kalles honnørkort.	
	\begin{figure}[H]
		\centering
		\includegraphics[scale=0.45]{kort}
	\end{figure}  

\parbox[l][][l]{0.5\linewidth}{Tenk at vi trekker opp et kort fra en blandet kortstokk. 
	Vi ønsker å finne sannsynligheten for at kortet som trekkes 
	er kløver \textsl{eller} honnør \textsl{eller} begge deler.
	Vi starter med å telle opp de gunstige utfallene for kløverkort og finner at antallet er 13.}\qquad
\parbox[r][][l]{0.4\linewidth}{\begin{shaded}%
		Et kort som kløver kong er et kløverkort, men det er også et honnørkort, og derfor er det begge deler: både kløverkort \textsl{og} honnørkor.\end{shaded}}
	
	\begin{figure}[H]
		\centering
		\includegraphics[scale=0.45]{kort1}
	\end{figure}
	Etterpå teller vi opp gunstige utfall for honnørkort og finner at 
	antallet er 16. \\
	\begin{figure}[H]
		\centering
		\includegraphics[scale=0.45]{kort2}
	\end{figure}
Så vi har altså telt opp ${13+16=29}$ gunstige utfall. Men nå støter vi på et problem. For da vi fant alle kløverkort, telte vi blant andre kløver knekt, dame, kong og ess. Disse fire kortene telte vi også da vi fant alle honnørkort, noe som betyr at vi har telt de samme kortene to ganger! \\
	\begin{figure}[H]
		\centering
		\includegraphics[scale=0.45]{kort4}
	\end{figure}
Men det finnes jo for eksempel ikke to kløver ess i en kortstokk, så skal vi regne ut hvor mange kort som oppfyller kravet om å være kløver, honnør \textsl{eller} begge deler $ - $ ja, så må vi trekke ifra antallet kort vi har telt dobbelt:
$$ 13+16-4=25 $$
	\begin{figure}[H]
		\centering
		\includegraphics[scale=0.45]{kort3}
	\end{figure}

La nå \textit{K} være hendelsen \textit{trekke et kløverkort} og \textit{H} være hendelsen \textit{trekke et honnørkort}. Siden det er 25 kort som er kløver, honnør \textsl{eller} begge deler av i alt 52, får vi:
$$P(K\cup\,H)=\frac{25}{52}$$

Siden vi har 13 kløverkort og 16 honnørkort, får vi videre at:
$$P(K)=\frac{13}{52} \text{\; og \;} P(H)=\frac{16}{52}$$
\begin{wrapfigure}[1]{r}{0.25\textwidth}
	\vs\vs
	\begin{shaded*}
		Symbolet $ \cap $ kalles \textit{snitt}.
	\end{shaded*}
\end{wrapfigure}
Vi har sett at fire kort er \textsl{både} kløver \textsl{og} honnørkort, dette skriver vi som: 
$$K\,\cap\,H=4$$
Vi sier da at \textit{K} og \textit{H} har fire \textit{felles utfall}.
Sannsynligheten for $K\,\cap\,H$ blir:
\[ P(K\,\cap\,H)=\frac{4}{52} \]

Nå som vi har funnet $ P(K), P(H)$ og $P(K\cup H)$ kan vi igjen finne  $P(K\,\cap\,H)$ på følgende måte:
\begin{align*}
P(K\,\cup\,H)&=P(K)+P(H)-P(K\,\cap\,H) \br
&= \frac{13}{52}+\frac{16}{52}-\frac{4}{52} \br
&= \frac{25}{52}
\end{align*}

\reg[Hendelser med felles utfall]{For to hendelser $ A $ og $ B $ er sannsynligheten for \textit{A \textsl{eller} B}, lik sannsynligheten for \textit{A} pluss sannsynligheten for \textit{B}, fratrukket sannynligheten for \textsl{både} $ A $ og $ B $:
	\[ P(A\,\cup\,B)=P(A)+P(B)-P(A\,\cap\,B) \]\vs\vs}
\eks{
I en klasse på 20 personer spiller 7 personer fotball (\textit{F}), og 10 personer spiller handball (\textit{H}). Av disse er det 4 som spiller både fotball og handball. Tenk deg at du trekker ut én person fra klassen. Hva er sjansen for at denne personen spiller fotball \textsl{eller} handball?
		
\sv \vs
\begin{itemize}
\item Sannsynligheten for at en person spiller fotball er: $ P(F)=\frac{7}{20}$
\item Sannsynligheten for at en person spiller handball er: $ P(H)=\frac{10}{20} $
\item Sannsynligheten for at en person spiller \textsl{både} fotball og handball er: $P(F\cap H)=\frac{4}{20} $	
\end{itemize}

Sannsynligheten for at en person spiller fotball \textsl{eller} handball er derfor: \alg{
P(F\cup H)&= P(F)+P(H)-P(F\cap H)\br
&=\frac{7}{20}+\frac{10}{20}-\frac{4}{20}\br
&=\frac{13}{20}
} 
Sjansen er altså $ \frac{13}{20} $.
}
\subsection{Venndiagram}
Noen ganger blir vi bedt om å sette opp informasjonen vi får i et \textit{venndiagram}. Målet med et venndiagram er å lage en figur som beskriver antallet av de enkelte utfallene og de \textit{felles} utfallene. La oss bruke eksempelet over til å lage en slik figur. For klassen der noen spiller fotball, noen handball og noen begge deler, kan vi lage et venndiagram som vist under.
\begin{figure}
	\centering
	\includegraphics[]{\fpath{venn}}
\end{figure}
Den grønne ellipsen (strekt sirkel) representerer de som spiller fotball ($ F $) og den blå de som spiller handball ($ H $). Fordi noen spiller \textit{begge} sportene ($ F\cap H $), har vi tegnet ellipsene litt over i hverandre. Nå vet vi at 7 spiller fotball, 10 spiller handball og fire av disse gjør \textit{begge }deler. Dette kan vi skrive inn i diagrammet vårt på følgende måte:
\begin{figure}
	\centering
	\includegraphics[]{\fpath{vennb}}
\end{figure}
Diagrammet forteller slik at tre personer spiller \textsl{bare} fotball og 6 spiller \textsl{bare} handball. Men så er det fire stykker som spiller \textsl{både} fotball og handball, til sammen er det derfor syv som spiller fotball og ti som spiller handball.
\eks[1]{I en matematikklasse på Akademiet VGS Ålesund er det 31 elever. I denne klassen er det 15 elever som tar buss til skolen og 9 elever som tar båt. Av disse er det 3 stykker som tar både buss og båt. \vsk

\textbf{a)} Sett opp et venndiagram som beskriver informasjonen som er gitt.\br

\textbf{b)} Én person trekkes tilfeldig ut av klassen. Hva er sannsynligheten for at denne personen tar buss eller båt til skolen?

\sv
\textbf{a)}
Siden 3 elever tar \textsl{både} buss og båt, er det \y{15-3=12} som \textsl{bare} tar buss og \y{9-3} som \textsl{bare} tar båt. Vi lar $ A $ bety \textit{tar buss} og $ B $ bety \textit{tar båt}, venndiagrammet vårt blir da seende slik ut:
\begin{figure}
	\centering
	\includegraphics[]{\fpath{venne}}
\end{figure}
\textbf{b)} Sannsynligheten for at en person tar buss eller båt kan vi skrive som \y{P(A\cup B) }. Siden 15 elever tar buss, 9 tar båt og 3 tar begge deler, er det i alt \y{15+9-3=21} elever som tar buss eller båt. Siden det er 31 elever i alt å velge mellom, får vi at:
\[ P(A\cup B)=\frac{21}{31} \]\vs
}
\eks[2]{
Om en klasse med 29 elever vet vi følgende:
\begin{itemize}
	\item 10 elever spiller fotball
	\item 8 elever spiller handball
	\item 6 elever spiller volleyball
	\item 2 elever spiller både fotball og handball, men ikke volleyball
	\item 3 elever spiller både fotball og volleyball, men ikke handball
	\item ingen elever spiller både handball og volleyball, men ikke fotball.	
	\item 1 elev spiller alle tre sportene.
\end{itemize}
\textbf{a)} Sett opp et venndiagram som beskriver fordelingen av de tre sportene i klassen. La $ F $ bety \textit{spiller fobtall}, $ H $ bety \textit{spiller handball} og $ V $ bety \textit{spiller volleyball}.\br

\textbf{b)} Én person trekkes tilfeldig ut av klassen. Hva er sannsynligheten for at denne personen spiller enten fotball, handball eller volleyball?\br

\textbf{c)} Personen som trekkes ut viser seg å spille fotball. Hva er sjansen for at denne personen også spiller handball?

\sv
Når vi skal lage et venndiagram er det lurt å skrive inn de felles utfallene først. Ut ifra fjerde til syvende punkt kan vi tegne dette:
\begin{figure}
	\centering
	\includegraphics[]{\fpath{venn3ea}}
\end{figure}
Da ser vi videre at \y{10-2-1-3=4} elever spiller \textsl{bare} fotball, \y{8-2-1=5} spiller \textsl{bare} handball og \y{6-3-1-0=2} spiller \textsl{bare} volleyball:
\begin{figure}
	\centering
	\includegraphics[]{\fpath{venn3e}}
\end{figure} 
\textbf{b)} Av diagrammet vårt ser vi at det er $ 5+2+4+3+1+2+0=17 $ uniker elever som spiller én eller flere av sportene. Sjansen for å trekke én av disse 17 i en klasse med 29 elever er $ \frac{17}{29} $. \br

\textbf{c)} Vi leser av diagrammet at av de totalt 10 som spiller fotball, er det \y{2+1=3} som også spiller handball. Sjansen for at personen som er trukket ut spiller handball er derfor $ \frac{3}{7} $.
}
\subsection{Krysstabell}
Når det er snakk om to hendelser kan vi også sette opp en \textit{krysstabell} for å skaffe oss oversikt. Si at det på en skole med 300 elever deles ut melk og epler til de elevene som ønsker det i lunsjen. Si videre at 220 av elevene får melk, mens 250 får eple. Av disse er det 180 som får både melk og eple. Hvis vi lar $ M $ bety \textit{får melk} og $ E $ bety \textit{får eple}, vil krysstabellen vår først se slik ut:
\begin{center}
	\renewcommand{\arraystretch}{1.5}
	\begin{tabular}{|c|c|c|c}
		& M &$ \bar{M} $ & sum \\
		\hline$ E $ & & \\
		\hline$ \bar{E} $ & &\\
		\hline sum & &
	\end{tabular}
\end{center}


Deretter fyller vi inn tabellen ut ifra informasjonene vi har:
\begin{itemize}
	\item får \textsl{både} melk og epple: \y{M\cap E = 180}
	\item får melk, men ikke eple: \y{M\cap \bar{E} = 220-180=40}
	\item får eple, men ikke melk: \y{E\cap M=250-180=70}
	\item får hverken melk eller eple: \y{\bar M \cap\bar{E}=300-180-40-70=10}
\end{itemize}

\begin{center}
	\renewcommand{\arraystretch}{1.5}
	\begin{tabular}{|c|c|c|c}
				& M &$ \bar{M} $ & sum \\
		\hline$ E $ & 180& 70&250 \\
		\hline$ \bar{E} $ & 40 &10&50\\
		\hline sum & 220& 80& 300
	\end{tabular}
\end{center}

\section{Gjentatte trekk \label{komb}}
\subsection{Kombinasjoner}
\begin{figure}[H]
	\centering
	\includegraphics[scale=0.8]{\fpath{bolle}}
	\vs
\end{figure}
La oss tenke oss at vi har en bolle med fire kuler som er nummererte fra 1 til 4. I et forsøk trekker vi opp én og én kule fram til vi har trukket opp tre kuler. Etterpå leser vi opp \textit{kombinasjonen} vi har fått. Hvis vi for eksempel først trekker kule 2, deretter kule 4, og så kule 3, får vi kombinasjonen $2\; 4\; 3$.
 
Legger vi kulene tilbake og foretar trekningen på nytt, får vi kanskje kombinasjonen $1\; 3\; 4$, eller kanskje $4\;1\; 2$, eller en helt annen kombinasjon. Så hvor mange forskjellige kombinasjoner kan vi få? La oss lage en figur som hjelper oss. Før første trekning ligger det fire kuler i bollen, vi kan derfor si at vi har 4 veier å gå. Enten trekker vi kule 1, eller kule 2, eller kule 3, eller kule 4:
\begin{figure}[H]
\centering
\includegraphics[scale=0.8]{\fpath{komb0}}
\end{figure}
Kula vi trekker opp legger vi ut av bollen og trekker så andre gang. For hver av de fire veiene vi kunne gå i første trekning får vi nå tre nye veier å følge \textendash \,altså har vi nå  $3\cdot4=12$ veier vi kan gå.\\

\begin{figure}[H]
	\centering
	\includegraphics[scale=0.8]{\fpath{komb1}}
\end{figure}
 
Den andre kula vi trekker opp legger vi også ut av bollen,  så  for hver av de 12 veiene fra 2. trekning, får vi nå to nye mulige veier å gå. Totalt antall veier blir derfor $12\cdot2=24$. 

\begin{figure}[H]
\centering
\includegraphics[scale=0.8]{\fpath{komb}}
\end{figure}

Denne utregningen kunne vi også ha skrevet slik:
\[ 4\cdot3\cdot2=24 \]
Dette fordi vi har fire veier å gå i første trekning, for hver av disse tre veier å gå i andre trekning og for hver av disse to veier å gå i tredje trekning. Vi sier da at vi har 24 mulige kombinasjoner.
\reg[Kombinasjoner]{Når vi foretar flere trekninger etter hverandre, finner vi alle mulige kombinasjoner ved å gange sammen de mulige utfallene i hver trekning.
}

\eks{	
	I en klasse med 15 personer trekker vi tilfeldig ut tre elever som skal danne et klasseråd. Hvor mange forskjellige klasseråd kan 
	dannes?

	\sv
	Å trekke ut tre personer kan sees på som en trekning hvor vi tilfeldig tar ut én og én, fram til vi har tre personer. Antall forskjellige klasseråd som da kan oppstå er:
	$$\underbrace{15}_{\substack{\text{mulige utfall}\\\text{1. trekning}}}\cdot\underbrace{14}_{\substack{\text{mulige utfall}\\\text{2. trekning}}}\cdot\underbrace{13}_{\substack{\text{mulige utfall}\\\text{3. trekning}}}=2730$$}
\eks[2]{
	Vi kaster om krone eller mynt fire ganger etter hverandre. Hvor mange mulige utfall har vi da?
	
	\sv	\vs \vs
	\[ 2\cdot2\cdot2\cdot2=16 \]\vs}


\subsection{Sannsynlighet ved gjentatte trekk}
\begin{figure}[H]
\centering
\includegraphics{\fpath{bolle2}}
\end{figure}
La oss tenke at vi har en med bolle sju kuler. Tre av dem er grønne, to er blå og to er røde. Si at vi tar opp først én kule av bollen, og deretter én til. Vi spør oss nå hva sannsynligheten er for at vi trekker opp to grønne kuler. Hvis vi lar $ G $ bety \textit{trekke en grønn kule}, kan vi skrive denne sannsynligheten som $ P(GG) $.

For å komme fram til et svar, starter vi med å spørre oss hvor mange \textit{gunstige} kombinasjoner vi har. Siden vi i første trekning har 3 gunstige utfall, og i andre trekning 2 gunstige utfall, har vi $3\cdot2=6$ gunstige kombinasjoner. Totalt velger vi blant 7 kuler i første trekning og 6 kuler i andre trekning. Antall \textit{mulige} kombinasjoner er derfor $7\cdot6=42$\,. Sannsynligheten for å få to grønne kuler blir da:
\begin{equation}
P(GG)=\frac{3\cdot2}{7\cdot6}=\frac{6}{42} \label{trekk}
\end{equation}
La oss nå i hver trekning se på sannsynligheten for å få en grønn kule, med krav om at dette skal skje i begge trekninger. I første trekning har vi 3 grønne av i alt 7 kuler, dermed får vi at:
\[ P(G \text{ i første trekning})=\frac{3}{7} \]
\begin{wrapfigure}[]{r}{0.6\linewidth}
	\vspace{-22 pt}
	\begin{shaded*}
		Symbolet $ | $ betyr \textit{\textsl{gitt} at ... har skjedd}. $ P(G|G) $ er derfor en forkortelse for \textit{sannsynligheten for å trekke en grønn kule, gitt at en grønn kule er trukket.}
	\end{shaded*}
\end{wrapfigure}
I andre trekning tas det for gitt at en grønn kule er plukket opp ved første trekning, og dermed er ute av bollen. Vi har da 2 av 6 kuler som er grønne:
\[ P(G|G)=\frac{2}{6} \]
Hvis vi nå ganger sammen sannsynligheten fra første trekning, med sannsynligheten fra andre trekning, ser vi at regnestykket blir det samme som i ligning (\ref{trekk}):
\[ P(GG)=\frac{3}{7}\cdot\frac{2}{6}=\frac{6}{42} \]

\reg[Sannsynlighet ved gjentatte trekk \label{kombsans}]{Sannsynligheten for at $ A $ vil skje, \textsl{gitt} at $ B$ har skjedd, skrives som \y{P(A|B)}. \vsk\\

Sannsynligheten for at $ A $  trekkes først, deretter $ B $, deretter $ C $ og så videre ($ ... $) er:
\[ P(ABC...)=P(A)\cdot P(B|A)\cdot P(C|AB)\cdot... \]
}

\eks{
	I en bolle ligger to blå og to røde kuler. Vi lar  $ B $ bety  \textit{blå kule trekkes} og  $ R $ bety \textit{rød kule trekkes}. Vi trekker én og én kule opp av bollen, fram til vi har hentet opp tre kuler. Hva er sannsynligheten for at vi først trekker en blå kule, deretter en rød, og til slutt en blå? 
		
		\sv 
		Sannsynligheten for først en blå, så en rød, så en blå kule, kan vi skrive som $P(BRB)$.
		\begin{itemize}
			\item 	Sannsynligheten for \textit{B} i første trekning er: $P(B)=\frac{2}{4}$.
			\item Sannsynligheten for \textit{R} i andre trekning, \textsl{gitt} \textit{B} i første er: $P(R|B)=\frac{2}{3}$.
			
			\item Sannsynligheten for \textit{B} i tredje trekning, \textsl{gitt} \textit{B} i første og \textit{R} i andre er:
			$P(B|RB)=\frac{1}{2}$.
		\end{itemize}
	  Vi får dermed:
		\begin{align*}
		P(BRB)&=P(B)\cdot P(R|B)\cdot P(B|RB)\br
		&= \frac{2}{4}\cdot\frac{2}{3}\cdot\frac{1}{2} \br
		&= \frac{4}{24}
		\end{align*}}

\subsection{Valgtre}
Vi kan utnytte ?? for å lage en hjelpetegning når vi har å gjøre med gjentatte trekk. Tegningen vi her skal ende opp med kalles et \textit{valgtre}. Vi tegner da en lignende figur som vi gjorde i delkapittel \ref{komb}, men langs alle veier skriver vi nå på sannsynligheten for utfallet veien leder oss til. 
\begin{figure}[H]
\centering
\includegraphics{\fpath{bolle2}}
\end{figure}
La oss igjen se på bollen med de syv kulene. Trekk av grønn, blå eller rød kule betegner vi henholdsvis med bokstavene \textit{G}, \textit{B} og \textit{R}.

Ved første trekning er sjansen for å trekke en grønn kule $ \frac{3}{7} $, derfor skriver vi denne brøken på veien som fører oss til \textit{G}. Gitt at vi har trukket en grønn kule, er sannsynligheten for også å trekke en grønn kule i andre trekning lik $ \frac{2}{6} $. Denne brøken skriver vi derfor langs veien som fører oss fra \textit{G} til \textit{G}.

Og sånn fortseter vi til vi har ført opp alle sannsynlighetene til hver vei:  \vs
\begin{figure}[H]
\centering
\includegraphics[]{\fpath{tree}}
\end{figure}

For å få en rask oversikt over de forskjellige kombinasjonene veiene fører til, kan det være lurt å skrive opp disse under hver ende av treet: 

La oss nå bruke valgtret over for å finne sannsynligheten for å trekke én grønn og én blå kule. \textit{GB} og \textit{BG} er da de gunstige kombinasjonene. Ved å gange sammen sannsynlighetene langs veien til \textit{GB}, finner vi at:
\[ P(GB)=\frac{3}{7}\cdot\frac{2}{6}=\frac{6}{42} \]
På samme måte kan vi finne \textit{P(BG)}:
\[ P(BG)=\frac{2}{7}\cdot\frac{3}{6}=\frac{6}{42} \]
Sannsynligheten for at \textit{GB eller BG} inntreffer blir (se \rref{uf}):
\begin{align*}
P(GB\cup BG)&=P(GB)+P(BG) \\
		&=\frac{6}{42}+\frac{6}{42} \\
		&= \frac{12}{42}
\end{align*}


\reg[Kobinasjoner på et valgtre]{	For å finne sannsynlighetene til en kombinasjon på et valgtre, ganger vi sammen sannsynlighetene langs veien vi må følge for å komme til kombinasjonen. }
\eks{I en bolle med 10 kuler er tre kuer grønne, to er blå og fem er røde. Du trekker to kuler ut av bollen. La $ G, B $ og $ R $ henholdsvis bety å trekke en blå, grønn eller rød kule.\vsk

\textbf{a)}  Tegn et valgtre som skisserer hvilke kombinasjoner av $ B $, $ G $ og $ R $ du kan få.	\br
\textbf{b)} Hva er sannsynligheten for at du trekker to røde kuler? \br
\textbf{c)} Hva er sannsynligheten for at du trekker én blå og én grønn kule? \br
\textbf{d)} Hva er sannsynligheten for at du trekker \textsl{minst} én blå eller én grønn kule?

\sv
\textbf{a)}
\begin{figure}[H]
	\centering
	\includegraphics{\fpath{treee}}
\end{figure}
\textbf{b)} Av valgtreet vårt ser vi at:
\alg{
	P(RR)&=\frac{2}{10}\cdot\frac{1}{9}\br
	&= \frac{2}{90} \br
	&= \frac{1}{45}
}
\textbf{c)} Både kombinasjonen $ GB $ og $ BG $ gir oss én blå og én grønn kule. Vi starter med å finne sannsynligheten for hver av dem:
\alg{
P(GB)&= \frac{3}{10}\cdot\frac{5}{9} \br
&= \frac{15}{90} \br
&= \frac{1}{6}
}
\alg{
P(BG) &= \frac{5}{10}\cdot\frac{3}{9} \br
&= \frac{1}{6}
}
Sannsynligheten for $ GB $ \textsl{eller} $ BG $ er summen av $ P(GB) $ og $ P(BG) $:
\alg{
P(GB\cup BG) &= P(GB)+P(BG) \\
&= \frac{1}{6}+\frac{1}{6} \br
&= \frac{2}{6} \br
&= \frac{1}{3}
}
\textbf{d)} For å svare på denne oppgaven kan vi selvsagt legge sammen sannsynligheten for kombinasjonene $ GG $, $ GB $, $ GR$, $ BG $, $ BB $, $ BR $, $ RG $ og $ RB $, men vi sparer oss veldig mye arbeid hvis vi merker oss dette: Å få \textsl{minst} én blå eller én grønn kule er det motsatte av å \textsl{bare} få røde kuler. Sjansen for dette, å få to røde kuler, fant vi i oppgave b). Av \rref{motsatt} har vi at:
\alg{
P(\bar{R}) &= 1-P(R) \br
&= 1- \frac{1}{45} \br
&= \frac{45}{45}-\frac{1}{45} \br
&= \frac{44}{45}
}
Sjasen for å få \textsl{minst} én blå eller én grønn kule er altså $ \frac{44}{45} $.
}

\end{document}


