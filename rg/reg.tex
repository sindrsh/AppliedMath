\documentclass[11pt,english]{report}
\usepackage[T1]{fontenc}
\usepackage[utf8]{luainputenc}
\usepackage{lmodern} % load a font with all the characters
\usepackage{geometry}
\geometry{verbose,paperwidth=16.1 cm, paperheight=24 cm, inner=2.3cm, outer=1.8 cm, bmargin=2cm, tmargin=1.8cm}
\setlength{\parindent}{0bp}
\usepackage{import}
\usepackage[subpreambles=false]{standalone}
\usepackage{amsmath}
\usepackage{amssymb}
\usepackage{esint}
\usepackage{babel}
\usepackage{tabu}
\usepackage[dvipsnames, table]{xcolor}
\makeatother
\makeatletter


%referances
\newcommand{\net}[2]{{\color{blue}\href{#1}{#2}}}

%Spaces
\newcommand{\vsk}{\\[12pt]}
\newcommand{\vs}{\vspace{-12pt}}

% Tabell for opplegg

\newcommand{\ovlist}[1]{
\vspace{-16pt}
\begin{itemize}
	#1
\end{itemize}
}

\newcommand{\lst}[5]{
\rule{\linewidth}{1pt}
\footnotesize
	\textbf{Øvingsområde}\\ #1 
	
	\textbf{Utstyr}\\ #2  \\
	
	\begin{tabular}{@{} p{4cm} l} 
		\textbf{Tid} & \textbf{Elevinndeling} \\
		#3  & #4
	\end{tabular} 

\rule{\linewidth}{1pt}	\vsk
\normalsize
	\textbf{Gjennomføring}\\ #5 \vsk
}
%

\newcounter{opl}
%\numberwithin{opl}{article}

\newcommand{\opl}[1]{
\newpage
{\refstepcounter{opl} %\phantomsection 
\large \textbf{\theopl \;#1} \vsk}
}

% Headlines
\newcommand{\fork}{\textbf{Forkunnskapar}\\}
\newcommand{\forb}{\textbf{Forberedelsar}\\}
\newcommand{\opgvr}{\textbf{Oppgaver}}

\usepackage{datetime2}
\usepackage[]{hyperref}
%\usepackage[T1]{fontenc}
\usepackage[utf8]{luainputenc}
\usepackage{lmodern} % load a font with all the characters
\usepackage{geometry}
\geometry{verbose,paperwidth=16.1 cm, paperheight=24 cm, inner=2.3cm, outer=1.8 cm, bmargin=2cm, tmargin=1.8cm}
\setlength{\parindent}{0bp}
\usepackage{import}
\usepackage[subpreambles=false]{standalone}
\usepackage{amsmath}
\usepackage{amssymb}
\usepackage{esint}
\usepackage{babel}
\usepackage{tabu}
\usepackage[dvipsnames, table]{xcolor}
\makeatother
\makeatletter


%referances
\newcommand{\net}[2]{{\color{blue}\href{#1}{#2}}}

%Spaces
\newcommand{\vsk}{\\[12pt]}
\newcommand{\vs}{\vspace{-12pt}}

% Tabell for opplegg

\newcommand{\ovlist}[1]{
\vspace{-16pt}
\begin{itemize}
	#1
\end{itemize}
}

\newcommand{\lst}[5]{
\rule{\linewidth}{1pt}
\footnotesize
	\textbf{Øvingsområde}\\ #1 
	
	\textbf{Utstyr}\\ #2  \\
	
	\begin{tabular}{@{} p{4cm} l} 
		\textbf{Tid} & \textbf{Elevinndeling} \\
		#3  & #4
	\end{tabular} 

\rule{\linewidth}{1pt}	\vsk
\normalsize
	\textbf{Gjennomføring}\\ #5 \vsk
}
%

\newcounter{opl}
%\numberwithin{opl}{article}

\newcommand{\opl}[1]{
\newpage
{\refstepcounter{opl} %\phantomsection 
\large \textbf{\theopl \;#1} \vsk}
}

% Headlines
\newcommand{\fork}{\textbf{Forkunnskapar}\\}
\newcommand{\forb}{\textbf{Forberedelsar}\\}
\newcommand{\opgvr}{\textbf{Oppgaver}}

\usepackage{datetime2}
\usepackage[]{hyperref}
\begin{document}


\section{Regnerekkefølge}
\rgm[Regneregler for \boldmath $ + $ og $ - $]{
\alg{
	+- &= - \\
	+\cdot- &= - \\
	-\cdot-&= -
	}
\vspace{-11 pt}
	}
\eks[1]{
	\algvv{ 3--2 &= 3+2 \\
	&= 5 }
	}
\rgm[Regnerekkefølge]{
\begin{enumerate}
	\item Regnestykker inni paranteser
	\item Potenser
	\item Ganging og deling
	\item Pluss og minus
\end{enumerate}
	}
\eks[1]{
\algvv{
2\cdot(-3)+4 &= -6+4 \\
&= -2	
	}	
	}
\eks[2]{
\vspace{-22 pt}
\begin{flalign*}
	&&4\cdot3^2 &= 4\cdot9 &&\text{Husk: } 3^2=3\cdot3=9\\
	&& &= 36
\end{flalign*}
	}
\eks[3]{
\algvv{
(2+4)\cdot(-3) &= 6\cdot(-3)\\
&= -18		
	}
	}
\eks[4]{
\begin{flalign*}
&&	&\quad\,(2--5)\cdot2^3-4\cdot3+2 && \text{Paranteser først} \\
&&	&= 7\cdot2^2-10:(-2)+2 && \text{Potenser etterpå} \\
&&	&= 7\cdot 4	-10:(-2)+2 && \text{Så} \cdot \text{og} : \\
&&	&= 28--5+2 && + \text{og} - \text{til slutt} \\
&&	&= 28+5+2 &&  \\
&&	&= 35
\end{flalign*}
	}
\section{Overslagsregning}
\rgm[Overslagsregning]{
\begin{eqnarray*}
+ &=& \text{Rund ett tall opp, ett tall ned}	\\
\cdot &=& \text{Rund ett tall opp, ett tall ned}	\\
- &=& \text{Rund begge tall opp, eller begge tall ned}	\\
: &=& \text{Rund begge tall opp, eller begge tall ned}
\end{eqnarray*}
Fremste siffer bør helst ikke endres med mer enn 1:
\begin{flalign*}
&& 8,5 &\approx 9 &&\color{ForestGreen} \text{O.K} \\
&& 8,5 & \approx 10 &&\color{red} \text{ikke O.K}
\end{flalign*}
	}
\eks[]{
Rund av og finn omtrentlig svar for regnestykkene.\\

	\begin{tabular}{l l}
		a) $ 23,1+174,7 $ &\quad b) $ 11,8\cdot107,2 $ \\
		c) $ 37,4-18,9 $  &\quad d) $ 2300:210,3 $	
	\end{tabular}



	}
\section{Brøkregning}
\rg[Brøk ganget med heltall]{
Når en brøk ganges med et heltall, ganger vi telleren med heltallet over brøkstreken og beholder nevneren under.
}
\eks[1]{
\algvv{
\frac{1}{3}\cdot 4&=\frac{1\cdot 4}{3} \br
&=\frac{4}{3}	
	}
}

\eks[2]{		
\algvv{
(-3)\cdot\frac{2}{5}&=\frac{(-3)\cdot 2}{5}\br
&=\frac{-6}{5} \br
&= -\frac{6}{5}
	}
}
\rg[Brøker ganget sammen]{Når brøker ganges sammen, setter vi alt på en felles brøkstrek; over ganger vi sammen tellerene og under ganger vi sammen nevnerene.}

\eks[1]{
\algv{\frac{1}{3}\cdot\frac{2}{5}&=\frac{1\cdot 2}{3\cdot 5}\br
	&=\frac{2}{15}}
}
\eks[2]{
\algv{
\frac{2}{3}\cdot\frac{1}{2}\cdot\frac{(-5)}{4}&=\frac{\bcancel{2}\cdot 1\cdot(-5)}{3\cdot \bcancel{2}\cdot 4}\br 	
&= -\frac{5}{12} \
	}	
	}
\rg[Utviding av brøk]{
Når vi utvider brøker, ganger vi det samme tallet både over og under brøkstreken. Verdien til brøken forblir den samme.
	}
\eks[1]{
\textbf{a)} Finn verdien til brøken $ \dfrac{1}{4} $. \\

\textbf{b)} Utvid brøken $ \dfrac{1}{4} $ med tallet $ 3 $. \\

\textbf{c)} Finn verdien til brøken du fant i oppgave \textit{b}.\\

\sv
\textbf{a)} Vi taster inn \texttt{1:4} på en kalkulator og får at:\[ \frac{1}{4}=0,25 \]
\textbf{b)}	\algv{\frac{1}{4}&=\frac{1 \cdot 3}{4\cdot 3}\br &=\frac{4}{12}}
\textbf{c)} Vi taster inn \texttt{4:12} på en kalkulator, og får at: \[ \frac{3}{12}=0,25 \]
}		
\eks[2]{
Utvid brøken $ \dfrac{2}{5} $ til en brøk med 15 som nevner.\\

\sv 
Siden den opprinnelig nevneren er 5, må vi gange med 3 for å få 15 som ny nevner. Dette må vi da gjøre både over og under brøkstreken:
\alg{\frac{2}{5}&=\frac{2\cdot3}{5\cdot3} \br
	&= \frac{6}{15}}	
	}
\rg[Stryking av tall i brøker]{Hvis vi har \textit{bare} gangetegn over og under en brøkstrek, og det samme tallet befinner seg både oppe og nede, da kan vi stryke dette tallet.}
\eks{
\algvv{
	\frac{2\cdot4\cdot5}{4\cdot3\cdot5} &= \frac{2\cdot\cancel{4}\cdot\cancel{5}}{\cancel{4}\cdot3\cdot\cancel{5}} \br
	&= \frac{2}{3}
}	
	}
\eks[2]{
\begin{flalign*}
&&	\frac{2}{4\cdot2} &= \frac{\cancel{2}}{4\cdot\cancel{2}}&& \br
&&	&= \frac{1}{4}	&& \text{1 over hvis alle andre tall oppe er strøket}
\end{flalign*}
	}
\subsection{Utviding av brøker}
Alle brøker har en verdi. Brøken $ \frac{1}{4} $ har for eksempel verdien $ 1:4=0,25 $. Men hva nå med brøken $ \frac{5}{20} $? Jo, vi kan regne ut $ 5:20=0,25 $, altså har $ \frac{5}{20} $ samme verdien som $ \frac{1}{4} $. Vi skal nå se litt på hvorfor det er slik.

Når vi ganger et tall med 1, vil ikke verdien endre seg. Ganger vi for eksempel $ \frac{1}{4} $ med 1, får vi fortsat 0,25 som verdi:
\[ \frac{1}{4}\cdot 1=0,25 \]
Men tallet 1 kan vi skrive på mange måter, for eksempel er jo $ \frac{5}{5}=1 $. Vi kan derfor skrive:
\alg{
\frac{1}{4}\cdot\underbrace{\frac{5}{5}}_1 &=0,25\br
\frac{1\cdot5}{4\cdot5}&=0,25 \\
\frac{5}{20} &= 0,25
	}
Vi har altså gjort om brøken $ \frac{1}{4} $ til $ \frac{5}{20} $, men ikke forandret \textit{verdien}. Når vi gjør dette og får et større tall enn det vi opprinnelig hadde som nevner, sier vi at vi har utvidet brøken.
\rg[Utviding av brøk]{
For å utvide en brøk, ganger vi med det samme tallet både over og under brøkstreken:	
	}
\eks{
Utvid brøken $ \frac{2}{3} $ med tallet 5. \\

\sv
\alg{
	\frac{2}{3} &= \frac{2\cdot5}{3\cdot5} \\
	&= \frac{10}{15}	
	}	
	}
\subsection{Regnestykker med brøker}
\rg[Fellesnevner]{
Når vi har et regnestykke hvor vi skal legge sammen eller trekke ifra flere brøker, må vi finne en fellesnevner. Dette er et tall som er delelig med alle nevnerene.	\\

En fellesnevner kan vi alltid finne ved å gange sammen alle ulike nevnere.
	}
\eks[1]{
Regn ut:
\[ \frac{4}{5}-\frac{2}{3} \]

\sv
Siden 15 er delelig med både 5 ($ 15:5=3 $) og 3 ($ 15:3=5 $), så er 15 en fellesnevner:
\begin{flalign*}
&&\frac{3}{5}-\frac{2}{3}	&= \frac{4\cdot 3}{5\cdot3}-\frac{2\cdot5}{3\cdot5} &&\text{Utvider til fellesnevner}\br 
&& &= \frac{12}{15}-\frac{10}{15} &&\text{Trekker sammen}\br
&& &= \frac{5}{15} && \text{Forkorter}\br 
&& &= \frac{1}{3} &&
\end{flalign*}	
	}	
\eks[2]{
Regn ut:
\[ \frac{1}{2}-\frac{7}{4}+\frac{5}{3} \]	
\sv
\textsl{Løsningsmetode 1:} \\
12 er fellesnevner siden tallet er delelig på 2, 3, og 4:
\alg{
-\frac{1}{2}-\frac{7}{4}+\frac{5}{3} &= -\frac{1\cdot6}{2\cdot6}-\frac{7\cdot3}{4\cdot3}+\frac{5\cdot4}{3\cdot4} \br 
&= \frac{6}{12}-\frac{21}{12}+\frac{20}{12}	\br
&= -\frac{5}{12}
	}
\textsl{Løsningsmetode 2:} \\
Hvis vi ganger sammen alle nevnerene, får vi $ 2\cdot3\cdot4=24 $. 24 er altså en fellesnevner:
\alg{
	-\frac{1}{2}-\frac{7}{4}+\frac{5}{3}+\frac{1}{3} &= -\frac{1\cdot12}{2\cdot12}-\frac{7\cdot6}{4\cdot6}+\frac{5\cdot8}{3\cdot8} \br 
	&= \frac{12}{24}-\frac{42}{24}+\frac{40}{24}	\br
	&= -\frac{10}{24}\br 
	&= -\frac{5}{12}
}
	}
\end{document}


