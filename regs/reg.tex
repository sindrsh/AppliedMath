\input{/home/sindre/P/doc}
\usepackage[T1]{fontenc}
\usepackage[utf8]{luainputenc}
\usepackage{lmodern} % load a font with all the characters
\usepackage{geometry}
\geometry{verbose,paperwidth=16.1 cm, paperheight=24 cm, inner=2.3cm, outer=1.8 cm, bmargin=2cm, tmargin=1.8cm}
\setlength{\parindent}{0bp}
\usepackage{import}
\usepackage[subpreambles=false]{standalone}
\usepackage{amsmath}
\usepackage{amssymb}
\usepackage{esint}
\usepackage{babel}
\usepackage{tabu}
\usepackage[dvipsnames, table]{xcolor}
\makeatother
\makeatletter


%referances
\newcommand{\net}[2]{{\color{blue}\href{#1}{#2}}}

%Spaces
\newcommand{\vsk}{\\[12pt]}
\newcommand{\vs}{\vspace{-12pt}}

% Tabell for opplegg

\newcommand{\ovlist}[1]{
\vspace{-16pt}
\begin{itemize}
	#1
\end{itemize}
}

\newcommand{\lst}[5]{
\rule{\linewidth}{1pt}
\footnotesize
	\textbf{Øvingsområde}\\ #1 
	
	\textbf{Utstyr}\\ #2  \\
	
	\begin{tabular}{@{} p{4cm} l} 
		\textbf{Tid} & \textbf{Elevinndeling} \\
		#3  & #4
	\end{tabular} 

\rule{\linewidth}{1pt}	\vsk
\normalsize
	\textbf{Gjennomføring}\\ #5 \vsk
}
%

\newcounter{opl}
%\numberwithin{opl}{article}

\newcommand{\opl}[1]{
\newpage
{\refstepcounter{opl} %\phantomsection 
\large \textbf{\theopl \;#1} \vsk}
}

% Headlines
\newcommand{\fork}{\textbf{Forkunnskapar}\\}
\newcommand{\forb}{\textbf{Forberedelsar}\\}
\newcommand{\opgvr}{\textbf{Oppgaver}}

\usepackage{datetime2}
\usepackage[]{hyperref}

\begin{document}
\tableofcontents
\newpage
\chapter{Tallregning og algebra}
\section{Regnerekkefølge}

\section{Overslagsregning}
\rgm[Overslagsregning]{
\begin{eqnarray*}
+ &=& \text{Rund ett tall opp, ett tall ned}	\\
\cdot &=& \text{Rund ett tall opp, ett tall ned}	\\
- &=& \text{Rund begge tall opp, eller begge tall ned}	\\
: &=& \text{Rund begge tall opp, eller begge tall ned}
\end{eqnarray*}
Fremste siffer bør helst ikke endres med mer enn 1:
\begin{flalign*}
&& 8,5 &\approx 9 &&\color{ForestGreen} \text{O.K} \\
&& 8,5 & \approx 10 &&\color{red} \text{ikke O.K}
\end{flalign*}
	}
\eks[]{
Rund av og finn omtrentlig svar for regnestykkene.\\

	\begin{tabular}{l l}
		\textbf{a)} $ 23,1+174,7 $ &\quad \textbf{b)} $ 11,8\cdot107,2 $ \\
		\textbf{c)} $ 37,4-18,9 $  &\quad \textbf{d)} $ 2300:210,3 $	
	\end{tabular}



	}
\section{Brøkregning}
\rg[Brøk ganget med heltall]{
Når en brøk ganges med et heltall, ganger vi telleren med heltallet over brøkstreken og beholder nevneren under.
}
\eks[1]{
\algvv{
\frac{1}{3}\cdot 4&=\frac{1\cdot 4}{3} \br
&=\frac{4}{3}	
	}
}

\eks[2]{		
\algvv{
(-3)\cdot\frac{2}{5}&=\frac{(-3)\cdot 2}{5}\br
&=\frac{-6}{5} \br
&= -\frac{6}{5}
	}
}
\rg[Stryking av tall i brøker]{Hvis vi har \textit{bare} gangetegn over og under en brøkstrek, og det samme tallet befinner seg både oppe og nede, da kan vi stryke dette tallet.}
\eks{
	\algvv{
		\frac{2\cdot4\cdot5}{4\cdot3\cdot5} &= \frac{2\cdot\cancel{4}\cdot\cancel{5}}{\cancel{4}\cdot3\cdot\cancel{5}} \br
		&= \frac{2}{3}
	}	
}
\eks[2]{
	\begin{flalign*}
		&&	\frac{2}{4\cdot2} &= \frac{\cancel{2}}{4\cdot\cancel{2}}&& \br
		&&	&= \frac{1}{4}	&& \text{Hvis alle tall er strøket, står vi igjen med 1}
	\end{flalign*}
}
\rg[Brøker ganget sammen]{Når brøker ganges sammen, setter vi alt på en felles brøkstrek. Over ganger vi sammen tellerene og under ganger vi sammen nevnerene.}

\eks[1]{
\algv{\frac{1}{3}\cdot\frac{2}{5}&=\frac{1\cdot 2}{3\cdot 5}\br
	&=\frac{2}{15}}
}
\eks[2]{
\algv{
\frac{2}{3}\cdot\frac{1}{2}\cdot\frac{(-5)}{4}&=\frac{\bcancel{2}\cdot 1\cdot(-5)}{3\cdot \bcancel{2}\cdot 4}\br 	
&= -\frac{5}{12} \
	}	
	}
\rg[Utviding av brøk]{
Når vi utvider brøker, ganger vi det samme tallet både over og under brøkstreken. Verdien til brøken forblir den samme.
	}
\eks[1]{
\textbf{a)} Finn verdien til brøken $ \dfrac{1}{4} $. \\

\textbf{b)} Utvid brøken $ \dfrac{1}{4} $ med tallet $ 3 $. \\

\textbf{c)} Finn verdien til brøken du fant i oppgave \textit{b}.\\

\sv
\textbf{a)} Vi taster inn \texttt{1:4} på en kalkulator og får at:\[ \frac{1}{4}=0,25 \]
\textbf{b)}	\algv{\frac{1}{4}&=\frac{1 \cdot 3}{4\cdot 3}\br &=\frac{4}{12}}
\textbf{c)} Vi taster inn \texttt{4:12} på en kalkulator, og får at: \[ \frac{3}{12}=0,25 \]
}		
\eks[2]{
Utvid brøken $ \dfrac{2}{5} $ til en brøk med 15 som nevner.\\

\sv 
Siden den opprinnelig nevneren er 5, må vi gange med 3 for å få 15 som ny nevner. Dette må vi da gjøre både over og under brøkstreken:
\alg{\frac{2}{5}&=\frac{2\cdot3}{5\cdot3} \br
	&= \frac{6}{15}}	
	}

\rg[Fellesnevner]{
Når vi har et regnestykke hvor vi skal legge sammen eller trekke ifra flere brøker, må vi finne en fellesnevner. Dette er et tall som er delelig med alle nevnerene.	\\

En fellesnevner kan vi alltid finne ved å gange sammen alle ulike nevnere.
	}
\eks[1]{
Regn ut:
\[ \frac{4}{5}-\frac{2}{3} \]

\sv
Siden 15 er delelig med både 5 ($ 15:5=3 $) og 3 ($ 15:3=5 $), så er 15 en fellesnevner:
\begin{flalign*}
&&\frac{3}{5}-\frac{2}{3}	&= \frac{4\cdot 3}{5\cdot3}-\frac{2\cdot5}{3\cdot5} &&\text{Utvider til fellesnevner}\br 
&& &= \frac{12}{15}-\frac{10}{15} &&\text{Trekker sammen}\br
&& &= \frac{5}{15} && \text{Forkorter}\br 
&& &= \frac{1}{3} &&
\end{flalign*}	
	}	
\eks[2]{
Regn ut:
\[ \frac{1}{2}-\frac{7}{4}+\frac{5}{3} \]	
\sv
\textsl{Løsningsmetode 1:} \\
12 er fellesnevner siden tallet er delelig på 2, 3, og 4:
\alg{
-\frac{1}{2}-\frac{7}{4}+\frac{5}{3} &= -\frac{1\cdot6}{2\cdot6}-\frac{7\cdot3}{4\cdot3}+\frac{5\cdot4}{3\cdot4} \br 
&= \frac{6}{12}-\frac{21}{12}+\frac{20}{12}	\br
&= -\frac{5}{12}
	}
\textsl{Løsningsmetode 2:} \\
Hvis vi ganger sammen alle nevnerene, får vi $ 2\cdot3\cdot4=24 $. 24 er altså en fellesnevner:
\alg{
	-\frac{1}{2}-\frac{7}{4}+\frac{5}{3}+\frac{1}{3} &= -\frac{1\cdot12}{2\cdot12}-\frac{7\cdot6}{4\cdot6}+\frac{5\cdot8}{3\cdot8} \br 
	&= \frac{12}{24}-\frac{42}{24}+\frac{40}{24}	\br
	&= -\frac{10}{24}\br 
	&= -\frac{5}{12}
}
	}
\section{Varibaler}
\rg[Lage uttrykk]{
Når vi skal sette opp uttrykk, må vi tenke oss hvordan vi ville regnet ut det vi ønsker \textit{hvis} vi hadde visst om alle tallene.	
	}
\eks{
En klasse skal på busstur. For å leie buss og sjåfør må de betale 5000 kr. I tillegg må de betale 100 kr per person som skal bli med. Sett opp et uttrykk som viser hvor mye de må betale hvis $ x $ elever blir med. \\

\sv
Vi må betale $ 100 $ ganget med antall personer, pluss leieprisen på 5000 kr. Regnestykket vårt blir derfor:
\[ \text{Å betale}=100\cdot\text{antall elever}+5000 \] 
I oppgaven har de sagt at ''antall elever'' skal hete $ x $. Derfor får vi:
\[  \text{Å betale}=100\cdot x+5000 \] 
Husk at $ 100\cdot x = 100x $, derfor skriver vi helst:
\[  \text{Å betale}=100x+5000 \] 
	}
\section{Ligninger}
\rg[Flytting av ledd over likhetstegnet \label{bytt}]{Når vi har en ligning, ønsker vi å samle alle $x$-er og alle kjente tall på hver sin side av likhetstegnet. Dersom et tall skifter side, må også fortegnet skiftes. 	}

\eks[1]{ \vspace{-20pt}
	\begin{flalign*}
&&	3x+3 &=2x+5 && \text{Flytter } 2x \text{ til venstre og }  3\text{ til høyre}\\
&&	3x-2x &=5-3 && \text{og skifter fortegn}\\ 
&&	x &=2 &&
	\end{flalign*}  \vspace{-20pt}}
	
	\eks[2]{ \vspace{-20pt}
		\begin{align*}
		-4x-3 &=-5x+12 \\
		-4x+5x &=12+3 \\
		x &=15
		\end{align*}  \vspace{-20pt}	}
\rg[Deling på begge sider av likhetstegnet]{	Når alle $x$-er og alle kjente tall er samlet på hver sin side av likhetstegnet, deler vi begge sider med tallet 
	foran $x$ for å finne endelig svar. }
\eks[1]{ \vspace{-20 pt}
	\begin{align*}
	4x &= 20 \\
	\frac{\bcancel{4}x}{\bcancel{4}}&=\frac{20}{4} \\
	x &=5
	\end{align*}
	\vspace{-20 pt}
}

\eks[2]{\vspace{-20 pt}
	\begin{flalign*}
	\qquad \qquad && 2x+6 &=3x-2 && \\
	&& 2x-3x &= -2-6 &&\\
	&&-x &= -8 &&\\
	&& \frac{\bcancel{-1}x}{\bcancel{-1}} &= \frac{-8x}{-1} &&
	\text{Husk at } -x=-1x \\
	&& x &= 8
	\end{flalign*}\vspace{-20 pt}}
\rg[Multiplikasjon med fellesnevner]{Når vi har ligninger med brøker, kan vi forenkle uttrykket ved å multiplisere venstre og høyre side med en fellesnevner. }

\eks{ Løs ligningen
	\[ \frac{1}{3}x+\frac{1}{6}=\frac{5}{12}x+2 \]
	\sv
	Vi forenkler ligningen ved å gange med fellesnevneren 12 på begge sider:
	\begin{align*}
	\frac{1}{3}x+\frac{1}{6}&=\frac{5}{12}x+2 \\
	12\left(\frac{1}{3}x+\frac{1}{6}\right)&=12\left(\frac{5}{12}x+2\right) \\
	4x+2 &= 5x+24 \\
	4x-5x &= 24-2 \\
	-x &= 22 \\
	\frac{\bcancel{-1}\,x}{-1} &= \frac{22}{-1} \\
	x &= -22
	\end{align*}\vspace{-20pt}}

\chapter{Prosentregning og forhold}
\section{Prosentregning}
\rg[Prosentdel av en mengde]{Prosentdelen av en mengde finner vi  ved å gange  mengden med prosentverdien, og dele på hundre:
	$$\mathrm{prosentdel}=\frac{\mathrm{mengde}\cdot\text{prosentverdi}}{100}$$
}

\eks[1]{
	Finn $\qquad\;\mathclap{\overbrace{50}^\text{prosentverdi}\qquad}$\%\; av $\overbrace{500}^\text{mengde}$. \\
	
	\sv \vs
	\[ \frac{500\cdot 50}{100}=\underbrace{250}_\mathrm{prosentdel} \]}

\eks[2]{
	Finn 2\% av 10. \\
	
	\sv \vs
	\[ \frac{10\cdot 2}{100}=0,2 \]}

\rg[Prosentfaktor]{Delingen på hundre når vi finner prosentdeler av en mengde 
	kan vi regne ut i hodet, og derfor skrive et mer
	kompakt uttrykk. Prosentverdien delt på 100 kaller vi for \textit{prosentfaktoren}}
\eks[1]{	
	\textbf{a)} Skriv 2\%  som prosentfaktor. \\
	\textbf{b)} Finn 2\% av 10. \\
	
	\sv 
	\textbf{a)} \[ 2\% = 0,02 \] \\
	\textbf{b)} \[ 10\cdot0,02 = 0,2 \]
}

\eks[2]{	Finn 50\% av 500. \\
	
	\sv \vs
	\[ 500\cdot0,5=250 \]
}



\rg[Prosentverdi av en mengde]{For å finne prosentverdien en delmengde utgjør av en mengde, må vi gange delmengden med 100, og deretter dele med mengden: 
	$$\frac{\mathrm{delmengde}\cdot 100}{\mathrm{mengde}}=\mathrm{prosentverdi}$$ \\
}
\eks[1]{Hvor mange prosent utgjør $\overbrace{40}^\mathrm{delmengde}$ av $\overbrace{200}^\mathrm{mengde}$? \\
	
	\sv \vs
	\[ \frac{40\cdot 100}{200}=20 \]
	40 utgjør altså 20\% av  
}

\eks[2]{	
	Hvor mange prosent utgjør 35 av 10? \\
	
	\sv \vs
	\[ \frac{35\cdot 100}{10}=30 \]}
\rg[Prosentvis økning]{	Når en mengde skal økes med en prosentverdi, finner vi den økte mengden ved å multiplisere mengden med 1 pluss prosentfaktoren til økningen:
	$$\text{økt mengde}=\mathrm{mengde}\cdot(1+\mathrm{prosentfaktor})$$}

\eks{En vare verd 1000 kr får en prisøkning på 20\%. Finn den nye prisen.\\
	
	\sv
	Prosentfaktoren til økningen er 0,2. Økt pris blir derfor lik $$1000\cdot(1+0.2)=1000\cdot1.2=1200$$
	Ny pris er altså 1200 kr.}

\rg[Prosentivs reduksjon]{	Når en mengde skal reduseres med en prosentverdi, finner vi den reduserte mengden ved å multiplisere mengden med 1 minus prosentfaktoren til økningen:
	$$\mathrm{redusert\; mengde}=\mathrm{mengde}\cdot(1-\mathrm{prosentfaktor})$$}

\eks{ En vare verd $\underbrace{1000}_\mathrm{mengde}$kr får en prisreduksjon på $\quad\mathclap{\underbrace{40}_{\text{prosentverdi til reduksjon}}}\;\;\;$\%. Finn den nye prisen.\\
	
	\textbf{Svar:} Prosentfaktoren til reduksjonen er 0,4. Redusert pris blir derfor lik $$1000\cdot(1-0.4)=1000\cdot0.6=600$$
	Ny pris er altså 600 kr.}

\section{Forhold}
\rg[Forholdet mellom to tall]{
	Forholdet mellom to tall $ a $ og $ b $ skrives som:
	\[ \frac{a}{b} \]
	eller:
	\[ a:b \]
	eller:
	\[ \text{''}a\text{ på }b\text{''}  \]
	Når man regner ut $ a:b $, sier vi at vi har funnet \textit{forholdstallet}.
}
\eks[]{I en blanding er forholdet mellom saft og vann 1:4. \\
	
	\textbf{a)} Hva er forholdstallet mellom saft og vann? \\
	
	I blandingen er det 16 deler vann.\\
	
	\textbf{b)} Hvor mange deler saft er det i blandingen? \\
	
	\sv
	\textbf{a)} Forholdstallet er:
	\[ 1:4 = 0.25 \]
	\textbf{b)} 16 deler vann må bety at:
	\alg{\frac{\text{saft}}{16}&=\frac{1}{4} \br
		\frac{\text{saft}}{\cancel{16}}\cdot\cancel{16}&=\frac{1}{4} \cdot16 \\
		\text{saft}&= 4
	}
	Det er altså 4 deler vann i blandingen.
}
\end{document}


