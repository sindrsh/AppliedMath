\input{/home/sindre/P/doc}
\usepackage[T1]{fontenc}
\usepackage[utf8]{luainputenc}
\usepackage{lmodern} % load a font with all the characters
\usepackage{geometry}
\geometry{verbose,paperwidth=16.1 cm, paperheight=24 cm, inner=2.3cm, outer=1.8 cm, bmargin=2cm, tmargin=1.8cm}
\setlength{\parindent}{0bp}
\usepackage{import}
\usepackage[subpreambles=false]{standalone}
\usepackage{amsmath}
\usepackage{amssymb}
\usepackage{esint}
\usepackage{babel}
\usepackage{tabu}
\usepackage[dvipsnames, table]{xcolor}
\makeatother
\makeatletter


%referances
\newcommand{\net}[2]{{\color{blue}\href{#1}{#2}}}

%Spaces
\newcommand{\vsk}{\\[12pt]}
\newcommand{\vs}{\vspace{-12pt}}

% Tabell for opplegg

\newcommand{\ovlist}[1]{
\vspace{-16pt}
\begin{itemize}
	#1
\end{itemize}
}

\newcommand{\lst}[5]{
\rule{\linewidth}{1pt}
\footnotesize
	\textbf{Øvingsområde}\\ #1 
	
	\textbf{Utstyr}\\ #2  \\
	
	\begin{tabular}{@{} p{4cm} l} 
		\textbf{Tid} & \textbf{Elevinndeling} \\
		#3  & #4
	\end{tabular} 

\rule{\linewidth}{1pt}	\vsk
\normalsize
	\textbf{Gjennomføring}\\ #5 \vsk
}
%

\newcounter{opl}
%\numberwithin{opl}{article}

\newcommand{\opl}[1]{
\newpage
{\refstepcounter{opl} %\phantomsection 
\large \textbf{\theopl \;#1} \vsk}
}

% Headlines
\newcommand{\fork}{\textbf{Forkunnskapar}\\}
\newcommand{\forb}{\textbf{Forberedelsar}\\}
\newcommand{\opgvr}{\textbf{Oppgaver}}

\usepackage{datetime2}
\usepackage[]{hyperref}

\begin{document}
\section{Titalls-systemet}
Hvor mange fingre har vi? Hvordan kan to personer vise tallet 15 med fingrene?\\

én tier, fem enere = 15\\
Tall kan bety så mye; 7 kroner, 15 hester, 9 sekunder osv.\\
Tall som ruter, tall på en tallinje.
\section{Regneartene}
Starter med en vekt. Viser 0 = 0. 
\subsection{Legge på tall}
\[ 0+5=5 \]
Tenk at vi har 4 ruter på vekten vår, og så legger vi på 3 til. Vi kan nå telle at det ligger 7 kuler på venstresiden. Når vi skal skrive at vi legger til noe, bruker vi symbolet ''$ + $''.
\[ 0+3+4=7 \]
\[ 3+4=7 \]
''Tre og fire lagt sammen''\\
'' Tre pluss fire''\\
''Tre tillagt fire''\\
'Tre lagt sammen med fire''

\subsection{Trekke ifra tall}
- Ta bort, 
- Til venstre på tallinjen.
\subsection{Gange sammen tall}
Det kan skje at vi skal legge til det samme tallet flere ganger, for eksempel:
\[ +2+2+2=+6 \]
Altså toere lagt til tre ganger. For å skrive regnestykket vårt på en kortere måte bruker vi symbolet ''$ \cdot $'':(Obs! Engelskmenn bruker ''$ \times $'').
\[ \begin{array}{c}
2+2+2=6 \br
\text{er det samme som:}\br
2\cdot3=6
\end{array} \]
\eks[]{blbal} \vsk
\textbf{\boldmath $ 2\cdot3$ er det samme som $3\cdot2 $} \os

\textbf{På tallinjen}\os
$ 2\cdot3 = \;$gå to bort tre ganger.
\subsection{Ganging med paranteser}
Planker med lengde 2 og 5, vi har 3 av hver.
\[ \text{2 og 5 lagt sammen, og etterpå ganget med 3}=(2+5)\cdot3 \]
\alg{
(2+5)\cdot3 &= 7\cdot3 \\
&= 21
}
\[ \text{2 ganget med 3 pluss 5 ganget med 3} \]
\alg{
2\cdot3 +5\cdot3 &= 6 + 15\\
&= 21
}
\textbf{Gangemetoden}
\alg{
 25\cdot3 &= (10\cdot2+1\cdot5)\cdot3 \\
 &=10\cdot2\cdot+1\cdot 5\cdot3 \\
 &= 10\cdot 6 + 1\cdot15 \\
 &= 10\cdot 6 + 10\cdot1+5\cdot1
}
7 tiere og 5 enere $ = $ 75.

\subsection{Dele med tall}
\begin{itemize}
	\item Hva blir 20 delt i 5 like biter?
	\item Hvor mange ganger ''går'' 5 på 20?
	\item Hvor mange femmere må jeg legge på for å få 20?
\end{itemize}
\textsl{Obs!} Siste punkt er det samme som ''Hva må jeg gange 5 med for å få 20?''
\subsection{Negative tall}
Motsatte. Ballonger går oppover.  0 Kan være så mye! $ 7-7 $
\[ +9+(-7)=9-7\]
\[ +5+(-8)=5-8=-3 \]

\[ +5+(-5)-8=-8 \]
\[ +5+(-5)-8-(-5)=+5-8 \]
\subsection{Ganging med negative tall}
\alg{
(-2)\cdot3& = (-2)+(-2)+(-2) \\
&= -6
}
\reg[Legge på neg tall]{Å legge på $ -2 $ er det samme som å trekke ifra 2:
\[ +(-2)=-2 \]
}
\alg{
2\cdot(-3)=-2-2-2
}
\subsection{Deling med negative tall}
\begin{itemize}
	\item Hva blir 20 delt i 5 like biter?
	\item Hvor mange ganger ''går'' 5 på 20?
	\item Hvor mange femmere må jeg legge på \textit{eller} trekke ifra for å få 20?
\end{itemize}
${ (-12):(-4)  = 3}$ ''hvor mange $ (-4) $-ere må jeg legge på eller trekke ifra for å få $ (-20) $? Svar: legge på tre $ (-4) $-ere:
\[ +(-4)+(-4)+(-4)=(-12) \]

$ {(-12):4  = (-3)}$ '' hvor mange $ 4 $-ere må jeg legge sammen eller trekke ifra for å få $ -(20) $? Svar: trekke ifra 3 4-ere.:
\[ -4-4-4=(-12) \tag{\color{blue}$ 4\cdot(-3)=-12  $}\]
$ {12:(-4)  = (-3)}$ '' hvor mange $ 4 $-ere må jeg legge sammen eller trekke ifra for å få $ -(20) $? Svar: trekke ifra tre 4-ere.:
\[ -4-4-4=-12 \]
\section*{Dele \textit{a} på \textit{b} for prosentfaktor}
\end{document}

