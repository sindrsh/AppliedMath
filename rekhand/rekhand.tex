\input{../doc}
\usepackage[T1]{fontenc}
\usepackage[utf8]{luainputenc}
\usepackage{lmodern} % load a font with all the characters
\usepackage{geometry}
\geometry{verbose,paperwidth=16.1 cm, paperheight=24 cm, inner=2.3cm, outer=1.8 cm, bmargin=2cm, tmargin=1.8cm}
\setlength{\parindent}{0bp}
\usepackage{import}
\usepackage[subpreambles=false]{standalone}
\usepackage{amsmath}
\usepackage{amssymb}
\usepackage{esint}
\usepackage{babel}
\usepackage{tabu}
\usepackage[dvipsnames, table]{xcolor}
\makeatother
\makeatletter


%referances
\newcommand{\net}[2]{{\color{blue}\href{#1}{#2}}}

%Spaces
\newcommand{\vsk}{\\[12pt]}
\newcommand{\vs}{\vspace{-12pt}}

% Tabell for opplegg

\newcommand{\ovlist}[1]{
\vspace{-16pt}
\begin{itemize}
	#1
\end{itemize}
}

\newcommand{\lst}[5]{
\rule{\linewidth}{1pt}
\footnotesize
	\textbf{Øvingsområde}\\ #1 
	
	\textbf{Utstyr}\\ #2  \\
	
	\begin{tabular}{@{} p{4cm} l} 
		\textbf{Tid} & \textbf{Elevinndeling} \\
		#3  & #4
	\end{tabular} 

\rule{\linewidth}{1pt}	\vsk
\normalsize
	\textbf{Gjennomføring}\\ #5 \vsk
}
%

\newcounter{opl}
%\numberwithin{opl}{article}

\newcommand{\opl}[1]{
\newpage
{\refstepcounter{opl} %\phantomsection 
\large \textbf{\theopl \;#1} \vsk}
}

% Headlines
\newcommand{\fork}{\textbf{Forkunnskapar}\\}
\newcommand{\forb}{\textbf{Forberedelsar}\\}
\newcommand{\opgvr}{\textbf{Oppgaver}}

\usepackage{datetime2}
\usepackage[]{hyperref}

\begin{document}
\section*{Addisjon}

\subsubsection{Oppstilling}
\parbox{0.3\linewidth}{
\eks[1]{
	\begin{figure}
		\centering
		\includegraphics[]{fig/plus1}
	\end{figure}
}
}\qquad
\parbox{0.3\linewidth}{
\eks[2]{
	\begin{figure}
		\centering
		\includegraphics[]{fig/plus2}
	\end{figure}
}
}\\[12pt]
\parbox{0.3\linewidth}{
\eks[3]{
	\begin{figure}
		\centering
		\includegraphics[]{fig/plus3}
	\end{figure}
}}\qquad
\parbox{0.3\linewidth}{
\eks[4]{
	\begin{figure}
		\centering
		\includegraphics[]{fig/plus4}
	\end{figure}
}}
\fork{Eksempel 1}{
\begin{figure}
	\centering
	\subfloat[]{\includegraphics{fig/plus1a}}\qquad
	\subfloat[]{\includegraphics{fig/plus1b}}\qquad
	\subfloat[]{\includegraphics{fig/plus1c}}
\end{figure}

\begin{enumerate}[label=\alph*)]
	\item Vi legger sammen enerne: $ 4+2=6 $
	\item Vi legger sammen tierne: $ 3+1=4 $
	\item Vi legger sammen hundrerne: $ 2+6=8 $
\end{enumerate}
} \regv
\fork{Eksempel 2}{
	\begin{figure}
		\centering
		\subfloat[]{\includegraphics{fig/plus2a}}\qquad
		\subfloat[]{\includegraphics{fig/plus2b}}\qquad
		\subfloat[]{\includegraphics{fig/plus2c}}
	\end{figure}
	
	\begin{enumerate}[label=\alph*)]
		\item Vi legger sammen enerne: $ 3+6=9 $
		\item Vi legger sammen tierne: $ {7+8=15} $. Siden 10 tiere er det samme som 100, legger vi til 1 på hundreplassen, og skriver opp de resterende 5 tierne på tierplassen.
		\item Vi legger sammen hundrerne: $ 1+2=3 $.
	\end{enumerate}
}
\section{Subtraksjon}
\subsubsection{Oppstilling}
\parbox{0.3\linewidth}{
\eks[1]{
	\begin{figure}
		\centering
		\includegraphics[]{fig/min1}
	\end{figure}
}} \qquad
\parbox{0.3\linewidth}{
\eks[2]{
	\begin{figure}
		\centering
		\includegraphics[]{fig/min2}
	\end{figure}
}} \\[12pt]
\parbox{0.3\linewidth}{
\eks[3]{
	\begin{figure}
		\centering
		\includegraphics[]{fig/min3}
	\end{figure}
}}\qquad
\parbox{0.3\linewidth}{
\eks[4]{
	\begin{figure}
		\centering
		\includegraphics[]{fig/min4}
	\end{figure}
}}
\subsubsection{Tabellmetoden}
\parbox{0.35\linewidth}{
\eks[1]{
$ \colb{789}-\colr{324}=\colc{465} $	 \vsk

\begin{tabular}{r|r}
	& \colr{324} \\ \hline
	6&330 \\
	70&400 \\
	389&\colb{789} \\ \hline
	\colc{465}
	
\end{tabular}
}} \qquad
\parbox{0.35\linewidth}{
	\eks[2]{
		$ \colb{83}-\colr{67}=\colc{16} $	 \vsk
		
		\begin{tabular}{r|r}
			& \colr{67} \\ \hline
			3&70 \\
			13&\colb{83} \\ \hline
			\colc{16}
		\end{tabular} \vspace{14pt}
}} \\[12pt]
\parbox{0.35\linewidth}{
	\eks[3]{
		$ 564-478=86 $	 \vsk
		
		\begin{tabular}{r|r}
			& 478 \\ \hline
			2&480 \\
			20&500 \\ 
			64&564\\ \hline
			86
		\end{tabular} \vspace{14pt}
}}

\section*{Ganging}
\eks[1]{
\begin{figure}
	\centering
	\includegraphics[]{fig/gang1}
\end{figure}
}
\eks[1]{
\begin{figure}
	\centering
	\includegraphics[]{fig/gfleirsif}
\end{figure}
}
\eks[2]{
	\begin{figure}
		\centering
		\includegraphics[]{fig/gfleirsifa}
	\end{figure}
}
\end{document}

