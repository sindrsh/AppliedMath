\input{../doc}
\usepackage[T1]{fontenc}
\usepackage[utf8]{luainputenc}
\usepackage{lmodern} % load a font with all the characters
\usepackage{geometry}
\geometry{verbose,paperwidth=16.1 cm, paperheight=24 cm, inner=2.3cm, outer=1.8 cm, bmargin=2cm, tmargin=1.8cm}
\setlength{\parindent}{0bp}
\usepackage{import}
\usepackage[subpreambles=false]{standalone}
\usepackage{amsmath}
\usepackage{amssymb}
\usepackage{esint}
\usepackage{babel}
\usepackage{tabu}
\usepackage[dvipsnames, table]{xcolor}
\makeatother
\makeatletter


%referances
\newcommand{\net}[2]{{\color{blue}\href{#1}{#2}}}

%Spaces
\newcommand{\vsk}{\\[12pt]}
\newcommand{\vs}{\vspace{-12pt}}

% Tabell for opplegg

\newcommand{\ovlist}[1]{
\vspace{-16pt}
\begin{itemize}
	#1
\end{itemize}
}

\newcommand{\lst}[5]{
\rule{\linewidth}{1pt}
\footnotesize
	\textbf{Øvingsområde}\\ #1 
	
	\textbf{Utstyr}\\ #2  \\
	
	\begin{tabular}{@{} p{4cm} l} 
		\textbf{Tid} & \textbf{Elevinndeling} \\
		#3  & #4
	\end{tabular} 

\rule{\linewidth}{1pt}	\vsk
\normalsize
	\textbf{Gjennomføring}\\ #5 \vsk
}
%

\newcounter{opl}
%\numberwithin{opl}{article}

\newcommand{\opl}[1]{
\newpage
{\refstepcounter{opl} %\phantomsection 
\large \textbf{\theopl \;#1} \vsk}
}

% Headlines
\newcommand{\fork}{\textbf{Forkunnskapar}\\}
\newcommand{\forb}{\textbf{Forberedelsar}\\}
\newcommand{\opgvr}{\textbf{Oppgaver}}

\usepackage{datetime2}
\usepackage[]{hyperref}

\begin{document}
\section{Addisjon}

\subsubsection{Oppstilling}
Addisjon med oppstilling baserer seg på plassverdisystemet, der man trinnvis rekner summen av enerne, tierne, hundrerne, o.l.
\begin{center}
	\parbox{0.3\linewidth}{
\eks[1]{
	\begin{figure}
		\centering
		\includegraphics[]{fig/plus1}
	\end{figure}
}
}\qquad
\parbox{0.3\linewidth}{
\eks[2]{
	\begin{figure}
		\centering
		\includegraphics[]{fig/plus2}
	\end{figure}
}
}\\[12pt]
\parbox{0.3\linewidth}{
\eks[3]{
	\begin{figure}
		\centering
		\includegraphics[]{fig/plus3}
	\end{figure}
}}\qquad
\parbox{0.3\linewidth}{
\eks[4]{
	\begin{figure}
		\centering
		\includegraphics[]{fig/plus4}
	\end{figure}
}}
\end{center}
\fork{Eksempel 1}{
\begin{figure}
	\centering
	\subfloat[]{\includegraphics{fig/plus1a}}\qquad
	\subfloat[]{\includegraphics{fig/plus1b}}\qquad
	\subfloat[]{\includegraphics{fig/plus1c}}
\end{figure}

\begin{enumerate}[label=\alph*)]
	\item Vi legger sammen enerne: $ 4+2=6 $
	\item Vi legger sammen tierne: $ 3+1=4 $
	\item Vi legger sammen hundrerne: $ 2+6=8 $
\end{enumerate}
} \newpage
\fork{Eksempel 2}{
	\begin{figure}
		\centering
		\subfloat[]{\includegraphics{fig/plus2a}}\qquad
		\subfloat[]{\includegraphics{fig/plus2b}}\qquad
		\subfloat[]{\includegraphics{fig/plus2c}}
	\end{figure}
	
	\begin{enumerate}[label=\alph*)]
		\item Vi legger sammen enerne: $ 3+6=9 $
		\item Vi legger sammen tierne: $ {7+8=15} $. Siden 10 tiere er det samme som 100, legger vi til 1 på hundreplassen, og skriver opp de resterende 5 tierne på tierplassen.
		\item Vi legger sammen hundrerne: $ 1+2=3 $.
	\end{enumerate}
}
\section{Subtraksjon}
\subsubsection{Oppstilling}
Subtraksjon med oppstilling baserer seg på plassverdisystemet, der man trinnvis rekner differansen mellom enerne, tierne, hundrerne, o.l. Metoden tar også utgangspunkt i et mengdeperspektiv, og tillater derfor ikke differanser med negativ verdi (se forklaringen til \textsl{Eksempel 2}).
\begin{center}
	\parbox{0.3\linewidth}{
\eks[1]{
	\begin{figure}
		\centering
		\includegraphics[]{fig/min1}
	\end{figure}
}} \qquad
\parbox{0.3\linewidth}{
\eks[2]{
	\begin{figure}
		\centering
		\includegraphics[]{fig/min2}
	\end{figure}
}} \\[12pt]
\parbox{0.3\linewidth}{
\eks[3]{
	\begin{figure}
		\centering
		\includegraphics[]{fig/min3}
	\end{figure}
}}\qquad
\parbox{0.3\linewidth}{
\eks[4]{
	\begin{figure}
		\centering
		\includegraphics[]{fig/min4}
	\end{figure}
}}

\end{center}
\fork{Eksempel 1}{
\begin{figure}
	\centering
	\subfloat[]{\includegraphics{fig/min1a}}\qquad
	\subfloat[]{\includegraphics{fig/min1b}}\qquad
	\subfloat[]{\includegraphics{fig/min1c}}
\end{figure}

\begin{enumerate}[label=\alph*)]
	\item Vi finner differansen mellom enerne: $ {9-4=5} $
	\item Vi finner differansen mellom tierne: $ {8-2=6} $. 
	\item Vi finner differansen mellom hundrerne: $ {7-3=4} $.
\end{enumerate}
}
\newpage
\fork{Eksempel 2}{
	\begin{figure}
		\centering
		\subfloat[]{\includegraphics{fig/min2a}}\qquad
		\subfloat[]{\includegraphics{fig/min2b}}
	\end{figure}
	
	\begin{enumerate}[label=\alph*)]
		\item Vi merker oss at 7 er større enn 3, derfor tar vi 1 tier fra de 8 på tierplassen. Dette markerer vi ved å sette en strek over 8. Så finner vi differansen mellom enerne: $ {13-7=6} $
		\item Siden vi tok 1 fra de 8 tierne, er der nå bare 7 tiere. Vi finner differansen mellom tierne: $ {7-6=1} $.
	\end{enumerate}
}
\subsubsection{Tabellmetoden}
Tabellmetoden for subtraksjon tar utgangspunkt i at subtraksjon er en omvendt operasjon av addisjon. For eksempel, svaret på spørsmålet ''Hva er $ 789-324 $?'' er det samme som svaret på spørsmålet ''Hvor mye må jeg legge til på 324 for å få 789?''. Med tabellmetoden følger du ingen spesiell regel underveis, men velger selv tallene du mener passer best for å nå målet.\\
\begin{center}
\parbox{0.35\linewidth}{
\eks[1]{
$ \colb{789}-\colr{324}=\colc{465} $	 \vsk

\begin{tabular}{r|r}
	& \colr{324} \\ \hline
	6&330 \\
	70&400 \\
	389&\colb{789} \\ \hline
	\colc{465}
\end{tabular}
}} \qquad
\parbox{0.35\linewidth}{
	\eks[2]{
		$ \colb{83}-\colr{67}=\colc{16} $	 \vsk
		
		\begin{tabular}{r|r}
			& \colr{67} \\ \hline
			3&70 \\
			13&\colb{83} \\ \hline
			\colc{16}
		\end{tabular} \vspace{14pt}
}} \\[12pt]
\parbox{0.35\linewidth}{
	\eks[3]{
		$ 564-478=86 $\vsk
		
		\begin{tabular}{r|r}
			& 478 \\ \hline
			2&480 \\
			20&500 \\ 
			64&564\\ \hline
			86
		\end{tabular}
}} \qquad 
\parbox{0.4\linewidth}{
	\eks[4]{
		$ {206,1-31,7=174,4} $\vsk
		
		\begin{tabular}{r|r}
			& 31,7 \\ \hline
			0,3& 32\phantom{,0} \\
			70\phantom{,0}&102\phantom{,0} \\ 
			104,1&206,1\\ \hline
			174,4
		\end{tabular}
}}
\end{center}
\fork{Eksempel 1}{
\begin{figure}
	\centering
	\subfloat[]{
	\begin{tabular}{r|r}
		& \colr{324} \\ \hline
		& \\
		& \\
		& \\ \hline
		&
	\end{tabular}	
} \qquad
	\subfloat[]{
	\begin{tabular}{r|r}
		& \colb{324} \\ \hline
	   \colb{6}& \colo{330}\\
		& \\
		& \\ \hline
		&
	\end{tabular}
}\qquad
	\subfloat[]{
	\begin{tabular}{r|r}
		& 324 \\ \hline
		6& \colb{330}\\
		\colb{70}& \colo{400} \\
		& \\ \hline
		&
	\end{tabular}	
}  \\[12pt]
\subfloat[]{
	\begin{tabular}{r|r}
		& 324 \\ \hline
		6& 330\\
		70& \colb{400} \\
		\colb{389}& \colo{789}\\ \hline
		&
	\end{tabular}	
}\qquad
\subfloat[]{
	\begin{tabular}{r|r}
		& 324 \\ \hline
		\colb{6}& 330\\
		\colb{70}& 400 \\
		\colb{389}& 789\\ \hline
		\colo{465}&
	\end{tabular}	
}
\end{figure}
\begin{enumerate}[label=(\alph*)]
	\item Vi starter med 324.
	\item Vi legger til 6, og får $ {324+6=330} $
	\item Vi legger til 70, og får $ {70+330=400} $
	\item Vi legger til 389, og får $ {389+400=789} $. Da er vi framme på 789.
	\item Vi adderer tallene vi har lagt til:  $ {6+70+389=465} $
\end{enumerate}
}
\section{Ganging}
\subsubsection{Ganging med 10, 100, 1\,000 osv.}
\subsubsection{Utvidet form}
Ganging på utvidet form baserer seg på distributiv lov (se \mb\,, s. 30).
\eks[1]{
\begin{figure}
	\centering
	\includegraphics[]{fig/gang1}
\end{figure}
}
\eks[2]{
\begin{figure}
	\centering
	\includegraphics[]{fig/gfleirsif}
\end{figure}
}
\subsubsection{Kompaktmetoden}
Kompaktmetoden bygger på de samme prinsippene som ganging på utvidet form, men har en skrivemåte som gjør utrekningen kortere.
\eks[1]{
	\begin{figure}
		\centering
		\includegraphics[]{fig/gfleirsifa}
	\end{figure}
}
\section{Divisjon}
\subsubsection{Deling med 10, 100, 1\,000 osv.}
\subsubsection{Oppstilling}
Divisjon med oppstilling baserer seg på divisjon tolket som inndeling av mengder (se \mb\,,s. 23)

\begin{center}
	\parbox{0.3\linewidth}{
	\eks[1]{ \vsb
		\begin{figure}
			\centering
			\includegraphics[]{fig/del1}
		\end{figure} \vspace{18pt}
	}
}\qquad
\parbox{0.45\linewidth}{
	\eks[1]{ \vspace{-5pt}
		\begin{figure}
			\centering
			\includegraphics[]{fig/del2}
		\end{figure}
	}
}
\end{center}
\subsubsection{Tabellmetoden}
Tabellmetoden baserer seg på divisjon som omvendt operasjon av ganging. For eksempel er svaret på spørsmålet ''Hva er $ {76:4} $'' det samme som svaret på spørsmålet ''Hvilket tall må jeg gange 4 med for å få 76?''. På samme vis som for tabellmetoden ved subtraksjon er det opp til en selv å velge passende tall for å nå målet.
\begin{center}
	\parbox{0.35\linewidth}{
		\eks[1]{
			$ \colg{76}:\colb{4}=\colo{19} $	 \vsk
			
			\begin{tabular}{r|r|r}
				$ \cdot\, \colb{4} $&\\ \hline
				10&40&40 \\
				9& 36 &\colg{76} \\ \hline
				\colo{19}&
			\end{tabular}
		\vspace{42pt}
	}} \qquad
\parbox{0.35\linewidth}{
	\eks[2]{
		$ \colg{894}:\colb{3}=\colg{298} $	 \vsk
		
		\begin{tabular}{r|r|r}
			$ \cdot\, \colb{3} $&\\ \hline
			200& 600 &600 \\
			60&120 &720 \\
			60&120 &840 \\
			10& 30&870 \\
			8&24 &\colg{894} \\ \hline
			\colo{298} &
		\end{tabular}
}}  \newpage
\parbox{0.415\linewidth}{
	\eks[2 \\ (annen utrekning)]{
		$ 894:3=298 $	 \vsk
		
		\begin{tabular}{r|r|r}
			$ \cdot\, 3 $&\\ \hline
			300& 900&900 \\
			$ -2 $& $ -6 $ &894 \\ \hline
			298&
		\end{tabular}
}}
\end{center}
\end{document}

