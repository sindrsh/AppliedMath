\input{../doc}
\usepackage[T1]{fontenc}
\usepackage[utf8]{luainputenc}
\usepackage{lmodern} % load a font with all the characters
\usepackage{geometry}
\geometry{verbose,paperwidth=16.1 cm, paperheight=24 cm, inner=2.3cm, outer=1.8 cm, bmargin=2cm, tmargin=1.8cm}
\setlength{\parindent}{0bp}
\usepackage{import}
\usepackage[subpreambles=false]{standalone}
\usepackage{amsmath}
\usepackage{amssymb}
\usepackage{esint}
\usepackage{babel}
\usepackage{tabu}
\usepackage[dvipsnames, table]{xcolor}
\makeatother
\makeatletter


%referances
\newcommand{\net}[2]{{\color{blue}\href{#1}{#2}}}

%Spaces
\newcommand{\vsk}{\\[12pt]}
\newcommand{\vs}{\vspace{-12pt}}

% Tabell for opplegg

\newcommand{\ovlist}[1]{
\vspace{-16pt}
\begin{itemize}
	#1
\end{itemize}
}

\newcommand{\lst}[5]{
\rule{\linewidth}{1pt}
\footnotesize
	\textbf{Øvingsområde}\\ #1 
	
	\textbf{Utstyr}\\ #2  \\
	
	\begin{tabular}{@{} p{4cm} l} 
		\textbf{Tid} & \textbf{Elevinndeling} \\
		#3  & #4
	\end{tabular} 

\rule{\linewidth}{1pt}	\vsk
\normalsize
	\textbf{Gjennomføring}\\ #5 \vsk
}
%

\newcounter{opl}
%\numberwithin{opl}{article}

\newcommand{\opl}[1]{
\newpage
{\refstepcounter{opl} %\phantomsection 
\large \textbf{\theopl \;#1} \vsk}
}

% Headlines
\newcommand{\fork}{\textbf{Forkunnskapar}\\}
\newcommand{\forb}{\textbf{Forberedelsar}\\}
\newcommand{\opgvr}{\textbf{Oppgaver}}

\usepackage{datetime2}
\usepackage[]{hyperref}

\begin{document}

\section{Brøkdeler av helheter} 
I \mb\;(s. 35\,-\,47) har vi sett hvordan brøker er definert ut ifra en inn-\\deling av 1. I hverdagen bruker vi også brøker for å snakke om inn-delinger av en helhet:
\begin{figure}
	\centering
	\subfloat[]{\includegraphics{\asym{br1}}}\qquad\qquad
	\subfloat[]{\includegraphics{\asym{br1a}}}\qquad \qquad
	\subfloat[]{\includegraphics{\asym{br1b}}}
\end{figure}
\begin{center}
	\begin{enumerate}[label=(\textsl{\alph*})]
		\item Helheten er 8 ruter. $ \frac{7}{8} $ av rutene er blå. 
		\item Helheten er et kvadrat. $ \frac{1}{4} $ av kvadratet er rødt.
		\item Helheten er 5 kuler. $ \frac{3}{5} $ av kulene er svarte.
	\end{enumerate}
\end{center}
\subsection*{Brøkdeler av tall}
Si at rektangelet under har verdien $ 12 $. 
\fig{br2}
Når vi sier ''$\frac{\colb{2}}{\colc{3}}$ av \colr{12}'' mener vi at vi skal
\begin{enumerate}[label=\alph*)]
	\item dele \colr{12} inn i \colc{3} like grupper
	\item finne ut hvor mye \colb{2} av disse gruppene utgjør til sammen.
\end{enumerate}
Vi har at
\begin{enumerate}[label=\alph*)]
	\item $ 12 $ delt inn i 3 grupper er lik $ 12:3=4 $.
	\fig{br2a}
	\item 2 grupper som begge har verdi 4 blir til sammen $ 2\cdot4=8 $.
	\fig{br2b}
\end{enumerate}
Altså er
\[ \frac{2}{3}\text{ av } 12= 8 \]
\newpage
For å finne $ \frac{2}{3} $ av 12, delte vi 12 med 3, og ganget kvotienten med 2. Dette er det samme som å gange $ 12 $ med $ \frac{2}{3} $ (se \mb, s. 45 og 50).\regv

\reg[Brøkdelen av et tall \label{brokdelavtall}]{
For å finne brøkdelen av et tall, ganger vi brøken med tallet.\os
\[ \frac{a}{b} \text{ av } c=\frac{a}{b}\cdot c \]
}
\eks[1]{
Finn $ \frac{2}{5} $ av 15.

\sv \vsb
\[ \frac{2}{5}\text{ av } 15=\frac{2}{5}\cdot 15= 6\]
}
\eks[2]{
	Finn $ \frac{7}{9} $ av $ \frac{5}{6} $.
	
	\sv \vsb
	\[ \frac{7}{9}\text{ av } \frac{5}{6}=\frac{7}{9}\cdot \frac{5}{6}= \frac{35}{54}\]
} 
\section{Prosent} \index{prosent}
\parbox[l][][l]{0.65\linewidth}{
Brøker er ypperlige til å oppgi andeler av en helhet fordi de gir et raskt bilde av hvor mye det er snakk om. For eksempel er lett å se (omtrent) hvor mye $ \frac{3}{5} $ eller $ \frac{7}{12} $ av en kake er. Men ofte er det ønskelig å raskt avgjøre hvilke andeler som utgjør \textsl{mest}, og da er det best om brøkene har samme nevner. }
\parbox[r][][l]{0.3\linewidth}{
	\begin{figure}
		\centering
		\includegraphics[scale=0.1]{\asym{kake}}
\end{figure}} \\[12pt]

Når andeler oppgis i det daglige, er det vanlig å bruke brøker med 100 i nevner. Brøker med denne nevneren er så mye brukt at de har fått sitt eget navn og symbol.
\reg[Prosenttall \label{prosenttall}]{ \vs
\[ a\% = \frac{a}{100} \]
}
\spr{
\sym{\%} uttales \textit{prosent}. Ordet kommer av det latinske \textit{per centum}, som betyr \textit{per hundre}.
}
\eks[1]{ \vs
\[ 43\%=\frac{43}{100} \]
}
\eks[2]{ \vs
\[ 12,7\% = \frac{12,7}{100} \]
\mer Det er kanskje litt uvant, men ikke noe galt med å ha et desimaltall i teller (eller nevner).
}
\newpage
\eks[3]{
Gjør om brøkene til prosenttall.\os
\textbf{a)} $ \dfrac{34}{100} $\\[12pt]
\textbf{b)} $ \dfrac{203}{100} $

\sv \vsk

\textbf{a)} $ \dfrac{34}{100}=34\% $\\[12pt]
\textbf{b)} $ \dfrac{203}{100}=203\% $
}

\eks[4]{
	Finn 50\% av 800.
	Av \rref{brokdelavtall} og \rref{prosenttall} har vi at
	
	\sv \vsb
	\[ 50\% \text{ av } 800=\frac{50}{100}\cdot 800=400 \]
}

\eks[5]{
	Finn 2\% av 7,4. 
	
	\sv \vsb 
	\[ 2\%\text{ av }7,4= \frac{2}{100}\cdot 7,4=0,148 \]
}
\info{Tips}{Å dele med 100 er såpass enkelt, at vi gjerne kan uttrykke prosenttall som desimaltall når vi foretar utrekninger. I \textsl{Eksempel 5} over kunne vi har reknet slik:
\[ 2\% \text{ av } 7,4 = 0,02\cdot 7,4 =0,148 \]
}
\newpage
\subsection*{Prosentdeler}
Hvor mange prosent utgjør 15 av 20?\vsk

15 er det samme som $ \frac{15}{20} $ av 20, så svaret på spørsmålet får vi ved å gjøre om $ \frac{15}{20} $ til en brøk med 100 i nevner. Siden $ 20\cdot\frac{100}{20}=100 $, utvider vi brøken vår med $ \frac{100}{20}=5 $:
\alg{
\frac{15\cdot5}{20\cdot 5}= \frac{75}{100}
}
15 utgjør altså 75\% av 20. Det er verdt å merke seg at vi kunne fått 75 direkte ved utrekningen
\[ 15\cdot \frac{100}{20}=75 \]
\reg[Antall prosent \boldmath $ a $ utgjør av $ b $ \label{proaavb}]{
	\vs
	\begin{equation}
		\text{Antall prosent \textit{a} utgjør av \textit{b}}=a\cdot \frac{100}{b}
	\end{equation}
}
\eks[1]{
Hvor mange prosent utgjør \colb{340} av \colc{400}? 

\sv \vsb
\[ \colb{340} \cdot \frac{100}{\colc{400}}=85 \]
340 utgjør 85\% av 400.
}
\eks[2]{
Hvor mange prosent utgjør 119 av 500?

\sv \vsb
\[ 119\cdot \frac{100}{500}=23,8 \]
119 utgjør 23,8\% av 500.
}
\info{Tips}{
Å gange med 100 er såpass enkelt å ta i hodet at man kan ta det bort fra selve utrekningen. \textsl{Eksempel 2} over kunne vi da reknet slik:
\[ \frac{119}{500}=0,238 \]
119 utgjør altså 23,8\% av 500. \\(Her rekner man i hodet at $ 0,238\cdot100=23,8 $.)
}


\section{Prosentvis endring og vekstfaktor \label{vekstfaktor}}
\subsubsection{Minkende størrelse} \label{skjorte} \vspace{-25pt}
\parbox[l][][l]{0.75\linewidth}{
I mange situasjoner har noe økt eller minket med en viss prosent. I en butikk kan man for eksempel komme over en skjorte som originalt kostet 500 kr, men selges med 40\% \textit{rabatt}. Dette betyr at vi skal trekke ifra 40\% av originalprisen. }
\parbox[r][][l]{0.2\linewidth}{
	\begin{figure}
		\centering
		\includegraphics[scale=0.4]{\asym{sale}}
\end{figure}} \\[-10pt]
Her er to måter vi kan tenke på for å finne prisen:
\begin{itemize}
	\item Vi starter med å finne det vi skal trekke ifra:
	\alg{
		\text{40\% av 500}&=\frac{40}{100}\cdot500\br
		&=200
	}
	Videre er
	\[ 500-200=300 \]
	Altså må vi betale 300 kr for skjorten.
	\item \phantom{}\\[-12pt]
	\prbxl{0.65}{Skal vi betale full pris, må vi betale 100\% av 500. Men får vi 40\% i rabatt, skal vi bare betale $60\%$ av 500:}\qquad
	\prbxr{0.25}{$ {100\%-40\%=60\% }$}
	\alg{
		\text{60\% av 500}&=\frac{60}{100}\cdot500 \br
		&= 300 
	}
	Svaret blir selvsagt det samme, vi må betale 300 kr for skjorten.\vsk
\end{itemize} 
\newpage
\subsubsection{Økende størrelse} \label{oreklokke}
\parbox[l][][l]{0.6\linewidth}{Det er ikke alltid vi er så heldige at vi får rabatt på et produkt, ofte må vi faktisk betale et tillegg. \net{https://www.skatteetaten.no/bedrift-og-organisasjon/avgifter/mva/slik-fungerer-mva/}{{Mervardiavgiften}} er et slikt tillegg. I Norge må vi betale 25\% i merverdiavgift på mange varer. Det betyr\vspace{1pt}}\qquad
\prbxr{0.3}{
		Merverdiavgift \\forkortes til mva.
} \\[-9pt]
\prbxl{0.65}{
 at vi må betale et tillegg på 25\%, altså $125\%$ av originalprisen.}\qquad
\prbxr{0.25}{$ {100\%+25\%=125\%} $}

\parbox[l][][l]{0.485\linewidth}{
	For eksempel koster øreklokkene på bildet 999,20 kr \textsl{eksludert} mva. Men \textsl{inkludert} mva. må vi betale
	\alg{
		\text{125\% av 999,20}&=\frac{125}{100}\cdot999,20\br
		&= 1249
	}
	Altså 1249 kr.\vsk
}\quad
\parbox[r][][l]{0.55\linewidth}{
	{\vspace{4pt}
		\includegraphics[scale=0.4]{\asym{peltor}}}}
\vsk
\subsubsection{Oppsummering}
\reg[Prosentvis endring]{\vs
	\begin{itemize}
		\item Når en størrelse synker med $ a $\%, ender vi opp med $ (100\% - a\%) $ av størrelsen.
		
		\item Når en størrelse øker med $ a $\%, ender vi opp med $ (100\% + a\%) $ av størrelsen. 
	\end{itemize}
}
\eks[1]{ \label{vekstfakteks}
Hva er \colb{210} senket med \colr{70}\%?

\sv
$ 100\%-\colr{70}\%=\colc{30}\% $, altså er
\alg{
\colb{210}\text{ senket med } \colr{70}\% &=\colc{30}\% \text{ av } \colb{210} \br &=\frac{\colc{30}}{100}\cdot\colb{210}\br
&=63 
}
}
\eks[2]{
Hva er 208,9 økt med 124,5\%?

\sv

$ 100\%+124,5\%=224,5\% $, altså er
\alg{
208,9 \text{ økt med } 124,5 &= 224,5\% \text{ av } 208,9 \br
&=\frac{224,5}{100}\cdot208,9
}
}
\section*{Vekstfaktor} \vspace{-12pt}
\prbxl{0.6}{På side \pageref{skjorte} ble prisen til en skjorte redusert med 40\%, og da endte vi opp med å betale 60\% av originalprisen. Vi sier da at \textit{vekstfaktoren} er 0,6. På side \pageref{oreklokke} måtte vi legge til 25\% mva., og da endte vi med å betale 125\% av originalprisen. Da er vekstfaktoren 1,25.}\qquad
\prbxr{0.3}{Mange stusser over at ordet vekstfaktor brukes selv om en størrelse \textsl{synker}, men slik er det. Kanskje et bedre ord ville være \textit{endringsfaktor}?}

\reg[Vekstfaktor I \label{vekstfaktordef}]{
Når en størrelse endres med $ a\% $, er vekstfaktoren verdien til\footnote{\sym{$ \pm $} betyr at man skal velge enten $ + $ eller $ - $.} $ {100\% \pm a\%} $.
}
\eks[1]{
En størrelse skal økes med 15\%. Hva er vekstfaktoren?

\sv
$ 100\%+15\% =115\% $, altså er vekstfaktoren 1,15.
}
\eks[2]{
En størrelse skal reduseres med 19,7\%. Hva er vekstfaktoren?

\sv
$ 100\%-19,7\%=80,3\% $, altså er vekstfaktoren 80,3\%
} \vsk
La oss se tilbake til \textsl{Eksempel 1} på side \pageref{vekstfakteks}, hvor 210 skulle senkes med 70\%. Da er vekstfaktoren 0,3. Videre er
\[ 0,3\cdot210=63 \]
Altså, for å finne ut hvor mye 210 senket med 70\% er, kan vi gange 210 med vekstfaktoren (forklar for deg selv hvorfor!).
\reg[Prosentvis endring med vekstfaktor \label{vekstfaktendr}]{ \vs
\[ \text{endret originalverdi}=\text{vekstfaktor}\cdot \text{originalverdi} \]	
}

\eks[1]{ En vare verd 1\,000 kr er rabattert med 20\%.\os
	\textbf{a)} Hva er vekstfaktoren?\os
	\textbf{b)} Finn den nye prisen.
	
	\sv 
	\textbf{a)} Siden det er 20\% rabbatt, må vi betale 
	\[ 100\%-20\%= 80\% \]
	av originalprisen. Vekstfaktoren er derfor 0,8. \os
	
	\textbf{b)} Vi har at
	\[ 0,8\cdot1000  = 800 \]
	Den nye prisen er altså 800 kr.
}
\eks[2]{En sjokolade koster 9,80 kr, ekskludert mva. På matvarer er det 15\% mva. \os
	\textbf{a)} Hva er vekstfaktoren?	\os
	\textbf{b)} Hva koster sjokoladen inkludert mva?
	
	\sv
	\textbf{a)} Med 15\% i tillegg må man betale
	\[ 100\%+15\%= 115\% \]
	av prisen eksludert mva. Vekstfaktoren er derfor 1,15.\vsk
	
	\textbf{b)} \vs
	\[ 1,15\cdot 9.90=12,25 \]
	Sjokoladen koster 12,25 kr inkludert mva.
} \vsk
Vi kan også omksrive likningen fra \rref{vekstfaktendr} for å få et uttrykk for vekstfaktoren:
\reg[Vekstfaktor II \label{vekstfaktexpr}]{ \vs
\[ \text{vekstfaktor}=\frac{\text{endret originalverdi}}{\text{originalverdi}} \]
}
\subsubsection{Å finne den prosentvise endringen}
Når man skal finne en prosentvis endring, er det viktig å være klar over at det er snakk om prosent \textsl{av} en helhet. Denne helheten man har som utgangspunkt er den originale verdien. \vsk

La oss som et eksempel si at Jakob tente 10\,000 kr i 2019 og 12\,000 kr i 2020. Vi kan da stille spørsmålet ''Hvor mye endret lønnen til Jakob seg med fra 2019 til 2020 i prosent?''. Spørsmålet tar utgangspunkt i lønnen fra 2019, dette betyr at 10\,000 er vår originale verdi. To måter å finne den prosentvise endringen på er disse (vi tar ikke med 'kr' i utrekningene):
\begin{itemize}
\item Lønnen til Jakob endret seg fra 10\,000 til 12\,000, en forandring på $12\,000-10\,000= 2\,000 $. Videre er (se \rref{proaavb})
\alg{
\text{Antall prosent 2\,000 utgjør av 10\,000}&=2\,000\cdot\frac{100}{10\,000} \\
&=20
}
Fra 2019 til 2020 økte altså lønnen til Jakob med 20\%. 
\item 
Vi har at
\alg{
\frac{12\,000}{10\,000}=1,2
}
Fra 2019 til 2020 økte altså lønnen til Jakob med en vekstfaktor lik 1,2 (se \rref{vekstfaktexpr}). Denne vekstfaktoren tilsvarer en endring lik 20\% (se \rref{vekstfaktordef}). Det betyr at lønnen økte med 20\%.
\end{itemize}
\reg[Prosentvis endring I \label{proendra}]{ \vs
\[ \text{prosentvis endring}=\frac{\text{endret originalverdi}-\text{originalverdi}}{\text{originalverdi}}\cdot100 \]
\footnotesize
Obs! Hvis 'endret originalverdi' er mindre enn 'original verdi', må man i steden rekne ut '$ \text{originalverdi}-\text{endret original} $'
} 
\info{Kommentar}{\rref{proendra} kan se litt voldsom ut, og er ikke nødvendigvis så lett å huske. Hvis du virkelig har forstått seksjon ??, kan du uten å bruke denne formelen finne prosentvise endringer trinnvis. I påfølgende eksempler viser vi derfor både en trinnvis løsningsmetode og en metode ved bruk av formel.} 
\eks[1]{
I 2019 hadde et fotballag 20 medlemmer. I 2020 hadde laget 12 medlemmer. Hvor mange prosent av medlemmene fra 2019 hadde sluttet i 2020?

\sv	

Vi starter med å merke oss at det det medlemstallet fra 2019 som er originalverdien vår.\vsk

\textit{Løsningsmetode 1; trinnvis metode}\os
Fotballaget gikk fra å ha 20 til 12 medlemmer, altså var det $ 20-12=8 $ som sluttet. Vi har at
\[ \text{Antall prosent 4 utgjør av 20}=8\cdot\frac{100}{20}=40 \]
I 2020 hadde altså 40\% av medlemmene fra 2019 sluttet. \vsk

\textit{Løsningsmetode 2; formel} \os
Vi legger merke til at originalverdien er større enn den endrede verdien, da har vi at
\alg{
\text{prosentvis endring}&=\frac{20-12}{20}\cdot100\br
&=\frac{8}{20}\cdot 100 \br
&=40
}
I 2020 hadde altså 40\% av medlemmene fra 2019 sluttet.
} \regv

\reg[Prosentvis endring II \label{proendrb}]{\vs
	\[ \text{prosentvis endring}=100\left(\frac{\text{endret originalverdi}}{\text{originalverdi}}-1\right) \]
	\footnotesize
	Obs! Hvis 'endret originalverdi' er mindre enn 'original verdi', må man ''snu'' reknestykket inni parantesen til $ 1-\dfrac{\text{endret originalverdi}}{\text{originalverdi}} $.
}
\info{Merk}{\rref{proendra} og \rref{proendrb} gir begge formler som kan brukes til å finne prosentvise endringer. Her er det opp til en selv å velge hvilken man liker best. Som allerede nevnt angående \rref{proendra}, er også \rref{proendrb} en litt kronglete formel, og man trenger den ikke hvis man har forstått seksjonn ?? og ??. Her vil vi også i påfølgende eksempler vise to løsningsmetoder. 
} 
\eks[1]{
	I 2019 hadde et fotballag 20 medlemmer. I 2020 hadde laget 12 medlemmer. Hvor mange prosent av medlemmene fra 2019 hadde sluttet i 2020?
	
	\sv	
	
	Vi starter med å merke oss at det det medlemstallet fra 2019 som er originalverdien vår. \vsk
	
	\textit{Løsningsmetode 1; trinnvis metode}\os
	Fotballaget gikk fra å ha 20 til 12 medlemmer, da har vi at (se \rref{vekstfaktexpr})
	\[ \text{vekstfaktor}=\frac{12}{20}=0,6 \]
	En vekstfaktor lik 0,6 tilsvarer en endring på 40\% (se \rref{vekstfaktordef}). 	I 2020 hadde altså 40\% av medlemmene fra 2019 sluttet. \vsk
	
	\textit{Løsningsmetode 2; formel} \os
	Vi legger merke til at originalverdien er større enn den endrede verdien, da har vi at
	\alg{
		\text{prosentvis endring}&=100\left(1-\frac{12}{20}\right)\\
		&=100\left(1-0,6\right) \\
		&=100\cdot0.4 \\
		&=40
	}
	I 2020 hadde altså 40\% av medlemmene fra 2019 sluttet.
}
\section{Prosentpoeng}
\prbxl{0.65}{Ofte snakker vi om mange størrelser samtidig, og når man da bruker prosent-ordet kan setninger bli veldig lange og knotete hvis man også snakker om forskjellige originalverdier. For å forenkle setningene har vi begrepet \textit{prosentpoeng}. La oss forsøke å forklare ordet med et eksempel.}
\fgbxr{0.25}{\begin{figure}
		\centering
		\includegraphics[scale=0.3]{\asym{sunglasses}}
\end{figure}} 
\prbxl{0.65}{
Tenk at et par solbriller først ble solgt med 30\% rabatt av originalprisen, men det solgt med 80\% rabatt av originalprisen. Da sier vi at rabatten har økt med 50 \textit{prosentpoeng}.
} \qquad
\prbxr{0.25}{
$ 80\%-30\%=50\% $ 
} \vsk

\textsl{Hvorfor kan vi ikke si at rabatten har økt med 50\%?}\vsk

Si at solbrillene hadde originalpris 1\,000 kr.
30\% rabatt på 1\,000 kr tilsvarer 300 kr i rabatt. 80\% rabatt på 1000 kr tilsvarer 800 kr i rabatt. Men hvis vi øker 300 med 50\% får vi $ 300\cdot1,5=450 $, og det er ikke det samme som 800! Saken er at vi snakker om to forskjellige originalverdier:

\prbxr{0.25}{ \vs
	\alg{
		80\%\cdot1000= 800 
	} \vsb
}
\prbxr{0.25}{ \vs
	\alg{
		20\%\cdot1000= 200 
	} \vsb}


\st{''Rabatten var først 30\%, så økte rabatten med 50 prosentpoeng. Da ble rabatten 80\%.'' \vsk

\textit{Forklaring:} ''Rabatten'' er en størrelse vi rekner ut i fra orignalprisen til solbrillene. Når vi sier ''prosentpoeng'' viser vi til at det fortsatt er originalprisen vi skal rekne prosentene \textsl{ut i fra}. Når prisen er 1\,000 kr, starter vi med $ {1\,000\enh{kr}\cdot0,3=300\enh{kr}} $  i rabatt. Når vi legger til 50 \textsl{prosentpoeng}, legger vi til 50\% av originalprisen, altså $ 1\,000\enh{kr}\cdot0,5=500\enh{kr} $. Totalt blir det 800\,kr i rabatt, som utgjør $ 80\% $ av originalprisen.
}
\st{''Rabatten var først 30\%, så økte rabatten med 50\%. Da ble rabatten 45\%.''\vsk

\textit{Forklaring:} ''Rabatten'' er en størrelse vi rekner ut i fra orignalprisen til solbrillene, men her rekner vi prosenten \textsl{ut ifra } rabatten. Når prisen er 1\,000\,kr, starter vi med 300\,kr i rabatt. Videre er
\[ 300\enh{kr} \text{ økt med } 50\%=300\enh{kr}\cdot1,5=450\enh{kr} \]
og 
\[ \text{Antall prosent 450 utgjør av 1\,000}=\frac{450}{100}=45 \]
Altså er den nye rabatten 45\%.
}
\info{Merk}{
I de to (gule) tekstboksene over regnet vi ut den økte rabatten ved å bruke en originalverdi lik 1\,000. Dette er streng tatt ikke nødvendig
}



\newpage

\end{document}


