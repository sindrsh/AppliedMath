\input{../doc}
\usepackage[T1]{fontenc}
\usepackage[utf8]{luainputenc}
\usepackage{lmodern} % load a font with all the characters
\usepackage{geometry}
\geometry{verbose,paperwidth=16.1 cm, paperheight=24 cm, inner=2.3cm, outer=1.8 cm, bmargin=2cm, tmargin=1.8cm}
\setlength{\parindent}{0bp}
\usepackage{import}
\usepackage[subpreambles=false]{standalone}
\usepackage{amsmath}
\usepackage{amssymb}
\usepackage{esint}
\usepackage{babel}
\usepackage{tabu}
\usepackage[dvipsnames, table]{xcolor}
\makeatother
\makeatletter


%referances
\newcommand{\net}[2]{{\color{blue}\href{#1}{#2}}}

%Spaces
\newcommand{\vsk}{\\[12pt]}
\newcommand{\vs}{\vspace{-12pt}}

% Tabell for opplegg

\newcommand{\ovlist}[1]{
\vspace{-16pt}
\begin{itemize}
	#1
\end{itemize}
}

\newcommand{\lst}[5]{
\rule{\linewidth}{1pt}
\footnotesize
	\textbf{Øvingsområde}\\ #1 
	
	\textbf{Utstyr}\\ #2  \\
	
	\begin{tabular}{@{} p{4cm} l} 
		\textbf{Tid} & \textbf{Elevinndeling} \\
		#3  & #4
	\end{tabular} 

\rule{\linewidth}{1pt}	\vsk
\normalsize
	\textbf{Gjennomføring}\\ #5 \vsk
}
%

\newcounter{opl}
%\numberwithin{opl}{article}

\newcommand{\opl}[1]{
\newpage
{\refstepcounter{opl} %\phantomsection 
\large \textbf{\theopl \;#1} \vsk}
}

% Headlines
\newcommand{\fork}{\textbf{Forkunnskapar}\\}
\newcommand{\forb}{\textbf{Forberedelsar}\\}
\newcommand{\opgvr}{\textbf{Oppgaver}}

\usepackage{datetime2}
\usepackage[]{hyperref}

\begin{document}

\section{Brøkdeler av helheter} 
I \mb\;(s. 35\,-\,47) har vi sett hvordan brøker er definert ut ifra en inn-\\deling av 1. I hverdagen bruker vi også brøker for å snakke om inn-delinger av en helhet:
\begin{figure}
	\centering
	\subfloat[]{\includegraphics{\asym{br1}}}\qquad\qquad
	\subfloat[]{\includegraphics{\asym{br1a}}}\qquad \qquad
	\subfloat[]{\includegraphics{\asym{br1b}}}
\end{figure}
\begin{center}
	\begin{enumerate}[label=(\textsl{\alph*})]
		\item Helheten er 8 ruter. $ \frac{7}{8} $ av rutene er blå. 
		\item Helheten er et kvadrat. $ \frac{1}{4} $ av kvadratet er rødt.
		\item Helheten er 5 kuler. $ \frac{3}{5} $ av kulene er svarte.
	\end{enumerate}
\end{center}
\subsection*{Brøkdeler av tall}
Si at rektangelet under har verdien $ 12 $. 
\fig{br2}
Når vi sier ''$\frac{\colb{2}}{\colc{3}}$ av \colr{12}'' mener vi at vi skal
\begin{enumerate}[label=\alph*)]
	\item dele \colr{12} inn i \colc{3} like grupper
	\item finne ut hvor mye \colb{2} av disse gruppene utgjør til sammen.
\end{enumerate}
Vi har at
\begin{enumerate}[label=\alph*)]
	\item $ 12 $ delt inn i 3 grupper er lik $ 12:3=4 $.
	\fig{br2a}
	\item 2 grupper som begge har verdi 4 blir til sammen $ 2\cdot4=8 $.
	\fig{br2b}
\end{enumerate}
Altså er
\[ \frac{2}{3}\text{ av } 12= 8 \]
\newpage
For å finne $ \frac{2}{3} $ av 12, delte vi 12 med 3, og ganget kvotienten med 2. Dette er det samme som å gange $ 12 $ med $ \frac{2}{3} $ (se \mb, s. 45 og 50).\regv

\reg[Brøkdelen av et tall \label{brokdelavtall}]{
For å finne brøkdelen av et tall, ganger vi brøken med tallet.\os
\[ \frac{a}{b} \text{ av } c=\frac{a}{b}\cdot c \]
}
\eks[1]{
Finn $ \frac{2}{5} $ av 15.

\sv \vsb
\[ \frac{2}{5}\text{ av } 15=\frac{2}{5}\cdot 15= 6\]
}
\eks[2]{
	Finn $ \frac{7}{9} $ av $ \frac{5}{6} $.
	
	\sv \vsb
	\[ \frac{7}{9}\text{ av } \frac{5}{6}=\frac{7}{9}\cdot \frac{5}{6}= \frac{35}{54}\]
} 
\section{Prosent} \index{prosent}
\parbox[l][][l]{0.65\linewidth}{
Brøker er ypperlige til å oppgi andeler av en helhet fordi de gir et raskt bilde av hvor mye det er snakk om. For eksempel er lett å se (omtrent) hvor mye $ \frac{3}{5} $ eller $ \frac{7}{12} $ av en kake er. Men ofte er det ønskelig å raskt avgjøre hvilke andeler som utgjør \textsl{mest}, og da er det best om brøkene har samme nevner. }
\parbox[r][][l]{0.3\linewidth}{
	\begin{figure}
		\centering
		\includegraphics[scale=0.1]{\asym{kake}}
\end{figure}} \\[12pt]

Når andeler oppgis i det daglige, er det vanlig å bruke brøker med 100 i nevner. Brøker med denne nevneren er så mye brukt at de har fått sitt eget navn og symbol.
\reg[Prosenttall \label{prosenttall}]{ \vs
\[ a\% = \frac{a}{100} \]
}
\spr{
\sym{\%} uttales \textit{prosent}. Ordet kommer av det latinske \textit{per centum}, som betyr \textit{per hundre}.
}
\eks[1]{ \vs
\[ 43\%=\frac{43}{100} \]
}
\eks[2]{ \vs
\[ 12,7\% = \frac{12,7}{100} \]
\mer Det er kanskje litt uvant, men ikke noe galt med å ha et desimaltall i teller (eller nevner).
}
\newpage
\eks[3]{
Gjør om brøkene til prosenttall.\os
\textbf{a)} $ \dfrac{34}{100} $\\[12pt]
\textbf{b)} $ \dfrac{203}{100} $

\sv \vsk

\textbf{a)} $ \dfrac{34}{100}=34\% $\\[12pt]
\textbf{b)} $ \dfrac{203}{100}=203\% $
}

\eks[4]{
	Finn 50\% av 800.
	Av \rref{brokdelavtall} og \rref{prosenttall} har vi at
	
	\sv \vsb
	\[ 50\% \text{ av } 800=\frac{50}{100}\cdot 800=400 \]
}

\eks[5]{
	Finn 2\% av 7,4. 
	
	\sv \vsb 
	\[ 2\%\text{ av }7,4= \frac{2}{100}\cdot 7,4=0,148 \]
}
\info{Tips}{Å dele med 100 er såpass enkelt, at vi gjerne kan uttrykke prosenttall som desimaltall når vi foretar utrekninger. I \textsl{Eksempel 5} over kunne vi har reknet slik:
\[ 2\% \text{ av } 7,4 = 0,02\cdot 7,4 =0,148 \]
}
\newpage
\subsection*{Prosentdeler}
Hvor mange prosent utgjør 15 av 20?\vsk

15 er det samme som $ \frac{15}{20} $ av 20, så svaret på spørsmålet får vi ved å gjøre om $ \frac{15}{20} $ til en brøk med 100 i nevner. Siden $ 20\cdot\frac{100}{20}=100 $, utvider vi brøken vår med $ \frac{100}{20}=5 $:
\alg{
\frac{15\cdot5}{20\cdot 5}= \frac{75}{100}
}
15 utgjør altså 75\% av 20. Det er verdt å merke seg at vi kunne fått 75 direkte ved utrekningen
\[ 15\cdot \frac{100}{20}=75 \]
\reg[Antall prosent \boldmath $ a $ utgjør av $ b $]{
	\vs
	\begin{equation}
		\text{Antall prosent \textit{a} utgjør av \textit{b}}=a\cdot \frac{100}{b}
	\end{equation}
}
\eks[1]{
Hvor mange prosent utgjør \colb{340} av \colc{400}? 

\sv \vsb
\[ \colb{340} \cdot \frac{100}{\colc{400}}=85 \]
340 utgjør 85\% av 400.
}
\eks[2]{
Hvor mange prosent utgjør 119 av 500?

\sv \vsb
\[ 119\cdot \frac{100}{500}=23,8 \]
119 utgjør 23,8\% av 500.
}
\info{Tips}{
Å gange med 100 er såpass enkelt å ta i hodet at man kan ta det bort fra selve utrekningen. \textsl{Eksempel 2} over kunne vi da reknet slik:
\[ \frac{119}{500}=0,238 \]
119 utgjør altså 23,8\% av 500. \\(Her rekner man i hodet at $ 0,238\cdot100=23,8 $.)
}


\section{Vekstfaktor \label{vekstfaktor}}
\subsubsection{Minkende størrelse} \vspace{-25pt}
\parbox[l][][l]{0.75\linewidth}{
	I mange situasjoner har noe økt eller minket med en viss prosent. I en butikk kan man for eksempel komme over en skjorte som originalt kostet 500 kr, men selges med 40\% rabatt. Dette betyr at vi skal trekke ifra 40\% av originalprisen når vi skal betale. }
\parbox[r][][l]{0.2\linewidth}{
	\begin{figure}
		\centering
		\includegraphics[scale=0.4]{\asym{sale}}
\end{figure}} \\[-10pt]
Her er to måter vi kan tenke på for å finne prisen:
\begin{itemize}
	\item Vi starter med å finne fratrekket:
	\alg{
		\text{40\% av 500}&=\frac{40}{100}\cdot500\br
		&=200
	}
	Videre er $ {500-200=300 }$, altså må vi betale 300 kr for skjorten.
	\item \phantom{}\\[-12pt]
	\prbxl{0.65}{Skal vi betale full pris, må vi betale 100\% av 500. Men får vi 40\% i rabatt, skal vi bare betale $60\%$ av 500:}\qquad
	\prbxr{0.25}{$ {100\%-40\%=60\% }$}
	\alg{
		\text{60\% av 500}&=\frac{60}{100}\cdot500 \br
		&= 300 
	}
	Svaret blir selvsagt det samme, vi må betale 300 kr for skjorten.\vsk
\end{itemize} 
\subsubsection{Økende størrelse}
\parbox[l][][l]{0.6\linewidth}{Det er ikke alltid vi er så heldige at vi får rabatt på et produkt, ofte må vi faktisk betale et tillegg. \net{https://www.skatteetaten.no/bedrift-og-organisasjon/avgifter/mva/slik-fungerer-mva/}{{Mervardiavgiften}} er et slikt tillegg. I Norge må vi betale 25\% i merverdiavgift på mange varer. Det betyr\vspace{1pt}}\qquad
\prbxr{0.3}{
		Merverdiavgift \\forkortes til mva.
} \\[-9pt]
\prbxl{0.65}{
 at vi må betale et tillegg på 25\%, altså $125\%$ av originalprisen.}\qquad
\prbxr{0.25}{$ {100\%+25\%=125\%} $}

\parbox[l][][l]{0.485\linewidth}{
	For eksempel koster øreklokkene på bildet 999,20 kr \textsl{eksludert} mva. Men \textsl{inkludert} mva. må vi betale
	\alg{
		\text{125\% av 999,20}&=\frac{125}{100}\cdot999,20\br
		&= 1249
	}
	Altså 1249 kr.\vsk
}\quad
\parbox[r][][l]{0.55\linewidth}{
	{\vspace{4pt}
		\includegraphics[scale=0.4]{\asym{peltor}}}}
\vsk
\subsubsection{Oppsummering}
\prbxl{0.6}{Vi har nå sett på to eksempler. I det ene sank prisen på en vare, mens i det andre økte den. Når prisen sank med 40\%, endte vi opp med å betale 60\% av originalprisen. Vi sier da at \textit{vekstfaktoren} er 0,6. Når prisen økte med 25\%, endte vi opp med å betale 125\% av originalprisen. Da er vekstfaktoren 1,25.}\qquad
\prbxr{0.3}{Mange stusser over at ordet vekstfaktor brukes selv om en størrelse \textsl{synker}, men slik er det. Kanskje et bedre ord ville være \textit{endringsfaktor}?}
\regv
\reg[Vekstfaktor \label{vekstfakt}]{\vs
	\begin{itemize}
		\item Når en størrelse synker med $ a $\%, ender vi opp med $ {100\% - a\%} $ av størrelsen.
		
		\item Når en størrelse øker med $ a $\%, ender vi opp med $ {100\% + a\%} $ av størrelsen. 
		
		\item Verdien til $ {100\% - a\%} $ eller $ {100\% + a\%} $ kalles vekstfaktoren.
	\end{itemize}
}
\eks[1]{ En vare verd 1000 kr er rabattert med 20\%.\os
	\textbf{a)} Hva er vekstfaktoren?\os
	\textbf{b)} Finn den nye prisen.
	
	\sv 
	\textbf{a)} Siden det er 20\% rabbatt må vi betale $ {100\%-20\%= 80\%} $ av originalprisen. Vekstfaktoren er derfor 0,8. \os
	
	\textbf{b)} Den nye prisen finner vi ved å gange vekstfaktoren med originalprisen:
	\[ 0,8\cdot1000  = 800 \]
	Den nye prisen er altså 800 kr.
}
\eks[2]{En sjokolade koster 9,80 kr, ekskludert mva. På matvarer er det 15\% mva. \os
	\textbf{a)} Hva er vekstfaktoren?	\os
	\textbf{b)} Hva koster sjokoladen inkludert mva?
	
	\sv
	\textbf{a)} Med 15\% i tillegg må man betale $ {100\%+15\%= 115\%} $ av prisen eksludert mva. Vekstfaktoren er derfor 1,15.\vsk
	
	\textbf{b)} \vs
	\[ 1,15\cdot 9.90=12,25 \]
	Sjokoladen koster 12,25 kr inkludert mva.
}
\section{Prosentpoeng}
Vi har akkurat sett på størrelser som økte eller minket med en viss prosent. Men hvis størrelsen selv er oppgitt i prosent, må vi holde tunga rett i munnen. La oss bruke størrelsen 10\% som et eksempel.\vsk


\prbxl{0.6}{Hvis 10\% øker med 5\%, får vi:
	\alg{
		\text{10\% økt med 5\%} &= 10\%\cdot1,05\\
		&= 10,5\%
	}
	Men hvis 10\% istedenfor øker med 5 \textit{prosentpoeng}, ender vi med:
}\qquad
\prbxr{0.3}{I forrige seksjon fant vi at å øke en størrelse med 5\% er det samme som å gange størrelsen med 1,05}
\alg{
	\text{10\% økt med 5 prosentpoeng} &= 10\%+5\%\\
	&= 15\%}
10,5\% og 15\% er to helt forskjellige størrelser!\regv
\reg[Prosentpoeng]{\vs \vs
	\alg{
		&\text{a\% økt med \textit{b} prosentpoeng} = a\% + b\% \br
		&\text{a\% minket med \textit{b} prosentpoeng} = a\% - b\%
	}
}
\eks{
	En dag var 5\% av elevene på en skole borte. Dagen etter var 7,5\% av elevene borte. \os
	\textbf{a)} Hvor mye økte fraværet i prosentpoeng?\os
	\textbf{b)} Hvor mye økte fraværet i prosent?
	
	\sv
	\textbf{a)} $ {7,5\%-5\%=2,5\%} $, derfor har fraværet økt med 2,5 prosentpoeng. \vsk
	
	\textbf{b)} Her må vi svare på hvor mye endringen, altså 2,5\%, utgjør av 5\%. Dette er det samme som å finne hvor mye 2,5 utgjør av 5. (Se tilbake til ligning \eqref{autgavb}). 1\% av 5 er 0,05, derfor får vi:
	\alg{
		\text{Antall prosent 2,5 utgjør av 5}&=\frac{2,5}{0,05} \\
		&= 50
	}
	Altså har fraværet økt med 50\%.
}
\info{Hva er egentlig forskjellen mellom prosent og prosentfaktor?}{Tenk på en skjorte som koster 200 kr. Tenk så at det er gitt 20\% rabatt på dene skjorten, altså får man $ {200\, \text{kr}\cdot0,2= 40\text{ kr}}$ i avslag. Men si at rabatten blir endret til 50\% av originalprisen, da blir avslaget $ {200\, \text{kr}\cdot0,5= 100\text{ kr}}$. \vsk
	
	Rabatten har da gått opp fra 20\% til 50\% av originalprisen, fra 40 kr til 100 kr. En økning på 60 kr. Og nå kommer poenget: Istedenfor å spørre \textit{hvor mange prosent av originalprisen har rabatten økt?}, bruker vi ordet prosentpoeng. Det samme spørsmålet blir da \textit{hvor mange prosentpoeng har rabatten økt?} Svaret blir $ {50\%-20\% = 30\%} $, altså 30 prosentpoeng. (60 utgjør 30\% av 200).\vsk
	
	Når vi isteden spør \textit{hvor mye har rabatten økt i prosent?}, mener vi \textit{hvor mange prosent av originalrabatten har rabatten økt?}. Dette kan vi finne på to måter:\vsk
	
	\textsl{Metode 1}:
	Originalrabatten var på 40 kr og økte med 60 kr. Hvor mange prosent 60 kr ugjør av 40 kr kan vi regne ut slik:
	\begin{flalign*}
		&& \frac{60}{0,4}& =150 && \llap{\color{blue}\text{1\% av 40 er 0,4}}
	\end{flalign*}
	Rabatten økte altså med 150\%.\vsk
	
	\textsl{Metode 2}: \\
	Økningen i prosentpoeng er 30, og startrabatten var 20\%. Hvor mange prosent 30 utgjør av 20 er:
	\begin{flalign*}
		&& \frac{30}{20}& =1,5 &&  \br
		&& & =150\% && \llap{\color{blue}\text{Se tipset på s \pageref{protips}}}
	\end{flalign*}
	Rabatten økte med 150\%.
}

\end{document}


