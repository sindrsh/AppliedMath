\input{/home/sindre/P/doc}
\usepackage[T1]{fontenc}
\usepackage[utf8]{luainputenc}
\usepackage{lmodern} % load a font with all the characters
\usepackage{geometry}
\geometry{verbose,paperwidth=16.1 cm, paperheight=24 cm, inner=2.3cm, outer=1.8 cm, bmargin=2cm, tmargin=1.8cm}
\setlength{\parindent}{0bp}
\usepackage{import}
\usepackage[subpreambles=false]{standalone}
\usepackage{amsmath}
\usepackage{amssymb}
\usepackage{esint}
\usepackage{babel}
\usepackage{tabu}
\usepackage[dvipsnames, table]{xcolor}
\makeatother
\makeatletter


%referances
\newcommand{\net}[2]{{\color{blue}\href{#1}{#2}}}

%Spaces
\newcommand{\vsk}{\\[12pt]}
\newcommand{\vs}{\vspace{-12pt}}

% Tabell for opplegg

\newcommand{\ovlist}[1]{
\vspace{-16pt}
\begin{itemize}
	#1
\end{itemize}
}

\newcommand{\lst}[5]{
\rule{\linewidth}{1pt}
\footnotesize
	\textbf{Øvingsområde}\\ #1 
	
	\textbf{Utstyr}\\ #2  \\
	
	\begin{tabular}{@{} p{4cm} l} 
		\textbf{Tid} & \textbf{Elevinndeling} \\
		#3  & #4
	\end{tabular} 

\rule{\linewidth}{1pt}	\vsk
\normalsize
	\textbf{Gjennomføring}\\ #5 \vsk
}
%

\newcounter{opl}
%\numberwithin{opl}{article}

\newcommand{\opl}[1]{
\newpage
{\refstepcounter{opl} %\phantomsection 
\large \textbf{\theopl \;#1} \vsk}
}

% Headlines
\newcommand{\fork}{\textbf{Forkunnskapar}\\}
\newcommand{\forb}{\textbf{Forberedelsar}\\}
\newcommand{\opgvr}{\textbf{Oppgaver}}

\usepackage{datetime2}
\usepackage[]{hyperref}

\begin{document}
\opgt
\op{br1}
Finn verdien til brøkene:\os
\begin{tabular}{@{}l l l l l}
	\textbf{a)} $ \dfrac{1}{2} $ & 	
	\textbf{b)} $ \dfrac{1}{4} $ & 	
	\textbf{c)} $ \dfrac{1}{5} $ &
	\textbf{d)} $ \dfrac{10}{5} $  & 
	\textbf{e)} $ \dfrac{6}{8} $
\end{tabular}
\nes

\op{br2}
Regn ut:\os
\begin{tabular}{@{}l l l l}
	\textbf{a)} $ \dfrac{4}{3}\cdot5 $ & 	\textbf{b)} $ \dfrac{3}{5}\cdot(-6) $ 
\end{tabular}

\op{br5}
Finn $ \frac{2}{3} $ av 9.

\nes
\op{br3}
Regn ut:\os
\begin{tabular}{@{}l l}
	\textbf{a)} $ \dfrac{4}{3}:5 $ & 	\textbf{b)} $ \dfrac{3}{5}:(-6) $
\end{tabular}

\nes
\op{br4}
Regn ut:\os
\begin{tabular}{@{}l l}
	\textbf{a)} $ \dfrac{4}{3}\cdot\dfrac{5}{9} $ & 	\textbf{b)} $ \dfrac{7}{8}\cdot\dfrac{2}{4} $ 
\end{tabular}

\op{br6}
Finn $ \frac{2}{3} $ av $ \frac{4}{5} $.

\nes
\nes

\op{br7}
Forkort brøkene:\os
\begin{tabular}{@{}l l}
	\textbf{a)} $ \dfrac{28}{16} $ & 	\textbf{b)} $ \dfrac{12}{36} $ 
\end{tabular}

\op{br8}
\textbf{a)} Uvid $ \frac{2}{3} $ til en brøk med 24 som nevner.\os
\textbf{b)} Utvid $ \frac{11}{9} $ til en brøk med 27 som nevner. 
\newpage
\nes
\op{br9}
Regn ut:\os
\begin{tabular}{@{}l l }
\textbf{a)}	$ \dfrac{2}{5}+\dfrac{5}{6} $&\textbf{b)} $ \dfrac{2}{3}+\dfrac{1}{2}-\dfrac{3}{4} $
\end{tabular}

\nes
\op{br10}
Regn ut:\os
\begin{tabular}{@{}l l }
	\textbf{a)}	$ \dfrac{2}{5}:\dfrac{5}{6} $&\textbf{b)} $ \dfrac{12}{3}\cdot\dfrac{3}{2} $
\end{tabular}

\op{br13}
Se tilbake til svarene i oppgave \ref{br1}a)-c). Fyll inn tallet som mangler der det står ''\_'' i setningene under:
{\renewcommand{\labelenumi}{(\alph{enumi})}
\begin{enumerate}
	\item Å dele med $ 0,5 $ er det samme som å gange med \_\,.
	\item Å dele med $ 0,25 $ er det samme som å gnage med \_\,.
	\item Å dele med $ 0,2 $ er det samme som å gnage med \_\,.	
	
\end{enumerate} }

\op{br11}
Regn ut:\os
\begin{tabular}{@{}l l l}
	\textbf{a)}	$ \dfrac{1}{2}\cdot h $
	& \textbf{b)} $ \dfrac{1}{2}: h  $ 
	& \textbf{c)} $ \dfrac{\pi}{2}\cdot\dfrac{a}{b} $
\end{tabular}

\op{br12}
Forkort brøken:
\[ \frac{2\pi a c}{4c\pi} \]

\nes
\op{forh2}
Finn forholdet og forholdstallet mellom antall hester og griser når vi har:\os
\begin{tabular}{@{}l l l}	
	\textbf{a)} 5 hester og 2 griser. &\textbf{b)} 12 griser og 4 hester.
\end{tabular}

\op{forh2}
Totaktsmotorer krever som regel bensin som er tilsatt en viss mengde motorolje. STHIL er en produsent av motorsager drevet av slike motorer, på deres hjemmesider kan vi lese dette:
\begin{figure}
	\includegraphics[]{stihl}
\end{figure}
Si at vi skal fylle på 2,5\,L bensin på motorsage vår, hvor mye olje må vi da tilsette?

\vsk \vspace{12pt}
\begin{comment}
Oppgave om hvilket dyr som er sterkest i forhold til vekten. Skaraben er verdens sterkeste.
\end{comment}
\textsl{Merk:} I de to neste oppgavene går vi ut ifra at både 1\,dL vann og 1\,dL saftsirup veier 100\,g.

\op{forh}
Coca-Cola inneholder 10\,g karbohydrater. En type saftsirup inneholder 44\,g karbohydrater per 100\,g. Saften skal lages med 2 deler sirup og 9 deler vann. \os

Inneholder saften mer eller mindre karbohydrater per 100\,g enn Coca-Cola?

\op{forh?}
På \textsl{Lærums solbærsirup} står det at 100\,g ferdig utblandet saft inneholder 12,5\,g sukker. Saften inneholder sirup og vann blandet i forholdet $ {1:5} $. \os

Hvor mye sukker inneholder 100\,g solbærsirup? (Rent vann inneholder ikke sukker i det hele tatt).
\end{document}

